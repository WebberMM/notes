% Created 2020-09-29 二 14:09
% Intended LaTeX compiler: pdflatex
\documentclass{article}
\usepackage[utf8]{inputenc}
\usepackage[T1]{fontenc}
\usepackage{graphicx}
\usepackage{grffile}
\usepackage{longtable}
\usepackage{wrapfig}
\usepackage{rotating}
\usepackage[normalem]{ulem}
\usepackage{amsmath}
\usepackage{textcomp}
\usepackage{amssymb}
\usepackage{capt-of}
\usepackage{hyperref}
\usepackage{minted}
%%%%%%%%%%%%%%%%%%%%%%%%%%%%%%%%%%%%%%
%% TIPS                                 %%
%%%%%%%%%%%%%%%%%%%%%%%%%%%%%%%%%%%%%%
% \substack{a\\b} for multiple lines text

\usepackage[utf8]{inputenc}

\usepackage[B1,T1]{fontenc}

% pdfplots will load xolor automatically without option
\usepackage[dvipsnames]{xcolor}
%%%%%%%%%%%%%%%%%%%%%%%%%%%%%%%%%%%%%%%
%% MATH related pacakge                  %%
%%%%%%%%%%%%%%%%%%%%%%%%%%%%%%%%%%%%%%%
% \usepackage{amsmath} mathtools loads the amsmath
\usepackage{amsmath}
\usepackage{mathtools}


\usepackage{amsthm}
\usepackage{amsbsy}

%\usepackage{commath}

\usepackage{amssymb}
\usepackage{mathrsfs}
%\usepackage{mathabx}
\usepackage{stmaryrd}
\usepackage{empheq}

\usepackage{scalerel}
\usepackage{stackengine}
\usepackage{stackrel}

\usepackage{nicematrix}
\usepackage{tensor}
\usepackage{blkarray}
\usepackage{siunitx}
\usepackage[f]{esvect}

\usepackage{unicode-math}
\setmainfont{TeX Gyre Pagella}
% \setmathfont{STIX}
% \setmathfont{texgyrepagella-math.otf}
% \setmathfont{Libertinus Math}
\setmathfont{Latin Modern Math}
\setmathfont[range={\mscra,\mscrb,\mscrc,\mscrd,\mscre,\mscrf,\mscrg,\mscrh,\mscri,\mscrj,\mscrk,\mscrl,\mscrm,\mscrn,\mscro,\mscrp,\mscrq,\mscrr,\mscrs,\mscrt,\mscru,\mscrv,\mscrw,\mscrx,\mscry,\mscrz,\mscrA,\mscrB,\mscrC,\mscrD,\mscrE,\mscrF,\mscrG,\mscrH,\mscrI,\mscrJ,\mscrK,\mscrL,\mscrM,\mscrN,\mscrO,\mscrP,\mscrQ,\mscrR,\mscrS,\mscrT,\mscrU,\mscrV,\mscrW,\mscrX,\mscrY,\mscrZ}]{Latin Modern Math}
\setmathfont[range={\smwhtdiamond,\enclosediamond,\varlrtriangle}]{Latin Modern Math}
\setmathfont[range={\rightrightarrows,\twoheadrightarrow,\leftrightsquigarrow,\triangledown}]{XITS Math}
\setmathfont[range={\int,\setminus}]{Libertinus Math}



%%%%%%%%%%%%%%%%%%%%%%%%%%%%%%%%%%%%%%%
%% TIKZ related packages                 %%
%%%%%%%%%%%%%%%%%%%%%%%%%%%%%%%%%%%%%%%

\usepackage{pgfplots}
\pgfplotsset{compat=1.15}
\usepackage{tikz}
\usepackage{tikz-cd}
\usepackage{tikz-qtree}

\usetikzlibrary{arrows,positioning,calc,fadings,decorations,matrix,decorations,shapes.misc}
%setting from geogebra
\definecolor{ccqqqq}{rgb}{0.8,0,0}


%%%%%%%%%%%%%%%%%%%%%%%%%%%%%%%%%%%%%%%
%% MISCLELLANEOUS packages               %%
%%%%%%%%%%%%%%%%%%%%%%%%%%%%%%%%%%%%%%%
\usepackage[most]{tcolorbox}
\usepackage{threeparttable}
\usepackage{tabularx}

\usepackage{enumitem}

% wrong with preview
\usepackage{subcaption}
\usepackage{caption}
% {\aunclfamily\Huge}
\usepackage{auncial}

\usepackage{float}

\usepackage{fancyhdr}

\usepackage{ifthen}
\usepackage{xargs}


\usepackage{imakeidx}
\usepackage{hyperref}
\usepackage{soul}


%\usepackage[xetex]{preview}
%%%%%%%%%%%%%%%%%%%%%%%%%%%%%%%%%%%%%%%
%% USEPACKAGES end                       %%
%%%%%%%%%%%%%%%%%%%%%%%%%%%%%%%%%%%%%%%

% \setlist{nosep}
% \numberwithin{equation}{subsection}
% \fancyhead{} % Clear the headers
% \renewcommand{\headrulewidth}{0pt} % Width of line at top of page
% \fancyhead[R]{\slshape\leftmark} % Mark right [R] of page with Chapter name [\leftmark]
% \pagestyle{fancy} % Set default style for all content pages (not TOC, etc)


% \newlength\shlength
% \newcommand\vect[2][0]{\setlength\shlength{#1pt}%
%   \stackengine{-5.6pt}{$#2$}{\smash{$\kern\shlength%
%     \stackengine{7.55pt}{$\mathchar"017E$}%
%       {\rule{\widthof{$#2$}}{.57pt}\kern.4pt}{O}{r}{F}{F}{L}\kern-\shlength$}}%
%       {O}{c}{F}{T}{S}}


\indexsetup{othercode=\small}
\makeindex[columns=2,options={-s /media/wu/file/stuuudy/notes/index_style.ist},intoc]
\makeatletter
\def\@idxitem{\par\hangindent 0pt}
\makeatother


%\newcounter{dummy} \numberwithin{dummy}{section}
\newtheorem{dummy}{dummy}[section]
\theoremstyle{definition}
\newtheorem{definition}[dummy]{Definition}
\theoremstyle{plain}
\newtheorem{corollary}[dummy]{Corollary}
\newtheorem{lemma}[dummy]{Lemma}
\newtheorem{proposition}[dummy]{Proposition}
\newtheorem{theorem}[dummy]{Theorem}
\theoremstyle{definition}
\newtheorem{examplle}{Example}[section]
\theoremstyle{remark}
\newtheorem*{remark}{Remark}
\newtheorem{exercise}{Exercise}[subsection]
\newtheorem{observation}{Observation}[section]


\newenvironment{claim}[1]{\par\noindent\textbf{Claim:}\space#1}{}

\makeatletter
\DeclareFontFamily{U}{tipa}{}
\DeclareFontShape{U}{tipa}{m}{n}{<->tipa10}{}
\newcommand{\arc@char}{{\usefont{U}{tipa}{m}{n}\symbol{62}}}%

\newcommand{\arc}[1]{\mathpalette\arc@arc{#1}}

\newcommand{\arc@arc}[2]{%
  \sbox0{$\m@th#1#2$}%
  \vbox{
    \hbox{\resizebox{\wd0}{\height}{\arc@char}}
    \nointerlineskip
    \box0
  }%
}
\makeatother

\setcounter{MaxMatrixCols}{20}
%%%%%%% ABS
\DeclarePairedDelimiter\abss{\lvert}{\rvert}%
\DeclarePairedDelimiter\normm{\lVert}{\rVert}%

% Swap the definition of \abs* and \norm*, so that \abs
% and \norm resizes the size of the brackets, and the
% starred version does not.
\makeatletter
\let\oldabs\abss
%\def\abs{\@ifstar{\oldabs}{\oldabs*}}
\newcommand{\abs}{\@ifstar{\oldabs}{\oldabs*}}
\newcommand{\norm}[1]{\left\lVert#1\right\rVert}
%\let\oldnorm\normm
%\def\norm{\@ifstar{\oldnorm}{\oldnorm*}}
%\renewcommand{norm}{\@ifstar{\oldnorm}{\oldnorm*}}
\makeatother

% \newcommand\what[1]{\ThisStyle{%
%     \setbox0=\hbox{$\SavedStyle#1$}%
%     \stackengine{-1.0\ht0+.5pt}{$\SavedStyle#1$}{%
%       \stretchto{\scaleto{\SavedStyle\mkern.15mu\char'136}{2.6\wd0}}{1.4\ht0}%
%     }{O}{c}{F}{T}{S}%
%   }
% }

% \newcommand\wtilde[1]{\ThisStyle{%
%     \setbox0=\hbox{$\SavedStyle#1$}%
%     \stackengine{-.1\LMpt}{$\SavedStyle#1$}{%
%       \stretchto{\scaleto{\SavedStyle\mkern.2mu\AC}{.5150\wd0}}{.6\ht0}%
%     }{O}{c}{F}{T}{S}%
%   }
% }

% \newcommand\wbar[1]{\ThisStyle{%
%     \setbox0=\hbox{$\SavedStyle#1$}%
%     \stackengine{.5pt+\LMpt}{$\SavedStyle#1$}{%
%       \rule{\wd0}{\dimexpr.3\LMpt+.3pt}%
%     }{O}{c}{F}{T}{S}%
%   }
% }

\newcommand{\bl}[1] {\boldsymbol{#1}}
\newcommand{\Wt}[1] {\stackrel{\sim}{\smash{#1}\rule{0pt}{1.1ex}}}
\newcommand{\wt}[1] {\widetilde{#1}}
\newcommand{\tf}[1] {\textbf{#1}}


%For boxed texts in align, use Aboxed{}
%otherwise use boxed{}

\DeclareMathSymbol{\widehatsym}{\mathord}{largesymbols}{"62}
\newcommand\lowerwidehatsym{%
  \text{\smash{\raisebox{-1.3ex}{%
    $\widehatsym$}}}}
\newcommand\fixwidehat[1]{%
  \mathchoice
    {\accentset{\displaystyle\lowerwidehatsym}{#1}}
    {\accentset{\textstyle\lowerwidehatsym}{#1}}
    {\accentset{\scriptstyle\lowerwidehatsym}{#1}}
    {\accentset{\scriptscriptstyle\lowerwidehatsym}{#1}}
  }


\newcommand{\cupdot}{\mathbin{\dot{\cup}}}
\newcommand{\bigcupdot}{\mathop{\dot{\bigcup}}}

\usepackage{graphicx}

\usepackage[toc,page]{appendix}

% text on arrow for xRightarrow
\makeatletter
%\newcommand{\xRightarrow}[2][]{\ext@arrow 0359\Rightarrowfill@{#1}{#2}}
\makeatother

% Arbitrary long arrow
\newcommand{\Rarrow}[1]{%
\parbox{#1}{\tikz{\draw[->](0,0)--(#1,0);}}
}

\newcommand{\LRarrow}[1]{%
\parbox{#1}{\tikz{\draw[<->](0,0)--(#1,0);}}
}


\makeatletter
\providecommand*{\rmodels}{%
  \mathrel{%
    \mathpalette\@rmodels\models
  }%
}
\newcommand*{\@rmodels}[2]{%
  \reflectbox{$\m@th#1#2$}%
}
\makeatother







\newcommand{\trcl}[1]{%
  \mathrm{trcl}{(#1)}
}



% Roman numerals
\makeatletter
\newcommand*{\rom}[1]{\expandafter\@slowromancap\romannumeral #1@}
\makeatother
% \\def \\b\([a-zA-Z]\) {\\boldsymbol{[a-zA-z]}}
% \\DeclareMathOperator{\\b\1}{\\textbf{\1}}


\DeclareMathOperator{\bx}{\textbf{x}}
\DeclareMathOperator{\bz}{\textbf{z}}
\DeclareMathOperator{\bff}{\textbf{f}}
\DeclareMathOperator{\ba}{\textbf{a}}
\DeclareMathOperator{\bk}{\textbf{k}}
\DeclareMathOperator{\bs}{\textbf{s}}
\DeclareMathOperator{\bh}{\textbf{h}}
\DeclareMathOperator{\bc}{\textbf{c}}
\DeclareMathOperator{\br}{\textbf{r}}
\DeclareMathOperator{\bi}{\textbf{i}}
\DeclareMathOperator{\bj}{\textbf{j}}
\DeclareMathOperator{\bn}{\textbf{n}}
\DeclareMathOperator{\be}{\textbf{e}}
\DeclareMathOperator{\bo}{\textbf{o}}
\DeclareMathOperator{\bU}{\textbf{U}}
\DeclareMathOperator{\bL}{\textbf{L}}
\DeclareMathOperator{\bV}{\textbf{V}}
\def \bzero {\mathbf{0}}
\def \btwo {\mathbf{2}}
\DeclareMathOperator{\bv}{\textbf{v}}
\DeclareMathOperator{\bp}{\textbf{p}}
\DeclareMathOperator{\bI}{\textbf{I}}
\DeclareMathOperator{\bM}{\textbf{M}}
\DeclareMathOperator{\bN}{\textbf{N}}
\DeclareMathOperator{\bK}{\textbf{K}}
\DeclareMathOperator{\bt}{\textbf{t}}
\DeclareMathOperator{\bb}{\textbf{b}}
\DeclareMathOperator{\bA}{\textbf{A}}
\DeclareMathOperator{\bX}{\textbf{X}}
\DeclareMathOperator{\bu}{\textbf{u}}
\DeclareMathOperator{\bS}{\textbf{S}}
\DeclareMathOperator{\bZ}{\textbf{Z}}
\DeclareMathOperator{\by}{\textbf{y}}
\DeclareMathOperator{\bw}{\textbf{w}}
\DeclareMathOperator{\bT}{\textbf{T}}
\DeclareMathOperator{\bF}{\textbf{F}}
\DeclareMathOperator{\bmm}{\textbf{m}}
\DeclareMathOperator{\bW}{\textbf{W}}
\DeclareMathOperator{\bR}{\textbf{R}}
\DeclareMathOperator{\bC}{\textbf{C}}
\DeclareMathOperator{\bD}{\textbf{D}}
\DeclareMathOperator{\bE}{\textbf{E}}
\DeclareMathOperator{\bQ}{\textbf{Q}}
\DeclareMathOperator{\bP}{\textbf{P}}
\DeclareMathOperator{\bY}{\textbf{Y}}
\DeclareMathOperator{\bH}{\textbf{H}}
\DeclareMathOperator{\bB}{\textbf{B}}
\DeclareMathOperator{\bG}{\textbf{G}}
\def \blambda {\symbf{\lambda}}
\def \boldeta {\symbf{\eta}}
\def \balpha {\symbf{\alpha}}
\def \bbeta {\symbf{\beta}}
\def \bgamma {\symbf{\gamma}}
\def \bxi {\symbf{\xi}}
\def \bLambda {\symbf{\Lambda}}

\newcommand{\bto}{{\boldsymbol{\to}}}
\newcommand{\Ra}{\Rightarrow}
\newcommand\und[1]{\underline{#1}}
\def \bPhi {\boldsymbol{\Phi}}
\def \btheta {\boldsymbol{\theta}}
\def \bTheta {\boldsymbol{\Theta}}
\def \bmu {\boldsymbol{\mu}}
\def \bphi {\boldsymbol{\phi}}
\def \bSigma {\boldsymbol{\Sigma}}
\def \lb {\left\{}
\def \rb {\right\}}
\def \la {\langle}
\def \ra {\rangle}
\def \caln {\mathcal{N}}
\def \dissum {\displaystyle\Sigma}
\def \dispro {\displaystyle\prod}
\def \E {\mathbb{E}}
\def \Q {\mathbb{Q}}
\def \N {\mathbb{N}}
\def \V {\mathbb{V}}
\def \R {\mathbb{R}}
\def \P {\mathbb{P}}
\def \A {\mathbb{A}}
\def \F {\mathbb{F}}
\def \Z {\mathbb{Z}}
\def \I {\mathbb{I}}
\def \C {\mathbb{C}}
\def \cala {\mathcal{A}}
\def \cale {\mathcal{E}}
\def \calb {\mathcal{B}}
\def \calq {\mathcal{Q}}
\def \calp {\mathcal{P}}
\def \cals {\mathcal{S}}
\def \calx {\mathcal{X}}
\def \caly {\mathcal{Y}}
\def \calg {\mathcal{G}}
\def \cald {\mathcal{D}}
\def \caln {\mathcal{N}}
\def \calr {\mathcal{R}}
\def \calt {\mathcal{T}}
\def \calm {\mathcal{M}}
\def \calw {\mathcal{W}}
\def \calc {\mathcal{C}}
\def \calv {\mathcal{V}}
\def \calf {\mathcal{F}}
\def \calk {\mathcal{K}}
\def \call {\mathcal{L}}
\def \calu {\mathcal{U}}
\def \calo {\mathcal{O}}
\def \calh {\mathcal{H}}
\def \cali {\mathcal{I}}

\def \bcup {\bigcup}

% set theory

\def \zfcc {\textbf{ZFC}^-}
\def \ac  {\textbf{AC}}
\def \gl  {\textbf{L }}
\def \gll {\textbf{L}}
\newcommand{\zfm}{$\textbf{ZF}^-$}

%\def \zfm {$\textbf{ZF}^-$}
\def \zfmm {\textbf{ZF}^-}
\def \wf {\textbf{WF }}
\def \on {\textbf{On }}
\def \cm {\textbf{M }}
\def \cn {\textbf{N }}
\def \cv {\textbf{V }}
\def \zc {\textbf{ZC }}
\def \zcm {\textbf{ZC}}
\def \zff {\textbf{ZF}}
\def \wfm {\textbf{WF}}
\def \onm {\textbf{On}}
\def \cmm {\textbf{M}}
\def \cnm {\textbf{N}}
\def \cvm {\textbf{V}}
\def \gchh {\textbf{GCH}}
\renewcommand{\restriction}{\mathord{\upharpoonright}}
\def \pred {\text{pred}}

\def \rank {\text{rank}}
\def \con {\text{Con}}
\def \deff {\text{Def}}


\def \uin {\underline{\in}}
\def \oin {\overline{\in}}
\def \uR {\underline{R}}
\def \oR {\overline{R}}
\def \uP {\underline{P}}
\def \oP {\overline{P}}

\def \Ra {\Rightarrow}

\def \e {\enspace}

\def \sgn {\operatorname{sgn}}
\def \gen {\operatorname{gen}}
\def \Hom {\operatorname{Hom}}
\def \hom {\operatorname{hom}}
\def \Sub {\operatorname{Sub}}

\def \supp {\operatorname{supp}}

\def \epiarrow {\twoheadarrow}
\def \monoarrow {\rightarrowtail}
\def \rrarrow {\rightrightarrows}

% \def \minus {\text{-}}
% \newcommand{\minus}{\scalebox{0.75}[1.0]{$-$}}
% \DeclareUnicodeCharacter{002D}{\minus}


\def \tril {\triangleleft}

\def \ACF {\text{ACF}}
\def \GL {\text{GL}}
\def \PGL {\text{PGL}}
\def \equal {=}
\def \deg {\text{deg}}
\def \degree {\text{degree}}
\def \app {\text{App}}
\def \FV {\text{FV}}
\def \conv {\text{conv}}
\def \cont {\text{cont}}
\DeclareMathOperator{\cl}{\textbf{CL}}
\DeclareMathOperator{\sg}{sg}
\DeclareMathOperator{\trdeg}{trdeg}
\def \Ord {\text{Ord}}

\DeclareMathOperator{\cf}{cf}
\DeclareMathOperator{\zfc}{ZFC}

%\DeclareMathOperator{\Th}{Th}
%\def \th {\text{Th}}
% \newcommand{\th}{\text{Th}}
\DeclareMathOperator{\type}{type}
\DeclareMathOperator{\zf}{\textbf{ZF}}
\def \fa {\mathfrak{a}}
\def \fb {\mathfrak{b}}
\def \fc {\mathfrak{c}}
\def \fd {\mathfrak{d}}
\def \fe {\mathfrak{e}}
\def \ff {\mathfrak{f}}
\def \fg {\mathfrak{g}}
\def \fh {\mathfrak{h}}
%\def \fi {\mathfrak{i}}
\def \fj {\mathfrak{j}}
\def \fk {\mathfrak{k}}
\def \fl {\mathfrak{l}}
\def \fm {\mathfrak{m}}
\def \fn {\mathfrak{n}}
\def \fo {\mathfrak{o}}
\def \fp {\mathfrak{p}}
\def \fq {\mathfrak{q}}
\def \fr {\mathfrak{r}}
\def \fs {\mathfrak{s}}
\def \ft {\mathfrak{t}}
\def \fu {\mathfrak{u}}
\def \fv {\mathfrak{v}}
\def \fw {\mathfrak{w}}
\def \fx {\mathfrak{x}}
\def \fy {\mathfrak{y}}
\def \fz {\mathfrak{z}}
\def \fA {\mathfrak{A}}
\def \fB {\mathfrak{B}}
\def \fC {\mathfrak{C}}
\def \fD {\mathfrak{D}}
\def \fE {\mathfrak{E}}
\def \fF {\mathfrak{F}}
\def \fG {\mathfrak{G}}
\def \fH {\mathfrak{H}}
\def \fI {\mathfrak{I}}
\def \fJ {\mathfrak{J}}
\def \fK {\mathfrak{K}}
\def \fL {\mathfrak{L}}
\def \fM {\mathfrak{M}}
\def \fN {\mathfrak{N}}
\def \fO {\mathfrak{O}}
\def \fP {\mathfrak{P}}
\def \fQ {\mathfrak{Q}}
\def \fR {\mathfrak{R}}
\def \fS {\mathfrak{S}}
\def \fT {\mathfrak{T}}
\def \fU {\mathfrak{U}}
\def \fV {\mathfrak{V}}
\def \fW {\mathfrak{W}}
\def \fX {\mathfrak{X}}
\def \fY {\mathfrak{Y}}
\def \fZ {\mathfrak{Z}}

\def \sfA {\textsf{A}}
\def \sfB {\textsf{B}}
\def \sfC {\textsf{C}}
\def \sfD {\textsf{D}}
\def \sfE {\textsf{E}}
\def \sfF {\textsf{F}}
\def \sfG {\textsf{G}}
\def \sfH {\textsf{H}}
\def \sfI {\textsf{I}}
\def \sfj {\textsf{J}}
\def \sfK {\textsf{K}}
\def \sfL {\textsf{L}}
\def \sfM {\textsf{M}}
\def \sfN {\textsf{N}}
\def \sfO {\textsf{O}}
\def \sfP {\textsf{P}}
\def \sfQ {\textsf{Q}}
\def \sfR {\textsf{R}}
\def \sfS {\textsf{S}}
\def \sfT {\textsf{T}}
\def \sfU {\textsf{U}}
\def \sfV {\textsf{V}}
\def \sfW {\textsf{W}}
\def \sfX {\textsf{X}}
\def \sfY {\textsf{Y}}
\def \sfZ {\textsf{Z}}
\def \sfa {\textsf{a}}
\def \sfb {\textsf{b}}
\def \sfc {\textsf{c}}
\def \sfd {\textsf{d}}
\def \sfe {\textsf{e}}
\def \sff {\textsf{f}}
\def \sfg {\textsf{g}}
\def \sfh {\textsf{h}}
\def \sfi {\textsf{i}}
\def \sfj {\textsf{j}}
\def \sfk {\textsf{k}}
\def \sfl {\textsf{l}}
\def \sfm {\textsf{m}}
\def \sfn {\textsf{n}}
\def \sfo {\textsf{o}}
\def \sfp {\textsf{p}}
\def \sfq {\textsf{q}}
\def \sfr {\textsf{r}}
\def \sfs {\textsf{s}}
\def \sft {\textsf{t}}
\def \sfu {\textsf{u}}
\def \sfv {\textsf{v}}
\def \sfw {\textsf{w}}
\def \sfx {\textsf{x}}
\def \sfy {\textsf{y}}
\def \sfz {\textsf{z}}



%\DeclareMathOperator{\ker}{ker}
\DeclareMathOperator{\im}{im}

\DeclareMathOperator{\inn}{Inn}
\DeclareMathOperator{\AC}{\textbf{AC}}
\DeclareMathOperator{\cod}{cod}
\DeclareMathOperator{\dom}{dom}
\DeclareMathOperator{\ran}{ran}
\DeclareMathOperator{\textd}{d}
\DeclareMathOperator{\td}{d}
\DeclareMathOperator{\id}{id}
\DeclareMathOperator{\LT}{LT}
\DeclareMathOperator{\Mat}{Mat}
\DeclareMathOperator{\Eq}{Eq}
\DeclareMathOperator{\irr}{irr}
\DeclareMathOperator{\Fr}{Fr}
\DeclareMathOperator{\Gal}{Gal}
\DeclareMathOperator{\lcm}{lcm}
\DeclareMathOperator{\alg}{\text{alg}}
\DeclareMathOperator{\Th}{Th}

\DeclareMathOperator{\DAG}{DAG}
\DeclareMathOperator{\ODAG}{ODAG}

% \varprod
\DeclareSymbolFont{largesymbolsA}{U}{txexa}{m}{n}
\DeclareMathSymbol{\varprod}{\mathop}{largesymbolsA}{16}
% \DeclareMathSymbol{\tonm}{\boldsymbol{\to}\textbf{Nm}}
\def \tonm {\bto\textbf{Nm}}
\def \tohm {\bto\textbf{Hm}}

% Category theory
\DeclareMathOperator{\Ab}{\textbf{Ab}}
\DeclareMathOperator{\Alg}{\textbf{Alg}}
\DeclareMathOperator{\Rng}{\textbf{Rng}}
\DeclareMathOperator{\Sets}{\textbf{Sets}}
\DeclareMathOperator{\Met}{\textbf{Met}}
\DeclareMathOperator{\Aut}{\textbf{Aut}}
\DeclareMathOperator{\RMod}{R-\textbf{Mod}}
\DeclareMathOperator{\RAlg}{R-\textbf{Alg}}
\DeclareMathOperator{\LF}{LF}
\DeclareMathOperator{\op}{op}
% Model theory
\DeclareMathOperator{\tp}{tp}
\DeclareMathOperator{\Diag}{Diag}
\DeclareMathOperator{\el}{el}
\DeclareMathOperator{\depth}{depth}
\DeclareMathOperator{\FO}{FO}
\DeclareMathOperator{\fin}{fin}
\DeclareMathOperator{\qr}{qr}
\DeclareMathOperator{\Mod}{Mod}
\DeclareMathOperator{\TC}{TC}
\DeclareMathOperator{\KH}{KH}
\DeclareMathOperator{\Part}{Part}
\DeclareMathOperator{\Infset}{\textsf{Infset}}
\DeclareMathOperator{\DLO}{\textsf{DLO}}
\DeclareMathOperator{\sfMod}{\textsf{Mod}}
\DeclareMathOperator{\AbG}{\textsf{AbG}}
\DeclareMathOperator{\sfACF}{\textsf{ACF}}
% Computability Theorem
\DeclareMathOperator{\Tot}{Tot}
\DeclareMathOperator{\graph}{graph}
\DeclareMathOperator{\Fin}{Fin}
\DeclareMathOperator{\Cof}{Cof}
\DeclareMathOperator{\lh}{lh}
% Commutative Algebra
\DeclareMathOperator{\ord}{ord}
\DeclareMathOperator{\Idem}{Idem}
\DeclareMathOperator{\zdiv}{z.div}
\DeclareMathOperator{\Frac}{Frac}
\DeclareMathOperator{\rad}{rad}
\DeclareMathOperator{\nil}{nil}
\DeclareMathOperator{\Ann}{Ann}
\DeclareMathOperator{\End}{End}
\DeclareMathOperator{\coim}{coim}
\DeclareMathOperator{\coker}{coker}
\DeclareMathOperator{\Bil}{Bil}
\DeclareMathOperator{\Tril}{Tril}
% Topology
\newcommand{\interior}[1]{%
  {\kern0pt#1}^{\mathrm{o}}%
}

% \makeatletter
% \newcommand{\vect}[1]{%
%   \vbox{\m@th \ialign {##\crcr
%   \vectfill\crcr\noalign{\kern-\p@ \nointerlineskip}
%   $\hfil\displaystyle{#1}\hfil$\crcr}}}
% \def\vectfill{%
%   $\m@th\smash-\mkern-7mu%
%   \cleaders\hbox{$\mkern-2mu\smash-\mkern-2mu$}\hfill
%   \mkern-7mu\raisebox{-3.81pt}[\p@][\p@]{$\mathord\mathchar"017E$}$}

% \newcommand{\amsvect}{%
%   \mathpalette {\overarrow@\vectfill@}}
% \def\vectfill@{\arrowfill@\relbar\relbar{\raisebox{-3.81pt}[\p@][\p@]{$\mathord\mathchar"017E$}}}

% \newcommand{\amsvectb}{%
% \newcommand{\vect}{%
%   \mathpalette {\overarrow@\vectfillb@}}
% \newcommand{\vecbar}{%
%   \scalebox{0.8}{$\relbar$}}
% \def\vectfillb@{\arrowfill@\vecbar\vecbar{\raisebox{-4.35pt}[\p@][\p@]{$\mathord\mathchar"017E$}}}
% \makeatother
% \bigtimes

\DeclareFontFamily{U}{mathx}{\hyphenchar\font45}
\DeclareFontShape{U}{mathx}{m}{n}{
      <5> <6> <7> <8> <9> <10>
      <10.95> <12> <14.4> <17.28> <20.74> <24.88>
      mathx10
      }{}
\DeclareSymbolFont{mathx}{U}{mathx}{m}{n}
\DeclareMathSymbol{\bigtimes}{1}{mathx}{"91}
% \odiv
\DeclareFontFamily{U}{matha}{\hyphenchar\font45}
\DeclareFontShape{U}{matha}{m}{n}{
      <5> <6> <7> <8> <9> <10> gen * matha
      <10.95> matha10 <12> <14.4> <17.28> <20.74> <24.88> matha12
      }{}
\DeclareSymbolFont{matha}{U}{matha}{m}{n}
\DeclareMathSymbol{\odiv}         {2}{matha}{"63}


\newcommand\subsetsim{\mathrel{%
  \ooalign{\raise0.2ex\hbox{\scalebox{0.9}{$\subset$}}\cr\hidewidth\raise-0.85ex\hbox{\scalebox{0.9}{$\sim$}}\hidewidth\cr}}}
\newcommand\simsubset{\mathrel{%
  \ooalign{\raise-0.2ex\hbox{\scalebox{0.9}{$\subset$}}\cr\hidewidth\raise0.75ex\hbox{\scalebox{0.9}{$\sim$}}\hidewidth\cr}}}

\newcommand\simsubsetsim{\mathrel{%
  \ooalign{\raise0ex\hbox{\scalebox{0.8}{$\subset$}}\cr\hidewidth\raise1ex\hbox{\scalebox{0.75}{$\sim$}}\hidewidth\cr\raise-0.95ex\hbox{\scalebox{0.8}{$\sim$}}\cr\hidewidth}}}
\newcommand{\stcomp}[1]{{#1}^{\mathsf{c}}}

\setlength{\baselineskip}{0.8in}

\stackMath
\newcommand\yrightarrow[2][]{\mathrel{%
  \setbox2=\hbox{\stackon{\scriptstyle#1}{\scriptstyle#2}}%
  \stackunder[0pt]{%
    \xrightarrow{\makebox[\dimexpr\wd2\relax]{$\scriptstyle#2$}}%
  }{%
   \scriptstyle#1\,%
  }%
}}
\newcommand\yleftarrow[2][]{\mathrel{%
  \setbox2=\hbox{\stackon{\scriptstyle#1}{\scriptstyle#2}}%
  \stackunder[0pt]{%
    \xleftarrow{\makebox[\dimexpr\wd2\relax]{$\scriptstyle#2$}}%
  }{%
   \scriptstyle#1\,%
  }%
}}
\newcommand\yRightarrow[2][]{\mathrel{%
  \setbox2=\hbox{\stackon{\scriptstyle#1}{\scriptstyle#2}}%
  \stackunder[0pt]{%
    \xRightarrow{\makebox[\dimexpr\wd2\relax]{$\scriptstyle#2$}}%
  }{%
   \scriptstyle#1\,%
  }%
}}
\newcommand\yLeftarrow[2][]{\mathrel{%
  \setbox2=\hbox{\stackon{\scriptstyle#1}{\scriptstyle#2}}%
  \stackunder[0pt]{%
    \xLeftarrow{\makebox[\dimexpr\wd2\relax]{$\scriptstyle#2$}}%
  }{%
   \scriptstyle#1\,%
  }%
}}

\newcommand\altxrightarrow[2][0pt]{\mathrel{\ensurestackMath{\stackengine%
  {\dimexpr#1-7.5pt}{\xrightarrow{\phantom{#2}}}{\scriptstyle\!#2\,}%
  {O}{c}{F}{F}{S}}}}
\newcommand\altxleftarrow[2][0pt]{\mathrel{\ensurestackMath{\stackengine%
  {\dimexpr#1-7.5pt}{\xleftarrow{\phantom{#2}}}{\scriptstyle\!#2\,}%
  {O}{c}{F}{F}{S}}}}

\newenvironment{bsm}{% % short for 'bracketed small matrix'
  \left[ \begin{smallmatrix} }{%
  \end{smallmatrix} \right]}

\newenvironment{psm}{% % short for ' small matrix'
  \left( \begin{smallmatrix} }{%
  \end{smallmatrix} \right)}

\newcommand{\bbar}[1]{\mkern 1.5mu\overline{\mkern-1.5mu#1\mkern-1.5mu}\mkern 1.5mu}

\newcommand{\bigzero}{\mbox{\normalfont\Large\bfseries 0}}
\newcommand{\rvline}{\hspace*{-\arraycolsep}\vline\hspace*{-\arraycolsep}}

\font\zallman=Zallman at 40pt
\font\elzevier=Elzevier at 40pt

\newcommand\isoto{\stackrel{\textstyle\sim}{\smash{\longrightarrow}\rule{0pt}{0.4ex}}}
\newcommand\embto{\stackrel{\textstyle\prec}{\smash{\longrightarrow}\rule{0pt}{0.4ex}}}
\usepackage[UTF8]{ctex}
\author{五狗砸}
\date{\today}
\title{考研题目本}
\hypersetup{
 pdfauthor={五狗砸},
 pdftitle={考研题目本},
 pdfkeywords={},
 pdfsubject={},
 pdfcreator={Emacs 26.3 (Org mode 9.4)}, 
 pdflang={English}}
\begin{document}

\maketitle
\tableofcontents \clearpage
\section{微积分}
\label{sec:orgfd07bc2}
\subsection{一元函数微分}
\label{sec:orgd9815c8}
\begin{examplle}[]
设\(f'(x)\)连续,\(f(0)=0,f'(0)\neq0\),求
\(\displaystyle\lim_{x\to0}\frac{\int_0^{x^2}f(x^2-t)dt}{x^3\int_0^1f(xt)dt}\)

令\(x^2-t=u,xt=u\)
\begin{align*}
\lim_{x\to0}\frac{\int_0^{x^2}f(x^2-t)dt}{x^3\int_0^1f(xt)dt}&=
\lim_{x\to0}\frac{-\int_{x^2}^0f(u)du}{x^3\int_0^xf(u)\frac{du}{x}}=
\lim_{x\to0}\frac{\int_0^{x^2}f(u)du}{x^2\int_0^xf(u)du}\\
&=\lim_{x\to0}\frac{2xf(x^2)}{2x\int_0^xf(u)du+x^2f(x)}\\
&=\lim_{x\to0}\frac{2f(x^2)}{2\int_0^xf(u)du+xf(x)}\\
&=\lim_{x\to0}\frac{4xf'(x^2)}{3f(x)+xf'(x)}\\
&=\lim_{x\to0}\frac{4f'(x^2)}{3\frac{f(x)-f(0)}{x}+f'(x)}=1
\end{align*}
\end{examplle}

\begin{examplle}[]
求\(\displaystyle\lim_{x\to0}\frac{\frac{x^2}{2}+1-\sqrt{1+x^2}}{(\cos x-e^{x^2})\sin
  x^2}\)

利用泰勒展开,\(\sqrt{1+x^2}=1+\frac{1}{2}x^2-\frac{1}{8}x^4+o(x^4)\),
\(\cos x=1-\frac{1}{2}x^2+o(x^2)\),\(e^{x^2}=1+x^2+o(x^2)\),因此
\begin{equation*}
\lim_{x\to0}\frac{\frac{x^2}{2}+1-\sqrt{1+x^2}}{(\cos x-e^{x^2})\sin
x^2}=\lim_{x\to0}\frac{\frac{x^4}{8}+o(x^4)}{-\frac{3}{2}x^4+o(x^4)}=-\frac{1}{12}
\end{equation*}
\end{examplle}

\begin{examplle}[]
求\(\displaystyle\lim_{n\to\infty}\tan^n(\frac{\pi}{4}+\frac{2}{n})\)

因为\(\lim_{x\to\infty}f(x)=A\Rightarrow\lim_{n\to\infty}f(n)=A\)
\end{examplle}

\begin{examplle}[]
suppose \(\displaystyle y_n=\left[\frac{(2n)!}{n!n^n}\right]^{\frac{1}{n+1}}\). Compute
\(\lim_{n\to\infty}y_n\)

\begin{align*}
\ln y_n&=\frac{1}{n+1}\ln\frac{(2n)!}{n!n^n}=
\frac{1}{n+1}\ln\frac{(2n)(2n-1)\dots(n+1)}{n^n}\\
&=\frac{1}{n+1}\sum_{k=1}^n\ln(1+\frac{k}{n})=
\frac{n}{n+1}\left(
\frac{1}{n}\sum_{k=1}^n\ln(1+\frac{k}{n})
\right)
\end{align*}
Hence
\begin{align*}
\lim_{n\to\infty}y_n&=\lim_{n\to\infty}\frac{n}{n+1}\left(
\frac{1}{n}\sum_{k=1}^n\ln(1+\frac{k}{n})
\right)\\
&=1\cdot\int_0^1\ln(1+x)dx=
x\ln(1+x)\rvert_0^1-\int_0^1\frac{x}{1+x}dx\\
&=\ln2-1+\ln2=\ln\frac{4}{e}
\end{align*}
\end{examplle}

\begin{examplle}[]
已知\(x\to0\)时,\(e^{-x^4}-\cos(\sqrt{2}x^2)\) 与\(ax^n\)是等价无穷小,试求
\(a,n\)
\begin{align*}
&e^{-x^4}=1-x^4+\frac{x^8}{2}+o(x^8)\\
&\cos(\sqrt{2}x^2)=1-x^4+\frac{x^8}{6}+o(x^8)
\end{align*}
Hence \(a=\frac{1}{3},n=8\)
\end{examplle}

\begin{examplle}[]
设\(\displaystyle f(x)=\frac{\sqrt{1+\sin x+\sin^2x}-(\alpha+\beta\sin x)}{\sin^2x}\),且点
\(x=0\)是\(f(x)\)的可去间断点,求\(\alpha,\beta\)

由极限存在可知,\(\alpha=1\),泰勒展开
\begin{align*}
&\frac{\sqrt{1+\sin x+\sin^2x}-(\alpha+\beta\sin x)}{\sin^2x}\\
&=\lim_{x\to0}\frac{1+\frac{1}{2}(\sin x+\sin^2x)-\frac{1}{8}(\sin x+\sin^2x)^2-(1+\beta\sin x)
+o(\sin^2x)}{\sin^2}\\
&=\lim_{x\to0}\frac{(\frac{1}{2}-\beta)\sin x+\frac{3}{8}\sin^2x}{\sin^2x}
\end{align*}
故\(\beta=\frac{1}{2}\)
\end{examplle}

\begin{examplle}[]
let \(\displaystyle f(x)=\lim_{n\to\infty}\frac{2x^n-3x^{-n}}{x^n+x^{-n}}\sin\frac{1}{x}\)

\begin{equation*}
f(x)=
\begin{cases}
2\sin\frac{1}{x}^x&x<-1\\
-\frac{1}{2}\sin\frac{1}{x}&x=-1\\
-3\sin\frac{1}{x}&-1<x<0\\
-3\sin\frac{1}{x}&0<x<1\\
-\frac{1}{2}\sin\frac{1}{x}&x=1\\
2\sin\frac{1}{x}^x&x>1
\end{cases}
\end{equation*}
\(x=0\)是第二类间断点,\(x=\pm1\)是第一类间断点
\end{examplle}

\begin{examplle}[]
设\(f(1)=0,f'(1)=a\),求极限\(\displaystyle
  \lim_{x\to0}\frac{\sqrt{1+2f(e^{x^2})}- \sqrt{1+f(1+\sin^2 x)}}{\ln\cos x}\)

由\(f(1)=0,f'(1)=a\)可知,\(\displaystyle
  f'(1)=\lim_{x\to1}\frac{f(x)-f(1)}{x-1}=\lim_{t\to0}\frac{f(1+t)}{t}=a\)

\begin{align*}
\lim_{x\to0}\frac{\sqrt{1+2f(e^{x^2})}- \sqrt{1+f(1+\sin^2 x)}}{\ln\cos x}&=
\frac{2f(e^{x^2})-f(1+\sin^2x)}{-\frac{1}{2}x^2
\left[\sqrt{1+2f(e^{x^2})}+\sqrt{1+f(1+\sin^2x)}
\right]}\\
&=\lim_{x\to0}\frac{f(1+\sin^2x)-f(e^{x^2})}{x^2}\\
&=\lim_{x\to0}\left[
\frac{f(1+\sin^2x)}{\sin^2x}\cdot\frac{\sin^2x}{x^2}-
\frac{f(e^{x^2})}{e^{x^2}-1}\cdot\frac{e^{x^2}-1}{x^2}
\right]\\
&=-a
\end{align*}
\end{examplle}

\begin{examplle}[]
设\(f(x)\)在\(x=0\)的某邻域内二阶可导,且
\(\displaystyle\lim_{x\to0}\frac{f(x)}{x}=0\),\(f''(0)\neq0\),
\(\displaystyle\lim_{x\to0^+}\frac{\int_0^xf(x)dt}{x^\alpha-\sin
  x}=\beta(\beta\neq0)\),求\(\alpha,\beta\)

因为\(\lim_{x\to0}\frac{f(x)}{x}=0\),\(f(0)=0,f'(0)=0\)

因为\(\lim_{x\to0^+}\int_0^xf(x)dt=0\),因此\(\lim_{x\to0^+}x^\alpha-\sin
  x=0\),因此\(\alpha>0\)
\begin{enumerate}
\item 若\(0<\alpha<1\)
\item 若\(\alpha>1\)
\item 若\(\alpha=1\)

\(\beta=f''(0)\)
\end{enumerate}
\end{examplle}

\begin{examplle}[]
设\(f(x)\)在\((-\infty,+\infty)\)上有定义,且\(f'(0)=1\),
\(f(x+y)=f(x)e^y+f(y)e^x\),求\(f(x)\)

\(f(0)=0\)

\begin{align*}
f'(x)&=\lim_{y\to0}\frac{f(x+y)-f(x)}{y}\\
&=\lim_{y\to0}\frac{f(x)e^y+f(y)e^x-f(x)}{y}\\
&=\lim_{y\to0}\left[
f(x)\frac{e^y-1}{y}+e^x\frac{f(y)-f(0)}{y}
\right]\\
&=f(x)+e^xf'(0)=f(x)+e^x
\end{align*}
即\(f'(x)-f(x)=e^x\),因此\(f(x)=e^x(x+C)\),又\(f(0)=0,C=0,f(x)=xe^x\)
\end{examplle}

\begin{examplle}[]
已知函数\(\displaystyle f(x)=\begin{cases}x&x\le0\\\frac{1}{n}&\frac{1}{n+1}
  <x\le\frac{1}{n}\end{cases}\)

\begin{equation*}
f_+'(0)=\lim_{x\to0^+}\frac{f(x)-f(0)}{x}=\lim_{x\to0^+}\frac{\frac{1}{n}}{x}
\left(\frac{1}{n+1}<x\le\frac{1}{n}
\right)
\end{equation*}
而\(1\le\frac{\frac{1}{n}}{x}<\frac{n+1}{n}\),由夹逼准则得\(f'_+(0)=1\),因此\(f'(0)=1\)
\end{examplle}

\begin{examplle}[]
设\(f(x)\)是可导的偶函数,它在\(x=0\)的某邻域内满足
\begin{equation*}
f(e^{x^2})-3f(1+\sin x^2)=2x^2+o(x^2)
\end{equation*}
求曲线\(y=f(x)\)在点\((-1,f(-1))\)处的切线方程

由
\begin{equation*}
\lim_{x\to0}\frac{f(e^{x^2})-3f(1+\sin x^2)-2x^2}{x^2}=0
\end{equation*}
得
\begin{equation*}
f(0)-3f(1)=0\Rightarrow f(1)=0
\end{equation*}
变形
\begin{equation*}
\lim_{x\to0}\left(
\frac{f(e^{x^2})}{e^{x^2}-1}\cdot\frac{e^{x^2}-1}{x^2}-
\frac{3f(1+\sin x^2)}{\sin x^2}\cdot\frac{\sin x^2}{x^2}-2
\right)=0
\end{equation*}
有\(f'(1)-3f'(1)-2=0\Rightarrow f'(1)=-1\)
\end{examplle}

\begin{examplle}[]
若\(y=f(x)\)存在单值反函数,且\(y'\neq0\),求\(\frac{d^2x}{dy^2}\)

根据反函数的求导法则\(\frac{dx}{dy}=\frac{1}{y'}\),于是
\begin{equation*}
\frac{d^2x}{dy^2}=\frac{d}{dy}\left(\frac{dx}{dy}\right)=
\frac{d}{dx}\left(\frac{dx}{dy}\right)\frac{dx}{dy}
\end{equation*}
因为\(\frac{1}{y'}\)是以\(x\)为变量的函数
\end{examplle}

\begin{examplle}[]
设函数\(f(x)=\arctan x-\frac{x}{1+ax^2}\),且\(f'''(0)=1\),求\(a\)

泰勒展开
\begin{align*}
f(x)&=\arctan x-\frac{x}{1+ax^2}=
\left(x-\frac{x^3}{3}+\dots
\right)-x(1-ax^2+\dots)\\
&=(a-\frac{1}{3})x^3+\dots
\end{align*}
因此\(f'''(0)/3!=a-1/3,a=1/2\)
\end{examplle}

\begin{examplle}[]
设\(f(x)\)在\([a,b]\)上连续且\(f(x)>0\),证明存在\(\xi\in(a,b)\)使得
\begin{equation*}
\int_a^\xi f(x)dx=\int_\xi^bf(x)dx=\frac{1}{2}\int_a^bf(x)dx
\end{equation*}

令\(F(x)=\int_a^xf(t)dt-\int_x^bf(t)dt\),则\(F(x)\)在\([a,b]\)上连续,且
\begin{equation*}
F(a)F(b)=-\left[\int_a^bf(t)dt\right]^2<0
\end{equation*}
故由连续函数的零点定理知:在\((a,b)\)内存在 \(\xi\) 使得\(F(\xi)=0\),即\(\int_a^\xi f(x)dx=\int_\xi^bf(x)dx\)
\end{examplle}

\begin{examplle}[]
设\(f(x),g(x)\)在\([a,b]\)上连续,证明存在\(\xi\in(a,b)\)使得
\begin{equation*}
g(\xi)\int_a^\xi f(x)dx=f(\xi)\int_\xi^bg(x)dx
\end{equation*}

令\(F'(x)=g(x)\int_a^x
  f(x)dx-f(x)\int_x^bg(x)dx=(\int^x_af(t)dt\int_b^xg(t)dt)'\),可取辅助函数
\(F(x)=\int_a^xf(t)dt\int_x^bg(t)dt\)。则\(F(a)=F(b)=0\),则存在
\(\xi\in(a,b)\)使得\(F'(\xi)=0\)
\end{examplle}

\begin{examplle}[]
设实数\(a_1,\dots,a_n\)满足关系式
\(a_1-\frac{a_2}{3}+\dots+(-1)^{n-1}\frac{a_n}{2n-1}=0\),证明方程
\(a_1\cos x+a_2\cos 3x+\dots+a_n\cos(2n-1)x=0\)在\((0,\frac{\pi}{2})\)内至少有一
实根

令\(f(x)=a_1\cos x+a_2\cos 3x+\dots+a_n\cos(2n-1)x\),但\(f(x)\)在
\([0,\frac{\pi}{2}]\)内不满足零点定理,因此考虑
\(f'(x)=a_1\cos x+a_2\cos 3x+\dots+a_n\cos(2n-1)x\),则
\(f(x)=a_1\cos x+\frac{a_2}{3}\sin 3x+\dots+\frac{a_n}{2n-1}\sin(2n-1)x\),则
\(f(0)=f(\pi/2)=0\)
\end{examplle}

\begin{examplle}[]
试确定方程\(e^x=ax^2(a>0)\)的根的个数,并指出每个根所在的范围

若直接令\(f(x)=e^x-ax^2\),\(f'(x)\)的符号不易判断。又\(x=0\)不是方程的根,于
是方程可化为等价方程\(\frac{e^x}{x^2}=a\)

令\(f(x)=\frac{e^x}{x^2}-a\),由\(f'(x)=\frac{x-2}{x^3}e^x=0\)得\(x=2\)
\end{examplle}

\begin{examplle}[]
已知方程\(\frac{1}{\ln(1+x)}-\frac{1}{x}=k\)在区间\((0,1)\)内有实根,确定常数
\(k\)的取值范围

令\(f(x)=\frac{1}{\ln(1+x)}-\frac{1}{x}-k\),\(x\in(0,1]\),则
\begin{equation*}
f'(x)=\frac{(1+x)\ln^2(1+x)-x^2}{x^2(1+x)\ln^2(1+x)}
\end{equation*}
因为\(x^2(1+x)\ln^2(1+x)>0\),因此只讨论\(g(x)=(1+x)\ln^2(1+x)-x^2\).
\begin{align*}
&g'(x)=\ln^2(1+x)+2\ln(1+x)-2x\\
&g''(x)=\frac{2\ln(1+x)}{1+x}+\frac{2}{1+x}-2=\frac{2\ln(1+x)-2x}{1+x}
\end{align*}
因此当\(x\in(0,1)\)时,\(g''(x)<0\),而\(g'(0)=0\),因此\(g(x)\)递减
\end{examplle}

\begin{examplle}[]
设\(f(x)\)在\([0,3]\)上连续,在\((0,3)\)内可导,且\(f(0)+f(1)+f(2)=3,f(3)=1\),
证明存在\(\xi\in(0,3)\)使得\(f'(\xi)=0\)

因为\(f(x)\)在\([0,3]\)上连续,所以在\([0,2]\)内必有最大值\(M\)和最小值\(m\),
于是\(m\le f(0)\le M,m\le f(1)\le M,m\le f(2)\le M\),故
\begin{equation*}
m\le\frac{f(0)+f(1)+f(2)}{3}\le M
\end{equation*}
由介值定理,至少存在一点\(\eta\in[0,2]\)使
\begin{equation*}
f(\eta)=\frac{f(0)+f(1)+f(2)}{3}=1
\end{equation*}
因此\(f(\eta)=f(3)=1\),由罗尔定理知,必存在\(\xi\in(\eta,3)\subset(0,3)\)使得\(f'(\xi)=0\)
\end{examplle}

\begin{examplle}[]
设\(f(x)\)在\([0,2]\)上连续,在\((0,2)\)内具有二阶导数且
\(\displaystyle\lim_{x\to\frac{1}{2}}\frac{f(x)}{\cos\pi x}=0\),
\(2\int_{1/2}^1f(x)dx=f(2)\),证明存在\(\xi\in(0,2)\)使得\(f''(\xi)=0\)

\(f(0.5)=0\),因此
\begin{equation*}
f'(0.5)=\lim_{x\to0.5}\frac{f(x)-f(0.5)}{x-0.5}=
\lim_{x\to0.5}\frac{f(x)}{\cos\pi x}\frac{\cos\pi x}{x-0.5}=
\lim_{x\to0.5}\frac{f(x)}{\cos\pi x}\lim_{x\to0.5}\frac{\cos\pi x}{x-0.5}=0
\end{equation*}
再由\(2\int_{0.5}^2f(x)dx=f(2)\),用积分中值定理\(\exists\xi_1\in[0.5,1]\)使得
\(2f(\xi_1)0.5=f(2)\),即\(f(\xi)=f(2)\),在\([\xi_1,2]\)上应用罗尔定理,
\(\exists\xi_2\in(\xi_1,2)\)使\(f'(\xi_2)=0\)

再在\([0.5,\xi_2]\)上对\(f'(x)\)应用罗尔定理,知\(\exists\xi\in(0.5,\xi_2)\),
使\(f''(\xi)=0\)
\end{examplle}

\begin{examplle}[]
设\(f(x)\)在\([0,1]\)上连续,\((0,1)\)内可导,且
\begin{equation*}
f(1)=k\int_0^{\frac{1}{k}}xe^{1-x}f(x)dx,k>1
\end{equation*}
证明:在\((0,1)\)内至少存在一点 \(\xi\) 使\(f'(\xi)=(1-\xi^{-1})f(\xi)\)

\begin{enumerate}
\item \(\xi\) 换为\(x\),\(f'(x)=(1-x^{-1})f(x)\)
\item 变形\(\frac{f'(x)}{f(x)}=1-x^{-1}\)
\item 两边积分\(\ln f(x)=x-\ln x+ \ln C\)
\item 分离常数\(\ln\frac{xf(x)}{e^x}=\ln C\),即\(xe^{-x}f(x)=C\),可令辅助函数
\(F(x)=xe^{-x}f(x)\)
\end{enumerate}


由积分中值定理,存在\(\xi_1\in[0,\frac{1}{k}]\)使得
\(f(1)=\xi_1e^{1-\xi_1}f(\xi_1)\),即\(1\times e^{-1}f(1)=\xi_1
  e^{-\xi_1}f(\xi_1)\)。因此\(F(x)\)满足在\([\xi_1,1]\)内的罗尔定理,因此
存在 \(\xi\) 使得 \(f'(\xi)=(1-\xi^{-1})f(\xi)\)
\end{examplle}

\begin{examplle}[]
设\(f(x)\)在\([a,b]\)上连续,在\((a,b)\)内可导,且\(f(a)=f(b)=\lambda\),证明
存在\(\xi\in(a,b)\)使得\(f'(\xi)+f(\xi)=\lambda\)

\begin{enumerate}
\item \(\xi\) 换为\(x\),\(f'(x)+f(x)=\lambda\)这是关于\(f(x)\)的一阶线性微分方程
\item 解微分方程\(f(x)=e^{-x}(\lambda e^x+C)\)
\item 分离常数\([f(x)-\lambda]e^x=C\),可令辅助函数\(F(x)=[f(x)-\lambda]e^x\)
\end{enumerate}


\(F(a)=F(b)=0\),因此存在\(\xi\in[a,b]\)使得\(F'(\xi)=0\)
\end{examplle}

\begin{examplle}[]
设\(f(x)\)在\([a,b]\)上连续,在\((a,b)\)上可导,求证:存在\(\xi\in(a,b)\)使得
\(f(b)-f(a)=\xi\ln\frac{b}{a}f'(\xi)\)

可变形为
\begin{equation*}
\frac{f(b)-f(a)}{\ln b-\ln a}=\xi f'(\xi)
\end{equation*}
令\(F(x)=\ln x\),由柯西中值定理,存在\(\xi\in(a,b)\)使得
\begin{equation*}
\frac{f(b)-f(a)}{\ln b-\ln a}=\frac{f'(\xi)}{F'(\xi)}=\xi f'(\xi)
\end{equation*}
\end{examplle}

\begin{examplle}[]
设\(f(x)\)在\([-1,1]\)上具有三阶连续导数,且\(f(-1)=0,f(1)=1,f'(0)=0\),证明:
在\((-1,1)\)内存在一点 \(\xi\) 使得\(f'''(\xi)=3\)

泰勒展开
\(f(x)=f(0)+f'(0)x+\frac{1}{2!}f''(0)x^2+\frac{1}{3!}f'''(\xi)x^3,\xi\in(0,x)\),
则
\begin{align*}
&0=f(-1)=f(0)+\frac{1}{2}f''(0)-\frac{1}{6}f'''(\xi_1),-1<\xi_1<0\\
&1=f(1)=f(0)+\frac{1}{2}f''(0)+\frac{1}{6}f'''(\xi_2),0<\xi_2<1
\end{align*}
两式相减得
\begin{equation*}
\frac{f'''(\xi_1)+f'''(\xi_2)}{2}=3
\end{equation*}
由介值定理可证存在\(\xi\in[\xi_1,\xi_2]\)有\(f'''(\xi)=\frac{f'''(\xi_1)+f'''(\xi_2)}{2}=3\)
\end{examplle}

\begin{examplle}[]
设\(f(x)\)在\([a,b]\)上连续,在\((a,b)\)内可导,\(0<a<b\),求证存在
\(\xi,\eta\in(a,b)\)使得\(f'(\xi)=\frac{f'(\eta)}{2\eta}(a+b)\)

根据拉格朗日中值定理至少存在一个\(\xi\in(a,b)\)使得
\begin{equation*}
f'(\xi)=\frac{f(b)-f(a)}{b-a}
\end{equation*}
只要再证存在\(\eta\in(a,b)\)使得
\(\frac{f(b)-f(a)}{b-a}=\frac{f'(\eta)}{2\eta}(a+b)\)即
\begin{equation*}
\frac{f(b)-f(a)}{b^2-a^2}=\frac{f'(\eta)}{2\eta}
\end{equation*}
只要用柯西中值定理
\end{examplle}

\begin{examplle}[]
已知函数\(f(x)\)在\([0,1]\)上连续,在\((0,1)\)内可导,且\(f(0)=0,f(1)=1\),证
明
\begin{enumerate}
\item 存在\(\xi\in(0,1)\)使得\(f(\xi)=1-\xi\)
\item 存在两个不同的点\(\eta,\zeta\in(0,1)\)使得\(f'(\eta)f'(\zeta)=1\)
\end{enumerate}


令\(F(x)=f(x)-1+x\),则\(F(0)=-1,F(1)=1\)

对\([0,\xi],[\xi,1]\)分别用拉格朗日中值定理,则
\begin{equation*}
f'(\eta)f'(\zeta)=\frac{f(\xi)-f(0)}{\xi-0}\frac{f(1)-f(\xi)}{1-\xi}=
\frac{f(\xi)}{\xi}\frac{1-f(\xi)}{1-\xi}=
\frac{1-\xi}{\xi}\frac{\xi}{1-\xi}=1
\end{equation*}
\end{examplle}

\begin{examplle}[]
求证\(\frac{\tan x}{x}>\frac{x}{\sin x},0<x<\frac{\pi}{2}\)

\begin{align*}
&f(x)=\sin x\tan x-x^2\\
&f'(x)=\sin x+\tan x\sec x-2x\\
&f''(x)=\cos x+\sec^3x+\tan^2x\sec x-2\\
&f'''(x)=-\sin x+5\sec^3x\tan x+\tan^3x\sec x=
\sin x(5\sec^4x-1)+\tan^3x\sec x>0
\end{align*}
\end{examplle}

\begin{examplle}[]
设\(a>0,b>0\),证明不等式
\begin{equation*}
a\ln a+b\ln b\ge(a+b)[\ln(a+b)-\ln2]
\end{equation*}

令\(f(x)=x\ln x\),则\(f'(x)=\ln x+1,f''(x)=\frac{1}{x}>0\),即曲线\(y=f(x)\)
在\((0,+\infty)\)是凹的,故对任意\(a>0,b>0\),有
\begin{equation*}
\frac{f(a)+f(b)}{2}\ge f(\frac{a+b}{2})
\end{equation*}
代入得
\begin{equation*}
\frac{a\ln a+b\ln b}{2}\ge\frac{a+b}{2}\ln\frac{a+b}{2}
\end{equation*}
\end{examplle}

\begin{examplle}[]
证明:对任意正整数\(n\),都有
\(\frac{1}{n+1}\le\ln(1+\frac{1}{n})<\frac{1}{n}\)

由拉格朗日定理,存在\(\xi\in(n,n+1)\)
\begin{gather*}
\ln(1+\frac{1}{n})=\ln(n+1)-\ln n=\frac{1}{\xi}\\
\frac{1}{n+1}<\frac{1}{\xi}<\frac{1}{n}
\end{gather*}
\end{examplle}

\begin{examplle}[]
设\(f(x)\)在\([0,1]\)上二阶可导,且\(f(0)=f(1)=0\),\(f(x)\)在\([0,1]\)上的最
小值等于\(-1\),证明:至少存在一点\(\xi\in(0,1)\)使\(f''(x)\ge8\)

存在\(a\in(0,1),f'(a)=0,f(a)=-1\),将\(f(x)\)在\(x=a\)泰勒展开
\begin{equation*}
f(x)=f(a)+f'(a)(x-a)+\frac{f''(\xi)}{2!}(x-a)^2=-1+\frac{f''(\xi)}{2}(x-a)^2(\xi\in(a,x)\text{ or }(x,a))
\end{equation*}
令\(x=0,x=1\)得
\begin{gather*}
f(0)=0=-1+\frac{f''(\xi_1)}{2}a^2,0<\xi_1<a\\
f(1)=0=-1+\frac{f''(\xi_2)}{2}(1-a)^2,a<\xi_2<1
\end{gather*}
若\(0<a<\frac{1}{2}\),则\(f''(\xi_1)>8\)


若\(\frac{1}{2}<a<1\),则\(f''(\xi_2)>8\)
\end{examplle}

\begin{examplle}[]
设函数\(f(x)\)在\([0,1]\)上二阶可导,且\(\int_0^1f(x)dx=0\),则当\(f''(x)>0\)
时

\begin{equation*}
f(x)=f(0.5)+f'(0.5)(x-0.5)+\frac{f''(\xi)}{2}(x-0.5)^2
\end{equation*}
积分
\begin{align*}
0&=f(0.5)+f'(0.5)\int_0^1(x-0.5)dx+\frac{f''(\xi)}{2}\int_0^(x-0.5)^2dx\\
&=f(0.5)+\frac{1}{2}f''(\xi)\int_0^1(x-0.5)^2dx
\end{align*}
因此\(f(0.5)<0\)
\end{examplle}

\begin{examplle}[]
设函数\(f(x)\)在点\(x=0\)可导,且\(f(0)=0\),求\(\lim_{x\to0}\frac{f(1-\cos
  x)}{\tan^2x}\)

\begin{align*}
\lim_{x\to0}\frac{f(1-\cos
x)}{\tan^2x}&=
\lim_{x\to0}\frac{f(1-\cos x)-f(0)}{1-\cos x}\frac{1-\cos x}{\tan2^x}\\
&=f'(0)\cdot\frac{1}{2}
\end{align*}
\end{examplle}

\begin{examplle}[]
设\(f(x)\)在\([a,b]\)上连续,在\((a,b)\)内可导,且\(f(a)\cdot f(b)>0,f(a)\cdot
  f(\frac{a+b}{2})<0\),证明: 对任意实数\(k\),存在\(\xi\in(a,b)\)使得$\backslash$(f'(\(\xi\))=kf(\(\xi\)))$\backslash$
\end{examplle}

\begin{examplle}[]
设\(f(x)\)在\([a,b]\)上连续,在\((a,b)\)内可导,且\(f(a)=f(b)=1\),证明:存在
两点\(\xi,\eta\in(a,b)\)使
\begin{equation*}
(e^{2a}+e^{a+b}+e^{2b})[f(\xi)+f'(\xi)]=3e^{3\eta-\xi}
\end{equation*}


\begin{align*}
&(e^{2a}+e^{a+b}+e^{2b})[f(\xi)+f'(\xi)]=3e^{3\eta-\xi}\\
&\Leftrightarrow (e^{2a}+e^{a+b}+e^{2b})[f(\xi)+f'(\xi)]e^{\xi}=3e^{3\eta}\\
&\Leftrightarrow(e^{2a}+e^{a+b}+e^{2b})[e^xf(x)]'|_{x=\xi}=
e^{3x}|_{x=\eta}
\end{align*}

令\(g(x)=e^{3x}\),则由拉格朗日中值定理
\begin{equation*}
g'(\eta)=\frac{g(b)-g(a)}{b-a}
\end{equation*}
即\(\displaystyle  3e^{3\eta}=\frac{e^{3b}-e^{3a}}{b-a}\). 令\(f(x)=e^xf(x)\),
由拉格朗日中值定理,存在\(\xi\in(a,b)\)使得
\begin{equation*}
\frac{e^bf(b)-e^af(a)}{b-a}=e^{\xi}[f(\xi)+f'(\xi)]=\frac{e^b-e^a}{b-a}
\end{equation*}
两边同乘\(e^{2a}+e^{a+b}+e^{2b}\)得
\begin{equation*}
\frac{e^{3b}-e^{3a}}{b-a}=(e^{2a}+e^{a+b}+e^{2b})e^{\xi}[f(\xi)+f'(\xi)]
\end{equation*}
\end{examplle}
\subsection{一元函数积分}
\label{sec:orgb7a70bb}
\begin{examplle}[]
求不定积分\(\displaystyle\int\frac{2^x\cdot 3^x}{9^x-4^x}dx\)

\begin{align*}
\int\frac{2^x\cdot 3^x}{9^x-4^x}dx&=
\int\frac{\left(\frac{3}{2}\right)^x}{\left(\frac{3}{2}\right)^{2x}-1}dx=
\frac{1}{\ln\frac{3}{2}}\int\frac{d\left[\left(\frac{3}{2}\right)^x\right]}
{\left[\left(\frac{3}{2}\right)^{2x}\right]-1}\\
&=\frac{1}{2(\ln3-\ln2)}\ln\abs{\frac{\left(\frac{3}{2}\right)^x-1}
{\left(\frac{3}{2}\right)^x+1}}
\end{align*}
\end{examplle}

\begin{examplle}[]
求\(\displaystyle\int\frac{dx}{\cos x\sqrt{\sin x}}\)

\begin{align*}
\int\frac{dx}{\cos x\sqrt{\sin x}}&=
\int\frac{\cos xdx}{(1-\sin^2x)\sqrt{\sin x}}=
2\int\frac{d(\sqrt{\sin x})}{1-(\sqrt{\sin x})^4}=2\int\frac{dt}{1-t^4}\\
&\int\left(\frac{1}{1+t^2}+\frac{1}{1-t^2}\right)dt
\end{align*}
\end{examplle}

\begin{examplle}[]
求\(\displaystyle\int\frac{dx}{\sqrt{x(4-x)}}\)

\begin{equation*}
\int\frac{dx}{\sqrt{x(4-x)}}=
\int\frac{2d(\sqrt{x})}{\sqrt{4-x}}=2\arcsin\frac{\sqrt{x}}{2}+C
\end{equation*}
\end{examplle}

\begin{examplle}[]
求\(\displaystyle\int\frac{1}{1+e^x}dx\)

\begin{equation*}
\int\frac{1}{1+e^x}dx=\int\frac{e^x}{e^x(1+e^x)}dx=
\int\left(\frac{1}{e^x}-\frac{1}{e^x+1}\right)de^x
\end{equation*}
\end{examplle}

\begin{examplle}[]
求\(\displaystyle\int\frac{xe^x}{\sqrt{e^x-1}}dx\)

令\(\sqrt{e^x-1}=t,x=\ln(1+t^2)\)
\begin{equation*}
\int\frac{xe^x}{\sqrt{e^x-1}}=2\int\ln(1+t^2)dt
\end{equation*}
\end{examplle}


\begin{examplle}[]
求\(\displaystyle\int\frac{dx}{x^4(1+x^2)}\)

\begin{align*}
\int\frac{dx}{x^4(1+x^2)}&=
\int\frac{1+x^2-x^2}{x^4(1+x^2)}dx
\end{align*}
\end{examplle}

\begin{examplle}[]
求\(\displaystyle\int\frac{3x^2-x+4}{x^3-x^2+2x-2}dx\)

\(x^3-x^2+2x-2=(x^2+2)(x-1)\),令
\begin{equation*}
\frac{3x^2-x+4}{x^3-x^2+2x-2}=
\frac{A}{x-1}+\frac{Bx+C}{x^2+2}
\end{equation*}
\end{examplle}

\begin{examplle}[]
求\(\displaystyle\int\frac{dx}{1+\sin x}\)

\begin{equation*}
\int\frac{dx}{1+\sin x}=\int\frac{1-\sin x}{\cos^2 x}=
\int\frac{dx}{\cos^2x}-\int\frac{\sin x}{\cos^2 x}=\tan x-\frac{1}{\cos x}+C
\end{equation*}
\end{examplle}

\begin{examplle}[]
求\(I_n=\int\tan^nxdx\)的递推公式

\begin{align*}
I_n&=\int\tan^{n-2}x(\sec^2x-1)dx=\int\tan^{n-2}x\sec^2 xdx-\int\tan^{n-2}xdx\\
&=\frac{1}{n-1}\tan^{n-1}x-I_{n-2}
\end{align*}
\end{examplle}

\begin{examplle}[]
求\(\displaystyle\lim_{n\to\infty}\int_0^1\frac{x^n}{1+x}dx\)

对于\(0\le x\le1\),有\(0\le\frac{x^n}{1+x}\le x\),则
\begin{equation*}
0\le\int_0^1\frac{x^n}{1+x}dx\le\int^1_0x^ndx=\frac{1}{n+1}
\end{equation*}
因此由夹逼定理,\(\displaystyle\lim_{n\to\infty}\int_0^1\frac{x^n}{1+x}dx=0\)
\end{examplle}

\begin{examplle}[]
求\(\displaystyle\lim_{n\to\infty}n(\frac{1}{1+n^2}+\dots+\frac{1}{n^2+n^2})\)

\begin{align*}
\lim_{n\to\infty}n(\frac{1}{1+n^2}+\dots+\frac{1}{n^2+n^2})&=
\lim_{n\to\infty}\left[
\frac{1}{(\frac{1}{n})^2+1}+\dots+\frac{1}{(\frac{n}{n})^2+1}
\right]\cdot\frac{1}{n}\\
&=\left.\int_0^1\frac{1}{1+x^2}dx=\arctan\right\rvert_0^1=\frac{\pi}{4}
\end{align*}
\end{examplle}

\begin{examplle}[]
证明下列不等式
\begin{equation*}
\frac{\sqrt{\pi}}{80}\pi^2<\int_0^{\frac{\pi}{4}}x\sqrt{\tan x}dx<
\frac{\pi^2}{32}
\end{equation*}

当\(0<x<\frac{\pi}{4}\)时,\(0<x<\tan x<1\),则
\begin{equation*}
\int_0^{\frac{\pi}{4}}x^{3/2}dx<\int_0^{\frac{\pi}{4}}x\sqrt{\tan x}dx
<\int^{\frac{\pi}{4}}_0xdx
\end{equation*}
\end{examplle}

\begin{examplle}[]
求\(\displaystyle\int_2^3\frac{\sqrt{3+2x-x^2}}{(x-1)^2}dx\)

\begin{align*}
\int_2^3\frac{\sqrt{3+2x-x^2}}{(x-1)^2}dx&=
\int_2^3\frac{\sqrt{4-(x-1)^2}}{(x-1)^2}dx=
\int^{\frac{\pi}{2}}_{\frac{\pi}{6}}\frac{\sqrt{4-4\sin^2t}}{4\sin^2t}2\cos tdt\\
&=\int^{\frac{\pi}{2}}_{\frac{\pi}{6}}\frac{\cos^2t}{\sin^t}dt=
\int^{\frac{\pi}{2}}_{\frac{\pi}{6}}(\csc^2t-1)dt=-\cot t\rvert^{\frac{\pi}{2}}_{\frac{\pi}{6}}
-t\rvert^{\frac{\pi}{2}}_{\frac{\pi}{6}}=\sqrt{3}-\frac{\pi}{3}
\end{align*}
\end{examplle}

\begin{examplle}[]
求\(\displaystyle\int_0^{\ln2}\sqrt{1-e^{-2x}}dx\)

令\(e^{-x}=\sin t\),则
\begin{align*}
\int_0^{\ln2}\sqrt{1-e^{-2x}}dx&=
\int_{\frac{\pi}{6}}^{\frac{\pi}{2}}\cos t\cdot\frac{\cos t}{\sin t}dt=
\int_{\frac{\pi}{6}}^{\frac{\pi}{2}}\frac{1}{\sin t}dt-
\int_{\frac{\pi}{6}}^{\frac{\pi}{2}}\sin tdt\\
&=-\ln(\csc t+\cot t)\rvert_{\frac{\pi}{6}}^{\frac{\pi}{2}}-\frac{\sqrt{3}}{2}
=\ln(2+\sqrt{3})-\frac{\sqrt{3}}{2}
\end{align*}
\end{examplle}

\begin{examplle}[]
求\(\displaystyle\int_0^3\arcsin\sqrt{\frac{x}{1+x}}dx\)

令\(\arcsin\sqrt{\frac{x}{1+x}}=t\),则
\(\sin^2u=\frac{x}{1+x},x\cos^2u=\sin^2u,x=\tan^2u\)
\begin{align*}
\int_0^3\arcsin\sqrt{\frac{x}{1+x}}dx&=
\left.\int_0^{\frac{\pi}{3}}ud(\tan^2u)=(u\cdot\tan^2u)\right\rvert_0^{\frac{\pi}{3}}
-\int_0^{\frac{\pi}{3}}1\cdot\tan^2udu\\
&\left.=\pi-\int_0^{\frac{\pi}{3}}(\sec^2u-1)du=\pi-\tan u\right\rvert_0^{\frac{\pi}{3}}
+\frac{\pi}{3}\\
&=\frac{4}{3}\pi-\sqrt{3}
\end{align*}
\end{examplle}

\begin{examplle}[]
求\(I=\displaystyle\int_{-\frac{\pi}{4}}^{\frac{\pi}{4}}\frac{\cos^2x}{1+e^{-x}}dx\)

令\(x=-t\),则
\(I=\displaystyle\int_{-\frac{\pi}{4}}^{\frac{\pi}{4}}\frac{\cos^2x}{1+e^{x}}dx\)。
因此
\begin{align*}
I&=\frac{1}{2}\int_{-\frac{\pi}{4}}^{\frac{\pi}{4}}
\left(\frac{\cos^2x}{1+e^{-x}}+\frac{\cos^2x}{1+e^{x}}
\right)dx=
\int^{\frac{\pi}{4}}_0
\left(\frac{1+e^{-x}+1+e^x}{(1+e^{-x})(1+e^x)}
\right)\cos^2xdx\\
&=\int^{\frac{\pi}{4}}_0\cos^2dx=\frac{\pi}{8}+\frac{1}{4}
\end{align*}
\end{examplle}

\begin{remark}
一般地,有如下结论:作变换\(x=a+b-t\)
\begin{equation*}
I=\int^b_af(x)dx=\int^b_af(a+b-t)dt
\end{equation*}
从而\(I=\frac{1}{2}\int^b_a[f(x)+f(a+b-x)]dx\)
\end{remark}

\begin{examplle}[]
求\(I=\displaystyle\int_0^{\frac{\pi}{2}}\frac{\sin^3x}{\sin x+\cos x}dx\)

令\(x=\frac{\pi}{2}-t\),则
\begin{align*}
I&=\int_0^{\frac{\pi}{2}}\frac{\sin^3x+\cos^3x}{\sin x+\cos x}dx=
\frac{1}{2}\int_0^{\frac{\pi}{2}}(\sin^2x-\sin x\cos x+\cos^2x)dx\\
&=\frac{1}{2}\int_0^{\frac{\pi}{2}}(1-\frac{1}{2}\sin 2x)dx=\frac{\pi-1}{4}
\end{align*}
\end{examplle}

\begin{remark}
要求\(I=\displaystyle\int^{\frac{\pi}{2}}_0f(\sin x,\cos x)dx\),可作变换
\(x=\frac{\pi}{2}-t\),则\(I=\displaystyle\int^{\frac{\pi}{2}}_0f(\cos x,\sin x)dx\)
\end{remark}

\begin{examplle}[]
求\(I=\int^\pi_0\frac{x\sin x}{1+\cos^2x}dx\)

令\(x=\pi-t\),则
\begin{align*}
I&=\int^\pi_0\frac{(\pi-t)\sin t}{1+\cos^2t}dt=
\pi\int^\pi_0\frac{\sin t}{1+\cos^2t}dt-I
\end{align*}
\end{examplle}

\begin{remark}
一般地,\(I=\int^\pi_0xf(\sin x)dx=\int^\pi_0(\pi-t)f(\sin
   t)dt=\pi\int^\pi_0f(\sin t)dt-I\)
\end{remark}

\begin{examplle}[]
求\(\int_0^1\frac{x^b-x^a}{\ln x}dx,a,b>0\)

\begin{align*}
\int_0^1\frac{x^b-x^a}{\ln x}dx,a,b>0&=
\int^1_0\left[f^b_ax^tdt
\right]dx=\int^b_a\left[\int^1_0x^tdx
\right]dt\\
&=\ln\frac{b+1}{a+1}
\end{align*}
\end{examplle}

\begin{examplle}[]
设\(\displaystyle f(x)=\int_0^x\frac{\sin t}{\pi-t}dt\),求
\(\int_0^\pi f(x)dx\)

\begin{align*}
\int^\pi_0f(x)dx&=\int_0^\pi f(x)d(x-\pi)\\
&=(x-\pi)f(x)|^\pi_0-\int_0^\pi(x-\pi)f'(x)dx\\
&=-\int_0^\pi (x-\pi)\frac{\sin x}{\pi-x}dx=2
\end{align*}
\end{examplle}

\begin{examplle}[]
证明\(\displaystyle\int_1^af(x^2+\frac{a^2}{x^2})\frac{dx}{x}=
   \int_1^af(x+\frac{a^2}{x})\frac{dx}{x}\)

\begin{align*}
\int_1^af(x^2+\frac{a^2}{x^2})\frac{dx}{x}&=\frac{1}{2}\int_1^{a^2}f(t+\frac{a^2}{t})\frac{dt}{t}\\
&=\frac{1}{2}\int_1^{a}f(t+\frac{a^2}{t})\frac{dt}{t}+
\frac{1}{2}\int_a^{a^2}f(t+\frac{a^2}{t})\frac{dt}{t}
\end{align*}
令\(t=\frac{a^2}{u}\)
\begin{align*}
\frac{1}{2}\int_a^{a^2}f(t+\frac{a^2}{t})\frac{dt}{t}&=
\int^1_af(\frac{a^2}{u}+u)\frac{u}{a^2}\left(-\frac{a^2}{u^2}\right)du\\
&=\int_1^af(u+\frac{a^2}{u})\frac{1}{u}du
\end{align*}
\end{examplle}

\begin{examplle}[]
设\(f(x)\)在\([a,b]\)上有二阶连续导数,又\(f(a)=f'(a)=0\),证明:
\begin{equation*}
\int_a^bf(x)dx=\frac{1}{2}\int_a^bf''(x)(x-b)^2dx
\end{equation*}

利用分部积分
\begin{align*}
\int_a^bf(x)dx&=\int_a^b f(x)d(x-b)=-\int_a^bf'(x)(x-b)d(x-b)\\
&=-\frac{1}{2}\int_a^bf'(x)d(x-b)^2=\frac{1}{2}\int_a^bf''(x)(x-b)^2dx
\end{align*}
\end{examplle}

\begin{examplle}[]
设\(f(x)\)在\([a,b]\)上有二阶连续导数且
\(f(a)=f(b)=0\),\(M=\displaystyle\max_{[a,b]}\abs{f''(x)}\),证明
\(\displaystyle\abs{\int^b_af(x)dx}\le\frac{(b-a)^2}{12}M\)

\begin{align*}
\int_a^bf(x)dx&=\int_a^bf(x)d(x-a)=-\int_a^bf'(x)(x-a)d(x-b)\\
&=\int_a^bf''(x)(x-a)(x-b)dx+\int_a^bf'(x)(x-b)dx\\
&=\int_a^bf''(x)(x-a)(x-b)dx+\int_a^b(x-b)df(x)\\
&=\int_a^bf''(x)(x-a)(x-b)dx-\int_a^bf(x)dx
\end{align*}
则
\begin{equation*}
\int_a^bf(x)dx=\frac{1}{2}\int_a^bf''(x)(x-a)(x-b)dx
\end{equation*}
因此
\begin{align*}
\abs{\int_a^bf(x)dx}&\le\frac{1}{2}M\int_a^b(x-a)(b-a)dx\\
&=\frac{1}{4}M\int_a^b(x-a)^2dx=\frac{(b-a)^3}{12}M
\end{align*}
\end{examplle}

\begin{examplle}[]
设\(f(x)\)在\([a,b]\)上连续且严格单调增,证明:
\begin{equation*}
(a+b)\int_a^bf(x)dx<2\int_a^bxf(x)dx
\end{equation*}

令\(F(x)=(a+x)\int^x_af(t)dt-2\int_a^xtf(t)dt,(a<x\le b)\)
\end{examplle}

\begin{examplle}[]
求
\(\displaystyle\int_{\frac{1}{2}}^{\frac{3}{2}}\frac{1}{\sqrt{\abs{x-x^2}}}dx\)

\begin{align*}
\int_{\frac{1}{2}}^{\frac{3}{2}}\frac{1}{\sqrt{\abs{x-x^2}}}dx&=
\int_{\frac{1}{2}}^1\frac{1}{\sqrt{x-x^2}}dx+
\int_{1}^{\frac{3}{2}}\frac{1}{\sqrt{x^2-x}}dx\\
&=\int_{\frac{1}{2}}^1\frac{1}{\sqrt{\frac{1}{4}-(x-\frac{1}{2})^2}}dx+
\int_{1}^{\frac{3}{2}}\frac{1}{\sqrt{(x-\frac{1}{2})^2-\frac{1}{4}}}dx\\
&=\arcsin(2x-1)\Big\rvert^1_{\frac{1}{2}}+\ln\left[
(x-\frac{1}{2})+\sqrt{(x-\frac{1}{2})-\frac{1}{4}}
\right]\Big\rvert^{\frac{3}{2}}_1
\end{align*}
\end{examplle}

\begin{examplle}[]
求\(\displaystyle\int e^x\frac{1+\sin x}{1+\cos x}dx\)

\begin{align*}
\int e^x\frac{1+\sin x}{1+\cos x}dx&=\int e^x(1+\sin x)\frac{1}{2\cos^2\frac{x}{2}}dx=
\int e^xd\tan\frac{x}{2}+\int e^x\tan\frac{x}{2}dx\\
&=e^x\tan\frac{x}{2}+C
\end{align*}
\end{examplle}

\begin{examplle}[]
设\(f(x)\)为非负连续函数,当\(x\ge0\)时,有\(\int_0^xf(x)f(x-t)dt=e^{2x}-1\),
求\(f(x)\)

\(f(x)\int)0^xf(u)du=e^{2x-1}\),令\(F(x)=\int_0^xf(t)dt\),则有
\(F'(x)F(x)=e^{2x-1},F(0)=0\),两边积分,得
\begin{equation*}
\frac{1}{2}F^2(x)=\frac{1}{2}e^{2x}-x+C
\end{equation*}
由\(F(0)=0\)得,\(C=-\frac{1}{2}\).因此\(F^2(x)=e^{2x}-x-1\),故
\begin{equation*}
f(x)=F'(x)=\frac{e^{2x}-1}{\sqrt{e^{2x}-2x-1}}
\end{equation*}
\end{examplle}

\begin{examplle}[]
设\(\displaystyle f(x)=\int_1^x\frac{\ln t}{1+t}dt(x>0)\),\(g(x)\)连续,且
\(f(x)+f(\frac{1}{x})=\int_0^1g(xt)dt\),求\(g(x)\)

\(\int_0^1g(xt)dt=\frac{1}{x}\int_0^xg(t)dt\),又
\begin{equation*}
f(\frac{1}{x})=\int_0^{\frac{1}{x}}\frac{\ln t}{1+t}dt=
\int_0^x\frac{\ln\frac{1}{u}}{1+\frac{1}{u}}(-\frac{1}{u^2})du=
\int_1^x\frac{\ln u}{u(1+u)}du
\end{equation*}
因此\(f(x)+f(\frac{1}{x})=\int_1^x\frac{\ln t}{t}dt\),于是
\(\int_0^xg(t)dt=x\int_1^x\frac{\ln t}{t}dt\),
\begin{equation*}
g(x)=\int_1^x\frac{\ln t}{t}dt+\ln x=\frac{1}{2}\ln^2x+\ln x
\end{equation*}
\end{examplle}

\begin{examplle}[]
设\(f(x)\)在\([0,+\infty)\)上连续且单调增加,证明:对任意\(a,b>0\),恒有
\begin{equation*}
\int_a^bxf(x)dx\ge\frac{1}{2}\left[
b\int_0^bf(x)dx-a\int_0^af(x)dx
\right]
\end{equation*}

令\(F(x)=x\int_0^xf(t)dt\),则\(F'(x)=\int_0^xf(t)dt+xf(x)\)
\begin{align*}
F(b)-F(a)&=\int_a^bF'(x)dx=\int_a^b
\left[\int_0^xf(t)dt+xf(x)
\right]dx\\
&\le\int_a^b[xf(x)+xf(x)]dx=2\int_a^bxf(x)dx
\end{align*}
\end{examplle}
\subsection{多元函数微积分学}
\label{sec:org40708f7}
\end{document}
