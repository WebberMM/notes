% Created 2020-09-27 日 09:24
% Intended LaTeX compiler: pdflatex
\documentclass[11pt]{article}
\usepackage[utf8]{inputenc}
\usepackage[T1]{fontenc}
\usepackage{graphicx}
\usepackage{grffile}
\usepackage{longtable}
\usepackage{wrapfig}
\usepackage{rotating}
\usepackage[normalem]{ulem}
\usepackage{amsmath}
\usepackage{textcomp}
\usepackage{amssymb}
\usepackage{capt-of}
\usepackage{hyperref}
\usepackage{minted}
%%%%%%%%%%%%%%%%%%%%%%%%%%%%%%%%%%%%%%
%% TIPS                                 %%
%%%%%%%%%%%%%%%%%%%%%%%%%%%%%%%%%%%%%%
% \substack{a\\b} for multiple lines text

\usepackage[utf8]{inputenc}

\usepackage[B1,T1]{fontenc}

% pdfplots will load xolor automatically without option
\usepackage[dvipsnames]{xcolor}
%%%%%%%%%%%%%%%%%%%%%%%%%%%%%%%%%%%%%%%
%% MATH related pacakge                  %%
%%%%%%%%%%%%%%%%%%%%%%%%%%%%%%%%%%%%%%%
% \usepackage{amsmath} mathtools loads the amsmath
\usepackage{amsmath}
\usepackage{mathtools}


\usepackage{amsthm}
\usepackage{amsbsy}

%\usepackage{commath}

\usepackage{amssymb}
\usepackage{mathrsfs}
%\usepackage{mathabx}
\usepackage{stmaryrd}
\usepackage{empheq}

\usepackage{scalerel}
\usepackage{stackengine}
\usepackage{stackrel}

\usepackage{nicematrix}
\usepackage{tensor}
\usepackage{blkarray}
\usepackage{siunitx}
\usepackage[f]{esvect}

\usepackage{unicode-math}
\setmainfont{TeX Gyre Pagella}
% \setmathfont{STIX}
% \setmathfont{texgyrepagella-math.otf}
% \setmathfont{Libertinus Math}
\setmathfont{Latin Modern Math}
\setmathfont[range={\mscra,\mscrb,\mscrc,\mscrd,\mscre,\mscrf,\mscrg,\mscrh,\mscri,\mscrj,\mscrk,\mscrl,\mscrm,\mscrn,\mscro,\mscrp,\mscrq,\mscrr,\mscrs,\mscrt,\mscru,\mscrv,\mscrw,\mscrx,\mscry,\mscrz,\mscrA,\mscrB,\mscrC,\mscrD,\mscrE,\mscrF,\mscrG,\mscrH,\mscrI,\mscrJ,\mscrK,\mscrL,\mscrM,\mscrN,\mscrO,\mscrP,\mscrQ,\mscrR,\mscrS,\mscrT,\mscrU,\mscrV,\mscrW,\mscrX,\mscrY,\mscrZ}]{Latin Modern Math}
\setmathfont[range={\smwhtdiamond,\enclosediamond,\varlrtriangle}]{Latin Modern Math}
\setmathfont[range={\rightrightarrows,\twoheadrightarrow,\leftrightsquigarrow,\triangledown}]{XITS Math}
\setmathfont[range={\int,\setminus}]{Libertinus Math}



%%%%%%%%%%%%%%%%%%%%%%%%%%%%%%%%%%%%%%%
%% TIKZ related packages                 %%
%%%%%%%%%%%%%%%%%%%%%%%%%%%%%%%%%%%%%%%

\usepackage{pgfplots}
\pgfplotsset{compat=1.15}
\usepackage{tikz}
\usepackage{tikz-cd}
\usepackage{tikz-qtree}

\usetikzlibrary{arrows,positioning,calc,fadings,decorations,matrix,decorations,shapes.misc}
%setting from geogebra
\definecolor{ccqqqq}{rgb}{0.8,0,0}


%%%%%%%%%%%%%%%%%%%%%%%%%%%%%%%%%%%%%%%
%% MISCLELLANEOUS packages               %%
%%%%%%%%%%%%%%%%%%%%%%%%%%%%%%%%%%%%%%%
\usepackage[most]{tcolorbox}
\usepackage{threeparttable}
\usepackage{tabularx}

\usepackage{enumitem}

% wrong with preview
\usepackage{subcaption}
\usepackage{caption}
% {\aunclfamily\Huge}
\usepackage{auncial}

\usepackage{float}

\usepackage{fancyhdr}

\usepackage{ifthen}
\usepackage{xargs}


\usepackage{imakeidx}
\usepackage{hyperref}
\usepackage{soul}


%\usepackage[xetex]{preview}
%%%%%%%%%%%%%%%%%%%%%%%%%%%%%%%%%%%%%%%
%% USEPACKAGES end                       %%
%%%%%%%%%%%%%%%%%%%%%%%%%%%%%%%%%%%%%%%

% \setlist{nosep}
% \numberwithin{equation}{subsection}
% \fancyhead{} % Clear the headers
% \renewcommand{\headrulewidth}{0pt} % Width of line at top of page
% \fancyhead[R]{\slshape\leftmark} % Mark right [R] of page with Chapter name [\leftmark]
% \pagestyle{fancy} % Set default style for all content pages (not TOC, etc)


% \newlength\shlength
% \newcommand\vect[2][0]{\setlength\shlength{#1pt}%
%   \stackengine{-5.6pt}{$#2$}{\smash{$\kern\shlength%
%     \stackengine{7.55pt}{$\mathchar"017E$}%
%       {\rule{\widthof{$#2$}}{.57pt}\kern.4pt}{O}{r}{F}{F}{L}\kern-\shlength$}}%
%       {O}{c}{F}{T}{S}}


\indexsetup{othercode=\small}
\makeindex[columns=2,options={-s /media/wu/file/stuuudy/notes/index_style.ist},intoc]
\makeatletter
\def\@idxitem{\par\hangindent 0pt}
\makeatother


%\newcounter{dummy} \numberwithin{dummy}{section}
\newtheorem{dummy}{dummy}[section]
\theoremstyle{definition}
\newtheorem{definition}[dummy]{Definition}
\theoremstyle{plain}
\newtheorem{corollary}[dummy]{Corollary}
\newtheorem{lemma}[dummy]{Lemma}
\newtheorem{proposition}[dummy]{Proposition}
\newtheorem{theorem}[dummy]{Theorem}
\theoremstyle{definition}
\newtheorem{examplle}{Example}[section]
\theoremstyle{remark}
\newtheorem*{remark}{Remark}
\newtheorem{exercise}{Exercise}[subsection]
\newtheorem{observation}{Observation}[section]


\newenvironment{claim}[1]{\par\noindent\textbf{Claim:}\space#1}{}

\makeatletter
\DeclareFontFamily{U}{tipa}{}
\DeclareFontShape{U}{tipa}{m}{n}{<->tipa10}{}
\newcommand{\arc@char}{{\usefont{U}{tipa}{m}{n}\symbol{62}}}%

\newcommand{\arc}[1]{\mathpalette\arc@arc{#1}}

\newcommand{\arc@arc}[2]{%
  \sbox0{$\m@th#1#2$}%
  \vbox{
    \hbox{\resizebox{\wd0}{\height}{\arc@char}}
    \nointerlineskip
    \box0
  }%
}
\makeatother

\setcounter{MaxMatrixCols}{20}
%%%%%%% ABS
\DeclarePairedDelimiter\abss{\lvert}{\rvert}%
\DeclarePairedDelimiter\normm{\lVert}{\rVert}%

% Swap the definition of \abs* and \norm*, so that \abs
% and \norm resizes the size of the brackets, and the
% starred version does not.
\makeatletter
\let\oldabs\abss
%\def\abs{\@ifstar{\oldabs}{\oldabs*}}
\newcommand{\abs}{\@ifstar{\oldabs}{\oldabs*}}
\newcommand{\norm}[1]{\left\lVert#1\right\rVert}
%\let\oldnorm\normm
%\def\norm{\@ifstar{\oldnorm}{\oldnorm*}}
%\renewcommand{norm}{\@ifstar{\oldnorm}{\oldnorm*}}
\makeatother

% \newcommand\what[1]{\ThisStyle{%
%     \setbox0=\hbox{$\SavedStyle#1$}%
%     \stackengine{-1.0\ht0+.5pt}{$\SavedStyle#1$}{%
%       \stretchto{\scaleto{\SavedStyle\mkern.15mu\char'136}{2.6\wd0}}{1.4\ht0}%
%     }{O}{c}{F}{T}{S}%
%   }
% }

% \newcommand\wtilde[1]{\ThisStyle{%
%     \setbox0=\hbox{$\SavedStyle#1$}%
%     \stackengine{-.1\LMpt}{$\SavedStyle#1$}{%
%       \stretchto{\scaleto{\SavedStyle\mkern.2mu\AC}{.5150\wd0}}{.6\ht0}%
%     }{O}{c}{F}{T}{S}%
%   }
% }

% \newcommand\wbar[1]{\ThisStyle{%
%     \setbox0=\hbox{$\SavedStyle#1$}%
%     \stackengine{.5pt+\LMpt}{$\SavedStyle#1$}{%
%       \rule{\wd0}{\dimexpr.3\LMpt+.3pt}%
%     }{O}{c}{F}{T}{S}%
%   }
% }

\newcommand{\bl}[1] {\boldsymbol{#1}}
\newcommand{\Wt}[1] {\stackrel{\sim}{\smash{#1}\rule{0pt}{1.1ex}}}
\newcommand{\wt}[1] {\widetilde{#1}}
\newcommand{\tf}[1] {\textbf{#1}}


%For boxed texts in align, use Aboxed{}
%otherwise use boxed{}

\DeclareMathSymbol{\widehatsym}{\mathord}{largesymbols}{"62}
\newcommand\lowerwidehatsym{%
  \text{\smash{\raisebox{-1.3ex}{%
    $\widehatsym$}}}}
\newcommand\fixwidehat[1]{%
  \mathchoice
    {\accentset{\displaystyle\lowerwidehatsym}{#1}}
    {\accentset{\textstyle\lowerwidehatsym}{#1}}
    {\accentset{\scriptstyle\lowerwidehatsym}{#1}}
    {\accentset{\scriptscriptstyle\lowerwidehatsym}{#1}}
  }


\newcommand{\cupdot}{\mathbin{\dot{\cup}}}
\newcommand{\bigcupdot}{\mathop{\dot{\bigcup}}}

\usepackage{graphicx}

\usepackage[toc,page]{appendix}

% text on arrow for xRightarrow
\makeatletter
%\newcommand{\xRightarrow}[2][]{\ext@arrow 0359\Rightarrowfill@{#1}{#2}}
\makeatother

% Arbitrary long arrow
\newcommand{\Rarrow}[1]{%
\parbox{#1}{\tikz{\draw[->](0,0)--(#1,0);}}
}

\newcommand{\LRarrow}[1]{%
\parbox{#1}{\tikz{\draw[<->](0,0)--(#1,0);}}
}


\makeatletter
\providecommand*{\rmodels}{%
  \mathrel{%
    \mathpalette\@rmodels\models
  }%
}
\newcommand*{\@rmodels}[2]{%
  \reflectbox{$\m@th#1#2$}%
}
\makeatother







\newcommand{\trcl}[1]{%
  \mathrm{trcl}{(#1)}
}



% Roman numerals
\makeatletter
\newcommand*{\rom}[1]{\expandafter\@slowromancap\romannumeral #1@}
\makeatother
% \\def \\b\([a-zA-Z]\) {\\boldsymbol{[a-zA-z]}}
% \\DeclareMathOperator{\\b\1}{\\textbf{\1}}


\DeclareMathOperator{\bx}{\textbf{x}}
\DeclareMathOperator{\bz}{\textbf{z}}
\DeclareMathOperator{\bff}{\textbf{f}}
\DeclareMathOperator{\ba}{\textbf{a}}
\DeclareMathOperator{\bk}{\textbf{k}}
\DeclareMathOperator{\bs}{\textbf{s}}
\DeclareMathOperator{\bh}{\textbf{h}}
\DeclareMathOperator{\bc}{\textbf{c}}
\DeclareMathOperator{\br}{\textbf{r}}
\DeclareMathOperator{\bi}{\textbf{i}}
\DeclareMathOperator{\bj}{\textbf{j}}
\DeclareMathOperator{\bn}{\textbf{n}}
\DeclareMathOperator{\be}{\textbf{e}}
\DeclareMathOperator{\bo}{\textbf{o}}
\DeclareMathOperator{\bU}{\textbf{U}}
\DeclareMathOperator{\bL}{\textbf{L}}
\DeclareMathOperator{\bV}{\textbf{V}}
\def \bzero {\mathbf{0}}
\def \btwo {\mathbf{2}}
\DeclareMathOperator{\bv}{\textbf{v}}
\DeclareMathOperator{\bp}{\textbf{p}}
\DeclareMathOperator{\bI}{\textbf{I}}
\DeclareMathOperator{\bM}{\textbf{M}}
\DeclareMathOperator{\bN}{\textbf{N}}
\DeclareMathOperator{\bK}{\textbf{K}}
\DeclareMathOperator{\bt}{\textbf{t}}
\DeclareMathOperator{\bb}{\textbf{b}}
\DeclareMathOperator{\bA}{\textbf{A}}
\DeclareMathOperator{\bX}{\textbf{X}}
\DeclareMathOperator{\bu}{\textbf{u}}
\DeclareMathOperator{\bS}{\textbf{S}}
\DeclareMathOperator{\bZ}{\textbf{Z}}
\DeclareMathOperator{\by}{\textbf{y}}
\DeclareMathOperator{\bw}{\textbf{w}}
\DeclareMathOperator{\bT}{\textbf{T}}
\DeclareMathOperator{\bF}{\textbf{F}}
\DeclareMathOperator{\bmm}{\textbf{m}}
\DeclareMathOperator{\bW}{\textbf{W}}
\DeclareMathOperator{\bR}{\textbf{R}}
\DeclareMathOperator{\bC}{\textbf{C}}
\DeclareMathOperator{\bD}{\textbf{D}}
\DeclareMathOperator{\bE}{\textbf{E}}
\DeclareMathOperator{\bQ}{\textbf{Q}}
\DeclareMathOperator{\bP}{\textbf{P}}
\DeclareMathOperator{\bY}{\textbf{Y}}
\DeclareMathOperator{\bH}{\textbf{H}}
\DeclareMathOperator{\bB}{\textbf{B}}
\DeclareMathOperator{\bG}{\textbf{G}}
\def \blambda {\symbf{\lambda}}
\def \boldeta {\symbf{\eta}}
\def \balpha {\symbf{\alpha}}
\def \bbeta {\symbf{\beta}}
\def \bgamma {\symbf{\gamma}}
\def \bxi {\symbf{\xi}}
\def \bLambda {\symbf{\Lambda}}

\newcommand{\bto}{{\boldsymbol{\to}}}
\newcommand{\Ra}{\Rightarrow}
\newcommand\und[1]{\underline{#1}}
\def \bPhi {\boldsymbol{\Phi}}
\def \btheta {\boldsymbol{\theta}}
\def \bTheta {\boldsymbol{\Theta}}
\def \bmu {\boldsymbol{\mu}}
\def \bphi {\boldsymbol{\phi}}
\def \bSigma {\boldsymbol{\Sigma}}
\def \lb {\left\{}
\def \rb {\right\}}
\def \la {\langle}
\def \ra {\rangle}
\def \caln {\mathcal{N}}
\def \dissum {\displaystyle\Sigma}
\def \dispro {\displaystyle\prod}
\def \E {\mathbb{E}}
\def \Q {\mathbb{Q}}
\def \N {\mathbb{N}}
\def \V {\mathbb{V}}
\def \R {\mathbb{R}}
\def \P {\mathbb{P}}
\def \A {\mathbb{A}}
\def \F {\mathbb{F}}
\def \Z {\mathbb{Z}}
\def \I {\mathbb{I}}
\def \C {\mathbb{C}}
\def \cala {\mathcal{A}}
\def \cale {\mathcal{E}}
\def \calb {\mathcal{B}}
\def \calq {\mathcal{Q}}
\def \calp {\mathcal{P}}
\def \cals {\mathcal{S}}
\def \calx {\mathcal{X}}
\def \caly {\mathcal{Y}}
\def \calg {\mathcal{G}}
\def \cald {\mathcal{D}}
\def \caln {\mathcal{N}}
\def \calr {\mathcal{R}}
\def \calt {\mathcal{T}}
\def \calm {\mathcal{M}}
\def \calw {\mathcal{W}}
\def \calc {\mathcal{C}}
\def \calv {\mathcal{V}}
\def \calf {\mathcal{F}}
\def \calk {\mathcal{K}}
\def \call {\mathcal{L}}
\def \calu {\mathcal{U}}
\def \calo {\mathcal{O}}
\def \calh {\mathcal{H}}
\def \cali {\mathcal{I}}

\def \bcup {\bigcup}

% set theory

\def \zfcc {\textbf{ZFC}^-}
\def \ac  {\textbf{AC}}
\def \gl  {\textbf{L }}
\def \gll {\textbf{L}}
\newcommand{\zfm}{$\textbf{ZF}^-$}

%\def \zfm {$\textbf{ZF}^-$}
\def \zfmm {\textbf{ZF}^-}
\def \wf {\textbf{WF }}
\def \on {\textbf{On }}
\def \cm {\textbf{M }}
\def \cn {\textbf{N }}
\def \cv {\textbf{V }}
\def \zc {\textbf{ZC }}
\def \zcm {\textbf{ZC}}
\def \zff {\textbf{ZF}}
\def \wfm {\textbf{WF}}
\def \onm {\textbf{On}}
\def \cmm {\textbf{M}}
\def \cnm {\textbf{N}}
\def \cvm {\textbf{V}}
\def \gchh {\textbf{GCH}}
\renewcommand{\restriction}{\mathord{\upharpoonright}}
\def \pred {\text{pred}}

\def \rank {\text{rank}}
\def \con {\text{Con}}
\def \deff {\text{Def}}


\def \uin {\underline{\in}}
\def \oin {\overline{\in}}
\def \uR {\underline{R}}
\def \oR {\overline{R}}
\def \uP {\underline{P}}
\def \oP {\overline{P}}

\def \Ra {\Rightarrow}

\def \e {\enspace}

\def \sgn {\operatorname{sgn}}
\def \gen {\operatorname{gen}}
\def \Hom {\operatorname{Hom}}
\def \hom {\operatorname{hom}}
\def \Sub {\operatorname{Sub}}

\def \supp {\operatorname{supp}}

\def \epiarrow {\twoheadarrow}
\def \monoarrow {\rightarrowtail}
\def \rrarrow {\rightrightarrows}

% \def \minus {\text{-}}
% \newcommand{\minus}{\scalebox{0.75}[1.0]{$-$}}
% \DeclareUnicodeCharacter{002D}{\minus}


\def \tril {\triangleleft}

\def \ACF {\text{ACF}}
\def \GL {\text{GL}}
\def \PGL {\text{PGL}}
\def \equal {=}
\def \deg {\text{deg}}
\def \degree {\text{degree}}
\def \app {\text{App}}
\def \FV {\text{FV}}
\def \conv {\text{conv}}
\def \cont {\text{cont}}
\DeclareMathOperator{\cl}{\textbf{CL}}
\DeclareMathOperator{\sg}{sg}
\DeclareMathOperator{\trdeg}{trdeg}
\def \Ord {\text{Ord}}

\DeclareMathOperator{\cf}{cf}
\DeclareMathOperator{\zfc}{ZFC}

%\DeclareMathOperator{\Th}{Th}
%\def \th {\text{Th}}
% \newcommand{\th}{\text{Th}}
\DeclareMathOperator{\type}{type}
\DeclareMathOperator{\zf}{\textbf{ZF}}
\def \fa {\mathfrak{a}}
\def \fb {\mathfrak{b}}
\def \fc {\mathfrak{c}}
\def \fd {\mathfrak{d}}
\def \fe {\mathfrak{e}}
\def \ff {\mathfrak{f}}
\def \fg {\mathfrak{g}}
\def \fh {\mathfrak{h}}
%\def \fi {\mathfrak{i}}
\def \fj {\mathfrak{j}}
\def \fk {\mathfrak{k}}
\def \fl {\mathfrak{l}}
\def \fm {\mathfrak{m}}
\def \fn {\mathfrak{n}}
\def \fo {\mathfrak{o}}
\def \fp {\mathfrak{p}}
\def \fq {\mathfrak{q}}
\def \fr {\mathfrak{r}}
\def \fs {\mathfrak{s}}
\def \ft {\mathfrak{t}}
\def \fu {\mathfrak{u}}
\def \fv {\mathfrak{v}}
\def \fw {\mathfrak{w}}
\def \fx {\mathfrak{x}}
\def \fy {\mathfrak{y}}
\def \fz {\mathfrak{z}}
\def \fA {\mathfrak{A}}
\def \fB {\mathfrak{B}}
\def \fC {\mathfrak{C}}
\def \fD {\mathfrak{D}}
\def \fE {\mathfrak{E}}
\def \fF {\mathfrak{F}}
\def \fG {\mathfrak{G}}
\def \fH {\mathfrak{H}}
\def \fI {\mathfrak{I}}
\def \fJ {\mathfrak{J}}
\def \fK {\mathfrak{K}}
\def \fL {\mathfrak{L}}
\def \fM {\mathfrak{M}}
\def \fN {\mathfrak{N}}
\def \fO {\mathfrak{O}}
\def \fP {\mathfrak{P}}
\def \fQ {\mathfrak{Q}}
\def \fR {\mathfrak{R}}
\def \fS {\mathfrak{S}}
\def \fT {\mathfrak{T}}
\def \fU {\mathfrak{U}}
\def \fV {\mathfrak{V}}
\def \fW {\mathfrak{W}}
\def \fX {\mathfrak{X}}
\def \fY {\mathfrak{Y}}
\def \fZ {\mathfrak{Z}}

\def \sfA {\textsf{A}}
\def \sfB {\textsf{B}}
\def \sfC {\textsf{C}}
\def \sfD {\textsf{D}}
\def \sfE {\textsf{E}}
\def \sfF {\textsf{F}}
\def \sfG {\textsf{G}}
\def \sfH {\textsf{H}}
\def \sfI {\textsf{I}}
\def \sfj {\textsf{J}}
\def \sfK {\textsf{K}}
\def \sfL {\textsf{L}}
\def \sfM {\textsf{M}}
\def \sfN {\textsf{N}}
\def \sfO {\textsf{O}}
\def \sfP {\textsf{P}}
\def \sfQ {\textsf{Q}}
\def \sfR {\textsf{R}}
\def \sfS {\textsf{S}}
\def \sfT {\textsf{T}}
\def \sfU {\textsf{U}}
\def \sfV {\textsf{V}}
\def \sfW {\textsf{W}}
\def \sfX {\textsf{X}}
\def \sfY {\textsf{Y}}
\def \sfZ {\textsf{Z}}
\def \sfa {\textsf{a}}
\def \sfb {\textsf{b}}
\def \sfc {\textsf{c}}
\def \sfd {\textsf{d}}
\def \sfe {\textsf{e}}
\def \sff {\textsf{f}}
\def \sfg {\textsf{g}}
\def \sfh {\textsf{h}}
\def \sfi {\textsf{i}}
\def \sfj {\textsf{j}}
\def \sfk {\textsf{k}}
\def \sfl {\textsf{l}}
\def \sfm {\textsf{m}}
\def \sfn {\textsf{n}}
\def \sfo {\textsf{o}}
\def \sfp {\textsf{p}}
\def \sfq {\textsf{q}}
\def \sfr {\textsf{r}}
\def \sfs {\textsf{s}}
\def \sft {\textsf{t}}
\def \sfu {\textsf{u}}
\def \sfv {\textsf{v}}
\def \sfw {\textsf{w}}
\def \sfx {\textsf{x}}
\def \sfy {\textsf{y}}
\def \sfz {\textsf{z}}



%\DeclareMathOperator{\ker}{ker}
\DeclareMathOperator{\im}{im}

\DeclareMathOperator{\inn}{Inn}
\DeclareMathOperator{\AC}{\textbf{AC}}
\DeclareMathOperator{\cod}{cod}
\DeclareMathOperator{\dom}{dom}
\DeclareMathOperator{\ran}{ran}
\DeclareMathOperator{\textd}{d}
\DeclareMathOperator{\td}{d}
\DeclareMathOperator{\id}{id}
\DeclareMathOperator{\LT}{LT}
\DeclareMathOperator{\Mat}{Mat}
\DeclareMathOperator{\Eq}{Eq}
\DeclareMathOperator{\irr}{irr}
\DeclareMathOperator{\Fr}{Fr}
\DeclareMathOperator{\Gal}{Gal}
\DeclareMathOperator{\lcm}{lcm}
\DeclareMathOperator{\alg}{\text{alg}}
\DeclareMathOperator{\Th}{Th}

\DeclareMathOperator{\DAG}{DAG}
\DeclareMathOperator{\ODAG}{ODAG}

% \varprod
\DeclareSymbolFont{largesymbolsA}{U}{txexa}{m}{n}
\DeclareMathSymbol{\varprod}{\mathop}{largesymbolsA}{16}
% \DeclareMathSymbol{\tonm}{\boldsymbol{\to}\textbf{Nm}}
\def \tonm {\bto\textbf{Nm}}
\def \tohm {\bto\textbf{Hm}}

% Category theory
\DeclareMathOperator{\Ab}{\textbf{Ab}}
\DeclareMathOperator{\Alg}{\textbf{Alg}}
\DeclareMathOperator{\Rng}{\textbf{Rng}}
\DeclareMathOperator{\Sets}{\textbf{Sets}}
\DeclareMathOperator{\Met}{\textbf{Met}}
\DeclareMathOperator{\Aut}{\textbf{Aut}}
\DeclareMathOperator{\RMod}{R-\textbf{Mod}}
\DeclareMathOperator{\RAlg}{R-\textbf{Alg}}
\DeclareMathOperator{\LF}{LF}
\DeclareMathOperator{\op}{op}
% Model theory
\DeclareMathOperator{\tp}{tp}
\DeclareMathOperator{\Diag}{Diag}
\DeclareMathOperator{\el}{el}
\DeclareMathOperator{\depth}{depth}
\DeclareMathOperator{\FO}{FO}
\DeclareMathOperator{\fin}{fin}
\DeclareMathOperator{\qr}{qr}
\DeclareMathOperator{\Mod}{Mod}
\DeclareMathOperator{\TC}{TC}
\DeclareMathOperator{\KH}{KH}
\DeclareMathOperator{\Part}{Part}
\DeclareMathOperator{\Infset}{\textsf{Infset}}
\DeclareMathOperator{\DLO}{\textsf{DLO}}
\DeclareMathOperator{\sfMod}{\textsf{Mod}}
\DeclareMathOperator{\AbG}{\textsf{AbG}}
\DeclareMathOperator{\sfACF}{\textsf{ACF}}
% Computability Theorem
\DeclareMathOperator{\Tot}{Tot}
\DeclareMathOperator{\graph}{graph}
\DeclareMathOperator{\Fin}{Fin}
\DeclareMathOperator{\Cof}{Cof}
\DeclareMathOperator{\lh}{lh}
% Commutative Algebra
\DeclareMathOperator{\ord}{ord}
\DeclareMathOperator{\Idem}{Idem}
\DeclareMathOperator{\zdiv}{z.div}
\DeclareMathOperator{\Frac}{Frac}
\DeclareMathOperator{\rad}{rad}
\DeclareMathOperator{\nil}{nil}
\DeclareMathOperator{\Ann}{Ann}
\DeclareMathOperator{\End}{End}
\DeclareMathOperator{\coim}{coim}
\DeclareMathOperator{\coker}{coker}
\DeclareMathOperator{\Bil}{Bil}
\DeclareMathOperator{\Tril}{Tril}
% Topology
\newcommand{\interior}[1]{%
  {\kern0pt#1}^{\mathrm{o}}%
}

% \makeatletter
% \newcommand{\vect}[1]{%
%   \vbox{\m@th \ialign {##\crcr
%   \vectfill\crcr\noalign{\kern-\p@ \nointerlineskip}
%   $\hfil\displaystyle{#1}\hfil$\crcr}}}
% \def\vectfill{%
%   $\m@th\smash-\mkern-7mu%
%   \cleaders\hbox{$\mkern-2mu\smash-\mkern-2mu$}\hfill
%   \mkern-7mu\raisebox{-3.81pt}[\p@][\p@]{$\mathord\mathchar"017E$}$}

% \newcommand{\amsvect}{%
%   \mathpalette {\overarrow@\vectfill@}}
% \def\vectfill@{\arrowfill@\relbar\relbar{\raisebox{-3.81pt}[\p@][\p@]{$\mathord\mathchar"017E$}}}

% \newcommand{\amsvectb}{%
% \newcommand{\vect}{%
%   \mathpalette {\overarrow@\vectfillb@}}
% \newcommand{\vecbar}{%
%   \scalebox{0.8}{$\relbar$}}
% \def\vectfillb@{\arrowfill@\vecbar\vecbar{\raisebox{-4.35pt}[\p@][\p@]{$\mathord\mathchar"017E$}}}
% \makeatother
% \bigtimes

\DeclareFontFamily{U}{mathx}{\hyphenchar\font45}
\DeclareFontShape{U}{mathx}{m}{n}{
      <5> <6> <7> <8> <9> <10>
      <10.95> <12> <14.4> <17.28> <20.74> <24.88>
      mathx10
      }{}
\DeclareSymbolFont{mathx}{U}{mathx}{m}{n}
\DeclareMathSymbol{\bigtimes}{1}{mathx}{"91}
% \odiv
\DeclareFontFamily{U}{matha}{\hyphenchar\font45}
\DeclareFontShape{U}{matha}{m}{n}{
      <5> <6> <7> <8> <9> <10> gen * matha
      <10.95> matha10 <12> <14.4> <17.28> <20.74> <24.88> matha12
      }{}
\DeclareSymbolFont{matha}{U}{matha}{m}{n}
\DeclareMathSymbol{\odiv}         {2}{matha}{"63}


\newcommand\subsetsim{\mathrel{%
  \ooalign{\raise0.2ex\hbox{\scalebox{0.9}{$\subset$}}\cr\hidewidth\raise-0.85ex\hbox{\scalebox{0.9}{$\sim$}}\hidewidth\cr}}}
\newcommand\simsubset{\mathrel{%
  \ooalign{\raise-0.2ex\hbox{\scalebox{0.9}{$\subset$}}\cr\hidewidth\raise0.75ex\hbox{\scalebox{0.9}{$\sim$}}\hidewidth\cr}}}

\newcommand\simsubsetsim{\mathrel{%
  \ooalign{\raise0ex\hbox{\scalebox{0.8}{$\subset$}}\cr\hidewidth\raise1ex\hbox{\scalebox{0.75}{$\sim$}}\hidewidth\cr\raise-0.95ex\hbox{\scalebox{0.8}{$\sim$}}\cr\hidewidth}}}
\newcommand{\stcomp}[1]{{#1}^{\mathsf{c}}}

\setlength{\baselineskip}{0.8in}

\stackMath
\newcommand\yrightarrow[2][]{\mathrel{%
  \setbox2=\hbox{\stackon{\scriptstyle#1}{\scriptstyle#2}}%
  \stackunder[0pt]{%
    \xrightarrow{\makebox[\dimexpr\wd2\relax]{$\scriptstyle#2$}}%
  }{%
   \scriptstyle#1\,%
  }%
}}
\newcommand\yleftarrow[2][]{\mathrel{%
  \setbox2=\hbox{\stackon{\scriptstyle#1}{\scriptstyle#2}}%
  \stackunder[0pt]{%
    \xleftarrow{\makebox[\dimexpr\wd2\relax]{$\scriptstyle#2$}}%
  }{%
   \scriptstyle#1\,%
  }%
}}
\newcommand\yRightarrow[2][]{\mathrel{%
  \setbox2=\hbox{\stackon{\scriptstyle#1}{\scriptstyle#2}}%
  \stackunder[0pt]{%
    \xRightarrow{\makebox[\dimexpr\wd2\relax]{$\scriptstyle#2$}}%
  }{%
   \scriptstyle#1\,%
  }%
}}
\newcommand\yLeftarrow[2][]{\mathrel{%
  \setbox2=\hbox{\stackon{\scriptstyle#1}{\scriptstyle#2}}%
  \stackunder[0pt]{%
    \xLeftarrow{\makebox[\dimexpr\wd2\relax]{$\scriptstyle#2$}}%
  }{%
   \scriptstyle#1\,%
  }%
}}

\newcommand\altxrightarrow[2][0pt]{\mathrel{\ensurestackMath{\stackengine%
  {\dimexpr#1-7.5pt}{\xrightarrow{\phantom{#2}}}{\scriptstyle\!#2\,}%
  {O}{c}{F}{F}{S}}}}
\newcommand\altxleftarrow[2][0pt]{\mathrel{\ensurestackMath{\stackengine%
  {\dimexpr#1-7.5pt}{\xleftarrow{\phantom{#2}}}{\scriptstyle\!#2\,}%
  {O}{c}{F}{F}{S}}}}

\newenvironment{bsm}{% % short for 'bracketed small matrix'
  \left[ \begin{smallmatrix} }{%
  \end{smallmatrix} \right]}

\newenvironment{psm}{% % short for ' small matrix'
  \left( \begin{smallmatrix} }{%
  \end{smallmatrix} \right)}

\newcommand{\bbar}[1]{\mkern 1.5mu\overline{\mkern-1.5mu#1\mkern-1.5mu}\mkern 1.5mu}

\newcommand{\bigzero}{\mbox{\normalfont\Large\bfseries 0}}
\newcommand{\rvline}{\hspace*{-\arraycolsep}\vline\hspace*{-\arraycolsep}}

\font\zallman=Zallman at 40pt
\font\elzevier=Elzevier at 40pt

\newcommand\isoto{\stackrel{\textstyle\sim}{\smash{\longrightarrow}\rule{0pt}{0.4ex}}}
\newcommand\embto{\stackrel{\textstyle\prec}{\smash{\longrightarrow}\rule{0pt}{0.4ex}}}
\DeclareMathOperator{\VAR}{VAR}
\DeclareMathOperator{\CON}{CON}
\DeclareMathOperator{\WE}{WE}
\author{L. T. E. Gamut}
\date{\today}
\title{Logic Language And Meaning}
\hypersetup{
 pdfauthor={L. T. E. Gamut},
 pdftitle={Logic Language And Meaning},
 pdfkeywords={},
 pdfsubject={},
 pdfcreator={Emacs 26.3 (Org mode 9.4)}, 
 pdflang={English}}
\begin{document}

\maketitle
\tableofcontents \clearpage\section{The Theory of Types and Categorical Grammar}
\label{sec:org54941d4}

\subsection{The Theory of Types}
\label{sec:org2380dcb}
\subsubsection{Type Distinctions in Natural Language}
\label{sec:org8016d3c}
\begin{enumerate}
\item If John is self-satisfied, then there is at least one thing he has in
common with Peter
\end{enumerate}


Sentence (1) contains quantification over properties.

\begin{enumerate}
\setcounter{enumi}{1}
\item Santa Claus has all the attributes of a sadist
\end{enumerate}


If we are to quantify not only over entities but also over properties of
entities, then we need to extend predicate logic by introducing variables
other than the ones we already have, which only range over entities. Besides
predicate letters, we need \textbf{predicate variables}, so that we can quantify over
this kind of variable in the syntax. Letting \(X\) be such a variable, (1)
and (2) may be represented as in (3) and (4):
\begin{enumerate}
\setcounter{enumi}{2}
\item \(Zj\to\exists X(Xj\wedge Xp)\)
\item \(\forall X(\forall x(Sx\to Xx)\to Xs)\)
\end{enumerate}


But second-order predicate logic does not exhaust the expressive power of
natural language. For not only are there natural language sentences which
quantify over properties of entities, but there are also sentences which
attribute properties to these properties of entities in turn.The predicate
\textbf{red} expresses a property of individuals, so the predicate \textbf{color} expresses a
property of properties of individuals. So in a sentence like \textbf{Red is a color},
which we represent as \(\calc(R)\), the second-order predicate \textbf{color} is
applied to the first-order predicate \textbf{red}. We can also quantify over these
properties of properties, as in \textbf{Red has something in common with green}.

Besides higher-order predicates, there are other kinds of expressions which
for linguistic purposes may usefully be added to predicate logic.

Our first class of examples is formed by expressions with \textbf{predicate
adverbials}
\begin{enumerate}
\setcounter{enumi}{4}
\item John is walking quickly
\end{enumerate}


The expression \textbf{quickly} is, form a linguistic perspective, a modifier acting
on the verb \textbf{is walking}. From a logical perspective, the property of walking
quickly is attributed to an entity, John. This property cannot be seen as a
conjunction of two properties, 'being quick' and 'walking'. For sentence (5)
does not mean the same thing as sentence (6):
\begin{enumerate}
\setcounter{enumi}{5}
\item John is walking and John is quick
\end{enumerate}


In logical terms, \textbf{quickly} is an expression which when applied to the
first-order predicate \textbf{walking} result in a new first-order predicate \textbf{walking
quickly}. From a logical point of view, the \textbf{relative adjectives} are
expressions of the same kind. Sentence (7) may be represented in first-order
predicate logic as formula (8)
\begin{enumerate}
\setcounter{enumi}{6}
\item Jumbo is a pink elephant
\item \(Ej\wedge Pj\)
\end{enumerate}


The adjective \textbf{pink} may, in other words, be represented as a standard
first-order predicate. But the same does not apply to relative adjectives
like \textbf{small}. Sentence (9) is the same kind of sentence as (7)
\begin{enumerate}
\setcounter{enumi}{8}
\item Jumbo is a small elephant
\end{enumerate}


But sentence (9) cannot be analyzed as a conjunction of two first-oder
predicates. The formula (10) which we would then obtain:
\begin{enumerate}
\setcounter{enumi}{9}
\item \(Ej\wedge Sj\)
\end{enumerate}


expresses something which is generally false. The relative adjective \textbf{small}
works the same way as the predicate adverbial \textbf{quickly}. When applied to the
predicate \textbf{elephant}, it results in a new predicate \textbf{small elephant}
\subsubsection{Syntax}
\label{sec:orge3ed1d3}
As our two basic types we have \(e\), which is the type of those expressions
which refer to entities, and \(t\), the type of those expressions which
refer to truth values.
\begin{definition}[]
\(\bT\), the set of types, is the smallest set s.t.
\begin{enumerate}
\item \(e,t\in\bT\)
\item if \(a,b\in\bT\), then \(\la a,b\ra\in\bT\)
\end{enumerate}
\end{definition}

An expression of type \(\la a,b\ra\) is an expression which when applied to
an expression of type \(a\) results in an expression of type \(b\). If \(\alpha\) is
an expression of type \(\la a,b\ra\) and \(\beta\) is an expression of type \(a\),
then \(\alpha(\beta)\) will be an expression of type \(b\). This process of applying
an \(\alpha\) of type \(\la a,b\ra\) to a \(\beta\) of type \(a\) is called
\textbf{(functional) application of \(\alpha\) to \(\beta\)}

The \textbf{vocabulary} of a type-theoretical language L contains some symbols which
are shared by all such languages and a number of symbols which are
characteristic of \(L\). The shared part consists of:
\begin{enumerate}
\item For every type \(a\), an infinite set \(\VAR_a\) of variables of type \(a\)
\item The usual connectives \(\wedge,\vee,\to,\neg,\leftrightarrow\)
\item The quantifiers \(\forall\) and \(\exists\)
\item Two brackets ( and )
\item The symbol for identity =
\end{enumerate}


The part of the vocabulary which is characteristic of \(L\) contains
\begin{enumerate}
\setcounter{enumi}{5}
\item for every type \(a\), a (possibly empty) set \(\CON_a^L\) of
constants of type \(a\)
\end{enumerate}


We will write \(v_a\) for variables of type \(a\) and \(c_a\) for constants
of type \(a\).

\begin{definition}[]
\begin{enumerate}
\item If \(\alpha\) is a variable or a constant of type \(a\) in \(L\), then \(\alpha\) is an
expression of type \(a\) in \(L\)
\item If \(\alpha\) is an expression of type \(\la a,b\ra\) in \(L\), and \(\beta\) is an
expression of type \(a\) in \(L\), then \((\alpha(\beta))\) is an expression of
type \(b\) in \(L\)
\item If \(\phi\) and \(\psi\) are expressions of type \(t\) in \(L\), then so are
\(\neg\phi,(\phi\wedge\psi),(\phi\vee\psi)\),\((\phi\to\psi)\) and \(\phi\leftrightarrow\psi\)
\item If \(\phi\) is an expression of type \(t\) in \(L\) and \(v\) is a variable (of
arbitrary type \(a\)), then \(\forall x\phi\) and \(\exists v\phi\) are
expressions of type \(t\) in \(L\)
\item If \(\alpha\) and \(\beta\) are expressions in \(L\) which belong to the same (arbitrary)
type, then \((\alpha=\beta)\) is an expression of type \(t\) in \(L\)
\item Every expression in \(L\) is to be constructed by means of (1) - (5) in a
finite number of steps
\end{enumerate}
\end{definition}

We refer to the set of all expressions in \(L\) of type \(a\) as \(\WE_a^L\)
or, if it is clear which \(L\) is meant, as \(\WE_a\). The \textbf{formulas} are the
elements of \(\WE_t\)
\subsubsection{Semantics}
\label{sec:org4e9dc02}
Given a domain \(D\), one-place predicates are interpreted as the
characteristic functions of subsets of that domain.

The domain of interpretation of expressions of type \(a\), given a domain
\(D\), is written as \(\bD_{a,D}\) and is defined as follows
\begin{definition}[]
\begin{enumerate}
\item \(\bD_{e,D}=D\)
\item \(\bD_{t,D}=\{0,1\}\)
\item \(\bD_{\la a,b\ra,D}=\bD_{b,D}^{\bD_{a,D}}\)
\end{enumerate}
\end{definition}

For example, in the theory of types, a two-place predicate \(L(loves)\) is
an expression of type \(\la e,\la e,t\ra\ra\). The corresponding
interpretation domain \(\bD_{\la e,\la e,t\ra\ra}\) is \((\{0,1\}^D)^D\)

Consider the second-order predicate \(\calc(color)\), which is of type
\(\la\la e,t\ra,t\ra\). The interpretation domain \(\bD_{\la\la
    e,t\ra,t\ra}\) is the set of functions \(\{0,1\}^{\{0,1\}^D}\)

A model \(\bM\) for an language \(L\) for the theory of types consists of a
nonempty domain set \(D\) together with an interpretation function \(I\).
For each type \(a\), \(I\) is a function from \(\CON_a^L\) into \(\bD_{a,D}\).

We must define the concept of \textbf{the interpretation of \(\alpha\) w.r.t. a model \(\bM\)}
\textbf{and an assignment} \(g\), to be written as
\(\llbracket\alpha\rrbracket_{\bM,g}\). The interpretation
function\(\llbracket\;\rrbracket_{\bM,g}\) can be seen as a function which
for all types \(a\), maps \(\WE^L_a\) into \(\bD_{a,D}\).

\begin{definition}[]
\begin{enumerate}
\item If \(\alpha\in \CON_a^l\), then
\(\llbracket\alpha\rrbracket_{\bM,g}=I(\alpha)\)

If \(\alpha\in \VAR_a\), then \(\llbracket\alpha\rrbracket_{\bM,g}=g(\alpha)\)

\item If \(\alpha\in \WE^L_{\la a,b\ra},\beta\in \WE^L_a\), then
\(\llbracket\alpha(\beta)\rrbracket_{\bM,g}=\llbracket\alpha\rrbracket_{\bM,g}(
       \llbracket\beta\rrbracket_{\bM,g})\)

\item If \(\phi,\psi\in \WE_t^L\), then

\(\llbracket\neg\phi\rrbracket_{\bM,g}=1\) iff
\(\llbracket\phi\rrbracket_{\bM,g}=0\)

\(\llbracket\phi\wedge\psi\rrbracket_{\bM,g}=1\) iff
\(\llbracket\phi\rrbracket_{\bM,g}=\llbracket\psi\rrbracket_{\bM,g}=1\)

\(\llbracket\phi\to\psi\rrbracket_{\bM,g}=0\) iff
\(\llbracket\phi\rrbracket_{\bM,g}=1\) and
\(\llbracket\psi\rrbracket_{\bM,g}=0\)

\(\llbracket\phi\leftrightarrow\psi\rrbracket_{\bM,g}\) iff
\(\llbracket\phi\rrbracket_{\bM,g}=\llbracket\psi\rrbracket_{\bM,g}\)

\item if \(\phi\in \WE_t^l,v\in \VAR_a\), then

\(\llbracket\forall v\phi\rrbracket_{\bM,g}=1\) iff for all
\(d\in\bD_{a,D}\):
\(\llbracket\phi\rrbracket_{M,g[v/d]}=1\)

\(\llbracket\exists v\phi\rrbracket_{\bM,g}=1\) iff there is at least one
\(d\in\bD_{a,D}\) s.t.: \(\llbracket\phi\rrbracket_{\bM,g[v/d]}=1\)

\item If \(\alpha,\beta\in \WE_a^L\), then
\(\llbracket\alpha=\beta\rrbracket_{\bM,g}=1\) iff
\(\llbracket\alpha\rrbracket_{\bM,g}=\llbracket\beta\rrbracket_{\bM,g}\)
\end{enumerate}
\end{definition}
\end{document}
