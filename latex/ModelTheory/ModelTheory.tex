% Created 2020-09-19 六 19:55
% Intended LaTeX compiler: pdflatex
\documentclass[11pt]{article}
\usepackage[utf8]{inputenc}
\usepackage[T1]{fontenc}
\usepackage{graphicx}
\usepackage{grffile}
\usepackage{longtable}
\usepackage{wrapfig}
\usepackage{rotating}
\usepackage[normalem]{ulem}
\usepackage{amsmath}
\usepackage{textcomp}
\usepackage{amssymb}
\usepackage{capt-of}
\usepackage{hyperref}
\usepackage{minted}
%%%%%%%%%%%%%%%%%%%%%%%%%%%%%%%%%%%%%%
%% TIPS                                 %%
%%%%%%%%%%%%%%%%%%%%%%%%%%%%%%%%%%%%%%
% \substack{a\\b} for multiple lines text

\usepackage[utf8]{inputenc}

\usepackage[B1,T1]{fontenc}

% pdfplots will load xolor automatically without option
\usepackage[dvipsnames]{xcolor}
%%%%%%%%%%%%%%%%%%%%%%%%%%%%%%%%%%%%%%%
%% MATH related pacakge                  %%
%%%%%%%%%%%%%%%%%%%%%%%%%%%%%%%%%%%%%%%
% \usepackage{amsmath} mathtools loads the amsmath
\usepackage{amsmath}
\usepackage{mathtools}


\usepackage{amsthm}
\usepackage{amsbsy}

%\usepackage{commath}

\usepackage{amssymb}
\usepackage{mathrsfs}
%\usepackage{mathabx}
\usepackage{stmaryrd}
\usepackage{empheq}

\usepackage{scalerel}
\usepackage{stackengine}
\usepackage{stackrel}

\usepackage{nicematrix}
\usepackage{tensor}
\usepackage{blkarray}
\usepackage{siunitx}
\usepackage[f]{esvect}

\usepackage{unicode-math}
\setmainfont{TeX Gyre Pagella}
% \setmathfont{STIX}
% \setmathfont{texgyrepagella-math.otf}
% \setmathfont{Libertinus Math}
\setmathfont{Latin Modern Math}
\setmathfont[range={\mscra,\mscrb,\mscrc,\mscrd,\mscre,\mscrf,\mscrg,\mscrh,\mscri,\mscrj,\mscrk,\mscrl,\mscrm,\mscrn,\mscro,\mscrp,\mscrq,\mscrr,\mscrs,\mscrt,\mscru,\mscrv,\mscrw,\mscrx,\mscry,\mscrz,\mscrA,\mscrB,\mscrC,\mscrD,\mscrE,\mscrF,\mscrG,\mscrH,\mscrI,\mscrJ,\mscrK,\mscrL,\mscrM,\mscrN,\mscrO,\mscrP,\mscrQ,\mscrR,\mscrS,\mscrT,\mscrU,\mscrV,\mscrW,\mscrX,\mscrY,\mscrZ}]{Latin Modern Math}
\setmathfont[range={\smwhtdiamond,\enclosediamond,\varlrtriangle}]{Latin Modern Math}
\setmathfont[range={\rightrightarrows,\twoheadrightarrow,\leftrightsquigarrow,\triangledown}]{XITS Math}
\setmathfont[range={\int,\setminus}]{Libertinus Math}



%%%%%%%%%%%%%%%%%%%%%%%%%%%%%%%%%%%%%%%
%% TIKZ related packages                 %%
%%%%%%%%%%%%%%%%%%%%%%%%%%%%%%%%%%%%%%%

\usepackage{pgfplots}
\pgfplotsset{compat=1.15}
\usepackage{tikz}
\usepackage{tikz-cd}
\usepackage{tikz-qtree}

\usetikzlibrary{arrows,positioning,calc,fadings,decorations,matrix,decorations,shapes.misc}
%setting from geogebra
\definecolor{ccqqqq}{rgb}{0.8,0,0}


%%%%%%%%%%%%%%%%%%%%%%%%%%%%%%%%%%%%%%%
%% MISCLELLANEOUS packages               %%
%%%%%%%%%%%%%%%%%%%%%%%%%%%%%%%%%%%%%%%
\usepackage[most]{tcolorbox}
\usepackage{threeparttable}
\usepackage{tabularx}

\usepackage{enumitem}

% wrong with preview
\usepackage{subcaption}
\usepackage{caption}
% {\aunclfamily\Huge}
\usepackage{auncial}

\usepackage{float}

\usepackage{fancyhdr}

\usepackage{ifthen}
\usepackage{xargs}


\usepackage{imakeidx}
\usepackage{hyperref}
\usepackage{soul}


%\usepackage[xetex]{preview}
%%%%%%%%%%%%%%%%%%%%%%%%%%%%%%%%%%%%%%%
%% USEPACKAGES end                       %%
%%%%%%%%%%%%%%%%%%%%%%%%%%%%%%%%%%%%%%%

% \setlist{nosep}
% \numberwithin{equation}{subsection}
% \fancyhead{} % Clear the headers
% \renewcommand{\headrulewidth}{0pt} % Width of line at top of page
% \fancyhead[R]{\slshape\leftmark} % Mark right [R] of page with Chapter name [\leftmark]
% \pagestyle{fancy} % Set default style for all content pages (not TOC, etc)


% \newlength\shlength
% \newcommand\vect[2][0]{\setlength\shlength{#1pt}%
%   \stackengine{-5.6pt}{$#2$}{\smash{$\kern\shlength%
%     \stackengine{7.55pt}{$\mathchar"017E$}%
%       {\rule{\widthof{$#2$}}{.57pt}\kern.4pt}{O}{r}{F}{F}{L}\kern-\shlength$}}%
%       {O}{c}{F}{T}{S}}


\indexsetup{othercode=\small}
\makeindex[columns=2,options={-s /media/wu/file/stuuudy/notes/index_style.ist},intoc]
\makeatletter
\def\@idxitem{\par\hangindent 0pt}
\makeatother


%\newcounter{dummy} \numberwithin{dummy}{section}
\newtheorem{dummy}{dummy}[section]
\theoremstyle{definition}
\newtheorem{definition}[dummy]{Definition}
\theoremstyle{plain}
\newtheorem{corollary}[dummy]{Corollary}
\newtheorem{lemma}[dummy]{Lemma}
\newtheorem{proposition}[dummy]{Proposition}
\newtheorem{theorem}[dummy]{Theorem}
\theoremstyle{definition}
\newtheorem{examplle}{Example}[section]
\theoremstyle{remark}
\newtheorem*{remark}{Remark}
\newtheorem{exercise}{Exercise}[subsection]
\newtheorem{observation}{Observation}[section]


\newenvironment{claim}[1]{\par\noindent\textbf{Claim:}\space#1}{}

\makeatletter
\DeclareFontFamily{U}{tipa}{}
\DeclareFontShape{U}{tipa}{m}{n}{<->tipa10}{}
\newcommand{\arc@char}{{\usefont{U}{tipa}{m}{n}\symbol{62}}}%

\newcommand{\arc}[1]{\mathpalette\arc@arc{#1}}

\newcommand{\arc@arc}[2]{%
  \sbox0{$\m@th#1#2$}%
  \vbox{
    \hbox{\resizebox{\wd0}{\height}{\arc@char}}
    \nointerlineskip
    \box0
  }%
}
\makeatother

\setcounter{MaxMatrixCols}{20}
%%%%%%% ABS
\DeclarePairedDelimiter\abss{\lvert}{\rvert}%
\DeclarePairedDelimiter\normm{\lVert}{\rVert}%

% Swap the definition of \abs* and \norm*, so that \abs
% and \norm resizes the size of the brackets, and the
% starred version does not.
\makeatletter
\let\oldabs\abss
%\def\abs{\@ifstar{\oldabs}{\oldabs*}}
\newcommand{\abs}{\@ifstar{\oldabs}{\oldabs*}}
\newcommand{\norm}[1]{\left\lVert#1\right\rVert}
%\let\oldnorm\normm
%\def\norm{\@ifstar{\oldnorm}{\oldnorm*}}
%\renewcommand{norm}{\@ifstar{\oldnorm}{\oldnorm*}}
\makeatother

% \newcommand\what[1]{\ThisStyle{%
%     \setbox0=\hbox{$\SavedStyle#1$}%
%     \stackengine{-1.0\ht0+.5pt}{$\SavedStyle#1$}{%
%       \stretchto{\scaleto{\SavedStyle\mkern.15mu\char'136}{2.6\wd0}}{1.4\ht0}%
%     }{O}{c}{F}{T}{S}%
%   }
% }

% \newcommand\wtilde[1]{\ThisStyle{%
%     \setbox0=\hbox{$\SavedStyle#1$}%
%     \stackengine{-.1\LMpt}{$\SavedStyle#1$}{%
%       \stretchto{\scaleto{\SavedStyle\mkern.2mu\AC}{.5150\wd0}}{.6\ht0}%
%     }{O}{c}{F}{T}{S}%
%   }
% }

% \newcommand\wbar[1]{\ThisStyle{%
%     \setbox0=\hbox{$\SavedStyle#1$}%
%     \stackengine{.5pt+\LMpt}{$\SavedStyle#1$}{%
%       \rule{\wd0}{\dimexpr.3\LMpt+.3pt}%
%     }{O}{c}{F}{T}{S}%
%   }
% }

\newcommand{\bl}[1] {\boldsymbol{#1}}
\newcommand{\Wt}[1] {\stackrel{\sim}{\smash{#1}\rule{0pt}{1.1ex}}}
\newcommand{\wt}[1] {\widetilde{#1}}
\newcommand{\tf}[1] {\textbf{#1}}


%For boxed texts in align, use Aboxed{}
%otherwise use boxed{}

\DeclareMathSymbol{\widehatsym}{\mathord}{largesymbols}{"62}
\newcommand\lowerwidehatsym{%
  \text{\smash{\raisebox{-1.3ex}{%
    $\widehatsym$}}}}
\newcommand\fixwidehat[1]{%
  \mathchoice
    {\accentset{\displaystyle\lowerwidehatsym}{#1}}
    {\accentset{\textstyle\lowerwidehatsym}{#1}}
    {\accentset{\scriptstyle\lowerwidehatsym}{#1}}
    {\accentset{\scriptscriptstyle\lowerwidehatsym}{#1}}
  }


\newcommand{\cupdot}{\mathbin{\dot{\cup}}}
\newcommand{\bigcupdot}{\mathop{\dot{\bigcup}}}

\usepackage{graphicx}

\usepackage[toc,page]{appendix}

% text on arrow for xRightarrow
\makeatletter
%\newcommand{\xRightarrow}[2][]{\ext@arrow 0359\Rightarrowfill@{#1}{#2}}
\makeatother

% Arbitrary long arrow
\newcommand{\Rarrow}[1]{%
\parbox{#1}{\tikz{\draw[->](0,0)--(#1,0);}}
}

\newcommand{\LRarrow}[1]{%
\parbox{#1}{\tikz{\draw[<->](0,0)--(#1,0);}}
}


\makeatletter
\providecommand*{\rmodels}{%
  \mathrel{%
    \mathpalette\@rmodels\models
  }%
}
\newcommand*{\@rmodels}[2]{%
  \reflectbox{$\m@th#1#2$}%
}
\makeatother







\newcommand{\trcl}[1]{%
  \mathrm{trcl}{(#1)}
}



% Roman numerals
\makeatletter
\newcommand*{\rom}[1]{\expandafter\@slowromancap\romannumeral #1@}
\makeatother
% \\def \\b\([a-zA-Z]\) {\\boldsymbol{[a-zA-z]}}
% \\DeclareMathOperator{\\b\1}{\\textbf{\1}}


\DeclareMathOperator{\bx}{\textbf{x}}
\DeclareMathOperator{\bz}{\textbf{z}}
\DeclareMathOperator{\bff}{\textbf{f}}
\DeclareMathOperator{\ba}{\textbf{a}}
\DeclareMathOperator{\bk}{\textbf{k}}
\DeclareMathOperator{\bs}{\textbf{s}}
\DeclareMathOperator{\bh}{\textbf{h}}
\DeclareMathOperator{\bc}{\textbf{c}}
\DeclareMathOperator{\br}{\textbf{r}}
\DeclareMathOperator{\bi}{\textbf{i}}
\DeclareMathOperator{\bj}{\textbf{j}}
\DeclareMathOperator{\bn}{\textbf{n}}
\DeclareMathOperator{\be}{\textbf{e}}
\DeclareMathOperator{\bo}{\textbf{o}}
\DeclareMathOperator{\bU}{\textbf{U}}
\DeclareMathOperator{\bL}{\textbf{L}}
\DeclareMathOperator{\bV}{\textbf{V}}
\def \bzero {\mathbf{0}}
\def \btwo {\mathbf{2}}
\DeclareMathOperator{\bv}{\textbf{v}}
\DeclareMathOperator{\bp}{\textbf{p}}
\DeclareMathOperator{\bI}{\textbf{I}}
\DeclareMathOperator{\bM}{\textbf{M}}
\DeclareMathOperator{\bN}{\textbf{N}}
\DeclareMathOperator{\bK}{\textbf{K}}
\DeclareMathOperator{\bt}{\textbf{t}}
\DeclareMathOperator{\bb}{\textbf{b}}
\DeclareMathOperator{\bA}{\textbf{A}}
\DeclareMathOperator{\bX}{\textbf{X}}
\DeclareMathOperator{\bu}{\textbf{u}}
\DeclareMathOperator{\bS}{\textbf{S}}
\DeclareMathOperator{\bZ}{\textbf{Z}}
\DeclareMathOperator{\by}{\textbf{y}}
\DeclareMathOperator{\bw}{\textbf{w}}
\DeclareMathOperator{\bT}{\textbf{T}}
\DeclareMathOperator{\bF}{\textbf{F}}
\DeclareMathOperator{\bmm}{\textbf{m}}
\DeclareMathOperator{\bW}{\textbf{W}}
\DeclareMathOperator{\bR}{\textbf{R}}
\DeclareMathOperator{\bC}{\textbf{C}}
\DeclareMathOperator{\bD}{\textbf{D}}
\DeclareMathOperator{\bE}{\textbf{E}}
\DeclareMathOperator{\bQ}{\textbf{Q}}
\DeclareMathOperator{\bP}{\textbf{P}}
\DeclareMathOperator{\bY}{\textbf{Y}}
\DeclareMathOperator{\bH}{\textbf{H}}
\DeclareMathOperator{\bB}{\textbf{B}}
\DeclareMathOperator{\bG}{\textbf{G}}
\def \blambda {\symbf{\lambda}}
\def \boldeta {\symbf{\eta}}
\def \balpha {\symbf{\alpha}}
\def \bbeta {\symbf{\beta}}
\def \bgamma {\symbf{\gamma}}
\def \bxi {\symbf{\xi}}
\def \bLambda {\symbf{\Lambda}}

\newcommand{\bto}{{\boldsymbol{\to}}}
\newcommand{\Ra}{\Rightarrow}
\newcommand\und[1]{\underline{#1}}
\def \bPhi {\boldsymbol{\Phi}}
\def \btheta {\boldsymbol{\theta}}
\def \bTheta {\boldsymbol{\Theta}}
\def \bmu {\boldsymbol{\mu}}
\def \bphi {\boldsymbol{\phi}}
\def \bSigma {\boldsymbol{\Sigma}}
\def \lb {\left\{}
\def \rb {\right\}}
\def \la {\langle}
\def \ra {\rangle}
\def \caln {\mathcal{N}}
\def \dissum {\displaystyle\Sigma}
\def \dispro {\displaystyle\prod}
\def \E {\mathbb{E}}
\def \Q {\mathbb{Q}}
\def \N {\mathbb{N}}
\def \V {\mathbb{V}}
\def \R {\mathbb{R}}
\def \P {\mathbb{P}}
\def \A {\mathbb{A}}
\def \F {\mathbb{F}}
\def \Z {\mathbb{Z}}
\def \I {\mathbb{I}}
\def \C {\mathbb{C}}
\def \cala {\mathcal{A}}
\def \cale {\mathcal{E}}
\def \calb {\mathcal{B}}
\def \calq {\mathcal{Q}}
\def \calp {\mathcal{P}}
\def \cals {\mathcal{S}}
\def \calx {\mathcal{X}}
\def \caly {\mathcal{Y}}
\def \calg {\mathcal{G}}
\def \cald {\mathcal{D}}
\def \caln {\mathcal{N}}
\def \calr {\mathcal{R}}
\def \calt {\mathcal{T}}
\def \calm {\mathcal{M}}
\def \calw {\mathcal{W}}
\def \calc {\mathcal{C}}
\def \calv {\mathcal{V}}
\def \calf {\mathcal{F}}
\def \calk {\mathcal{K}}
\def \call {\mathcal{L}}
\def \calu {\mathcal{U}}
\def \calo {\mathcal{O}}
\def \calh {\mathcal{H}}
\def \cali {\mathcal{I}}

\def \bcup {\bigcup}

% set theory

\def \zfcc {\textbf{ZFC}^-}
\def \ac  {\textbf{AC}}
\def \gl  {\textbf{L }}
\def \gll {\textbf{L}}
\newcommand{\zfm}{$\textbf{ZF}^-$}

%\def \zfm {$\textbf{ZF}^-$}
\def \zfmm {\textbf{ZF}^-}
\def \wf {\textbf{WF }}
\def \on {\textbf{On }}
\def \cm {\textbf{M }}
\def \cn {\textbf{N }}
\def \cv {\textbf{V }}
\def \zc {\textbf{ZC }}
\def \zcm {\textbf{ZC}}
\def \zff {\textbf{ZF}}
\def \wfm {\textbf{WF}}
\def \onm {\textbf{On}}
\def \cmm {\textbf{M}}
\def \cnm {\textbf{N}}
\def \cvm {\textbf{V}}
\def \gchh {\textbf{GCH}}
\renewcommand{\restriction}{\mathord{\upharpoonright}}
\def \pred {\text{pred}}

\def \rank {\text{rank}}
\def \con {\text{Con}}
\def \deff {\text{Def}}


\def \uin {\underline{\in}}
\def \oin {\overline{\in}}
\def \uR {\underline{R}}
\def \oR {\overline{R}}
\def \uP {\underline{P}}
\def \oP {\overline{P}}

\def \Ra {\Rightarrow}

\def \e {\enspace}

\def \sgn {\operatorname{sgn}}
\def \gen {\operatorname{gen}}
\def \Hom {\operatorname{Hom}}
\def \hom {\operatorname{hom}}
\def \Sub {\operatorname{Sub}}

\def \supp {\operatorname{supp}}

\def \epiarrow {\twoheadarrow}
\def \monoarrow {\rightarrowtail}
\def \rrarrow {\rightrightarrows}

% \def \minus {\text{-}}
% \newcommand{\minus}{\scalebox{0.75}[1.0]{$-$}}
% \DeclareUnicodeCharacter{002D}{\minus}


\def \tril {\triangleleft}

\def \ACF {\text{ACF}}
\def \GL {\text{GL}}
\def \PGL {\text{PGL}}
\def \equal {=}
\def \deg {\text{deg}}
\def \degree {\text{degree}}
\def \app {\text{App}}
\def \FV {\text{FV}}
\def \conv {\text{conv}}
\def \cont {\text{cont}}
\DeclareMathOperator{\cl}{\textbf{CL}}
\DeclareMathOperator{\sg}{sg}
\DeclareMathOperator{\trdeg}{trdeg}
\def \Ord {\text{Ord}}

\DeclareMathOperator{\cf}{cf}
\DeclareMathOperator{\zfc}{ZFC}

%\DeclareMathOperator{\Th}{Th}
%\def \th {\text{Th}}
% \newcommand{\th}{\text{Th}}
\DeclareMathOperator{\type}{type}
\DeclareMathOperator{\zf}{\textbf{ZF}}
\def \fa {\mathfrak{a}}
\def \fb {\mathfrak{b}}
\def \fc {\mathfrak{c}}
\def \fd {\mathfrak{d}}
\def \fe {\mathfrak{e}}
\def \ff {\mathfrak{f}}
\def \fg {\mathfrak{g}}
\def \fh {\mathfrak{h}}
%\def \fi {\mathfrak{i}}
\def \fj {\mathfrak{j}}
\def \fk {\mathfrak{k}}
\def \fl {\mathfrak{l}}
\def \fm {\mathfrak{m}}
\def \fn {\mathfrak{n}}
\def \fo {\mathfrak{o}}
\def \fp {\mathfrak{p}}
\def \fq {\mathfrak{q}}
\def \fr {\mathfrak{r}}
\def \fs {\mathfrak{s}}
\def \ft {\mathfrak{t}}
\def \fu {\mathfrak{u}}
\def \fv {\mathfrak{v}}
\def \fw {\mathfrak{w}}
\def \fx {\mathfrak{x}}
\def \fy {\mathfrak{y}}
\def \fz {\mathfrak{z}}
\def \fA {\mathfrak{A}}
\def \fB {\mathfrak{B}}
\def \fC {\mathfrak{C}}
\def \fD {\mathfrak{D}}
\def \fE {\mathfrak{E}}
\def \fF {\mathfrak{F}}
\def \fG {\mathfrak{G}}
\def \fH {\mathfrak{H}}
\def \fI {\mathfrak{I}}
\def \fJ {\mathfrak{J}}
\def \fK {\mathfrak{K}}
\def \fL {\mathfrak{L}}
\def \fM {\mathfrak{M}}
\def \fN {\mathfrak{N}}
\def \fO {\mathfrak{O}}
\def \fP {\mathfrak{P}}
\def \fQ {\mathfrak{Q}}
\def \fR {\mathfrak{R}}
\def \fS {\mathfrak{S}}
\def \fT {\mathfrak{T}}
\def \fU {\mathfrak{U}}
\def \fV {\mathfrak{V}}
\def \fW {\mathfrak{W}}
\def \fX {\mathfrak{X}}
\def \fY {\mathfrak{Y}}
\def \fZ {\mathfrak{Z}}

\def \sfA {\textsf{A}}
\def \sfB {\textsf{B}}
\def \sfC {\textsf{C}}
\def \sfD {\textsf{D}}
\def \sfE {\textsf{E}}
\def \sfF {\textsf{F}}
\def \sfG {\textsf{G}}
\def \sfH {\textsf{H}}
\def \sfI {\textsf{I}}
\def \sfj {\textsf{J}}
\def \sfK {\textsf{K}}
\def \sfL {\textsf{L}}
\def \sfM {\textsf{M}}
\def \sfN {\textsf{N}}
\def \sfO {\textsf{O}}
\def \sfP {\textsf{P}}
\def \sfQ {\textsf{Q}}
\def \sfR {\textsf{R}}
\def \sfS {\textsf{S}}
\def \sfT {\textsf{T}}
\def \sfU {\textsf{U}}
\def \sfV {\textsf{V}}
\def \sfW {\textsf{W}}
\def \sfX {\textsf{X}}
\def \sfY {\textsf{Y}}
\def \sfZ {\textsf{Z}}
\def \sfa {\textsf{a}}
\def \sfb {\textsf{b}}
\def \sfc {\textsf{c}}
\def \sfd {\textsf{d}}
\def \sfe {\textsf{e}}
\def \sff {\textsf{f}}
\def \sfg {\textsf{g}}
\def \sfh {\textsf{h}}
\def \sfi {\textsf{i}}
\def \sfj {\textsf{j}}
\def \sfk {\textsf{k}}
\def \sfl {\textsf{l}}
\def \sfm {\textsf{m}}
\def \sfn {\textsf{n}}
\def \sfo {\textsf{o}}
\def \sfp {\textsf{p}}
\def \sfq {\textsf{q}}
\def \sfr {\textsf{r}}
\def \sfs {\textsf{s}}
\def \sft {\textsf{t}}
\def \sfu {\textsf{u}}
\def \sfv {\textsf{v}}
\def \sfw {\textsf{w}}
\def \sfx {\textsf{x}}
\def \sfy {\textsf{y}}
\def \sfz {\textsf{z}}



%\DeclareMathOperator{\ker}{ker}
\DeclareMathOperator{\im}{im}

\DeclareMathOperator{\inn}{Inn}
\DeclareMathOperator{\AC}{\textbf{AC}}
\DeclareMathOperator{\cod}{cod}
\DeclareMathOperator{\dom}{dom}
\DeclareMathOperator{\ran}{ran}
\DeclareMathOperator{\textd}{d}
\DeclareMathOperator{\td}{d}
\DeclareMathOperator{\id}{id}
\DeclareMathOperator{\LT}{LT}
\DeclareMathOperator{\Mat}{Mat}
\DeclareMathOperator{\Eq}{Eq}
\DeclareMathOperator{\irr}{irr}
\DeclareMathOperator{\Fr}{Fr}
\DeclareMathOperator{\Gal}{Gal}
\DeclareMathOperator{\lcm}{lcm}
\DeclareMathOperator{\alg}{\text{alg}}
\DeclareMathOperator{\Th}{Th}

\DeclareMathOperator{\DAG}{DAG}
\DeclareMathOperator{\ODAG}{ODAG}

% \varprod
\DeclareSymbolFont{largesymbolsA}{U}{txexa}{m}{n}
\DeclareMathSymbol{\varprod}{\mathop}{largesymbolsA}{16}
% \DeclareMathSymbol{\tonm}{\boldsymbol{\to}\textbf{Nm}}
\def \tonm {\bto\textbf{Nm}}
\def \tohm {\bto\textbf{Hm}}

% Category theory
\DeclareMathOperator{\Ab}{\textbf{Ab}}
\DeclareMathOperator{\Alg}{\textbf{Alg}}
\DeclareMathOperator{\Rng}{\textbf{Rng}}
\DeclareMathOperator{\Sets}{\textbf{Sets}}
\DeclareMathOperator{\Met}{\textbf{Met}}
\DeclareMathOperator{\Aut}{\textbf{Aut}}
\DeclareMathOperator{\RMod}{R-\textbf{Mod}}
\DeclareMathOperator{\RAlg}{R-\textbf{Alg}}
\DeclareMathOperator{\LF}{LF}
\DeclareMathOperator{\op}{op}
% Model theory
\DeclareMathOperator{\tp}{tp}
\DeclareMathOperator{\Diag}{Diag}
\DeclareMathOperator{\el}{el}
\DeclareMathOperator{\depth}{depth}
\DeclareMathOperator{\FO}{FO}
\DeclareMathOperator{\fin}{fin}
\DeclareMathOperator{\qr}{qr}
\DeclareMathOperator{\Mod}{Mod}
\DeclareMathOperator{\TC}{TC}
\DeclareMathOperator{\KH}{KH}
\DeclareMathOperator{\Part}{Part}
\DeclareMathOperator{\Infset}{\textsf{Infset}}
\DeclareMathOperator{\DLO}{\textsf{DLO}}
\DeclareMathOperator{\sfMod}{\textsf{Mod}}
\DeclareMathOperator{\AbG}{\textsf{AbG}}
\DeclareMathOperator{\sfACF}{\textsf{ACF}}
% Computability Theorem
\DeclareMathOperator{\Tot}{Tot}
\DeclareMathOperator{\graph}{graph}
\DeclareMathOperator{\Fin}{Fin}
\DeclareMathOperator{\Cof}{Cof}
\DeclareMathOperator{\lh}{lh}
% Commutative Algebra
\DeclareMathOperator{\ord}{ord}
\DeclareMathOperator{\Idem}{Idem}
\DeclareMathOperator{\zdiv}{z.div}
\DeclareMathOperator{\Frac}{Frac}
\DeclareMathOperator{\rad}{rad}
\DeclareMathOperator{\nil}{nil}
\DeclareMathOperator{\Ann}{Ann}
\DeclareMathOperator{\End}{End}
\DeclareMathOperator{\coim}{coim}
\DeclareMathOperator{\coker}{coker}
\DeclareMathOperator{\Bil}{Bil}
\DeclareMathOperator{\Tril}{Tril}
% Topology
\newcommand{\interior}[1]{%
  {\kern0pt#1}^{\mathrm{o}}%
}

% \makeatletter
% \newcommand{\vect}[1]{%
%   \vbox{\m@th \ialign {##\crcr
%   \vectfill\crcr\noalign{\kern-\p@ \nointerlineskip}
%   $\hfil\displaystyle{#1}\hfil$\crcr}}}
% \def\vectfill{%
%   $\m@th\smash-\mkern-7mu%
%   \cleaders\hbox{$\mkern-2mu\smash-\mkern-2mu$}\hfill
%   \mkern-7mu\raisebox{-3.81pt}[\p@][\p@]{$\mathord\mathchar"017E$}$}

% \newcommand{\amsvect}{%
%   \mathpalette {\overarrow@\vectfill@}}
% \def\vectfill@{\arrowfill@\relbar\relbar{\raisebox{-3.81pt}[\p@][\p@]{$\mathord\mathchar"017E$}}}

% \newcommand{\amsvectb}{%
% \newcommand{\vect}{%
%   \mathpalette {\overarrow@\vectfillb@}}
% \newcommand{\vecbar}{%
%   \scalebox{0.8}{$\relbar$}}
% \def\vectfillb@{\arrowfill@\vecbar\vecbar{\raisebox{-4.35pt}[\p@][\p@]{$\mathord\mathchar"017E$}}}
% \makeatother
% \bigtimes

\DeclareFontFamily{U}{mathx}{\hyphenchar\font45}
\DeclareFontShape{U}{mathx}{m}{n}{
      <5> <6> <7> <8> <9> <10>
      <10.95> <12> <14.4> <17.28> <20.74> <24.88>
      mathx10
      }{}
\DeclareSymbolFont{mathx}{U}{mathx}{m}{n}
\DeclareMathSymbol{\bigtimes}{1}{mathx}{"91}
% \odiv
\DeclareFontFamily{U}{matha}{\hyphenchar\font45}
\DeclareFontShape{U}{matha}{m}{n}{
      <5> <6> <7> <8> <9> <10> gen * matha
      <10.95> matha10 <12> <14.4> <17.28> <20.74> <24.88> matha12
      }{}
\DeclareSymbolFont{matha}{U}{matha}{m}{n}
\DeclareMathSymbol{\odiv}         {2}{matha}{"63}


\newcommand\subsetsim{\mathrel{%
  \ooalign{\raise0.2ex\hbox{\scalebox{0.9}{$\subset$}}\cr\hidewidth\raise-0.85ex\hbox{\scalebox{0.9}{$\sim$}}\hidewidth\cr}}}
\newcommand\simsubset{\mathrel{%
  \ooalign{\raise-0.2ex\hbox{\scalebox{0.9}{$\subset$}}\cr\hidewidth\raise0.75ex\hbox{\scalebox{0.9}{$\sim$}}\hidewidth\cr}}}

\newcommand\simsubsetsim{\mathrel{%
  \ooalign{\raise0ex\hbox{\scalebox{0.8}{$\subset$}}\cr\hidewidth\raise1ex\hbox{\scalebox{0.75}{$\sim$}}\hidewidth\cr\raise-0.95ex\hbox{\scalebox{0.8}{$\sim$}}\cr\hidewidth}}}
\newcommand{\stcomp}[1]{{#1}^{\mathsf{c}}}

\setlength{\baselineskip}{0.8in}

\stackMath
\newcommand\yrightarrow[2][]{\mathrel{%
  \setbox2=\hbox{\stackon{\scriptstyle#1}{\scriptstyle#2}}%
  \stackunder[0pt]{%
    \xrightarrow{\makebox[\dimexpr\wd2\relax]{$\scriptstyle#2$}}%
  }{%
   \scriptstyle#1\,%
  }%
}}
\newcommand\yleftarrow[2][]{\mathrel{%
  \setbox2=\hbox{\stackon{\scriptstyle#1}{\scriptstyle#2}}%
  \stackunder[0pt]{%
    \xleftarrow{\makebox[\dimexpr\wd2\relax]{$\scriptstyle#2$}}%
  }{%
   \scriptstyle#1\,%
  }%
}}
\newcommand\yRightarrow[2][]{\mathrel{%
  \setbox2=\hbox{\stackon{\scriptstyle#1}{\scriptstyle#2}}%
  \stackunder[0pt]{%
    \xRightarrow{\makebox[\dimexpr\wd2\relax]{$\scriptstyle#2$}}%
  }{%
   \scriptstyle#1\,%
  }%
}}
\newcommand\yLeftarrow[2][]{\mathrel{%
  \setbox2=\hbox{\stackon{\scriptstyle#1}{\scriptstyle#2}}%
  \stackunder[0pt]{%
    \xLeftarrow{\makebox[\dimexpr\wd2\relax]{$\scriptstyle#2$}}%
  }{%
   \scriptstyle#1\,%
  }%
}}

\newcommand\altxrightarrow[2][0pt]{\mathrel{\ensurestackMath{\stackengine%
  {\dimexpr#1-7.5pt}{\xrightarrow{\phantom{#2}}}{\scriptstyle\!#2\,}%
  {O}{c}{F}{F}{S}}}}
\newcommand\altxleftarrow[2][0pt]{\mathrel{\ensurestackMath{\stackengine%
  {\dimexpr#1-7.5pt}{\xleftarrow{\phantom{#2}}}{\scriptstyle\!#2\,}%
  {O}{c}{F}{F}{S}}}}

\newenvironment{bsm}{% % short for 'bracketed small matrix'
  \left[ \begin{smallmatrix} }{%
  \end{smallmatrix} \right]}

\newenvironment{psm}{% % short for ' small matrix'
  \left( \begin{smallmatrix} }{%
  \end{smallmatrix} \right)}

\newcommand{\bbar}[1]{\mkern 1.5mu\overline{\mkern-1.5mu#1\mkern-1.5mu}\mkern 1.5mu}

\newcommand{\bigzero}{\mbox{\normalfont\Large\bfseries 0}}
\newcommand{\rvline}{\hspace*{-\arraycolsep}\vline\hspace*{-\arraycolsep}}

\font\zallman=Zallman at 40pt
\font\elzevier=Elzevier at 40pt

\newcommand\isoto{\stackrel{\textstyle\sim}{\smash{\longrightarrow}\rule{0pt}{0.4ex}}}
\newcommand\embto{\stackrel{\textstyle\prec}{\smash{\longrightarrow}\rule{0pt}{0.4ex}}}
\author{C. C. Chang \& H. Jerome Keisler}
\date{\today}
\title{Model Theory}
\hypersetup{
 pdfauthor={C. C. Chang \& H. Jerome Keisler},
 pdftitle={Model Theory},
 pdfkeywords={},
 pdfsubject={},
 pdfcreator={Emacs 26.3 (Org mode 9.4)}, 
 pdflang={English}}
\begin{document}

\maketitle
\tableofcontents \clearpage
\section{Models Constructed From Constants}
\label{sec:orgcf8117f}

\subsection{Completeness and Compactness}
\label{sec:orgb2fa140}
\begin{definition}[]
Let \(T\) be a set of sentences of \(\call\) and let \(C\) be a set of
constant symbols of \(\call\). We say that \(C\) is a \textbf{set of witnesses} for
\(T\) iff for every formula \(\varphi\) of \(\call\) with at most one free variable,
say ,\(x\), there is a constant \(c\in C\) s.t.
\begin{equation*}
T\vdash(\exists x)\varphi \to\varphi(c)
\end{equation*}
We say that \(T\) \textbf{has witnesses} in \(\call\) iff \(T\) has some set \(C\) of
witness in \(\call\)
\end{definition}

\begin{lemma}[]
\label{lemma2.1.1}
Let \(T\) be a consistent set of sentences of \(\call\). Let \(C\) be a set
of new constant symbols of power \(\abs{C}=\norm{\call}\), and let
\(\bbar{\call}=\call\cup C\) be the simple extension of \(\call\) formed by
adding \(C\). Then \(T\) can be extended to a consistent set of sentences
\(\bbar{T}\) in \(\bbar{\call}\) which has \(C\) as a set of witnesses in \(\bbar{\call}\)
\end{lemma}

\begin{proof}
Let \(\alpha=\norm{\call}\). For each \(\beta<\alpha\), let \(c_\beta\) be a
constant symbol which does not occur in \(\call\) and s.t. \(\c_\beta\neq
   c_\gamma\) if \(\beta<\gamma<\alpha\). Let \(C=\{c_\beta:\beta<\alpha\}\),
\(\bbar{\call}=\call\cup C\). Clearly \(\norm{\bbar{\call}}=\alpha\), so we
may arrange all formulas of \(\bbar{\call}\) with at most one free variable
in a sequence \(\varphi_\xi,\xi<\alpha\). We now define an increasing sequence
of sets of sentences of \(\bbar{\call}\):
\begin{equation*}
T=T_0\subset T_1\subset\dots\subset T_\xi\subset\dots,\quad\xi<\alpha
\end{equation*}
and a sequence \(d_\xi,\xi<\alpha\) of constants from \(C\) s.t.
\begin{enumerate}
\item each \(T_\xi\) is consistent in \(\bbar{\call}\)
\item if \(\xi=\xi+1\), then \(T_\xi=T_\zeta\cup\{(\exists
      x_\zeta)\varphi_\zeta\to\varphi_\zeta(d_\zeta)\}\); \(\xi_\zeta\) is the
free variable in \(\varphi_\zeta\) if it has one, otherwise \(x_\xi=v_0\)
\end{enumerate}
3.if \(\xi\) is a limit ordinal different from 0, then
\(T_\xi=\bigcup_{\zeta<\xi}T_\zeta\)


Let \(d_\zeta\) be the first element of \(C\) which has not yet occurred in
\(T_\zeta\). We show that
\begin{equation*}
T_{\zeta+1}=T_\zeta\cup\{(\exists x_\zeta)\varphi_\zeta\to\varphi_\zeta(d_\zeta)\}
\end{equation*}
is consistent. If this were not the case, then
\begin{equation*}
   T_\zeta\vdash\neg((\exists x_\zeta)\varphi_\zeta\to\varphi_\zeta(d_\zeta))
\end{equation*}
By propositional logic
\begin{equation*}
T_\zeta\vdash(\exists x_\zeta)\varphi_\zeta\wedge\neg\varphi_\zeta(d_\zeta)
\end{equation*}
As \(d_\zeta\) does not occur in \(T_\zeta\), we have by predicate logic
\begin{gather*}
T_\zeta\vdash(\forall x_\zeta)((\exists x_\zeta)\varphi_\zeta\wedge\neg\varphi_\zeta
(x_\zeta))\\
T_\zeta\vdash(\exists x_\zeta)\varphi_\zeta\wedge\neg(\exists x_\zeta)\varphi_\zeta
\end{gather*}
which contradicts the consistency of \(T_\zeta\). If \(\xi\) is a nonzero limit
ordinal, and each member of the increasing chain \(T_\zeta,\zeta<\xi\) is
consistent, then \(T_\xi\) is consistent.

Now let \(\bbar{T}=\bigcup_{\xi<\alpha}T_\xi\). Suppose \(\varphi\) is a formula of
\(\bbar{\call}\) with at most the variable \(x\) free. Then we may assume
that \(\varphi=\varphi_xi\) and \(x=x_\xi\) for some \(\xi<\alpha\). Whence the
sentence
\begin{equation*}
(\exists x_\xi)\varphi_xi\to\varphi_\xi(d_\xi)
\end{equation*}
belongs to \(T_{\xi+1}\) and so to \(\bbar{T}\)
\end{proof}

\begin{lemma}[]
\label{lemma2.1.2}
Let \(T\) be a consistent set of sentences and \(C\) be a set of witnesses
for \(T\) in \(\call\). Then \(T\) has a model \(\fA\) s.t. every element of
\(\fA\) is an interpretation of a constant \(c\in C\)
\end{lemma}

\begin{proof}
If a set of sentences \(T\) has a set \(C\) of witnesses in \(\call\), then
\(C\) is also a set of witnesses for every extension of \(T\). Second, if an
extension of \(T\) has a model \(\fA\), then \(fA\) is also a model of \(T\).
So we may assume that \(T\) is maximal consistent in \(\call\)

For two constants \(c,d\in C\), define
\begin{equation*}
c\sim d \quad\text{ iff }\quad
c\equiv d\in T
\end{equation*}
Because \(T\) is maximal consistent, we see that \(\sim\) is an equivalence
relation on \(C\). For each \(c\in C\), let
\begin{equation*}
\widetilde{c}=\{d\in C:d\sim c\}
\end{equation*}
be the equivalence class of \(c\). We propose to construct a model \(\fA\)
whose set of elements \(A\) is the set of all these equivalence classes
\(\widetilde{c}\), for \(c\in C\); so we define
\begin{enumerate}
\item \(A=\{\widetilde{c}:c\in C\}\)
\end{enumerate}



\begin{enumerate}
\setcounter{enumi}{1}
\item For each \(n\)-placed relation symbol \(P\) in \(\call\), we define an
\(n\)-placed relation \(R'\) on the set \(C\) by: for all \(c_1,\dots,c_n\in
      C\)

\(R'(c_1,\dots, c_n)\) iff \(P(c_1,\dots,c_n)\in T\)

By our axioms of identity, we have

\begin{equation*}
\vdash P(c_1,\dots,c_n)\wedge c_1\equiv d_1\wedge\dots\wedge c_n\equiv d_n\to
P(d_1,\dots,d_n)
\end{equation*}
So \(\sim\) is what is called a \textbf{congruence relation}.
\(R(\widetilde{c}_1,\dots,\widetilde{c}_n)\) iff
\(P(c_1,\dots,c_n)\in T\)
\item Now consider a constant symbol \(d\) of \(\call\). From predicate logic, we
have
\begin{equation*}
\vdash(\exists v_0)(d\equiv v_0)
\end{equation*}
So \((\exists v_0)(d\equiv v_0)\in T\), and because \(T\) has witnesses,
there is a constant \(c\in C\) s.t.
\begin{equation*}
(d\equiv c)\in T
\end{equation*}

the constant \(c\) may not be unique, but its equivalence class is unique
 because
\begin{equation*}
\vdash(d\equiv c\wedge d\equiv c'\to c\equiv c')
\end{equation*}

\item Let \(F\) be any \(m\)-placed function symbol of \(\call\), and let
\(c_1,\dots,c_m\in C\). We have
\begin{equation*}
(\exists v_0)(F(c_1,\dots,c_m)\equiv v_0)\in T
\end{equation*}
hence there is a constant \(c\in C\) s.t.
\begin{equation*}
(F(c_1,\dots ,c_m)\equiv c)\in T
\end{equation*}
We use our axioms of identity to obtain
\begin{equation*}
\vdash (F(c_1\dots c_m)\equiv c\wedge
c_1\equiv d_1\wedge\dots\wedge c_m\equiv d_m\wedge c\equiv d)\to
F(d_1\dots d_m)\equiv d
\end{equation*}
Hence we define

\(G(\widetilde{c}_1\dots\widetilde{c}_m)\) iff
\((F(c_1\dots c_m)\equiv c)\in T\)

By induction
\begin{equation*}
\fA\models t\equiv c \quad\text{ iff }\quad
(t\equiv c)\in T
\end{equation*}
Since \(C\) is a set of witness for \(T\), we have: for any terms
\(t_1,t_2\) of \(\call\) with no free variables
\begin{equation*}
\fA\models t_1\equiv t_2 \quad\text{ iff }\quad
(t_1\equiv t_2)\in T
\end{equation*}

for any atomic formula \(P(t_1\dots t_n)\) of \(\call\) containing no free
variables
\begin{equation*}
\fA\models P(t_1\dots t_n) \quad\text{ iff }\quad
P(t_1\dots t_n)\in T
\end{equation*}

Hence for any sentence \(\varphi\) of \(\call\)
\begin{equation*}
\fA\models\varphi \quad\text{ iff }\quad
\varphi\in T
\end{equation*}

Suppose \(\varphi=(\exists x)\psi\). If \(fA\models\varphi\), then for
some \(\widetilde{c}\in A,\fA\models\psi[\widetilde{c}]\). This means that
\(\fA\models\psi(c)\). So \(\psi(c)\in T\) and because
\begin{equation*}
\vdash\psi(c)\to(\exists x)\psi
\end{equation*}
we have \(\varphi\in T\). On the other hand, if \(\varphi\in T\), then
because \(T\) has witnesses, there exists a constant \(c\in C\) s.t.
\(\psi(c)\in T\), so \(\fA\models\psi(c)\). This gives
\(\fA\models\psi[\widetilde{c}]\) and \(\fA\models\varphi\)
\end{enumerate}
\end{proof}

\begin{lemma}[]
\label{lemma2.1.3}
Let \(C\) be a set of constant symbols of \(\call\), and let \(T\) be a
set of sentences of \(\call\). If \(T\) has a model \(\fA\) s.t. every
element of \(\fA\) is an interpretation of some constant \(c\in C\), then
\(T\) can be extended to a consistent \(\bbar{T}\) in \(\call\) for which
\(C\) is a set of witnesses
\end{lemma}

\begin{proof}
Let \(\bbar{T}\) be the sentences of \(\call\) true in \(\fA\)
\end{proof}

\begin{theorem}[Extended Completeness Theorem]
   \label{thm1.3.21} 
Let \(\Sigma\) be a set of sentences of \(\call\). Then \(\Sigma\) is consistent iff \(\Sigma\) has a model
\end{theorem}

\begin{proof}
Assume \(\Sigma\) is consistent.  By Lemma \ref{lemma2.1.1} we consider extensions
\(\bbar{\Sigma}\) of \(\Sigma\) and \(\bbar{\call}\) of \(\call\), so that \(\bbar{\Sigma}\) has
witnesses in \(\bbar{\call}\). By Lemma \ref{lemma2.1.2} let \(\fA\) be the
model of \(\bbar{\Sigma}\). Let \(\fB\) be the model for \(\call\) which is the
reduct of \(\fA\) to \(\call\).
\end{proof}

\begin{corollary}[Downward Löwenheim–Skolem Theorem]


Every consistent theory \(T\) in \(\call\) has a model of power at most \(\norm{\call}\)
\end{corollary}

\begin{proof}
Choose \(\fA\) so that every element is a constant.

\(\abs{B}=\abs{A}\le\norm{\bbar{\call}}=\norm{\call}\)
\end{proof}

\begin{theorem}[Gödel's Completeness Theorem]
\label{thm1.3.20}
A sentence of \(\call\) is a theorem of \(\call\) iff it is valid
\end{theorem}

\begin{proof}
If a sentence \(\sigma\) is not a theorem of \(\call\), then \(\{\neg\sigma\}\) is
consistent in \(\call\). By Theorem \ref{thm1.3.21}, \(\{\neg\sigma\}\) will
have a model where \(\sigma\) cannot hold. Hence \(\sigma\) is not valid
\end{proof}

\begin{theorem}[Compactness Theorem]
\label{thm1.3.22}
A set of sentences \(\Sigma\) has a model iff every finite subset of \(\Sigma\) has a model
\end{theorem}

\begin{proof}
If every finite subset of \(\Sigma\) has a model, then every finite subset of \(\Sigma\) is
consistent. So \(\Sigma\) is consistent and has a model by Theorem \ref{thm1.3.21}
\end{proof}

\begin{corollary}[]
If a theory \(T\) has arbitrarily large finite models, then it has an
infinite model
\end{corollary}

\begin{proof}
Consider the expansion \(\call'=\call\cup\{c_n:n\in\omega\}\)  where \(c_n\)
is a list of distinct constant symbols not in \(\call\). Consider the set \(\Sigma\)
of \(\call'\) defined by
\begin{equation*}
\Sigma=T\cup\{\neg(c_n\equiv c_m):n<m<\omega\}
\end{equation*}
Any finite subset \(\Sigma'\) of \(\Sigma\) will involve at most the constants
\(c_0,\dots,c_m\) for some \(m\). Let \(\fA\) be  a model of \(T\) with at
least \(m+1\) elements, and let \(a_0,\dots,a_m\) be a list of \(m+1\)
distinct elements of \(\fA\). The model \((\fA,a_0,\dots,a_m)\) for the
finite expansion \(\call''=\call\cup\{c_0,\dots,c_m\}\) of \(\call\) is a
model of (\(\Sigma\)'). So by Theorem \ref{thm1.3.22} \(\Sigma\) has a model.
\end{proof}

\begin{corollary}[Upward Löwenheim–Skolem-Tarski Theorem]
If \(T\) has infinite models, then it has infinite models of any given power \(\alpha\ge\norm{\call}\)
\end{corollary}


\textbf{Method of diagrams}. Let \(\fA\) be a model of \(\call\). We expand the
language \(\call\) to a new language
\begin{equation*}
\call_A=\call\cup\{c_a:a\in A\}
\end{equation*}
by If \(a\neq b\) and \(c_a,c_b\) are different symbols, we may then expand
\(\fA\) to the model
\begin{equation*}
\fA_A=(\fA,a)_{a\in A}
\end{equation*}
The \textbf{diagram of} \(\fA\), denote by \(\varlrtriangle_{\fA}\), is the set of all
atomic sentences and negations of atomic sentences of \(\call_A\) which hold
in the model \(\fA_A\)

If \(X\) is a subset of \(A\), then we let \(\call_X=\call\cup\{c_a:a\in
   X\}\) and \(\fA_X=(\fA,a)_{a\in X}\). If \(f\) is a mapping from \(X\) into
the set of elements \(B\) of a model \(\fB\) for \(\call\), then
\((\fB,fa)_{a\in X}\) is the expansion of \(\fB\) to a model for \(\call_X\)

\begin{proposition}[]
\label{prop2.1.8}
Let \(\fA,\fB\) be models for \(\call\) and let \(f:A\to B\). Then the
following are equivalent:
\begin{enumerate}
\item \(f\) is an isomorphic embedding of \(\fA\) into \(\fB\)
\item There is an extension \(\fC\supset\fA\) and an isomorphism
\(g:\fC\cong\fB\) s.t. \(g\supset f\)
\item \((\fB,fa)_{a\in A}\) is a model of the diagram of \(\fA\)
\end{enumerate}
\end{proposition}

\begin{proof}
\(1\to2\). Extend the set \(A\) to a set \(C\) and extend the function \(f\)
to a one-to-one function \(g\) from \(C\) onto \(B\). Then define the
relations
\begin{equation*}
\fC\models R[c_1\dots c_n] \quad\text{ iff }\quad
\fB\models R[gc_1\dots gc_n]
\end{equation*}

\(1\leftrightarrow2\). For each formula \(\varphi(x_1\dots x_n)\) and all
\(a_1,\dots,a_n\in A\)
\begin{equation*}
\fA\models\varphi[a_1\dots a_n] \quad\text{ iff }\quad
\fA_A\models\varphi(a_1\dots a_n)
\end{equation*}
and
\begin{equation*}
\fB\models\varphi[fa_1\dots fa_n] \quad\text{ iff }\quad
(\fB,fa)_{a\in A}\models\varphi(a_1\dots a_n)
\end{equation*}
\end{proof}

\begin{corollary}[]
Suppose that \(\call\) has no function or constant symbols. Let \(T\) be a
theory in \(\call\) and \(\fA\) be a model for \(\call\). Then \(\fA\) is
isomorphically embedded in some model of \(T\) iff every finite submodel of
\(\fA\) is isomorphically embedded in some model of \(T\)
\end{corollary}

\begin{proof}
Suppose every finite submodel of \(\fA\) is isomorphically embedded in some
model of \(T\). We show that the set \(\Sigma=T\cup\varlrtriangle_{\fA}\) is
consistent. Every finite subset \(\Sigma'\) of \(\Sigma\) contains at most a finite
number of the new constants, say \(c_{a_1},\dots,c_{a_m}\). Because the
language \(\call\) has no function or constant symbols, the finite set
\(A'=\{a_1,\dots,a_m\}\) generates a finite submodel \(\fA'\) of \(\fA\). Let
\(\fB'\) be a model of \(T\) where \(\fA'\) is isomorphically embedded. Since
\(\Sigma'\subset\Sigma\), by Proposition \ref{prop2.1.8} \(\fB'\) can be
extended to a model of \(\Sigma'\), and hence \(\Sigma'\) has a model. By
campactness, \(\Sigma\) has a model \(\fB\). By Proposition \ref{prop2.1.8} the reduct
of \(\fB\) to \(\call\) gives a mode lof \(T\)
\end{proof}

\subsection{Refinements of the method. Omitting types and interpolation theorems}
\label{sec:orgb7029bc}
\begin{equation*}
\fA\models\Sigma[a_1\dots a_n]
\end{equation*}
for every \(\sigma\in\Sigma,a_1,\dots,a_n\)satisfies \(\sigma\) in \(\fA\); in this
case we say that \(a_1,\dots,a_n\) \textbf{satisfies}, or \textbf{realizes} \(\Sigma\) in \(\fA\).

\(\fA\) \textbf{realizes} \(\Sigma\) iff some \(n\)-tuple of elements of \(A\) satisfies
\(\Sigma\) in \(\fA\). \(\fA\) \textbf{omits} \(\Sigma\) iff \(\fA\) does not realize \(\Sigma\).
\(\Sigma\) is \textbf{satisfiable in \(\fA\)} iff \(\fA\) realizes \(\Sigma\). \(\Sigma\) is \textbf{consistent} iff its satisfiable

By a \textbf{type} \(\Gamma(x_1\dots x_n)\) in the variables \(x_1,\dots,x_n\) we mean a
maximal consistent set of formulas of \(\call\) in these variables. Given any
model \(\fA\) and \(n\)-tuple \(a_1,\dots,a_n\in A\), the set \(\Gamma(x_1\dots
   x_n)\) of all formulas \(\gamma(x_1\dots x_n)\) satisfied by \(a_1,\dots,a_n\) is
a type and is the unique type realized by \(a_1,\dots,a_n\). It is called the
\textbf{type of} \(a_1,\dots,a_n\) in \(\fA\)

\begin{proposition}[]
Let \(T\) be a theory and let \(\Sigma=\Sigma(x_1\dots x_n)\). The following
are equivalent
\begin{enumerate}
\item \(T\) has a model which realizes \(\Sigma\)
\item Every finite subset of \(\Sigma\) is realized in some model of \(T\)
\item \(T\cup\{(\exists x_1\dots x_n)(\sigma_1\wedge\dots\wedge
      \sigma_m):m<\omega,\sigma_1,\dots,\sigma_m\in\Sigma\}\) is consistent
\end{enumerate}
\end{proposition}

Let \(\Sigma=\Sigma(x_1\dots x_n)\) be a set of formulas of \(\call\). A
theory \(T\) in \(\call\) is said to \textbf{locally realize} \(\Sigma\) iff there is a
formula \(\varphi(x_1\dots x_n)\) in \(\call\) s.t.
\begin{enumerate}
\item \(\varphi\) is consistent with \(T\)
\item For all \(\sigma\in\Sigma,T\models\varphi\to\sigma\)
\end{enumerate}


That is, every \(n\)-tuple in a model of \(T\) which satisfies \(\varphi\) realizes \(\Sigma\)

\(T\) \textbf{locally omits} \(\Sigma\) iff \(T\) does not locally realize \(\Sigma\). Thus
\(T\) locally omits \(\Sigma\) iff for every formula \(\varphi(x_1\dots x_n)\) which is
consistent with \(T\), there exists \(\sigma\in\Sigma\) s.t.
\(\varphi\wedge\neg\sigma\) is consistent with \(T\)

\begin{proposition}[]
Let \(T\) be a complete theory in \(\call\), and let \(\Sigma=\Sigma(x_1\dots
   x_n)\) be a set of formulas of \(\call\). If \(T\) has a model which omits \(\Sigma\),
then \(T\) locally omits \(\Sigma\)
\end{proposition}

\begin{proof}
If \(T\) locally realizes \(\Sigma\), then every model of \(T\) realizes \(\Sigma\)
\end{proof}

\begin{theorem}[Omitting Types Theorem]
Let \(T\) be a consistent theory in a countable language \(\call\), and let
\(\Sigma(x_1\dots x_n)\) be a set of formulas. If \(T\) locally omits \(\Sigma\), then
\(T\) has a countable model which omits \(\Sigma\)
\end{theorem}

\begin{proof}
Suppose \(T\) locally omits \(\Sigma(x)\). Let \(C=\{c_0,c_1,\dots\}\) be a
countable set of new constant symbols not already in \(\call\) and let
\(\call'=\call\cup C\). Then \(\call'\) is countable. Arrange all the
sentences of \(\call'\) in a list \(\varphi_0,\varphi_1,\dots\). We shall
construct an increasing sequence of consistent theories
\begin{equation*}
T=T_0\subset T_0\subset\dots\subset T_m\subset\dots
\end{equation*}
s.t.
\begin{enumerate}
\item Each \(T_m\) is a consistent theory of \(\call'\) which is a finite
extension of \(T\)
\item Either \(\varphi_m\in T_{m+1}\) or \((\neg\varphi_m)\in T_{m+1}\)
\item If \(\varphi_m=(\exists x)\psi(x)\) and \(\varphi_m\in T_{m+1}\), then
\(\psi(c_p)\in T_{m+1}\) where \(c_p\) is the first constant not occuring in
\(T_m\) or \(\varphi_m\)
\item There is a formula \(\sigma(x)\in\Sigma(x)\) s.t. \((\neg\sigma(c_m))\in
      T_{m+1}\)
\end{enumerate}


Assuming we already have the theory \(T_m\), we construct \(T_{m+1}\) as
follows: Let \(T_m=T\cup\{\theta_1,\dots,\theta_r\},r>0\) and let
\(\theta=\theta_1\wedge\dots\wedge\theta_r\) . Let \(c_0,\dots,c_n\) contain
all the constants from \(C\) occuring in \(\theta\). For the formula \(\theta(x_m)\) of
\(\call\) by replacing each constant \(c_i\) by \(x_i\)(renaming bound
variables if necessary) and prefixing by \(\exists x_i,i\not\equiv m\). Then
\(\theta(x_m)\) is consistent with \(T\). Therefore for some \(\sigma(x)\in\Sigma(x)\),
\(\theta(x_m)\wedge\neg\sigma(x_m)\) is consistent with \(T\). Put the sentence
\(\neg\sigma(c_m)\) into \(T_{m+1}\). This makes (4) hold

If \(\varphi_m\) is consistent with \(T_m\cup\{\neg\sigma(c_m)\}\), put
\(\varphi_m\) into \(T_{m+1}\). Otherwise put \((\neg\varphi_m)\) into
\(T_{m+1}\). This take care of (2). If \(\varphi_m=(\exists x)\psi(x)\) is
consistent with \(T_m\cup\{\neg(\sigma(c_m))\}\), put \(\psi(c_p)\) into \(T_{m+1}\).
This take care of (3). The theory \(T_{m+1}\) is a consistent finite
extension of \(T_m\). Thus (1) - (4) hold for \(T_{m+1}\)

Let \(T_\omega=\bigcup_{n<\omega}T_n\). From (1) and (2) we see that \(T_\omega\)
is a maximal consistent theory in \(\call'\). Let
\(\fB'=(\fB,b_0,b_1,\dots)\) be a countable model of \(T_\omega\), and let
\(\fA'=(\fA,b_0,b_1,\dots)\) be the submodel of \(\fB'\) generated by the
constants \(b_0,b_1,\dots\) We then see from (3) that
\begin{equation*}
A=\{b_0,b_1,\dots\}
\end{equation*}
Moreover, using (3) and the completeness of \(T_\omega\), we can show by
induction on the complexity of a sentence \(\varphi\) in \(\call'\) that
\begin{equation*}
\fA'\models\varphi,\quad\fB'\models\varphi,\quad T_\omega\models\varphi
\end{equation*}
are all equivalent. Thus \(\fA'\) is a model of \(T_\omega\) and hence
\(\fA\) is a model of \(T\). Finally condition (4) ensures that \(\fA\) omits \(\Sigma\)
\end{proof}

\begin{corollary}[]
Let \(\call\) be countable. A theory \(T\) has a (countable) model omitting
\(\Sigma(x_1\dots x_n)\) iff some complete extension of \(T\) locally omits
\(\Sigma(x_1\dots x_n)\)
\end{corollary}

\begin{examplle}[]
Consider the language \(\call=\{+,\cdot,S,0\}\). We abbreviate
\(1=S0,2=SS0,3=SSS0,\dots\). By an \textbf{\(\omega\)-model} we mean a model \(\fA\)
in which
\begin{equation*}
A=\{0,1,2,3,\dots\}
\end{equation*}
that is, \(\fA\) omits the set \(\{x\not\equiv0,x\not\equiv1,\dots\}\). A
theory \(T\) in \(\call\) is said to be \textbf{\(\omega\)-consistent} iff there is no
formula \(\varphi(x)\) of \(\call\) s.t.
\begin{equation*}
T\models\varphi(0),\quad T\models\varphi(1),\quad T\models\varphi(2),\dots
\end{equation*}
and
\begin{equation*}
T\models(\exists x)\neg\varphi(x)
\end{equation*}
\(T\) is said to be \textbf{\(\omega\)-complete} iff for every formula \(\varphi(x)\) of
\(\call\) we have
\begin{equation*}
T\models\varphi(0),T\models\varphi(1),T\models\varphi(2),\dots\text{ implies }
T\models\(\forall x\)\varphi(x)
\end{equation*}
\end{examplle}

If follows from the omitting types theorem that
\begin{proposition}[]
Let \(T\) be a consistent theory in \(\call\)
\begin{enumerate}
\item If \(T\) is \(\omega\)-complete, then \(T\) has an \(\omega\)-model
\item If \(T\) has an \(\omega\)-model, then \(T\) is \(\omega\)-consistent
\end{enumerate}
\end{proposition}
\end{document}
