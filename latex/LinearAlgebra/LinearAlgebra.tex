% Created 2020-09-06 日 21:12
% Intended LaTeX compiler: pdflatex
\documentclass[11pt]{article}
\usepackage[utf8]{inputenc}
\usepackage[T1]{fontenc}
\usepackage{graphicx}
\usepackage{grffile}
\usepackage{longtable}
\usepackage{wrapfig}
\usepackage{rotating}
\usepackage[normalem]{ulem}
\usepackage{amsmath}
\usepackage{textcomp}
\usepackage{amssymb}
\usepackage{capt-of}
\usepackage{hyperref}
\usepackage{minted}
%%%%%%%%%%%%%%%%%%%%%%%%%%%%%%%%%%%%%%
%% TIPS                                 %%
%%%%%%%%%%%%%%%%%%%%%%%%%%%%%%%%%%%%%%
% \substack{a\\b} for multiple lines text

\usepackage[utf8]{inputenc}

\usepackage[B1,T1]{fontenc}

% pdfplots will load xolor automatically without option
\usepackage[dvipsnames]{xcolor}
%%%%%%%%%%%%%%%%%%%%%%%%%%%%%%%%%%%%%%%
%% MATH related pacakge                  %%
%%%%%%%%%%%%%%%%%%%%%%%%%%%%%%%%%%%%%%%
% \usepackage{amsmath} mathtools loads the amsmath
\usepackage{amsmath}
\usepackage{mathtools}


\usepackage{amsthm}
\usepackage{amsbsy}

%\usepackage{commath}

\usepackage{amssymb}
\usepackage{mathrsfs}
%\usepackage{mathabx}
\usepackage{stmaryrd}
\usepackage{empheq}

\usepackage{scalerel}
\usepackage{stackengine}
\usepackage{stackrel}

\usepackage{nicematrix}
\usepackage{tensor}
\usepackage{blkarray}
\usepackage{siunitx}
\usepackage[f]{esvect}

\usepackage{unicode-math}
\setmainfont{TeX Gyre Pagella}
% \setmathfont{STIX}
% \setmathfont{texgyrepagella-math.otf}
% \setmathfont{Libertinus Math}
\setmathfont{Latin Modern Math}
\setmathfont[range={\mscra,\mscrb,\mscrc,\mscrd,\mscre,\mscrf,\mscrg,\mscrh,\mscri,\mscrj,\mscrk,\mscrl,\mscrm,\mscrn,\mscro,\mscrp,\mscrq,\mscrr,\mscrs,\mscrt,\mscru,\mscrv,\mscrw,\mscrx,\mscry,\mscrz,\mscrA,\mscrB,\mscrC,\mscrD,\mscrE,\mscrF,\mscrG,\mscrH,\mscrI,\mscrJ,\mscrK,\mscrL,\mscrM,\mscrN,\mscrO,\mscrP,\mscrQ,\mscrR,\mscrS,\mscrT,\mscrU,\mscrV,\mscrW,\mscrX,\mscrY,\mscrZ}]{Latin Modern Math}
\setmathfont[range={\smwhtdiamond,\enclosediamond,\varlrtriangle}]{Latin Modern Math}
\setmathfont[range={\rightrightarrows,\twoheadrightarrow,\leftrightsquigarrow,\triangledown}]{XITS Math}
\setmathfont[range={\int,\setminus}]{Libertinus Math}



%%%%%%%%%%%%%%%%%%%%%%%%%%%%%%%%%%%%%%%
%% TIKZ related packages                 %%
%%%%%%%%%%%%%%%%%%%%%%%%%%%%%%%%%%%%%%%

\usepackage{pgfplots}
\pgfplotsset{compat=1.15}
\usepackage{tikz}
\usepackage{tikz-cd}
\usepackage{tikz-qtree}

\usetikzlibrary{arrows,positioning,calc,fadings,decorations,matrix,decorations,shapes.misc}
%setting from geogebra
\definecolor{ccqqqq}{rgb}{0.8,0,0}


%%%%%%%%%%%%%%%%%%%%%%%%%%%%%%%%%%%%%%%
%% MISCLELLANEOUS packages               %%
%%%%%%%%%%%%%%%%%%%%%%%%%%%%%%%%%%%%%%%
\usepackage[most]{tcolorbox}
\usepackage{threeparttable}
\usepackage{tabularx}

\usepackage{enumitem}

% wrong with preview
\usepackage{subcaption}
\usepackage{caption}
% {\aunclfamily\Huge}
\usepackage{auncial}

\usepackage{float}

\usepackage{fancyhdr}

\usepackage{ifthen}
\usepackage{xargs}


\usepackage{imakeidx}
\usepackage{hyperref}
\usepackage{soul}


%\usepackage[xetex]{preview}
%%%%%%%%%%%%%%%%%%%%%%%%%%%%%%%%%%%%%%%
%% USEPACKAGES end                       %%
%%%%%%%%%%%%%%%%%%%%%%%%%%%%%%%%%%%%%%%

% \setlist{nosep}
% \numberwithin{equation}{subsection}
% \fancyhead{} % Clear the headers
% \renewcommand{\headrulewidth}{0pt} % Width of line at top of page
% \fancyhead[R]{\slshape\leftmark} % Mark right [R] of page with Chapter name [\leftmark]
% \pagestyle{fancy} % Set default style for all content pages (not TOC, etc)


% \newlength\shlength
% \newcommand\vect[2][0]{\setlength\shlength{#1pt}%
%   \stackengine{-5.6pt}{$#2$}{\smash{$\kern\shlength%
%     \stackengine{7.55pt}{$\mathchar"017E$}%
%       {\rule{\widthof{$#2$}}{.57pt}\kern.4pt}{O}{r}{F}{F}{L}\kern-\shlength$}}%
%       {O}{c}{F}{T}{S}}


\indexsetup{othercode=\small}
\makeindex[columns=2,options={-s /media/wu/file/stuuudy/notes/index_style.ist},intoc]
\makeatletter
\def\@idxitem{\par\hangindent 0pt}
\makeatother


%\newcounter{dummy} \numberwithin{dummy}{section}
\newtheorem{dummy}{dummy}[section]
\theoremstyle{definition}
\newtheorem{definition}[dummy]{Definition}
\theoremstyle{plain}
\newtheorem{corollary}[dummy]{Corollary}
\newtheorem{lemma}[dummy]{Lemma}
\newtheorem{proposition}[dummy]{Proposition}
\newtheorem{theorem}[dummy]{Theorem}
\theoremstyle{definition}
\newtheorem{examplle}{Example}[section]
\theoremstyle{remark}
\newtheorem*{remark}{Remark}
\newtheorem{exercise}{Exercise}[subsection]
\newtheorem{observation}{Observation}[section]


\newenvironment{claim}[1]{\par\noindent\textbf{Claim:}\space#1}{}

\makeatletter
\DeclareFontFamily{U}{tipa}{}
\DeclareFontShape{U}{tipa}{m}{n}{<->tipa10}{}
\newcommand{\arc@char}{{\usefont{U}{tipa}{m}{n}\symbol{62}}}%

\newcommand{\arc}[1]{\mathpalette\arc@arc{#1}}

\newcommand{\arc@arc}[2]{%
  \sbox0{$\m@th#1#2$}%
  \vbox{
    \hbox{\resizebox{\wd0}{\height}{\arc@char}}
    \nointerlineskip
    \box0
  }%
}
\makeatother

\setcounter{MaxMatrixCols}{20}
%%%%%%% ABS
\DeclarePairedDelimiter\abss{\lvert}{\rvert}%
\DeclarePairedDelimiter\normm{\lVert}{\rVert}%

% Swap the definition of \abs* and \norm*, so that \abs
% and \norm resizes the size of the brackets, and the
% starred version does not.
\makeatletter
\let\oldabs\abss
%\def\abs{\@ifstar{\oldabs}{\oldabs*}}
\newcommand{\abs}{\@ifstar{\oldabs}{\oldabs*}}
\newcommand{\norm}[1]{\left\lVert#1\right\rVert}
%\let\oldnorm\normm
%\def\norm{\@ifstar{\oldnorm}{\oldnorm*}}
%\renewcommand{norm}{\@ifstar{\oldnorm}{\oldnorm*}}
\makeatother

% \newcommand\what[1]{\ThisStyle{%
%     \setbox0=\hbox{$\SavedStyle#1$}%
%     \stackengine{-1.0\ht0+.5pt}{$\SavedStyle#1$}{%
%       \stretchto{\scaleto{\SavedStyle\mkern.15mu\char'136}{2.6\wd0}}{1.4\ht0}%
%     }{O}{c}{F}{T}{S}%
%   }
% }

% \newcommand\wtilde[1]{\ThisStyle{%
%     \setbox0=\hbox{$\SavedStyle#1$}%
%     \stackengine{-.1\LMpt}{$\SavedStyle#1$}{%
%       \stretchto{\scaleto{\SavedStyle\mkern.2mu\AC}{.5150\wd0}}{.6\ht0}%
%     }{O}{c}{F}{T}{S}%
%   }
% }

% \newcommand\wbar[1]{\ThisStyle{%
%     \setbox0=\hbox{$\SavedStyle#1$}%
%     \stackengine{.5pt+\LMpt}{$\SavedStyle#1$}{%
%       \rule{\wd0}{\dimexpr.3\LMpt+.3pt}%
%     }{O}{c}{F}{T}{S}%
%   }
% }

\newcommand{\bl}[1] {\boldsymbol{#1}}
\newcommand{\Wt}[1] {\stackrel{\sim}{\smash{#1}\rule{0pt}{1.1ex}}}
\newcommand{\wt}[1] {\widetilde{#1}}
\newcommand{\tf}[1] {\textbf{#1}}


%For boxed texts in align, use Aboxed{}
%otherwise use boxed{}

\DeclareMathSymbol{\widehatsym}{\mathord}{largesymbols}{"62}
\newcommand\lowerwidehatsym{%
  \text{\smash{\raisebox{-1.3ex}{%
    $\widehatsym$}}}}
\newcommand\fixwidehat[1]{%
  \mathchoice
    {\accentset{\displaystyle\lowerwidehatsym}{#1}}
    {\accentset{\textstyle\lowerwidehatsym}{#1}}
    {\accentset{\scriptstyle\lowerwidehatsym}{#1}}
    {\accentset{\scriptscriptstyle\lowerwidehatsym}{#1}}
  }


\newcommand{\cupdot}{\mathbin{\dot{\cup}}}
\newcommand{\bigcupdot}{\mathop{\dot{\bigcup}}}

\usepackage{graphicx}

\usepackage[toc,page]{appendix}

% text on arrow for xRightarrow
\makeatletter
%\newcommand{\xRightarrow}[2][]{\ext@arrow 0359\Rightarrowfill@{#1}{#2}}
\makeatother

% Arbitrary long arrow
\newcommand{\Rarrow}[1]{%
\parbox{#1}{\tikz{\draw[->](0,0)--(#1,0);}}
}

\newcommand{\LRarrow}[1]{%
\parbox{#1}{\tikz{\draw[<->](0,0)--(#1,0);}}
}


\makeatletter
\providecommand*{\rmodels}{%
  \mathrel{%
    \mathpalette\@rmodels\models
  }%
}
\newcommand*{\@rmodels}[2]{%
  \reflectbox{$\m@th#1#2$}%
}
\makeatother







\newcommand{\trcl}[1]{%
  \mathrm{trcl}{(#1)}
}



% Roman numerals
\makeatletter
\newcommand*{\rom}[1]{\expandafter\@slowromancap\romannumeral #1@}
\makeatother
% \\def \\b\([a-zA-Z]\) {\\boldsymbol{[a-zA-z]}}
% \\DeclareMathOperator{\\b\1}{\\textbf{\1}}


\DeclareMathOperator{\bx}{\textbf{x}}
\DeclareMathOperator{\bz}{\textbf{z}}
\DeclareMathOperator{\bff}{\textbf{f}}
\DeclareMathOperator{\ba}{\textbf{a}}
\DeclareMathOperator{\bk}{\textbf{k}}
\DeclareMathOperator{\bs}{\textbf{s}}
\DeclareMathOperator{\bh}{\textbf{h}}
\DeclareMathOperator{\bc}{\textbf{c}}
\DeclareMathOperator{\br}{\textbf{r}}
\DeclareMathOperator{\bi}{\textbf{i}}
\DeclareMathOperator{\bj}{\textbf{j}}
\DeclareMathOperator{\bn}{\textbf{n}}
\DeclareMathOperator{\be}{\textbf{e}}
\DeclareMathOperator{\bo}{\textbf{o}}
\DeclareMathOperator{\bU}{\textbf{U}}
\DeclareMathOperator{\bL}{\textbf{L}}
\DeclareMathOperator{\bV}{\textbf{V}}
\def \bzero {\mathbf{0}}
\def \btwo {\mathbf{2}}
\DeclareMathOperator{\bv}{\textbf{v}}
\DeclareMathOperator{\bp}{\textbf{p}}
\DeclareMathOperator{\bI}{\textbf{I}}
\DeclareMathOperator{\bM}{\textbf{M}}
\DeclareMathOperator{\bN}{\textbf{N}}
\DeclareMathOperator{\bK}{\textbf{K}}
\DeclareMathOperator{\bt}{\textbf{t}}
\DeclareMathOperator{\bb}{\textbf{b}}
\DeclareMathOperator{\bA}{\textbf{A}}
\DeclareMathOperator{\bX}{\textbf{X}}
\DeclareMathOperator{\bu}{\textbf{u}}
\DeclareMathOperator{\bS}{\textbf{S}}
\DeclareMathOperator{\bZ}{\textbf{Z}}
\DeclareMathOperator{\by}{\textbf{y}}
\DeclareMathOperator{\bw}{\textbf{w}}
\DeclareMathOperator{\bT}{\textbf{T}}
\DeclareMathOperator{\bF}{\textbf{F}}
\DeclareMathOperator{\bmm}{\textbf{m}}
\DeclareMathOperator{\bW}{\textbf{W}}
\DeclareMathOperator{\bR}{\textbf{R}}
\DeclareMathOperator{\bC}{\textbf{C}}
\DeclareMathOperator{\bD}{\textbf{D}}
\DeclareMathOperator{\bE}{\textbf{E}}
\DeclareMathOperator{\bQ}{\textbf{Q}}
\DeclareMathOperator{\bP}{\textbf{P}}
\DeclareMathOperator{\bY}{\textbf{Y}}
\DeclareMathOperator{\bH}{\textbf{H}}
\DeclareMathOperator{\bB}{\textbf{B}}
\DeclareMathOperator{\bG}{\textbf{G}}
\def \blambda {\symbf{\lambda}}
\def \boldeta {\symbf{\eta}}
\def \balpha {\symbf{\alpha}}
\def \bbeta {\symbf{\beta}}
\def \bgamma {\symbf{\gamma}}
\def \bxi {\symbf{\xi}}
\def \bLambda {\symbf{\Lambda}}

\newcommand{\bto}{{\boldsymbol{\to}}}
\newcommand{\Ra}{\Rightarrow}
\newcommand\und[1]{\underline{#1}}
\def \bPhi {\boldsymbol{\Phi}}
\def \btheta {\boldsymbol{\theta}}
\def \bTheta {\boldsymbol{\Theta}}
\def \bmu {\boldsymbol{\mu}}
\def \bphi {\boldsymbol{\phi}}
\def \bSigma {\boldsymbol{\Sigma}}
\def \lb {\left\{}
\def \rb {\right\}}
\def \la {\langle}
\def \ra {\rangle}
\def \caln {\mathcal{N}}
\def \dissum {\displaystyle\Sigma}
\def \dispro {\displaystyle\prod}
\def \E {\mathbb{E}}
\def \Q {\mathbb{Q}}
\def \N {\mathbb{N}}
\def \V {\mathbb{V}}
\def \R {\mathbb{R}}
\def \P {\mathbb{P}}
\def \A {\mathbb{A}}
\def \F {\mathbb{F}}
\def \Z {\mathbb{Z}}
\def \I {\mathbb{I}}
\def \C {\mathbb{C}}
\def \cala {\mathcal{A}}
\def \cale {\mathcal{E}}
\def \calb {\mathcal{B}}
\def \calq {\mathcal{Q}}
\def \calp {\mathcal{P}}
\def \cals {\mathcal{S}}
\def \calx {\mathcal{X}}
\def \caly {\mathcal{Y}}
\def \calg {\mathcal{G}}
\def \cald {\mathcal{D}}
\def \caln {\mathcal{N}}
\def \calr {\mathcal{R}}
\def \calt {\mathcal{T}}
\def \calm {\mathcal{M}}
\def \calw {\mathcal{W}}
\def \calc {\mathcal{C}}
\def \calv {\mathcal{V}}
\def \calf {\mathcal{F}}
\def \calk {\mathcal{K}}
\def \call {\mathcal{L}}
\def \calu {\mathcal{U}}
\def \calo {\mathcal{O}}
\def \calh {\mathcal{H}}
\def \cali {\mathcal{I}}

\def \bcup {\bigcup}

% set theory

\def \zfcc {\textbf{ZFC}^-}
\def \ac  {\textbf{AC}}
\def \gl  {\textbf{L }}
\def \gll {\textbf{L}}
\newcommand{\zfm}{$\textbf{ZF}^-$}

%\def \zfm {$\textbf{ZF}^-$}
\def \zfmm {\textbf{ZF}^-}
\def \wf {\textbf{WF }}
\def \on {\textbf{On }}
\def \cm {\textbf{M }}
\def \cn {\textbf{N }}
\def \cv {\textbf{V }}
\def \zc {\textbf{ZC }}
\def \zcm {\textbf{ZC}}
\def \zff {\textbf{ZF}}
\def \wfm {\textbf{WF}}
\def \onm {\textbf{On}}
\def \cmm {\textbf{M}}
\def \cnm {\textbf{N}}
\def \cvm {\textbf{V}}
\def \gchh {\textbf{GCH}}
\renewcommand{\restriction}{\mathord{\upharpoonright}}
\def \pred {\text{pred}}

\def \rank {\text{rank}}
\def \con {\text{Con}}
\def \deff {\text{Def}}


\def \uin {\underline{\in}}
\def \oin {\overline{\in}}
\def \uR {\underline{R}}
\def \oR {\overline{R}}
\def \uP {\underline{P}}
\def \oP {\overline{P}}

\def \Ra {\Rightarrow}

\def \e {\enspace}

\def \sgn {\operatorname{sgn}}
\def \gen {\operatorname{gen}}
\def \Hom {\operatorname{Hom}}
\def \hom {\operatorname{hom}}
\def \Sub {\operatorname{Sub}}

\def \supp {\operatorname{supp}}

\def \epiarrow {\twoheadarrow}
\def \monoarrow {\rightarrowtail}
\def \rrarrow {\rightrightarrows}

% \def \minus {\text{-}}
% \newcommand{\minus}{\scalebox{0.75}[1.0]{$-$}}
% \DeclareUnicodeCharacter{002D}{\minus}


\def \tril {\triangleleft}

\def \ACF {\text{ACF}}
\def \GL {\text{GL}}
\def \PGL {\text{PGL}}
\def \equal {=}
\def \deg {\text{deg}}
\def \degree {\text{degree}}
\def \app {\text{App}}
\def \FV {\text{FV}}
\def \conv {\text{conv}}
\def \cont {\text{cont}}
\DeclareMathOperator{\cl}{\textbf{CL}}
\DeclareMathOperator{\sg}{sg}
\DeclareMathOperator{\trdeg}{trdeg}
\def \Ord {\text{Ord}}

\DeclareMathOperator{\cf}{cf}
\DeclareMathOperator{\zfc}{ZFC}

%\DeclareMathOperator{\Th}{Th}
%\def \th {\text{Th}}
% \newcommand{\th}{\text{Th}}
\DeclareMathOperator{\type}{type}
\DeclareMathOperator{\zf}{\textbf{ZF}}
\def \fa {\mathfrak{a}}
\def \fb {\mathfrak{b}}
\def \fc {\mathfrak{c}}
\def \fd {\mathfrak{d}}
\def \fe {\mathfrak{e}}
\def \ff {\mathfrak{f}}
\def \fg {\mathfrak{g}}
\def \fh {\mathfrak{h}}
%\def \fi {\mathfrak{i}}
\def \fj {\mathfrak{j}}
\def \fk {\mathfrak{k}}
\def \fl {\mathfrak{l}}
\def \fm {\mathfrak{m}}
\def \fn {\mathfrak{n}}
\def \fo {\mathfrak{o}}
\def \fp {\mathfrak{p}}
\def \fq {\mathfrak{q}}
\def \fr {\mathfrak{r}}
\def \fs {\mathfrak{s}}
\def \ft {\mathfrak{t}}
\def \fu {\mathfrak{u}}
\def \fv {\mathfrak{v}}
\def \fw {\mathfrak{w}}
\def \fx {\mathfrak{x}}
\def \fy {\mathfrak{y}}
\def \fz {\mathfrak{z}}
\def \fA {\mathfrak{A}}
\def \fB {\mathfrak{B}}
\def \fC {\mathfrak{C}}
\def \fD {\mathfrak{D}}
\def \fE {\mathfrak{E}}
\def \fF {\mathfrak{F}}
\def \fG {\mathfrak{G}}
\def \fH {\mathfrak{H}}
\def \fI {\mathfrak{I}}
\def \fJ {\mathfrak{J}}
\def \fK {\mathfrak{K}}
\def \fL {\mathfrak{L}}
\def \fM {\mathfrak{M}}
\def \fN {\mathfrak{N}}
\def \fO {\mathfrak{O}}
\def \fP {\mathfrak{P}}
\def \fQ {\mathfrak{Q}}
\def \fR {\mathfrak{R}}
\def \fS {\mathfrak{S}}
\def \fT {\mathfrak{T}}
\def \fU {\mathfrak{U}}
\def \fV {\mathfrak{V}}
\def \fW {\mathfrak{W}}
\def \fX {\mathfrak{X}}
\def \fY {\mathfrak{Y}}
\def \fZ {\mathfrak{Z}}

\def \sfA {\textsf{A}}
\def \sfB {\textsf{B}}
\def \sfC {\textsf{C}}
\def \sfD {\textsf{D}}
\def \sfE {\textsf{E}}
\def \sfF {\textsf{F}}
\def \sfG {\textsf{G}}
\def \sfH {\textsf{H}}
\def \sfI {\textsf{I}}
\def \sfj {\textsf{J}}
\def \sfK {\textsf{K}}
\def \sfL {\textsf{L}}
\def \sfM {\textsf{M}}
\def \sfN {\textsf{N}}
\def \sfO {\textsf{O}}
\def \sfP {\textsf{P}}
\def \sfQ {\textsf{Q}}
\def \sfR {\textsf{R}}
\def \sfS {\textsf{S}}
\def \sfT {\textsf{T}}
\def \sfU {\textsf{U}}
\def \sfV {\textsf{V}}
\def \sfW {\textsf{W}}
\def \sfX {\textsf{X}}
\def \sfY {\textsf{Y}}
\def \sfZ {\textsf{Z}}
\def \sfa {\textsf{a}}
\def \sfb {\textsf{b}}
\def \sfc {\textsf{c}}
\def \sfd {\textsf{d}}
\def \sfe {\textsf{e}}
\def \sff {\textsf{f}}
\def \sfg {\textsf{g}}
\def \sfh {\textsf{h}}
\def \sfi {\textsf{i}}
\def \sfj {\textsf{j}}
\def \sfk {\textsf{k}}
\def \sfl {\textsf{l}}
\def \sfm {\textsf{m}}
\def \sfn {\textsf{n}}
\def \sfo {\textsf{o}}
\def \sfp {\textsf{p}}
\def \sfq {\textsf{q}}
\def \sfr {\textsf{r}}
\def \sfs {\textsf{s}}
\def \sft {\textsf{t}}
\def \sfu {\textsf{u}}
\def \sfv {\textsf{v}}
\def \sfw {\textsf{w}}
\def \sfx {\textsf{x}}
\def \sfy {\textsf{y}}
\def \sfz {\textsf{z}}



%\DeclareMathOperator{\ker}{ker}
\DeclareMathOperator{\im}{im}

\DeclareMathOperator{\inn}{Inn}
\DeclareMathOperator{\AC}{\textbf{AC}}
\DeclareMathOperator{\cod}{cod}
\DeclareMathOperator{\dom}{dom}
\DeclareMathOperator{\ran}{ran}
\DeclareMathOperator{\textd}{d}
\DeclareMathOperator{\td}{d}
\DeclareMathOperator{\id}{id}
\DeclareMathOperator{\LT}{LT}
\DeclareMathOperator{\Mat}{Mat}
\DeclareMathOperator{\Eq}{Eq}
\DeclareMathOperator{\irr}{irr}
\DeclareMathOperator{\Fr}{Fr}
\DeclareMathOperator{\Gal}{Gal}
\DeclareMathOperator{\lcm}{lcm}
\DeclareMathOperator{\alg}{\text{alg}}
\DeclareMathOperator{\Th}{Th}

\DeclareMathOperator{\DAG}{DAG}
\DeclareMathOperator{\ODAG}{ODAG}

% \varprod
\DeclareSymbolFont{largesymbolsA}{U}{txexa}{m}{n}
\DeclareMathSymbol{\varprod}{\mathop}{largesymbolsA}{16}
% \DeclareMathSymbol{\tonm}{\boldsymbol{\to}\textbf{Nm}}
\def \tonm {\bto\textbf{Nm}}
\def \tohm {\bto\textbf{Hm}}

% Category theory
\DeclareMathOperator{\Ab}{\textbf{Ab}}
\DeclareMathOperator{\Alg}{\textbf{Alg}}
\DeclareMathOperator{\Rng}{\textbf{Rng}}
\DeclareMathOperator{\Sets}{\textbf{Sets}}
\DeclareMathOperator{\Met}{\textbf{Met}}
\DeclareMathOperator{\Aut}{\textbf{Aut}}
\DeclareMathOperator{\RMod}{R-\textbf{Mod}}
\DeclareMathOperator{\RAlg}{R-\textbf{Alg}}
\DeclareMathOperator{\LF}{LF}
\DeclareMathOperator{\op}{op}
% Model theory
\DeclareMathOperator{\tp}{tp}
\DeclareMathOperator{\Diag}{Diag}
\DeclareMathOperator{\el}{el}
\DeclareMathOperator{\depth}{depth}
\DeclareMathOperator{\FO}{FO}
\DeclareMathOperator{\fin}{fin}
\DeclareMathOperator{\qr}{qr}
\DeclareMathOperator{\Mod}{Mod}
\DeclareMathOperator{\TC}{TC}
\DeclareMathOperator{\KH}{KH}
\DeclareMathOperator{\Part}{Part}
\DeclareMathOperator{\Infset}{\textsf{Infset}}
\DeclareMathOperator{\DLO}{\textsf{DLO}}
\DeclareMathOperator{\sfMod}{\textsf{Mod}}
\DeclareMathOperator{\AbG}{\textsf{AbG}}
\DeclareMathOperator{\sfACF}{\textsf{ACF}}
% Computability Theorem
\DeclareMathOperator{\Tot}{Tot}
\DeclareMathOperator{\graph}{graph}
\DeclareMathOperator{\Fin}{Fin}
\DeclareMathOperator{\Cof}{Cof}
\DeclareMathOperator{\lh}{lh}
% Commutative Algebra
\DeclareMathOperator{\ord}{ord}
\DeclareMathOperator{\Idem}{Idem}
\DeclareMathOperator{\zdiv}{z.div}
\DeclareMathOperator{\Frac}{Frac}
\DeclareMathOperator{\rad}{rad}
\DeclareMathOperator{\nil}{nil}
\DeclareMathOperator{\Ann}{Ann}
\DeclareMathOperator{\End}{End}
\DeclareMathOperator{\coim}{coim}
\DeclareMathOperator{\coker}{coker}
\DeclareMathOperator{\Bil}{Bil}
\DeclareMathOperator{\Tril}{Tril}
% Topology
\newcommand{\interior}[1]{%
  {\kern0pt#1}^{\mathrm{o}}%
}

% \makeatletter
% \newcommand{\vect}[1]{%
%   \vbox{\m@th \ialign {##\crcr
%   \vectfill\crcr\noalign{\kern-\p@ \nointerlineskip}
%   $\hfil\displaystyle{#1}\hfil$\crcr}}}
% \def\vectfill{%
%   $\m@th\smash-\mkern-7mu%
%   \cleaders\hbox{$\mkern-2mu\smash-\mkern-2mu$}\hfill
%   \mkern-7mu\raisebox{-3.81pt}[\p@][\p@]{$\mathord\mathchar"017E$}$}

% \newcommand{\amsvect}{%
%   \mathpalette {\overarrow@\vectfill@}}
% \def\vectfill@{\arrowfill@\relbar\relbar{\raisebox{-3.81pt}[\p@][\p@]{$\mathord\mathchar"017E$}}}

% \newcommand{\amsvectb}{%
% \newcommand{\vect}{%
%   \mathpalette {\overarrow@\vectfillb@}}
% \newcommand{\vecbar}{%
%   \scalebox{0.8}{$\relbar$}}
% \def\vectfillb@{\arrowfill@\vecbar\vecbar{\raisebox{-4.35pt}[\p@][\p@]{$\mathord\mathchar"017E$}}}
% \makeatother
% \bigtimes

\DeclareFontFamily{U}{mathx}{\hyphenchar\font45}
\DeclareFontShape{U}{mathx}{m}{n}{
      <5> <6> <7> <8> <9> <10>
      <10.95> <12> <14.4> <17.28> <20.74> <24.88>
      mathx10
      }{}
\DeclareSymbolFont{mathx}{U}{mathx}{m}{n}
\DeclareMathSymbol{\bigtimes}{1}{mathx}{"91}
% \odiv
\DeclareFontFamily{U}{matha}{\hyphenchar\font45}
\DeclareFontShape{U}{matha}{m}{n}{
      <5> <6> <7> <8> <9> <10> gen * matha
      <10.95> matha10 <12> <14.4> <17.28> <20.74> <24.88> matha12
      }{}
\DeclareSymbolFont{matha}{U}{matha}{m}{n}
\DeclareMathSymbol{\odiv}         {2}{matha}{"63}


\newcommand\subsetsim{\mathrel{%
  \ooalign{\raise0.2ex\hbox{\scalebox{0.9}{$\subset$}}\cr\hidewidth\raise-0.85ex\hbox{\scalebox{0.9}{$\sim$}}\hidewidth\cr}}}
\newcommand\simsubset{\mathrel{%
  \ooalign{\raise-0.2ex\hbox{\scalebox{0.9}{$\subset$}}\cr\hidewidth\raise0.75ex\hbox{\scalebox{0.9}{$\sim$}}\hidewidth\cr}}}

\newcommand\simsubsetsim{\mathrel{%
  \ooalign{\raise0ex\hbox{\scalebox{0.8}{$\subset$}}\cr\hidewidth\raise1ex\hbox{\scalebox{0.75}{$\sim$}}\hidewidth\cr\raise-0.95ex\hbox{\scalebox{0.8}{$\sim$}}\cr\hidewidth}}}
\newcommand{\stcomp}[1]{{#1}^{\mathsf{c}}}

\setlength{\baselineskip}{0.8in}

\stackMath
\newcommand\yrightarrow[2][]{\mathrel{%
  \setbox2=\hbox{\stackon{\scriptstyle#1}{\scriptstyle#2}}%
  \stackunder[0pt]{%
    \xrightarrow{\makebox[\dimexpr\wd2\relax]{$\scriptstyle#2$}}%
  }{%
   \scriptstyle#1\,%
  }%
}}
\newcommand\yleftarrow[2][]{\mathrel{%
  \setbox2=\hbox{\stackon{\scriptstyle#1}{\scriptstyle#2}}%
  \stackunder[0pt]{%
    \xleftarrow{\makebox[\dimexpr\wd2\relax]{$\scriptstyle#2$}}%
  }{%
   \scriptstyle#1\,%
  }%
}}
\newcommand\yRightarrow[2][]{\mathrel{%
  \setbox2=\hbox{\stackon{\scriptstyle#1}{\scriptstyle#2}}%
  \stackunder[0pt]{%
    \xRightarrow{\makebox[\dimexpr\wd2\relax]{$\scriptstyle#2$}}%
  }{%
   \scriptstyle#1\,%
  }%
}}
\newcommand\yLeftarrow[2][]{\mathrel{%
  \setbox2=\hbox{\stackon{\scriptstyle#1}{\scriptstyle#2}}%
  \stackunder[0pt]{%
    \xLeftarrow{\makebox[\dimexpr\wd2\relax]{$\scriptstyle#2$}}%
  }{%
   \scriptstyle#1\,%
  }%
}}

\newcommand\altxrightarrow[2][0pt]{\mathrel{\ensurestackMath{\stackengine%
  {\dimexpr#1-7.5pt}{\xrightarrow{\phantom{#2}}}{\scriptstyle\!#2\,}%
  {O}{c}{F}{F}{S}}}}
\newcommand\altxleftarrow[2][0pt]{\mathrel{\ensurestackMath{\stackengine%
  {\dimexpr#1-7.5pt}{\xleftarrow{\phantom{#2}}}{\scriptstyle\!#2\,}%
  {O}{c}{F}{F}{S}}}}

\newenvironment{bsm}{% % short for 'bracketed small matrix'
  \left[ \begin{smallmatrix} }{%
  \end{smallmatrix} \right]}

\newenvironment{psm}{% % short for ' small matrix'
  \left( \begin{smallmatrix} }{%
  \end{smallmatrix} \right)}

\newcommand{\bbar}[1]{\mkern 1.5mu\overline{\mkern-1.5mu#1\mkern-1.5mu}\mkern 1.5mu}

\newcommand{\bigzero}{\mbox{\normalfont\Large\bfseries 0}}
\newcommand{\rvline}{\hspace*{-\arraycolsep}\vline\hspace*{-\arraycolsep}}

\font\zallman=Zallman at 40pt
\font\elzevier=Elzevier at 40pt

\newcommand\isoto{\stackrel{\textstyle\sim}{\smash{\longrightarrow}\rule{0pt}{0.4ex}}}
\newcommand\embto{\stackrel{\textstyle\prec}{\smash{\longrightarrow}\rule{0pt}{0.4ex}}}
\usepackage[UTF8]{ctex}
\author{wu}
\date{\today}
\title{高等代数简明教程}
\hypersetup{
 pdfauthor={wu},
 pdftitle={高等代数简明教程},
 pdfkeywords={},
 pdfsubject={},
 pdfcreator={Emacs 26.3 (Org mode 9.4)}, 
 pdflang={English}}
\begin{document}

\maketitle
\tableofcontents \clearpage\section{代数学的经典课题}
\label{sec:org20f8d76}
\subsection{线性方程组}
\label{sec:org28f2c23}
\begin{equation*}
\begin{cases}
a_{11}x_1+\dots+a_{1n}x_n=b_1,\\
a_{21}x_1+\dots+a_{2n}x_n=b_2,\\
\dots\\
a_{m1}x_1+\dots+a_{mn}x_n=b_m
\end{cases}
\end{equation*}
设在线性方程组中,未知量的系数\(a_{ij}\)和常数项\(b_1,\dots,b_m\)都属于数域
\(K\),则称它是 \textbf{数域\(K\)上的线性方程组}

\begin{definition}[]
\textbf{初等变换}
\begin{enumerate}
\item 互换两个方程的位置
\item 把某一个方程两边同乘数域\(K\)内的一个非零常数\(c\)
\item 把第\(j\)个方程加上第\(i\)个方程的\(k\)倍,这里\(k\in K\)且\(i\neq  j\)
\end{enumerate}
\end{definition}

\begin{proposition}[]
设方程组经过某一初等变换后变为另一个方程组,则新方程组与原方程组同解
\end{proposition}

\begin{definition}[]
给定数域\(K\)上\(mn\)个数\(a_{ij}(i=1,2,\dots,m;j=1,\dots,n)\),把它们按一定
次序排成一个\(m\)行\(n\)列的长方形表格
\begin{equation*}
A=
\begin{bmatrix}
a_{11}&a_{12}&\dots&a_{1n}\\
a_{21}&a_{22}&\dots&a_{2n}\\
\vdots&\vdots&&\vdots\\
a_{m1}&a_{m2}&\dots&a_{mn}
\end{bmatrix}
\end{equation*}
称为数域\(K\)上的一个 \textbf{\(m\)行\(n\)列矩阵} ,简称为 \(m\times n\) \textbf{矩阵}
\end{definition}

方程组中未知量的系数就可以排成一个矩阵,称\(A\)为方程组的 \textbf{系数矩阵} ,如果添加
常数项,则有
\begin{equation*}
\bbar{A}=
\begin{bmatrix}
a_{11}&a_{12}&\dots&a_{1n}&b_1\\
a_{21}&a_{22}&\dots&a_{2n}&b_2\\
\vdots&\vdots&&\vdots\\
a_{m1}&a_{m2}&\dots&a_{mn}&b_m
\end{bmatrix}
\end{equation*}
矩阵\(\bbar{A}\)称为方程组的 \textbf{增广矩阵}

在数域 \(K\)上的线性方程组,如果常数项 \(b_1=b_2=\dots=b_m=0\) ,则称为数域
\(K\)上的一个 \textbf{齐次线性方程组} ,这类方程的一般形式是
\begin{equation*}
\begin{cases}
a_{11}x_1+\dots+a_{1n}x_n=0\\
a_{21}x_1+\dots+a_{2n}x_n=0\\
\dots\\
a_{m1}x_1+\dots+a_{mn}x_n=0
\end{cases}
\end{equation*}
方程组显然有一组解
\begin{equation*}
x_1=0,\dots,x_n
\end{equation*}
这组解称为 \textbf{零解} 或 \textbf{平凡解} ,除此之外的解称为 \textbf{非零解} 或 \textbf{非平凡解}

\begin{proposition}[]
数域\(K\)上的齐次线性方程组中,如果方程个数\(m\)小于未知量个数\(n\),则它必有
非零解
\end{proposition}

\begin{proof}
对方程个数\(m\)作归纳

当\(m=1\),若\(a_{11}=0\),则令\(x_1=1,x_2=\dots=x_n=0\)即为一组非零解;否则,
因\(n\iffalse<\fi>m=1\),取\(x_1=-a_{12},x_2=a_{11},x_3=\dots=x_n\)。现设有\(m-1\)个方程
的齐次线性方程组

若方程组中\(x_1\)的系数全为零,则取\(x_1=1,x_2=\dots=x_n=0\)。否则调换方程的
次序,总可使第一个方程\(x_1\)的系数不为 0,因而不妨设\(a_{11}\neq0\),方程组可
化为
\begin{equation*}
\begin{cases}
a_{11}x_1+a_{12}x_2+\dots+a_{1n}x_n=0\\
\hspace{1.1cm}b_{22}x_2+\dots+b_{2n}x_n=0\\
\hspace{1.1cm}\dots\\
\hspace{1.1cm}b_{m2}x_2+\dots+b_{mn}x_n=0
\end{cases}
\end{equation*}
上述方程后面\(m-1\)个方程是有\(n-1\)个未知量\(x_2,\dots,x_n\)和\(m-1\)个方程
的齐次线性方程组,故有非零解
\end{proof}
\section{向量空间和矩阵}
\label{sec:org9407bc6}
\subsection{\(m\)维向量空间}
\label{sec:org7e86828}
\begin{definition}[]
设\(K\)是一个数域,\(K\)中\(m\)个数\(a_1,\dots,a_m\)所组成的一个\(m\)元有序数
组
\begin{equation*}
\alpha=
\begin{bmatrix}
a_1\\a_2\\\vdots\\a_m
\end{bmatrix}(a_i\in K, i=1,2,\dots,m)
\end{equation*}
称为一个 \textbf{\(m\)维向量} , \(a_i\) 称为它的第 \(i\)个 \textbf{分量} 或 \textbf{坐标} 。 \(K\)上的
全体\(m\)维向量所组成的集合记为 \(K^m\),在\(K^m\)内定义两个向量的 \textbf{加法} 如下
\begin{equation*}
\begin{bmatrix}
a_1\\a_2\\\vdots\\a_m
\end{bmatrix}+
\begin{bmatrix}
b_1\\b_2\\\vdots\\b_m
\end{bmatrix}=
\begin{bmatrix}
a_1+b_1\\a_2+b_2\\\vdots\\a_m+b_m
\end{bmatrix}\in K^m
\end{equation*}
又设 \(k\in K\),定义 \(k\) 与\(K^m\)中向量的 \textbf{数乘} 如下
\begin{equation*}
k
\begin{bmatrix}
a_1\\a_2\\\vdots\\a_m
\end{bmatrix}=
\begin{bmatrix}
ka_1\\ka_2\\\vdots\\ka_m
\end{bmatrix}\in K^m
\end{equation*}
集合 \(K^m\) 和上面定义的加法,数乘运算这一系统称为数域 \(K\)上的 \textbf{\(m\)维向
量空间}
\end{definition}

\begin{proposition}[]
\(K^m\)中向量加法、数乘满足如下八条运算性质
\begin{enumerate}
\item 加法结合律: \(\alpha+(\beta+\gamma)=(\alpha+\beta)+\gamma\)
\item 加法交换律: \(\alpha+\beta=\beta+\alpha\)
\item 称\((0,0,\dots,0)\)为\(m\)维 \textbf{零向量} ,记为 0,对任意\(m\)维向量 \(\alpha\) ,有
\(0+\alpha=\alpha=\alpha+0\)
\item 任给\(\alpha=(a_1,\dots,a_m)\),记\(-\alpha=(-a_1,\dots,-a_m)\),称其为 \(\alpha\)
的 \textbf{负向量} ,它满足 \(\alpha+(-\alpha)=(-\alpha+0)=0\)
\item 对数 1,有\(1\cdot\alpha=\alpha\)
\item 对\(K\)内任意数\(k,l\),有\((kl)\alpha=k(l\alpha)\)
\item 对\(K\)内任意数\(k,l\),有\((k+l)\alpha=k\alpha+l\alpha\)
\item 对\(K\)内任意数\(k\),有\(k(\alpha+\beta)=k\alpha+k\beta\)
\end{enumerate}
\end{proposition}

\begin{definition}[]
给定\(K^m\)内的向量组\(\alpha_1,\dots,\alpha_s\),又给定数域\(K\)内\(s\)个数
\(k_1,\dots,k_s\),称向量\(k_1\alpha_1+k_2\alpha_2+\dots+k_s\alpha_s\)为向量
组\(\alpha_1,\dots,\alpha_s\)的一个 \textbf{线性组合}
\end{definition}

\begin{definition}[]
给定 \(K^m\)内向量组\(\alpha_1,\dots,\alpha_s\),设 \(\beta\) 是 \(K^m\) 内一个向量,
如果存在数域 \(K\) 内 \(s\)个数\(k_1,\dots,k_s\)使
\begin{equation*}
\beta=k_1\alpha_1+\dots+k_s\alpha_s
\end{equation*}
则称 \(\beta\) 可被向量组 \(\alpha_1,\dots,\alpha_s\) \textbf{线性表示}
\end{definition}

给定数域\(K\)上的线性方程组
\begin{equation*}
\begin{cases}
a_{11}x_1+\dots +a_{1n}x_n=b_1\\
\dots\\
a_{m1}x_1+\dots+a_{mn}x_n=b_m
\end{cases}
\end{equation*}

考虑 \(K^m\)中的\(n+1\)个向量
\begin{equation*}
\alpha_1=
\begin{bmatrix}
a_{11}\\a_{21}\\\vdots\\a_{m1}
\end{bmatrix},\dots
\alpha_n=
\begin{bmatrix}
a_{1n}\\a_{2n}\\\vdots\\a_{mn}
\end{bmatrix},
\beta=
\begin{bmatrix}
b_{1}\\b_{2}\\\vdots\\b_{m}
\end{bmatrix}
\end{equation*}
应用\(m\)维向量 d 额加法和数乘运算,方程组可改写成
\begin{equation*}
x_1\alpha_1+\dots+x_n\alpha_n=\beta
\end{equation*}
如果方程组有一组解
\begin{equation*}
x_1=k_1,\dots,x_n=k_n(k_i\in K)
\end{equation*}
则
\begin{equation*}
\beta=k_1\alpha+\dots+k_n\alpha
\end{equation*}
即 \(\beta\) 能被向量组 \(\alpha_1,\dots,\alpha_n\) 线性表示。反之,若 \(\beta\) 能被向量组
\(\alpha_1,\dots,\alpha_n\)线性表示,则表示的系数就是方程组的一组解,于是有
\begin{enumerate}
\item 方程组有解当且仅当 \(\beta\) 能被向量组\(\alpha_1,\dots,\alpha_n\)线性表示
\item 方程组的解数等于线性表示法的组数
\end{enumerate}
\begin{definition}
给定\(K^m\)中的一个向量组
\begin{equation*}
\alpha_1=
\begin{bmatrix}
a_{11}\\a_{21}\\\vdots\\a_{m1}
\end{bmatrix}   ,\dots
\alpha_s=
\begin{bmatrix}
a_{1s}\\a_{2s}\\\vdots\\a_{ms}
\end{bmatrix}
\end{equation*}
如果齐次线性方程组
\begin{equation*}
\begin{cases}
a_{11}x_1+\dots+a_{1s}x_s=0\\
\dots\\
a_{m1}x_1+\dots+a_{ms}x_s=0
\end{cases}
\end{equation*}
有非零解,则称向量组\(\alpha_1,\dots,\alpha_s\) \textbf{线性相关} ,如果齐次线性方程组
只有零解,则称此向量组 \textbf{线性无关}
\end{definition}

\begin{proposition}[]
给定\(K^5\)内向量组
\begin{alignat*}{2}
&\alpha_1=(7,0,0,0,0)&&\alpha_2=(-1,3,4,0,0)\\
&\alpha_3=(1,0,1,1,0)\quad&&\alpha_4=(0,0,1,1,-1)
\end{alignat*}
判断它们是否线性相关
\end{proposition}

\begin{proof}
把它们竖起来排成一个\(5\times4\)矩阵
\begin{equation*}
A=
\begin{bmatrix}
7&-1&1&0\\
0&3&0&0\\
0&4&1&1\\
0&0&1&1\\
0&0&0&-1
\end{bmatrix}
\end{equation*}
用矩阵消元法把\(A\)化为阶梯形
\begin{equation*}
\begin{bmatrix}
7&-1&1&0\\
0&1&0&0\\
0&0&1&1\\
0&0&0&1\\
0&0&0&0
\end{bmatrix}
\end{equation*}
最后的阶梯形矩阵对应的齐次线性方程组显然只有零解,故以 \(A\) 为系数矩阵的齐次
线性方程组也只有零解,即\(\alpha_1,\alpha_2,\alpha_3,\alpha_4\)线性无关
\end{proof}

\begin{definition}[]
给定\(K^m\)内向量组\(\alpha_1,\dots,\alpha_s\),如果存在\(K\)内不全为零的数
\(k_1,\dots,k_s\)使
\begin{equation*}
k_1\alpha_1+\dots+k_s\alpha_s=0
\end{equation*}
则称向量组线性相关,否则称为线性无关
\end{definition}

\begin{proposition}[]
\(K^m\)内 l 向量组\(\alpha_1,\dots,\alpha_s(s\ge2)\)线性相关的 g 充分必要条件是其
中存在一个向量能被其余向量线性表示
\end{proposition}

\begin{corollary}[]
如果\(K^m\)内向量组\(\alpha_1,\dots,\alpha_s(s\ge2)\)中任一向量都不能被其余向
量线性表示,则此向量组线性无关
\end{corollary}

给定\(K^n\)中如下\(n\)个向量
\begin{align*}
&\epsilon_1=(1,0,\dots,0),\\
&\epsilon_2=(0,1,\dots,0),\\
&\dots\\
&\epsilon_n=(0,0,\dots,1)
\end{align*}
称之为数域\(K\)上\(n\)维向量空间的\(n\)个 \textbf{坐标向量}

\begin{definition}[]
给定\(K^m\)内两个向量组
\begin{align}
&   \alpha_1,\alpha_2,\dots,\alpha_r\label{eq1.1}\\
&\beta_1,\beta_2,\dots,\beta_s\label{eq1.2}\\
\end{align}
如果向量组 \eqref{eq1.2} 中每一个向量都能被向量组 \eqref{eq1.1} 线性表示,反过来
也成立,则称向量组 \eqref{eq1.1} 和向量组 \eqref{eq1.2} \textbf{线性等价}
\end{definition}

\begin{proposition}[]
给定\(K^m\)内两个向量组
\begin{align}
&\alpha_1,\alpha_2,\dots,\alpha_r\label{eq1.11}\\
&\beta_1,\beta_2,\dots,\beta_s\label{eq1.22}
\end{align}
且 \eqref{eq1.22} 中每一个向量 \(\beta_i\) 均能被向量组 \eqref{eq1.11} 线性表示,
那么当向量 \(\gamma\) 能被向量组 \eqref{eq1.22} 线性表示时,它也能被向量组 \eqref{eq1.11}
线性表示
\end{proposition}

线性等价:自反性、对称性、传递性

\begin{definition}[]
给定\(K^m\)内向量组
\begin{equation*}
\alpha_1,\dots,\alpha_s
\end{equation*}
如果它的一个部分组
\begin{equation*}
\alpha_{i_1},\dots,\alpha_{i_r}
\end{equation*}
满足
\begin{enumerate}
\item 向量组能被部分组线性表示
\item 部分组线性无关
\end{enumerate}


则称部分组是 \textbf{极大线性无关部分组}
\end{definition}

\begin{proposition}[]
给定\(K^m\)内两个向量组
\begin{align}
&\alpha_1,\dots,\alpha_r\label{eq1.1.4.1}\\
&\beta_1,\dots,\beta_s\label{eq1.1.4.2}
\end{align}
如果向量组 \eqref{eq1.1.4.1} 中每个向量都能被 \eqref{eq1.1.4.2} 线性表示,且
\(r\iffalse<\fi>s\),则向量组 \eqref{eq1.1.4.1} 线性相关
\end{proposition}

\begin{definition}[]
给定\(K^m\)内向量组
\begin{equation*}
\alpha_1,\dots,\alpha_s
\end{equation*}
设它的某一个极大线性无关部分组
\begin{equation*}
\alpha_{i_1},\dots,\alpha_{i_r}
\end{equation*}
又有另一个向量组
\begin{equation*}
\beta_1,\dots,\beta_t
\end{equation*}
设它的某一个极大线性无关部分组为
\begin{equation*}
\beta_{j_1},\dots,\beta_{j_l}
\end{equation*}
若\((\alpha)\)与\((\beta)\)线性等价,则\(r=l\)
\end{definition}

\begin{proof}
线性等价的传递性,两个部分组也等价
\end{proof}

\begin{corollary}[]
一个向量组的任意两个极大线性无关部分组中包含的向量个数相同
\end{corollary}

\begin{definition}[]
一个向量组的极大线性无关部分组中包含的向量个数称为该向量组的 \textbf{秩} ,全由零向量
组成的向量组的秩为零
\end{definition}


\begin{corollary}[]
两个线性等价的向量组的秩相等
\end{corollary}

\begin{proposition}[]
给定\(K^m\)内向量组
\begin{equation*}
\alpha_1,\dots,\alpha_n
\end{equation*}
其中\(\alpha_1\neq0\),作如下筛选:保持\(\alpha_1\)不懂,若\(\alpha_2\)可被
\(\alpha_1\)线性表示,则去掉\(\alpha_2\),否则保留,若\(\alpha_i\)可被前面保
留下来的向量线性表示,则去掉,否则保留,经\(n\)此筛选后,得到的向量组是
\begin{equation*}
\alpha_{i_1}=\alpha_1,\alpha_{i_2},\dots,\alpha_{i_r}
\end{equation*}
则\(\alpha_{i_1}\)是一个极大线性无关部分组
\end{proposition}
\subsection{矩阵的秩}
\label{sec:orgf2ae0b2}
给定数域\(K\)上一个\(m\times n\)矩阵\(A\),它的每一列可以看成一个\(m\)维向量,
它有\(n\)列,组成一个\(m\)维向量组,我们称之为矩阵\(A\)的 \textbf{列向量组} ,同样,它
的每一行可以看作一个\(n\)维向量,称为\(A\)的 \textbf{行向量组}

\begin{definition}[]
一个矩阵的行向量组的秩称为 \textbf{行秩} ,列向量组的秩称为 \textbf{列秩}
\end{definition}

设矩阵\(A\)的列向量组为\(\alpha_1,\dots,\alpha_n\),则\(A\)可写成
\begin{equation*}
A=(\alpha_1,\dots,\alpha_n)
\end{equation*}

\begin{definition}[]
对数域\(K\)上的\(m\times n\)矩阵\(A\)的行(列)作如下变换
\begin{enumerate}
\item 互换两行(列)的位置
\item 把某一行(列)乘以\(K\)内一个非零常数\(c\)
\item 把第\(j\)行(列)加上第\(i\)行(列)的\(k\)倍,这里\(k\in K\)且\(i\neq j\)
\end{enumerate}


上述三种变换的每一种都称为矩阵\(A\)的 \textbf{初等行(列)变换}
\end{definition}

\begin{proposition}[]
矩阵\(A\)的行秩在初等行变换下保持不变,列秩在初等列变化下保持不变
\end{proposition}

把矩阵\(A\)的行与列互换后得到\(A'\)称为\(A\)的转置矩阵

\begin{proposition}[]
矩阵的行秩在初等列变化下保持不变;矩阵的列秩在初等列变化下保持不变
\end{proposition}

\begin{proof}
证\(A\)的列秩在初等行变换下保持不变。设\(A\)的列向量组为
\(\alpha_1,\dots,\alpha_n\),其列秩为\(r\)。不妨设\(\alpha_1,\dots,\alpha_r\)
为列向量组的一个极大线性无关部分组,假定\(A\)经过初等行变换后所得新矩阵的列向
量组为\(\alpha_1',\dots,\alpha_n'\),只要证\(\alpha_1',\dots,\alpha_r'\)是极
大线性无关
\begin{enumerate}
\item \(\alpha_1',\dots,\alpha_r'\)线性无关。以\(\alpha_1,\dots,\alpha_r\)为列向
量排成一矩阵\(B\)。因\(\alpha_1,\dots,\alpha_r\)线性无关,故以\(B\)为系数
矩阵的齐次线性方程组只有零解;另一方面,以\(\alpha_1',\dots,\alpha_r'\)为
列向量排成矩阵\(B_1\),\(B_1\)是\(B\)经过初等行变换得到的,它们同解
\item 考察以\(\alpha_1,\dots,\alpha_r,\alpha_i\)为列向量组的矩阵\(\bbar{B}\)。因
\(\alpha_i\)可被\(\alpha_1,\dots,\alpha_r\)线性表示,故以\(\bbar{B}\)为增
广矩阵的线性方程组有解;另一方面,以
\(\alpha_1',\dots,\alpha_r',\alpha_i'\)为列向量组的矩阵\(\bbar{B_1}\)可由
\(\bbar{B}\)经初等行变换得到,说明\(\alpha_i'\)可被
\(\alpha_1',\dots,\alpha_r'\)线性表示
\end{enumerate}
\end{proof}

\begin{corollary}[]
设\(A\)是数域\(K\)上的\(m\times m\)矩阵,\(A\)经若干此初等行变换化为矩阵\(B\),
设\(A\)的列向量组是\(\alpha_1,\dots,\alpha_n\),\(B\)的列向量组是
\(\alpha_1',\dots,\alpha_n'\),我们有
\begin{enumerate}
\item 如果\(\alpha_{i_1},\dots,\alpha_{i_r}\)是\(A\)的列向量组的一个极大线性无关
部分组,则\(\alpha_{i_1}',\dots,\alpha_{i_r}'\)是\(B\)的列向量组的一个极大
线性无关部分组。而且当
\begin{equation*}
\alpha_i=k_1\alpha_{i_1}+\dots+k_r\alpha_{i_r}
\end{equation*}
时,有\(\alpha_i'=k_1\alpha_{i_1}'+\dots+k_r\alpha_{i_r}'\)
\item 如果\(\alpha_{i_1}‘,\dots,\alpha_{i_r}’\)是\(B\)的列向量组的一个极大线性无关
部分组,则\(\alpha_{i_1},\dots,\alpha_{i_r}\)是\(A\)的列向量组的一个极大
线性无关部分组。而且当
\begin{equation*}
\alpha_i'=k_1\alpha_{i_1}'+\dots+k_r\alpha_{i_r}'
\end{equation*}
时,有\(\alpha_i=k_1\alpha_{i_1}+\dots+k_r\alpha_{i_r}\)
\end{enumerate}
\end{corollary}

如果一个\(m\times n\)矩阵其所有元素都是 0,则称为 \textbf{零矩阵} ,记作 0。下面设
\(A\neq0\)
\begin{enumerate}
\item 在矩阵\(A\)中,如果\(a_{11}=0\),我们就在矩阵中找一个不为零的元素,设为
\(a_{ij}\),现对换\(1,i\)两行,再对换\(1,j\)两列
\item 若\(a_{11}\neq0\),利用初等行变换把\(A\)变成如下形状
\begin{equation*}
A\to
\begin{bmatrix}
a_{11}&a_{12}&\dots&a_{1n}\\
0&b_{22}&\dots&b_{2n}\\
\vdots&\vdots&&\vdots\\
0&b_{m2}&\dots&b_{mn}
\end{bmatrix}
\end{equation*}
再利用初等列变换把\(A\)进一步变为
\begin{equation*}
A\to
\begin{bmatrix}
1&0&\dots&0\\
0&b_{22}&\dots&b_{2n}\\
\vdots&\vdots&&\vdots\\
0&b_{m2}&\dots&b_{mn}
\end{bmatrix}
\end{equation*}
如此继续对右下角\((m-1)\times(n-1)\)矩阵重复
\end{enumerate}



如果连续施行上述初等行、列变换之后,矩阵\(A\)可化成

 \begin{tabular}{p{4.8cm} p{3.4cm} p{3cm}}
\begin{equation*}
  \begin{bmatrix}
    \begin{matrix}
      1 &&\\&\ddots&\\&&1\\
    \end{matrix}&&\bigzero\\
    \bigzero&
    \begin{matrix}
      0&&\\&\ddots&\\&&0\\
    \end{matrix}&
    \begin{matrix}&\\&\\\dots&0
    \end{matrix}
  \end{bmatrix}
\end{equation*}&
\begin{equation*}
  \begin{bmatrix}
    \begin{matrix}
      1 &&\\&\ddots&\\&&1\\
    \end{matrix}&\bigzero\\
    \bigzero&
    \begin{matrix}
      0&&\\&\ddots&\\&&0\\
    \end{matrix}
  \end{bmatrix}
\end{equation*}&
\begin{equation*}
  \begin{bmatrix}
    \begin{matrix}
      1 &&\\&\ddots&\\&&1\\
    \end{matrix}&\bigzero\\&
    \begin{matrix}
      0&&\\
      &\ddots&\\
      &&0\\
    \end{matrix}\\\bigzero&
    \begin{matrix}
      0&\ddots &\vdots\\& &0
    \end{matrix}
  \end{bmatrix}
\end{equation*}\\
  $n>m$&$n=m$&$n<m$
 \end{tabular}
\end{document}
