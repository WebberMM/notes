% Created 2020-11-04 三 15:33
% Intended LaTeX compiler: pdflatex
\documentclass[11pt]{article}
\usepackage[utf8]{inputenc}
\usepackage[T1]{fontenc}
\usepackage{graphicx}
\usepackage{grffile}
\usepackage{longtable}
\usepackage{wrapfig}
\usepackage{rotating}
\usepackage[normalem]{ulem}
\usepackage{amsmath}
\usepackage{textcomp}
\usepackage{amssymb}
\usepackage{capt-of}
\usepackage{hyperref}
%%%%%%%%%%%%%%%%%%%%%%%%%%%%%%%%%%%%%%
%% TIPS                                 %%
%%%%%%%%%%%%%%%%%%%%%%%%%%%%%%%%%%%%%%
% \substack{a\\b} for multiple lines text

\usepackage[utf8]{inputenc}

\usepackage[B1,T1]{fontenc}

% pdfplots will load xolor automatically without option
\usepackage[dvipsnames]{xcolor}
%%%%%%%%%%%%%%%%%%%%%%%%%%%%%%%%%%%%%%%
%% MATH related pacakge                  %%
%%%%%%%%%%%%%%%%%%%%%%%%%%%%%%%%%%%%%%%
% \usepackage{amsmath} mathtools loads the amsmath
\usepackage{amsmath}
\usepackage{mathtools}


\usepackage{amsthm}
\usepackage{amsbsy}

%\usepackage{commath}

\usepackage{amssymb}
\usepackage{mathrsfs}
%\usepackage{mathabx}
\usepackage{stmaryrd}
\usepackage{empheq}

\usepackage{scalerel}
\usepackage{stackengine}
\usepackage{stackrel}

\usepackage{nicematrix}
\usepackage{tensor}
\usepackage{blkarray}
\usepackage{siunitx}
\usepackage[f]{esvect}

\usepackage{unicode-math}
\setmainfont{TeX Gyre Pagella}
% \setmathfont{STIX}
% \setmathfont{texgyrepagella-math.otf}
% \setmathfont{Libertinus Math}
\setmathfont{Latin Modern Math}
\setmathfont[range={\mscra,\mscrb,\mscrc,\mscrd,\mscre,\mscrf,\mscrg,\mscrh,\mscri,\mscrj,\mscrk,\mscrl,\mscrm,\mscrn,\mscro,\mscrp,\mscrq,\mscrr,\mscrs,\mscrt,\mscru,\mscrv,\mscrw,\mscrx,\mscry,\mscrz,\mscrA,\mscrB,\mscrC,\mscrD,\mscrE,\mscrF,\mscrG,\mscrH,\mscrI,\mscrJ,\mscrK,\mscrL,\mscrM,\mscrN,\mscrO,\mscrP,\mscrQ,\mscrR,\mscrS,\mscrT,\mscrU,\mscrV,\mscrW,\mscrX,\mscrY,\mscrZ}]{Latin Modern Math}
\setmathfont[range={\smwhtdiamond,\enclosediamond,\varlrtriangle}]{Latin Modern Math}
\setmathfont[range={\rightrightarrows,\twoheadrightarrow,\leftrightsquigarrow,\triangledown}]{XITS Math}
\setmathfont[range={\int,\setminus}]{Libertinus Math}



%%%%%%%%%%%%%%%%%%%%%%%%%%%%%%%%%%%%%%%
%% TIKZ related packages                 %%
%%%%%%%%%%%%%%%%%%%%%%%%%%%%%%%%%%%%%%%

\usepackage{pgfplots}
\pgfplotsset{compat=1.15}
\usepackage{tikz}
\usepackage{tikz-cd}
\usepackage{tikz-qtree}

\usetikzlibrary{arrows,positioning,calc,fadings,decorations,matrix,decorations,shapes.misc}
%setting from geogebra
\definecolor{ccqqqq}{rgb}{0.8,0,0}


%%%%%%%%%%%%%%%%%%%%%%%%%%%%%%%%%%%%%%%
%% MISCLELLANEOUS packages               %%
%%%%%%%%%%%%%%%%%%%%%%%%%%%%%%%%%%%%%%%
\usepackage[most]{tcolorbox}
\usepackage{threeparttable}
\usepackage{tabularx}

\usepackage{enumitem}

% wrong with preview
\usepackage{subcaption}
\usepackage{caption}
% {\aunclfamily\Huge}
\usepackage{auncial}

\usepackage{float}

\usepackage{fancyhdr}

\usepackage{ifthen}
\usepackage{xargs}


\usepackage{imakeidx}
\usepackage{hyperref}
\usepackage{soul}


%\usepackage[xetex]{preview}
%%%%%%%%%%%%%%%%%%%%%%%%%%%%%%%%%%%%%%%
%% USEPACKAGES end                       %%
%%%%%%%%%%%%%%%%%%%%%%%%%%%%%%%%%%%%%%%

% \setlist{nosep}
% \numberwithin{equation}{subsection}
% \fancyhead{} % Clear the headers
% \renewcommand{\headrulewidth}{0pt} % Width of line at top of page
% \fancyhead[R]{\slshape\leftmark} % Mark right [R] of page with Chapter name [\leftmark]
% \pagestyle{fancy} % Set default style for all content pages (not TOC, etc)


% \newlength\shlength
% \newcommand\vect[2][0]{\setlength\shlength{#1pt}%
%   \stackengine{-5.6pt}{$#2$}{\smash{$\kern\shlength%
%     \stackengine{7.55pt}{$\mathchar"017E$}%
%       {\rule{\widthof{$#2$}}{.57pt}\kern.4pt}{O}{r}{F}{F}{L}\kern-\shlength$}}%
%       {O}{c}{F}{T}{S}}


\indexsetup{othercode=\small}
\makeindex[columns=2,options={-s /media/wu/file/stuuudy/notes/index_style.ist},intoc]
\makeatletter
\def\@idxitem{\par\hangindent 0pt}
\makeatother


%\newcounter{dummy} \numberwithin{dummy}{section}
\newtheorem{dummy}{dummy}[section]
\theoremstyle{definition}
\newtheorem{definition}[dummy]{Definition}
\theoremstyle{plain}
\newtheorem{corollary}[dummy]{Corollary}
\newtheorem{lemma}[dummy]{Lemma}
\newtheorem{proposition}[dummy]{Proposition}
\newtheorem{theorem}[dummy]{Theorem}
\theoremstyle{definition}
\newtheorem{examplle}{Example}[section]
\theoremstyle{remark}
\newtheorem*{remark}{Remark}
\newtheorem{exercise}{Exercise}[subsection]
\newtheorem{observation}{Observation}[section]


\newenvironment{claim}[1]{\par\noindent\textbf{Claim:}\space#1}{}

\makeatletter
\DeclareFontFamily{U}{tipa}{}
\DeclareFontShape{U}{tipa}{m}{n}{<->tipa10}{}
\newcommand{\arc@char}{{\usefont{U}{tipa}{m}{n}\symbol{62}}}%

\newcommand{\arc}[1]{\mathpalette\arc@arc{#1}}

\newcommand{\arc@arc}[2]{%
  \sbox0{$\m@th#1#2$}%
  \vbox{
    \hbox{\resizebox{\wd0}{\height}{\arc@char}}
    \nointerlineskip
    \box0
  }%
}
\makeatother

\setcounter{MaxMatrixCols}{20}
%%%%%%% ABS
\DeclarePairedDelimiter\abss{\lvert}{\rvert}%
\DeclarePairedDelimiter\normm{\lVert}{\rVert}%

% Swap the definition of \abs* and \norm*, so that \abs
% and \norm resizes the size of the brackets, and the
% starred version does not.
\makeatletter
\let\oldabs\abss
%\def\abs{\@ifstar{\oldabs}{\oldabs*}}
\newcommand{\abs}{\@ifstar{\oldabs}{\oldabs*}}
\newcommand{\norm}[1]{\left\lVert#1\right\rVert}
%\let\oldnorm\normm
%\def\norm{\@ifstar{\oldnorm}{\oldnorm*}}
%\renewcommand{norm}{\@ifstar{\oldnorm}{\oldnorm*}}
\makeatother

% \newcommand\what[1]{\ThisStyle{%
%     \setbox0=\hbox{$\SavedStyle#1$}%
%     \stackengine{-1.0\ht0+.5pt}{$\SavedStyle#1$}{%
%       \stretchto{\scaleto{\SavedStyle\mkern.15mu\char'136}{2.6\wd0}}{1.4\ht0}%
%     }{O}{c}{F}{T}{S}%
%   }
% }

% \newcommand\wtilde[1]{\ThisStyle{%
%     \setbox0=\hbox{$\SavedStyle#1$}%
%     \stackengine{-.1\LMpt}{$\SavedStyle#1$}{%
%       \stretchto{\scaleto{\SavedStyle\mkern.2mu\AC}{.5150\wd0}}{.6\ht0}%
%     }{O}{c}{F}{T}{S}%
%   }
% }

% \newcommand\wbar[1]{\ThisStyle{%
%     \setbox0=\hbox{$\SavedStyle#1$}%
%     \stackengine{.5pt+\LMpt}{$\SavedStyle#1$}{%
%       \rule{\wd0}{\dimexpr.3\LMpt+.3pt}%
%     }{O}{c}{F}{T}{S}%
%   }
% }

\newcommand{\bl}[1] {\boldsymbol{#1}}
\newcommand{\Wt}[1] {\stackrel{\sim}{\smash{#1}\rule{0pt}{1.1ex}}}
\newcommand{\wt}[1] {\widetilde{#1}}
\newcommand{\tf}[1] {\textbf{#1}}


%For boxed texts in align, use Aboxed{}
%otherwise use boxed{}

\DeclareMathSymbol{\widehatsym}{\mathord}{largesymbols}{"62}
\newcommand\lowerwidehatsym{%
  \text{\smash{\raisebox{-1.3ex}{%
    $\widehatsym$}}}}
\newcommand\fixwidehat[1]{%
  \mathchoice
    {\accentset{\displaystyle\lowerwidehatsym}{#1}}
    {\accentset{\textstyle\lowerwidehatsym}{#1}}
    {\accentset{\scriptstyle\lowerwidehatsym}{#1}}
    {\accentset{\scriptscriptstyle\lowerwidehatsym}{#1}}
  }


\newcommand{\cupdot}{\mathbin{\dot{\cup}}}
\newcommand{\bigcupdot}{\mathop{\dot{\bigcup}}}

\usepackage{graphicx}

\usepackage[toc,page]{appendix}

% text on arrow for xRightarrow
\makeatletter
%\newcommand{\xRightarrow}[2][]{\ext@arrow 0359\Rightarrowfill@{#1}{#2}}
\makeatother

% Arbitrary long arrow
\newcommand{\Rarrow}[1]{%
\parbox{#1}{\tikz{\draw[->](0,0)--(#1,0);}}
}

\newcommand{\LRarrow}[1]{%
\parbox{#1}{\tikz{\draw[<->](0,0)--(#1,0);}}
}


\makeatletter
\providecommand*{\rmodels}{%
  \mathrel{%
    \mathpalette\@rmodels\models
  }%
}
\newcommand*{\@rmodels}[2]{%
  \reflectbox{$\m@th#1#2$}%
}
\makeatother







\newcommand{\trcl}[1]{%
  \mathrm{trcl}{(#1)}
}



% Roman numerals
\makeatletter
\newcommand*{\rom}[1]{\expandafter\@slowromancap\romannumeral #1@}
\makeatother
% \\def \\b\([a-zA-Z]\) {\\boldsymbol{[a-zA-z]}}
% \\DeclareMathOperator{\\b\1}{\\textbf{\1}}


\DeclareMathOperator{\bx}{\textbf{x}}
\DeclareMathOperator{\bz}{\textbf{z}}
\DeclareMathOperator{\bff}{\textbf{f}}
\DeclareMathOperator{\ba}{\textbf{a}}
\DeclareMathOperator{\bk}{\textbf{k}}
\DeclareMathOperator{\bs}{\textbf{s}}
\DeclareMathOperator{\bh}{\textbf{h}}
\DeclareMathOperator{\bc}{\textbf{c}}
\DeclareMathOperator{\br}{\textbf{r}}
\DeclareMathOperator{\bi}{\textbf{i}}
\DeclareMathOperator{\bj}{\textbf{j}}
\DeclareMathOperator{\bn}{\textbf{n}}
\DeclareMathOperator{\be}{\textbf{e}}
\DeclareMathOperator{\bo}{\textbf{o}}
\DeclareMathOperator{\bU}{\textbf{U}}
\DeclareMathOperator{\bL}{\textbf{L}}
\DeclareMathOperator{\bV}{\textbf{V}}
\def \bzero {\mathbf{0}}
\def \btwo {\mathbf{2}}
\DeclareMathOperator{\bv}{\textbf{v}}
\DeclareMathOperator{\bp}{\textbf{p}}
\DeclareMathOperator{\bI}{\textbf{I}}
\DeclareMathOperator{\bM}{\textbf{M}}
\DeclareMathOperator{\bN}{\textbf{N}}
\DeclareMathOperator{\bK}{\textbf{K}}
\DeclareMathOperator{\bt}{\textbf{t}}
\DeclareMathOperator{\bb}{\textbf{b}}
\DeclareMathOperator{\bA}{\textbf{A}}
\DeclareMathOperator{\bX}{\textbf{X}}
\DeclareMathOperator{\bu}{\textbf{u}}
\DeclareMathOperator{\bS}{\textbf{S}}
\DeclareMathOperator{\bZ}{\textbf{Z}}
\DeclareMathOperator{\by}{\textbf{y}}
\DeclareMathOperator{\bw}{\textbf{w}}
\DeclareMathOperator{\bT}{\textbf{T}}
\DeclareMathOperator{\bF}{\textbf{F}}
\DeclareMathOperator{\bmm}{\textbf{m}}
\DeclareMathOperator{\bW}{\textbf{W}}
\DeclareMathOperator{\bR}{\textbf{R}}
\DeclareMathOperator{\bC}{\textbf{C}}
\DeclareMathOperator{\bD}{\textbf{D}}
\DeclareMathOperator{\bE}{\textbf{E}}
\DeclareMathOperator{\bQ}{\textbf{Q}}
\DeclareMathOperator{\bP}{\textbf{P}}
\DeclareMathOperator{\bY}{\textbf{Y}}
\DeclareMathOperator{\bH}{\textbf{H}}
\DeclareMathOperator{\bB}{\textbf{B}}
\DeclareMathOperator{\bG}{\textbf{G}}
\def \blambda {\symbf{\lambda}}
\def \boldeta {\symbf{\eta}}
\def \balpha {\symbf{\alpha}}
\def \bbeta {\symbf{\beta}}
\def \bgamma {\symbf{\gamma}}
\def \bxi {\symbf{\xi}}
\def \bLambda {\symbf{\Lambda}}

\newcommand{\bto}{{\boldsymbol{\to}}}
\newcommand{\Ra}{\Rightarrow}
\newcommand\und[1]{\underline{#1}}
\def \bPhi {\boldsymbol{\Phi}}
\def \btheta {\boldsymbol{\theta}}
\def \bTheta {\boldsymbol{\Theta}}
\def \bmu {\boldsymbol{\mu}}
\def \bphi {\boldsymbol{\phi}}
\def \bSigma {\boldsymbol{\Sigma}}
\def \lb {\left\{}
\def \rb {\right\}}
\def \la {\langle}
\def \ra {\rangle}
\def \caln {\mathcal{N}}
\def \dissum {\displaystyle\Sigma}
\def \dispro {\displaystyle\prod}
\def \E {\mathbb{E}}
\def \Q {\mathbb{Q}}
\def \N {\mathbb{N}}
\def \V {\mathbb{V}}
\def \R {\mathbb{R}}
\def \P {\mathbb{P}}
\def \A {\mathbb{A}}
\def \F {\mathbb{F}}
\def \Z {\mathbb{Z}}
\def \I {\mathbb{I}}
\def \C {\mathbb{C}}
\def \cala {\mathcal{A}}
\def \cale {\mathcal{E}}
\def \calb {\mathcal{B}}
\def \calq {\mathcal{Q}}
\def \calp {\mathcal{P}}
\def \cals {\mathcal{S}}
\def \calx {\mathcal{X}}
\def \caly {\mathcal{Y}}
\def \calg {\mathcal{G}}
\def \cald {\mathcal{D}}
\def \caln {\mathcal{N}}
\def \calr {\mathcal{R}}
\def \calt {\mathcal{T}}
\def \calm {\mathcal{M}}
\def \calw {\mathcal{W}}
\def \calc {\mathcal{C}}
\def \calv {\mathcal{V}}
\def \calf {\mathcal{F}}
\def \calk {\mathcal{K}}
\def \call {\mathcal{L}}
\def \calu {\mathcal{U}}
\def \calo {\mathcal{O}}
\def \calh {\mathcal{H}}
\def \cali {\mathcal{I}}

\def \bcup {\bigcup}

% set theory

\def \zfcc {\textbf{ZFC}^-}
\def \ac  {\textbf{AC}}
\def \gl  {\textbf{L }}
\def \gll {\textbf{L}}
\newcommand{\zfm}{$\textbf{ZF}^-$}

%\def \zfm {$\textbf{ZF}^-$}
\def \zfmm {\textbf{ZF}^-}
\def \wf {\textbf{WF }}
\def \on {\textbf{On }}
\def \cm {\textbf{M }}
\def \cn {\textbf{N }}
\def \cv {\textbf{V }}
\def \zc {\textbf{ZC }}
\def \zcm {\textbf{ZC}}
\def \zff {\textbf{ZF}}
\def \wfm {\textbf{WF}}
\def \onm {\textbf{On}}
\def \cmm {\textbf{M}}
\def \cnm {\textbf{N}}
\def \cvm {\textbf{V}}
\def \gchh {\textbf{GCH}}
\renewcommand{\restriction}{\mathord{\upharpoonright}}
\def \pred {\text{pred}}

\def \rank {\text{rank}}
\def \con {\text{Con}}
\def \deff {\text{Def}}


\def \uin {\underline{\in}}
\def \oin {\overline{\in}}
\def \uR {\underline{R}}
\def \oR {\overline{R}}
\def \uP {\underline{P}}
\def \oP {\overline{P}}

\def \Ra {\Rightarrow}

\def \e {\enspace}

\def \sgn {\operatorname{sgn}}
\def \gen {\operatorname{gen}}
\def \Hom {\operatorname{Hom}}
\def \hom {\operatorname{hom}}
\def \Sub {\operatorname{Sub}}

\def \supp {\operatorname{supp}}

\def \epiarrow {\twoheadarrow}
\def \monoarrow {\rightarrowtail}
\def \rrarrow {\rightrightarrows}

% \def \minus {\text{-}}
% \newcommand{\minus}{\scalebox{0.75}[1.0]{$-$}}
% \DeclareUnicodeCharacter{002D}{\minus}


\def \tril {\triangleleft}

\def \ACF {\text{ACF}}
\def \GL {\text{GL}}
\def \PGL {\text{PGL}}
\def \equal {=}
\def \deg {\text{deg}}
\def \degree {\text{degree}}
\def \app {\text{App}}
\def \FV {\text{FV}}
\def \conv {\text{conv}}
\def \cont {\text{cont}}
\DeclareMathOperator{\cl}{\textbf{CL}}
\DeclareMathOperator{\sg}{sg}
\DeclareMathOperator{\trdeg}{trdeg}
\def \Ord {\text{Ord}}

\DeclareMathOperator{\cf}{cf}
\DeclareMathOperator{\zfc}{ZFC}

%\DeclareMathOperator{\Th}{Th}
%\def \th {\text{Th}}
% \newcommand{\th}{\text{Th}}
\DeclareMathOperator{\type}{type}
\DeclareMathOperator{\zf}{\textbf{ZF}}
\def \fa {\mathfrak{a}}
\def \fb {\mathfrak{b}}
\def \fc {\mathfrak{c}}
\def \fd {\mathfrak{d}}
\def \fe {\mathfrak{e}}
\def \ff {\mathfrak{f}}
\def \fg {\mathfrak{g}}
\def \fh {\mathfrak{h}}
%\def \fi {\mathfrak{i}}
\def \fj {\mathfrak{j}}
\def \fk {\mathfrak{k}}
\def \fl {\mathfrak{l}}
\def \fm {\mathfrak{m}}
\def \fn {\mathfrak{n}}
\def \fo {\mathfrak{o}}
\def \fp {\mathfrak{p}}
\def \fq {\mathfrak{q}}
\def \fr {\mathfrak{r}}
\def \fs {\mathfrak{s}}
\def \ft {\mathfrak{t}}
\def \fu {\mathfrak{u}}
\def \fv {\mathfrak{v}}
\def \fw {\mathfrak{w}}
\def \fx {\mathfrak{x}}
\def \fy {\mathfrak{y}}
\def \fz {\mathfrak{z}}
\def \fA {\mathfrak{A}}
\def \fB {\mathfrak{B}}
\def \fC {\mathfrak{C}}
\def \fD {\mathfrak{D}}
\def \fE {\mathfrak{E}}
\def \fF {\mathfrak{F}}
\def \fG {\mathfrak{G}}
\def \fH {\mathfrak{H}}
\def \fI {\mathfrak{I}}
\def \fJ {\mathfrak{J}}
\def \fK {\mathfrak{K}}
\def \fL {\mathfrak{L}}
\def \fM {\mathfrak{M}}
\def \fN {\mathfrak{N}}
\def \fO {\mathfrak{O}}
\def \fP {\mathfrak{P}}
\def \fQ {\mathfrak{Q}}
\def \fR {\mathfrak{R}}
\def \fS {\mathfrak{S}}
\def \fT {\mathfrak{T}}
\def \fU {\mathfrak{U}}
\def \fV {\mathfrak{V}}
\def \fW {\mathfrak{W}}
\def \fX {\mathfrak{X}}
\def \fY {\mathfrak{Y}}
\def \fZ {\mathfrak{Z}}

\def \sfA {\textsf{A}}
\def \sfB {\textsf{B}}
\def \sfC {\textsf{C}}
\def \sfD {\textsf{D}}
\def \sfE {\textsf{E}}
\def \sfF {\textsf{F}}
\def \sfG {\textsf{G}}
\def \sfH {\textsf{H}}
\def \sfI {\textsf{I}}
\def \sfj {\textsf{J}}
\def \sfK {\textsf{K}}
\def \sfL {\textsf{L}}
\def \sfM {\textsf{M}}
\def \sfN {\textsf{N}}
\def \sfO {\textsf{O}}
\def \sfP {\textsf{P}}
\def \sfQ {\textsf{Q}}
\def \sfR {\textsf{R}}
\def \sfS {\textsf{S}}
\def \sfT {\textsf{T}}
\def \sfU {\textsf{U}}
\def \sfV {\textsf{V}}
\def \sfW {\textsf{W}}
\def \sfX {\textsf{X}}
\def \sfY {\textsf{Y}}
\def \sfZ {\textsf{Z}}
\def \sfa {\textsf{a}}
\def \sfb {\textsf{b}}
\def \sfc {\textsf{c}}
\def \sfd {\textsf{d}}
\def \sfe {\textsf{e}}
\def \sff {\textsf{f}}
\def \sfg {\textsf{g}}
\def \sfh {\textsf{h}}
\def \sfi {\textsf{i}}
\def \sfj {\textsf{j}}
\def \sfk {\textsf{k}}
\def \sfl {\textsf{l}}
\def \sfm {\textsf{m}}
\def \sfn {\textsf{n}}
\def \sfo {\textsf{o}}
\def \sfp {\textsf{p}}
\def \sfq {\textsf{q}}
\def \sfr {\textsf{r}}
\def \sfs {\textsf{s}}
\def \sft {\textsf{t}}
\def \sfu {\textsf{u}}
\def \sfv {\textsf{v}}
\def \sfw {\textsf{w}}
\def \sfx {\textsf{x}}
\def \sfy {\textsf{y}}
\def \sfz {\textsf{z}}



%\DeclareMathOperator{\ker}{ker}
\DeclareMathOperator{\im}{im}

\DeclareMathOperator{\inn}{Inn}
\DeclareMathOperator{\AC}{\textbf{AC}}
\DeclareMathOperator{\cod}{cod}
\DeclareMathOperator{\dom}{dom}
\DeclareMathOperator{\ran}{ran}
\DeclareMathOperator{\textd}{d}
\DeclareMathOperator{\td}{d}
\DeclareMathOperator{\id}{id}
\DeclareMathOperator{\LT}{LT}
\DeclareMathOperator{\Mat}{Mat}
\DeclareMathOperator{\Eq}{Eq}
\DeclareMathOperator{\irr}{irr}
\DeclareMathOperator{\Fr}{Fr}
\DeclareMathOperator{\Gal}{Gal}
\DeclareMathOperator{\lcm}{lcm}
\DeclareMathOperator{\alg}{\text{alg}}
\DeclareMathOperator{\Th}{Th}

\DeclareMathOperator{\DAG}{DAG}
\DeclareMathOperator{\ODAG}{ODAG}

% \varprod
\DeclareSymbolFont{largesymbolsA}{U}{txexa}{m}{n}
\DeclareMathSymbol{\varprod}{\mathop}{largesymbolsA}{16}
% \DeclareMathSymbol{\tonm}{\boldsymbol{\to}\textbf{Nm}}
\def \tonm {\bto\textbf{Nm}}
\def \tohm {\bto\textbf{Hm}}

% Category theory
\DeclareMathOperator{\Ab}{\textbf{Ab}}
\DeclareMathOperator{\Alg}{\textbf{Alg}}
\DeclareMathOperator{\Rng}{\textbf{Rng}}
\DeclareMathOperator{\Sets}{\textbf{Sets}}
\DeclareMathOperator{\Met}{\textbf{Met}}
\DeclareMathOperator{\Aut}{\textbf{Aut}}
\DeclareMathOperator{\RMod}{R-\textbf{Mod}}
\DeclareMathOperator{\RAlg}{R-\textbf{Alg}}
\DeclareMathOperator{\LF}{LF}
\DeclareMathOperator{\op}{op}
% Model theory
\DeclareMathOperator{\tp}{tp}
\DeclareMathOperator{\Diag}{Diag}
\DeclareMathOperator{\el}{el}
\DeclareMathOperator{\depth}{depth}
\DeclareMathOperator{\FO}{FO}
\DeclareMathOperator{\fin}{fin}
\DeclareMathOperator{\qr}{qr}
\DeclareMathOperator{\Mod}{Mod}
\DeclareMathOperator{\TC}{TC}
\DeclareMathOperator{\KH}{KH}
\DeclareMathOperator{\Part}{Part}
\DeclareMathOperator{\Infset}{\textsf{Infset}}
\DeclareMathOperator{\DLO}{\textsf{DLO}}
\DeclareMathOperator{\sfMod}{\textsf{Mod}}
\DeclareMathOperator{\AbG}{\textsf{AbG}}
\DeclareMathOperator{\sfACF}{\textsf{ACF}}
% Computability Theorem
\DeclareMathOperator{\Tot}{Tot}
\DeclareMathOperator{\graph}{graph}
\DeclareMathOperator{\Fin}{Fin}
\DeclareMathOperator{\Cof}{Cof}
\DeclareMathOperator{\lh}{lh}
% Commutative Algebra
\DeclareMathOperator{\ord}{ord}
\DeclareMathOperator{\Idem}{Idem}
\DeclareMathOperator{\zdiv}{z.div}
\DeclareMathOperator{\Frac}{Frac}
\DeclareMathOperator{\rad}{rad}
\DeclareMathOperator{\nil}{nil}
\DeclareMathOperator{\Ann}{Ann}
\DeclareMathOperator{\End}{End}
\DeclareMathOperator{\coim}{coim}
\DeclareMathOperator{\coker}{coker}
\DeclareMathOperator{\Bil}{Bil}
\DeclareMathOperator{\Tril}{Tril}
% Topology
\newcommand{\interior}[1]{%
  {\kern0pt#1}^{\mathrm{o}}%
}

% \makeatletter
% \newcommand{\vect}[1]{%
%   \vbox{\m@th \ialign {##\crcr
%   \vectfill\crcr\noalign{\kern-\p@ \nointerlineskip}
%   $\hfil\displaystyle{#1}\hfil$\crcr}}}
% \def\vectfill{%
%   $\m@th\smash-\mkern-7mu%
%   \cleaders\hbox{$\mkern-2mu\smash-\mkern-2mu$}\hfill
%   \mkern-7mu\raisebox{-3.81pt}[\p@][\p@]{$\mathord\mathchar"017E$}$}

% \newcommand{\amsvect}{%
%   \mathpalette {\overarrow@\vectfill@}}
% \def\vectfill@{\arrowfill@\relbar\relbar{\raisebox{-3.81pt}[\p@][\p@]{$\mathord\mathchar"017E$}}}

% \newcommand{\amsvectb}{%
% \newcommand{\vect}{%
%   \mathpalette {\overarrow@\vectfillb@}}
% \newcommand{\vecbar}{%
%   \scalebox{0.8}{$\relbar$}}
% \def\vectfillb@{\arrowfill@\vecbar\vecbar{\raisebox{-4.35pt}[\p@][\p@]{$\mathord\mathchar"017E$}}}
% \makeatother
% \bigtimes

\DeclareFontFamily{U}{mathx}{\hyphenchar\font45}
\DeclareFontShape{U}{mathx}{m}{n}{
      <5> <6> <7> <8> <9> <10>
      <10.95> <12> <14.4> <17.28> <20.74> <24.88>
      mathx10
      }{}
\DeclareSymbolFont{mathx}{U}{mathx}{m}{n}
\DeclareMathSymbol{\bigtimes}{1}{mathx}{"91}
% \odiv
\DeclareFontFamily{U}{matha}{\hyphenchar\font45}
\DeclareFontShape{U}{matha}{m}{n}{
      <5> <6> <7> <8> <9> <10> gen * matha
      <10.95> matha10 <12> <14.4> <17.28> <20.74> <24.88> matha12
      }{}
\DeclareSymbolFont{matha}{U}{matha}{m}{n}
\DeclareMathSymbol{\odiv}         {2}{matha}{"63}


\newcommand\subsetsim{\mathrel{%
  \ooalign{\raise0.2ex\hbox{\scalebox{0.9}{$\subset$}}\cr\hidewidth\raise-0.85ex\hbox{\scalebox{0.9}{$\sim$}}\hidewidth\cr}}}
\newcommand\simsubset{\mathrel{%
  \ooalign{\raise-0.2ex\hbox{\scalebox{0.9}{$\subset$}}\cr\hidewidth\raise0.75ex\hbox{\scalebox{0.9}{$\sim$}}\hidewidth\cr}}}

\newcommand\simsubsetsim{\mathrel{%
  \ooalign{\raise0ex\hbox{\scalebox{0.8}{$\subset$}}\cr\hidewidth\raise1ex\hbox{\scalebox{0.75}{$\sim$}}\hidewidth\cr\raise-0.95ex\hbox{\scalebox{0.8}{$\sim$}}\cr\hidewidth}}}
\newcommand{\stcomp}[1]{{#1}^{\mathsf{c}}}

\setlength{\baselineskip}{0.8in}

\stackMath
\newcommand\yrightarrow[2][]{\mathrel{%
  \setbox2=\hbox{\stackon{\scriptstyle#1}{\scriptstyle#2}}%
  \stackunder[0pt]{%
    \xrightarrow{\makebox[\dimexpr\wd2\relax]{$\scriptstyle#2$}}%
  }{%
   \scriptstyle#1\,%
  }%
}}
\newcommand\yleftarrow[2][]{\mathrel{%
  \setbox2=\hbox{\stackon{\scriptstyle#1}{\scriptstyle#2}}%
  \stackunder[0pt]{%
    \xleftarrow{\makebox[\dimexpr\wd2\relax]{$\scriptstyle#2$}}%
  }{%
   \scriptstyle#1\,%
  }%
}}
\newcommand\yRightarrow[2][]{\mathrel{%
  \setbox2=\hbox{\stackon{\scriptstyle#1}{\scriptstyle#2}}%
  \stackunder[0pt]{%
    \xRightarrow{\makebox[\dimexpr\wd2\relax]{$\scriptstyle#2$}}%
  }{%
   \scriptstyle#1\,%
  }%
}}
\newcommand\yLeftarrow[2][]{\mathrel{%
  \setbox2=\hbox{\stackon{\scriptstyle#1}{\scriptstyle#2}}%
  \stackunder[0pt]{%
    \xLeftarrow{\makebox[\dimexpr\wd2\relax]{$\scriptstyle#2$}}%
  }{%
   \scriptstyle#1\,%
  }%
}}

\newcommand\altxrightarrow[2][0pt]{\mathrel{\ensurestackMath{\stackengine%
  {\dimexpr#1-7.5pt}{\xrightarrow{\phantom{#2}}}{\scriptstyle\!#2\,}%
  {O}{c}{F}{F}{S}}}}
\newcommand\altxleftarrow[2][0pt]{\mathrel{\ensurestackMath{\stackengine%
  {\dimexpr#1-7.5pt}{\xleftarrow{\phantom{#2}}}{\scriptstyle\!#2\,}%
  {O}{c}{F}{F}{S}}}}

\newenvironment{bsm}{% % short for 'bracketed small matrix'
  \left[ \begin{smallmatrix} }{%
  \end{smallmatrix} \right]}

\newenvironment{psm}{% % short for ' small matrix'
  \left( \begin{smallmatrix} }{%
  \end{smallmatrix} \right)}

\newcommand{\bbar}[1]{\mkern 1.5mu\overline{\mkern-1.5mu#1\mkern-1.5mu}\mkern 1.5mu}

\newcommand{\bigzero}{\mbox{\normalfont\Large\bfseries 0}}
\newcommand{\rvline}{\hspace*{-\arraycolsep}\vline\hspace*{-\arraycolsep}}

\font\zallman=Zallman at 40pt
\font\elzevier=Elzevier at 40pt

\newcommand\isoto{\stackrel{\textstyle\sim}{\smash{\longrightarrow}\rule{0pt}{0.4ex}}}
\newcommand\embto{\stackrel{\textstyle\prec}{\smash{\longrightarrow}\rule{0pt}{0.4ex}}}
\setcounter{secnumdepth}{2}
\setcounter{tocdepth}{2}
\author{Katin Tent \& Martin Ziegler}
\date{\today}
\title{A Course in Model Theory}
\hypersetup{
 pdfauthor={Katin Tent \& Martin Ziegler},
 pdftitle={A Course in Model Theory},
 pdfkeywords={},
 pdfsubject={},
 pdfcreator={Emacs 27.1 (Org mode 9.3)}, 
 pdflang={English}}
\begin{document}

\maketitle
\tableofcontents


\section{The Basics}
\label{sec:org1be1bef}

\subsection{Structures}
\label{sec:org2aa21d0}
\begin{definition}[]
Let \(\fA,\fB\) be \(L\)-structures. A map \(h:A\to B\) is called a
\textbf{homomorphism} if for all \(a_1,\dots,a_n\in A\)
\begin{equation*}
 \begin{array}{rcl}
 h(c^{\fA})&=&c^{\fB}\\
 h(f^{\fA}(a_1,\dots,a_n))&=&f^{\fB}(h(a_1),\dots,h(a_n))\\
 R^{\fA}(a_1,\dots,a_n)&\Rightarrow&R^{\fB}(h(a_1),\dots,h(a_n))
 \end{array}
\end{equation*}

We denote this by
\begin{equation*}
 h:\fA\to\fB
\end{equation*}

If in addition \(h\) is injective and
\begin{equation*}
 R^{\fA}(a_1,\dots,a_n)\Leftrightarrow R^{\fB}(h(a_1),\dots,h(a_n))
\end{equation*}
for all \(a_1,\dots,a_n\in A\), then \(h\) is called an (isomorphic)
\textbf{embedding}. An \textbf{isomorphism} is a surjective embedding
\end{definition}

\begin{lemma}[]
\label{lemma1.1.8}
Let \(h:\fA \isoto\fA'\) be an isomorphism and \(\fB\) an
extension of \(\fA\). Then there exists an extension \(\fB'\) of \(\fA'\) and
an isomorphism \(g:\fB \isoto\fB'\) extending \(h\)
\end{lemma}

For any family \(\fA_i\) of substructures of \(\fB\), the intersection of the
\(A_i\) is either empty or a substructure of \(\fB\). Therefore if \(S\) is
any non-empty subset of \(\fB\), then there exists a smallest substructure
\(\fA=\la S\ra^{\fB}\) which contains \(S\). We call the \(\fA\) the
substructure \textbf{generated} by \(S\)

\begin{lemma}[]
If \(\fa=\la S\ra\), then every homomorphism \(h:\fA\to\fB\) is determined by
its values on \(S\)
\end{lemma}

\begin{definition}[]
Let \((I,\le)\) be a \textbf{directed partial order}. This means that for all
\(i,j\in I\) there exists a \(k\in I\) s.t. \(i\le k\) and \(j\le k\). A
family \((\fA_i)_{i\in I}\) of \(L\)-structures is called \textbf{directed} if
\begin{equation*}
i\le j\Rightarrow\fA_i\subseteq\fA_j
\end{equation*}
If \(I\) is linearly ordered, we call \((\fA_i)_{i\in I}\) a \textbf{chain}
\end{definition}

If a structure \(\fA_1\) is isomorphic to a substructure \(\fA_0\) of itself,
\begin{equation*}
 h_0:\fA_0\isoto\fA_1
\end{equation*}
then Lemma \ref{lemma1.1.8} gives an extension
\begin{equation*}
 h_1:\fA_1\isoto\fA_2
\end{equation*}
Continuing in this way we obtain a chain 
\(\fA_0\subseteq \fA_1\subseteq\fA_2\subseteq\dots\)
and an increasing sequence
\(h_i:\fA_i\isoto\fA_{i+1}\) of isomorphism

\begin{lemma}[]
Let \((\fA_i)_{i\in I}\) be a directed family of \(L\)-structures. Then
\(A=\bigcup_{i\in I}A_i\) is the universe of a (uniquely determined)
\(L\)-structure
\begin{equation*}
\fA=\bigcup_{i\in I}\fA_i
\end{equation*}
which is an extension of all \(\fA_i\)
\end{lemma}

A subset \(K\) of \(L\) is called a \textbf{sublanguage}. An \(L\)-structure becomes a
\(K\)-structure, the \textbf{reduct}.
\begin{equation*}
\fA\restriction K=(A,(Z^{\fA})_{Z\in K})
\end{equation*}
Conversely we call \(\fA\) an \textbf{expansion} of \(\fA\restriction K\).
\begin{enumerate}
\item Let \(B\subseteq A\) , we obtain a new language
\begin{equation*}
L(B)=L\cup B
\end{equation*}
and the \(L(B)\)-structure 
\begin{equation*}
\fA_B=(\fA,b)_{b\in B}
\end{equation*}
Note that \(\Aut(\fA_B)\) is the group of automorphisms of \(\fA\) fixing
\(B\) elementwise. We denote this group by \(\Aut(\fA/B)\)
\end{enumerate}


Let \(S\) be a set, which we call the set of sorts. An \(S\)-sorted
language \(L\) is given by a set of constants for each sort in \(S\), and
typed function and relations. For any tuple \((s_1,\dots,s_n)\) and
\((s_1,\dots,s_n,t)\) there is a set of relation symbols and function
symbols respectively. An \(S\)-sorted structure is a pair
\(\fA=(A,(Z^{\fA})_{Z\in L})\), where 
\begin{alignat*}{2}      
&A&&\text{if a family $(A_s)_{s\in S}$ of non-empty sets}\\
&Z^{\fA}\in A_s&&\text{if $Z$ is a constant of sort $s\in S$}\\
&Z^{\fA}:A_{s_1}\times\dots\times A_{s_n}\to A_t&&\text{if $Z$ is a
function symbol of type $(s_1,\dots,s_n,t)$}\\
&Z^{\fA}\subseteq A_{s_1}\times\dots\times A_{s_n}&&\text{if $Z$ is a
relation symbol of type $(s_1,\dots,s_n)$}
\end{alignat*}

\begin{examplle}[]
Consider the two-sorted language \(L_{Perm}\) for permutation groups with a
sort \(x\) for the set and a sort \(g\) for the group. The constants and
function symbols for \(L_{Perm}\) are those of \(L_{Group}\) restricted to
the sort \(g\) and an additional function symbol \(\varphi\) of type \((x,g,x)\). Thus
an \(L_{Perm}\)-structure \((X,G)\) is given by a set \(X\) and an
\(L_{Group}\)-structure \(G\) together with a function \(X\times G\to X\)
\end{examplle}

\subsection{Language}
\label{sec:orgfd19562}
\begin{lemma}[]
\label{lemma1.2.11}
Suppose \(\vv{b}\) and \(\vv{c}\) agree on all variables which are free
in \(\varphi\). Then 
\begin{equation*}
\fA\models\varphi[\vv{b}]\Leftrightarrow\fA\models\varphi[\vv{c}]
\end{equation*}
\end{lemma}

We define
\begin{equation*}
\fA\models\varphi[a_1,\dots,a_n]
\end{equation*}
by \(\fA\models\varphi[\vv{b}]\), where \(\vv{b}\) is an assignment
satisfying \(\vv{b}(x_i)=a_i\). Because of Lemma \ref{lemma1.2.11} this is
well defined.




Thus \(\varphi(x_1,\dots,x_n)\) defines an \(n\)-ary relation
\begin{equation*}
\varphi(\fA)=\{\bbar{a}\mid\fA\models\varphi[\bbar{a}]\}
\end{equation*}
on \(A\), the \textbf{realisation set} of \(\varphi\). Such realisation sets are called
\textbf{0-definable subsets} of \(A^n\), or 0-definable relations

Let \(B\) be a subset of \(A\). A \textbf{\(B\)-definable} subset of \(\fA\) is a set
of the form \(\varphi(\fA)\) for an \(L(B)\)-formula \(\varphi(x)\). We also say that \(\varphi\)
are defined \textbf{over} \(B\) and that the set \(\varphi(\fA)\) is defined by \(\varphi\). We call
two formulas \textbf{equivalent} if in every structure they define the same set.

Atomic formulas and their negations are called \textbf{basic}. Formulas without
quantifiers are Boolean combinations of basic formulas. It is convenient to
allow the empty conjunction and the empty disjunction. For that we introduce
two new formulas: the formula \(\top\), which is always true, and the formula
\(\bot\), which is always false. We define
\begin{align*}
&\bigwedge_{i<0}\pi_i=\top\\
&\bigvee_{i<0}\pi_i=\bot
\end{align*}

A formula is in \textbf{negation normal form} if it is built from basic formulas using
\(\wedge,\vee,exists,\forall\)

\begin{definition}[]
A formula in negation normal form which does not contain any existential
quantifier is called \textbf{universal}. Formulas in negation normal form without
universal quantifiers are called \textbf{existential}
\end{definition}


Let \(\fA\) be an \(L\)-structure. The \textbf{atomic diagram} of \(\fA\) is
\begin{equation*}
\Diag(\fA)=\{\varphi\text{ basic $L(A)$-sentence}\mid\fA_A\models\varphi\}
\end{equation*}

\begin{lemma}[]
The models of \(\Diag(\fa)\) are precisely those structures
\((\fB,h(a))_{a\in A}\) for embeddings \(h:\fA\to\fB\)
\end{lemma}

\begin{exercise}
\label{ex1.2.3}
Every formula is equivalent to a formula in prenex normal form:
\begin{equation*}
Q_1x_1\dots Q_nx_n\varphi
\end{equation*}
The \(Q_i\) are quantifiers and \(\varphi\) is quantifier-free
\end{exercise}

\begin{proof}
\begin{align*}
&(\forall x)\phi\wedge\psi\models\rmodels
\forall x(\phi\wedge\psi)\text{ if }\exists x\top(\text{at least one individual exists})\\
&(\forall x\phi)\vee\psi\models\rmodels\forall x(\phi\vee\psi)\\
&(\exists x\phi)\wedge\psi\models\rmodels\exists x(\phi\wedge\psi)\\
&(\exists x\phi)\vee\psi\models\rmodels\exists x(\phi\vee\psi)\text{ if }\exists x\top\\
&\neg\exists x\phi\models\rmodels\forall x\neg\phi\\
&\neg\forall x\phi\models\rmodels\exists x\neg\phi\\
&(\forall x\phi)\to\psi\models\rmodels\exists x(\phi\to\psi)\text{ if }\exists x\top\\
&(\exists x\phi)\to\psi\models\rmodels\forall x(\phi\to\psi)\\
&\phi\to(\exists x\psi)\models\rmodels\exists x(\phi\to\psi)\text{ if }\exists x\top\\
&\phi\to(\forall x\psi)\models\rmodels\forall x(\phi\to\psi)
\end{align*}
\end{proof}

\subsection{Theories}
\label{sec:org0264cf8}
\begin{definition}[]
An \textbf{\(L\)-theory} \(T\) is a set of \(L\)-sentences
\end{definition}

A theory which has a model is a \textbf{consistent} theory. We call a set \(\Sigma\) of
\(L\)-formulas \textbf{consistent} if there is an \(L\)-structure and \textbf{an assignment}
\(\vv{b}\) \textbf{s.t.} \(\fA\models[\vv{b}]\) for all \(\varphi\in\Sigma\)

\begin{lemma}[]
Let \(T\) be an \(L\)-theory and \(L'\) be an extension of \(L\). Then \(T\)
is consistent as an \(L\)-theory iff \(T\) is consistent as a \(L'\)-theory
\end{lemma}


\begin{lemma}[]
\label{lemma1.3.4}
\begin{enumerate}
\item If \(T\models\varphi\) and \(T\models(\varphi\to\psi)\), then \(T\models\psi\)
\item If \(T\models\varphi(c_1,\dots,c_n)\) and the constants \(c_1,\dots,c_n\)
occur neither in \(T\) nor in \(\varphi(x_1,\dots,x_n)\), then \(T\models\forall
      x_1\dots x_n\varphi(x_1,\dots,x_n)\)
\end{enumerate}
\end{lemma}

\begin{proof}
\begin{enumerate}
\setcounter{enumi}{1}
\item Let \(L'=L\setminus\{c_1,\dots,c_n\}\). If the \(L'\)-structure is a
model of \(T\) and \(a_1,\dots,a_n\) are arbitrary elements, then
\((\fA,a_1,\dots,a_n)\models\varphi(c_1,\dots,c_n)\). This means
\(\fA\models\forall x_1\dots x_n\varphi(x_1,\dots,x_n)\).
\end{enumerate}
\end{proof}

\(S\) and \(T\) are called \textbf{equivalent}, \(S\equiv T\), if \(S\) and \(T\) have
the same models

\begin{definition}[]
A consistent \(L\)-theory \(T\) is called \textbf{complete} if for all \(L\)-sentences
\(\varphi\)
\begin{equation*}
T\models\varphi \quad\text{ or }\quad T\models\neg\varphi
\end{equation*}
\end{definition}

\begin{definition}[]
For a complete theory \(T\) we define
\begin{equation*}
\abs{T}=\max(\abs{L},\aleph_0)
\end{equation*}
\end{definition}

The typical example of a complete theory is the theory of a structure \(\fA\)
\begin{equation*}
\Th(\fA)=\{\varphi\mid\fA\models\varphi\}
\end{equation*}

\begin{lemma}[]
A consistent theory is complete iff it is maximal consistent, i.e., if it is
equivalent to every consistent extension
\end{lemma}

\begin{definition}[]
Two \(L\)-structures \(\fA\) and \(\fB\) are called \textbf{elementary equivalent}
\begin{equation*}
\fA\equiv\fB
\end{equation*}
if they have the same theory
\end{definition}

\begin{lemma}[]
Let \(T\) be a consistent theory. Then the following are equivalent
\begin{enumerate}
\item \(T\) is complete
\item All models of \(T\) are elemantarily equivalent
\item There exists a structure \(\fA\) with \(T\equiv\Th(\fA)\)
\end{enumerate}
\end{lemma}

\begin{proof}
\(1\to3\to2\to1\)
\end{proof}





\section{Elementary Extensions and Compactness}
\label{sec:org2818568}
\subsection{Elementary substructures}
\label{sec:org7e20027}
Let \(\fA,\fB\) be two \(L\)-structures. A map \(h:A\to B\) is called
\textbf{elementary} if for all \(a_1,\dots,a_n\in A\) we have
\begin{equation*}
\fA\models\varphi[a_1,\dots,a_n]\Leftrightarrow
\fB\models\varphi[h(a_1),\dots,h(a_n)]
\end{equation*}
We write
\begin{equation*}
h:\fA\embto\fB
\end{equation*}
\begin{lemma}[]
\label{lemma2.1.1}
The models of \(\Th(\fA_A)\) are exactly the structures of the form
\((\fB,h(a))_{a\in A}\) for elementary embeddings \(h:\fA\embto\fB\)
\end{lemma}

We call \(\Th(\fA_A)\) the \textbf{elemantary diagram} of \(\fA\)

A substructure \(\fA\) of \(\fB\) is called \textbf{elementary} if the inclusion map
is elementary. In this case we write
\begin{equation*}
\fA\prec\fB
\end{equation*}

\begin{theorem}[Tarski's Test]
\label{thm2.1.2}
Let \(\fB\) be an \(L\)-structure and \(A\) a subset of \(B\). Then \(A\) is
the universe of an elementary substructure iff every \(L(A)\)-formula
\(\varphi(x)\) which is satisfiable in \(\fB\) can be satisfied by an element of \(A\)
\end{theorem}

We use Tarski's Test to construct small elementary substructures

\begin{corollary}[]
Suppose \(S\) is a subset of the \(L\)-structure \(\fB\). Then \(\fB\) has a
elementary substructure \(\fA\) containing \(S\) and of cardinality at most
\begin{equation*}
\max(\abs{S},\abs{L},\aleph_0)
\end{equation*}
\end{corollary}

\begin{proof}
We construct \(A\) as the union of an ascending sequence \(S_0\subseteq
   S_1\subseteq\dots\) of subsets of \(B\). We start with \(S_0=S\). If \(S_i\)
is already defined, we choose an element \(a_\varphi\in B\) for every
\(L(S_i)\)-formula \(\varphi(x)\) which is satisfiable in \(\fB\) and define
\(S_{i+1}\) to be \(S_i\) together with these \(a_{\varphi}\).

An \(L\)-formula is a finite sequence of symbols from \(L\), auxiliary
symbols and logical symbols. These are
\(\abs{L}+\aleph_0=\max(\abs{L},\aleph_0)\) many symbols and there are
exactly\(\max(\abs{L},\aleph_0)\) many \(L\)-formulas

Let \(\kappa=\max(\abs{S},\abs{L},\aleph_0)\). There are \(\kappa\) many
\(L(S)\)-formulas: therefore \(\abs{S_1}\le\kappa\). Inductively it follows
for every \(i\) that \(\abs{S_i}\le\kappa\). Finally we have \(\abs{A}\le\kappa\cdot\aleph_0=\kappa\)
\end{proof}

A directed family \((\fA_i)_{i\in I}\) of structures is \textbf{elementary} if
\(\fA_i\prec\fA_j\) for all \(i\le j\)

\begin{theorem}[Tarski's Chain Lemma]
\label{thm2.1.4}
The union of an elementary directed family is an elementary extension of all
its members
\end{theorem}

\begin{proof}
Let \(\fA=\bigcup_{i\in I}(\fA_i)_{i\in I}\). We prove by induction on
\(\varphi(\bbar{x})\) that for all \(i\) and \(\bbar{a}\in\fA_i\)
\begin{equation*}
\fA_i\models\varphi(\bbar{a})\Leftrightarrow\fA\models\varphi(\bbar{a})
\end{equation*}
\end{proof}


\subsection{The Compactness Theorem}
\label{sec:org9637eae}
\begin{theorem}[Compactness Theorem]
Finitely satisfiable theories are consistent
\end{theorem}

Let \(L\) be a language and \(C\) a set of new constants. An \(L(C)\)-theory
\(T'\) is called a \textbf{Henkin theory} if for every \(L(C)\)-formula \(\varphi(x)\) there
is a constant \(c\in C\) s.t.
\begin{equation*}
\exists x\varphi(x)\to\varphi(c)\in T'
\end{equation*}

\begin{lemma}[]
Every finitely satisfiable \(L\)-theory \(T\) can be extended to a finitely
complete Henkin Theory \(T^*\)
\end{lemma}

\begin{lemma}[]
\label{lemma2.2.3}
Every finitely satisfiable \(L\)-theory \(T\) can be extended to a finitely
complete Henkin theory \(T^*\)
\end{lemma}

\begin{lemma}[]
Every finitely complete Henkin theory \(T^*\) has a model \(\fA\) (unique up
to isomorphism) consisting of constants; i.e.,
\begin{equation*}
(\fA,a_c)_{c\in C}\models T^*
\end{equation*}
with \(A=\{a_c\mid c\in C\}\)
\end{lemma}



\begin{corollary}[]
A set of formulas \(\Sigma(x_1,\dots,x_n)\) is consistent with \(T\) if and only
if every finite subset of \(\Sigma\) is consistent with \(T\)
\end{corollary}

\begin{proof}
Introduce new constants \(c_1,\dots,c_n\). Then \(\Sigma\) is consistent with \(T\) is
and only if \(T\cup\Sigma(c_1,\dots,c_n)\) is consistent. Now apply the
Compactness Theorem
\end{proof}

\begin{definition}[]
Let \(\fA\) be an \(L\)-structure and \(B\subseteq A\). Then \(a\in A\)
\textbf{realises} a set of \(L(B)\)-formulas \(\Sigma(x)\) if \(a\) satisfied all formulas
from \(\Sigma\). We write 
\begin{equation*}
\fA\models\Sigma(a)
\end{equation*}

We call \(\Sigma(x)\) \textbf{finitely satisfiable} in \(\fA\) if every finite subset of \(\Sigma\)
is realised in \(\fA\)
\end{definition}

\begin{lemma}[]
\label{lemma2.2.7}
The set \(\Sigma(x)\) is finitely satisfiable in \(\fA\) iff there is an
elementary extension of \(\fA\) in which \(\Sigma(x)\) is realised
\end{lemma}

\begin{proof}
By Lemma \ref{lemma2.1.1} \(\Sigma\) is realised in an elementary extension of \(\fA\)
iff \(\Sigma\) is consistent with \(\Th(\fA_A)\). So the lemma follows from the
observation that a finite set of \(L(A)\)-formulas is consistent with
\(\Th(\fA_A)\) iff it is realised in \(\fA\)
\end{proof}

\begin{definition}[]
Let \(\fA\) be an \(L\)-structure and \(B\) a subset of \(A\). A set \(p(x)\)
of \(L(B)\)-formulas is a \textbf{type} over \(B\) if \(p(x)\) is maximal finitely
satisfiable in \(\fA\). We call \(B\) the \textbf{domain} of \(p\). Let
\begin{equation*}
S(B)=S^{\fA}(B)
\end{equation*}
denote the set of types over \(B\).
\end{definition}

Every element \(a\) of \(\fA\) determines a type
\begin{equation*}
\tp(a/B)=tp^{\fA}(a/B)=\{\varphi(x)\mid\fA\models\varphi(a),\varphi\text{ an $L(B)$-formula}\}
\end{equation*}
So an element \(a\) realises the type \(p\in S(B)\) exactly if
\(p=\tp(a/B)\). If \(\fA'\) is an elementary extension of \(\fA\), then
\begin{equation*}
S^{\fA}(B)=S^{\fA'}(B)\quad\text{ and }\quad
\tp^{\fA'}(a/B)=\tp^{\fA}(a/B)
\end{equation*}
If \(\fA'\models p(x)\) then \(\fA'\models\exists xp(x)\), so
\(\fA\models\exists xp(x)\).

We use the notation \(\tp(a)\) for \(\tp(a/\emptyset)\)

Maximal finitely satisfiable sets of formulas in \(x_1,\dots,x_n\) are called
\textbf{\(n\)-types} and
\begin{equation*}
S_n(B)=S_N^{\fA}(B)
\end{equation*}
denotes the set of \(n\)-types over \(B\).
\begin{equation*}
\tp(C/B)=\{\varphi(x_{c_1},\dots,x_{c_n})\mid\fA\models\varphi(c_1,\dots,c_n),\varphi
\text{ an $L(B)$-formula}\}
\end{equation*}

\begin{corollary}[]
Every structure \(\fA\) has an elementary extension \(\fB\) in which all
types over \(A\) are realised
\end{corollary}

\begin{proof}
We choose for every \(p\in S(A)\) a new constant \(c_p\). We have to find a
model of
\begin{equation*}
\Th(\fA_A)\cup\bigcup_{p\in S(A)}p(c_p)
\end{equation*}
This theory is finitely satisfiable since every \(p\) is finitely satisfiable
in \(\fA\).

Or use Lemma \ref{lemma2.2.7}. Let \((p_\alpha)_{\alpha<\lambda}\) be an enumeration of
\(S(A)\). Construct an elementary chain
\begin{equation*}
\fA=\fA_0\prec\fA_1\prec\dots\prec\fA_\beta\prec\dots(\beta\le\lambda)
\end{equation*}
s.t. each \(p_\alpha\) is realised in \(\fA_{\alpha+1}\) (by recursion
theorem on ordinal numbers)

Suppose that the elementary chain \((\fA_{\alpha'})_{\alpha'<\beta}\) is already
constructed. If \(\beta\) is a limit ordinal, we let
\(\fA_\beta=\bigcup_{\alpha<\beta}\fA_\alpha\), which is elementary by Lemma \ref{thm2.1.4}. If
\(\beta=\alpha+1\) we  first note that \(p_\alpha\) is also finitely
satisfiable in \(\fA_\alpha\), therefore we can realise \(p_\alpha\) in a
suitable elementary extension \(\fA_\beta\succ\fA_\alpha\) by Lemma
\ref{lemma2.2.7}. Then \(\fB=\fA_\lambda\) is the model we were looking for
\end{proof}
\section{Quantifier Elimination}
\label{sec:orgc5cd17e}
\subsection{Preservation theorems}
\label{sec:orgdca917f}
\begin{lemma}[Separation Lemma]
Let \(T_1, T_2\) be two theories. Assume \(\calh\) is a set of sentences
which is closed under \(\wedge,\vee\) and contains \(\bot\) and \(\top\).
Then the following are equivalent
\begin{enumerate}
\item There is a sentence \(\varphi\in\calh\) which separates \(T_1\) from
\(T_2\). This means
\begin{equation*}
  T_1\models\varphi \quad\text{ and }\quad
  T_2\models\neg\varphi
\end{equation*}
\item All models \(\fA_1\) of \(T_1\) can be separated from all models \(\fA_2\)
of \(T_2\) by a sentence \(\varphi\in\calh\). This means
\begin{equation*}
  \fA_1\models\varphi \quad\text{ and }\quad\fA_2\models\neg\varphi
\end{equation*}
\end{enumerate}
\end{lemma}

\begin{proof}
\(2\to1\). For any model \(\fA_1\) of \(T_1\) let \(\calh_{\fA_1}\) be the
set of all sentences from \(\calh\) which are true in \(\fA_1\). (2) implies
that \(\calh_{\fA_1}\) and \(T_2\) cannot have a common model. By the
Compactness Theorem there is a finite conjunction \(\varphi_{\fA_1}\) of
sentences from \(\calh_{\fA_1}\) inconsistent with \(T_2\). Clearly
\begin{equation*}
 T_1\cup\{\neg\varphi_{\fA_1}\mid\fA_1\models T_1\}
\end{equation*}
is inconsistent. Again by compactness \(T_1\) implies a disjunction \(\varphi\) of
finitely many of the \(\varphi_{\fA_1}\)
\end{proof}

For structures \(\fA,\fB\) and a map \(f:A\to B\) preserving all formulas
from a set of formulas \(\Delta\), we use the notation
\begin{equation*}
f:\fA\to_\Delta\fB
\end{equation*}
We also write
\begin{equation*}
\quad\fA\Rightarrow_\Delta\fB
\end{equation*}
to express that all sentences from \(\Delta\) true in \(\fA\) are also true in \(\fB\)

\begin{lemma}[]
\label{lemma3.1.2}
Let \(T\) be a theory, \(\fA\) a structure and \(\Delta\) a set of formulas, closed
under existential quantification, conjunction and substitution of variables.
Then the following are equivalent
\begin{enumerate}
\item All sentences \(\varphi\in\Delta\) which are true in \(\fA\) are
consistent with \(T\) (There is a model \(\fB\models\Th_\Delta(\fA_A)\cup T\) and \(\fA\Rightarrow_\Delta\fB\))
\item There is a model \(\fB\models T\) and a map \(f:\fA\to_\Delta\fB\)
\end{enumerate}
\end{lemma}

\begin{proof}
\(1\to2\). Consider \(\Th_\Delta(\fA_A)\), the set of all sentences
\(\delta(\bbar{a})\)
(\(\delta(\bbar{x})\in\Delta\)), which are true in \(\fA_A\). The models
\((\fB,f(a)_{a\in A})\) of this theory correspond to maps
\(f:\fA\to_\Delta\fB\). \textbf{This means that we have to find a model of}
\(T\cup\Th_\Delta(\fA_A)\). To show finite satisfiability it is enough to
show that \(T\cup D\) is consistent for every finite subset \(D\) of
\(\Th_\Delta(\fA_A)\). Let \(\delta(\bbar{a})\) be the conjunction of the elements
of \(D\). Then \(T\) has a model   \(\fB\) which is also a model of \(\varphi=\exists\bbar{x}\delta(\bbar{x})\)
\end{proof}

Lemma \ref{lemma3.1.2} applied to \(T=\Th(\fB)\) shows that
\(\fA\Rightarrow_\Delta\fB\) iff there exists a map \(f\) and a structure
\(\fB'\equiv\fB\) s.t. \(f:\fA\to_\Delta\fB'\)

\begin{theorem}[]
\label{thm3.1.3}
Let \(T_1\) and \(T_2\) be two theories. Then the following are equivalent
\begin{enumerate}
\item There is a universal sentence which separates \(T_1\) from \(T_2\)
\item No model of \(T_2\) is a substructure of a model of \(T_1\)
\end{enumerate}
\end{theorem}

\begin{proof}
\(2\to1\). If \(T_1\) and \(T_2\) cannot be separated by a universal
sentence, then they have models \(\fA_1\) and \(\fA_2\) which cannot be separated
by a universal sentence. This can be denoted by
\begin{equation*}
\fA_2\Rightarrow_\exists\fA_1
\end{equation*}
 Now Lemma \ref{lemma3.1.2} implies that \(\fA_2\) there is a map
 \(\fA_2\to_\exists\fA_1'\) where \(\fA_1'\models T_1\)
. Hence \(\fA_2\) has an extension \(\fA_2'\) s.t. \(\fA_2'\equiv\fA_1'\).
Then \(\fA'\) is gain a model of \(T_1\) contradicting (2)
\end{proof}

\begin{definition}[]
For any \(L\)-theory \(T\), the formulas \(\varphi(\bbar{x}),\psi(\bbar{x})\) are said
to be \textbf{equivalent} modulo \(T\) (or relative to \(T\)) if \(T\models\forall\bbar{x}(\varphi(\bbar{x})\leftrightarrow\psi(\bbar{x}))\)
\end{definition}

\begin{corollary}[]
\label{cor3.1.5}
Let \(T\) be a theory
\begin{enumerate}
\item Consider a formula \(\varphi(x_1,\dots,x_n)\). The following are equivalent
\begin{enumerate}
\item \(\varphi(x_1,\dots,x_n)\) is, modulo \(T\), equivalent to a universal formula
\item If \(\fA\subseteq\fB\) are models of \(T\) and \(a_1,\dots,a_n\in A\),
then \(\fB\models\varphi(a_1,\dots,a_n)\) implies \(\fA\models\varphi(a_1,\dots,a_n)\)
\end{enumerate}
\item We say that a theory which consists of universal sentences is universal.
Then \(T\) is equivalent to a universal theory iff all substructures of
models of \(T\) are again models of \(T\)
\end{enumerate}
\end{corollary}

\begin{proof}
\begin{enumerate}
\item Assume (2). We extend \(L\) by an \(n\)-tuple \(\bbar{c}\) of new
constants \(c_1,\dots,c_n\) and consider theory
\begin{equation*}
T_1=T\cup\{\varphi(\bbar{c})\}\quad\text{ and }\quad
T_2=T\cup\{\neg\varphi(\bbar{c})\}
\end{equation*}
Then (2) says the substructures of models of \(T_1\) cannot be models of
\(T_2\). By Theorem \ref{thm3.1.3} \(T_1\) and \(T_2\) can be separated by a
universal \(L(\bbar{c})\)-sentence \(\psi(\bbar{c})\). By Lemma
\ref{lemma1.3.4}, \(T_1\models\psi(\bbar{c})\) implies
\begin{equation*}
T\models\forall\bbar{x}(\varphi(\bbar{x})\to\psi(\bbar{x}))
\end{equation*}
and from \(T_2\models\neg\psi(\bbar{c})\) we see
\begin{equation*}
T\models\forall\bbar{x}(\neg\varphi(\bbar{x})\to\neg\psi(\bbar{x}))
\end{equation*}
\item Suppose a theory \(T\) has this property. Let \(\varphi\) be an axiom of \(T\). If
\(\fA\) is a substructure of \(\fB\), it is not possible for \(\fB\) to be
a model of \(T\) and for \(\fA\) to be a model of \(\neg\psi\) at the same
time. By Theorem \ref{thm3.1.3} there is a universal sentence \(\psi\) with
\(T\models\psi\) and \(\neg\varphi\models\neg\psi\). Hence all axioms of
\(T\) follow from
\begin{equation*}
T_\forall=\{\psi\mid T\models\psi,\psi\text{ universal}\}
\end{equation*}
\end{enumerate}
\end{proof}

An \(\forall\exists\)-formula is of the form
\begin{equation*}
\forall x_1\dots x_n\psi
\end{equation*}
where \(\psi\) is existential
\begin{lemma}[]
Suppose \(\varphi\) is an \(\forall\exists\)-sentence, \((\fA_i)_{i\in I}\) is a
directed family of models of \(\varphi\) and \(\fB\) the union of the \(\fA_i\). Then
\(\fB\) is also a model of \(\varphi\).
\end{lemma}

\begin{proof}
Write
\begin{equation*}
\varphi=\forall\bbar{x}\psi(\bbar{x})
\end{equation*}
where \(\psi\) is existential. For any \(\bbar{a}\in B\) there is an \(A_i\)
containing \(\bbar{a}\), clearly \(\psi(\bbar{a})\) holds in \(\fA_i\). As
\(\psi(\bbar{a})\) is existential it must also hold in \(\fB\)
\end{proof}

\begin{definition}[]
We call a theory \(T\) \textbf{inductive} if the union of any directed family of
models of \(T\) is again a model
\end{definition}

\begin{theorem}[]
\label{thm3.1.8}
Let \(T_1\) and \(T_2\) be two theories. Then the following are equivalent
\begin{enumerate}
\item there is an \(\forall\exists\)-sentence which separates \(T_1\) and \(T_2\)
\item No model of \(T_2\) is the union of a chain (or of a directed family) of
models of \(T_1\)
\end{enumerate}
\end{theorem}


\begin{proof}
\(2\to1\). If (1) is not true, \(T_1,T_2\) have models which cannot be
separated by an \(\forall\exists\)-sentence. Since
\(\exists\forall\)-formulas are equivalent to negated
\(\forall\exists\)-formulas (since \(\forall\) is too strong), we have
\begin{equation*}
\fB^0\Rightarrow_{\exists\forall}\fA
\end{equation*}
By Lemma \ref{lemma3.1.2} there is a map
\begin{equation*}
f:\fB^0\to_{\forall}\fA^0
\end{equation*}
with \(\fA^0\equiv\fA\) (since \(\fB^0\to_{\exists\forall}\fA^0\)). We can assume that\(\fB^0\subseteq\fA^0\) and \(f\)
is the inclusion map. Then
\begin{equation*}
\fA_B^0\Rightarrow_{\exists}\fB^0_B
\end{equation*}

Applying Lemma \ref{lemma3.1.2} again, we obtain an extension \(\fB_B^1\) of
\(\fA_B^0\) with \(\fB_B^1\equiv\fB_B^0\), i.e. \(\fB^0\prec\fB^1\). Hence we
have an infinite chain
\begin{gather*}
\fB^0\subseteq\fA^0\subseteq^1\fB^1\subseteq\fA^1\subseteq\fB^2\subseteq\cdots\\
\fB^0\prec\fB^1\prec\fB^2\prec\cdots\\
\fA^i\equiv\fA
\end{gather*}
Let \(\fB\) be the union of the \(\fA^i\).  Since \(\fB\) is also the union
of the elementary chain of the \(\fB^i\), it is an elementary extension of
\(\fB^0\) and hence a model of \(T_2\). But the \(\fA^i\) are models of
\(T_1\), so (2) does not hold
\end{proof}

\begin{corollary}[]
Let \(T\) be a theory
\begin{enumerate}
\item For each sentence \(\varphi\) the following are equivalent
\begin{enumerate}
\item \(\varphi\) is, modulo \(T\), equivalent to an \(\forall\exists\)-sentence
\item If
\begin{equation*}
\fA^0\subseteq\fA^1\subseteq\cdots
\end{equation*}
and their union \(\fB\) are models of \(T\), then \(\varphi\) holds in \(\fB\) if
it is true in all the \(\fA^i\)
\end{enumerate}
\item \(T\) is inductive iff it can be axiomatised by \(\forall\exists\)-sentences
\end{enumerate}
\end{corollary}

\begin{proof}
\begin{enumerate}
\item Theorem \ref{thm3.1.8} shows that \(\forall\exists\)-formulas are preserved
by unions of chains. Hence (a)\(\Rightarrow\)(b). For the converse
consider the theories
\begin{equation*}
T_1=T\cup\{\varphi\} \quad\text{ and }\quad T_2=T\cup\{\neg\varphi\}
\end{equation*}
Part (b) says that the union of a chain of models of \(T_1\) cannot be a
model of \(T_2\). By Theorem \ref{thm3.1.8} we can separate \(T_1\) and
\(T_2\) by an \(\forall\exists\)-sentence \(\psi\). Hence
\(T\cup\{\varphi\}\models\psi\) and
\(T\cup\{\neg\varphi\}\models\neg\psi\)
\item Clearly \(\forall\exists\)-axiomatised theories are inductive. For the
converse assume that \(T\) is inductive and \(\varphi\) is an axiom of \(T\). If
\(\fB\) is a union of models of \(T\), it cannot be a model of
\(\neg\varphi\). By Theorem \ref{thm3.1.8} there is an
\(\forall\exists\)-sentence \(\psi\) with \(T\models\psi\) and
\(\neg\varphi\models\neg\psi\). Hence all axioms of \(T\) follows from
\begin{equation*}
T_{\forall\exists}=\{\psi\mid T\models\psi,\psi\text{ $\forall\exists$-formula}\}
\end{equation*}
\end{enumerate}
\end{proof}
\subsection{Quantifier elimination}
\label{sec:org9e25767}
\begin{definition}[]
A theory \(T\) has \textbf{quantifier elimination} if every \(L\)-formula
\(\varphi(x_1,\dots,x_n)\) in the theory is equivalent modulo \(T\) to some
quantifier-free formula \(\rho(x_1,\dots,x_n)\)
\end{definition}

It's easy to transform any theory \(T\) into a theory with quantifier
elimination if one is willing to expand the language: just enlarge \(L\) by
adding an \(n\)-place relation symbol \(R_{\varphi}\) for every \(L\)-formula
\(\varphi(x_1,\dots,x_n)\) and \(T\) by adding all axioms
\begin{equation*}
\forall x_1,\dots,x_n(R_\varphi(x_1,\dots,x_n)\leftrightarrow\varphi(x_1,\dots,x_n))
\end{equation*}
The resulting theory, the \textbf{Morleyisation} \(T^m\) of \(T\), has quantifier
elimination

A \textbf{prime structure} of \(T\) is a structure which embeds into all models of
\(T\)

\begin{lemma}[]
A consistent theory \(T\) with quantifier elimination which posseses a prime
structure is complete
\end{lemma}

\begin{proof}
If \(\fM,\fN\models T\) and \(\fM\models\varphi\) and
\(\fN\models\neg\varphi\). The prime structure is \(\fH\). Then we have
\(h_1:\fH\to\fM\) and \(h_2:\fH\to\fN\). If \(\varphi\) doesn't contain existential
quantification, then there is a contradiction.
\end{proof}

\begin{definition}[]
A \textbf{simple existential formula} has the form
\begin{equation*}
\varphi=\exists y\rho
\end{equation*}
for a quantifier-free formula \(\rho\). If \(\rho\) is a conjunction of basic formulas, \(\varphi\)
is called \textbf{primitive existential}
\end{definition}

\begin{lemma}[]
\label{lemma3.2.4}
The theory \(T\) has quantifier elimination iff every primitive existential
formula is, modulo \(T\), equivalent to a quantifier-free formula
\end{lemma}

\begin{proof}
We can write every simple existential formula in the form \(\exists
   y\bigvee_{i<n}\rho_i\) for \(\rho_i\) which are conjunctions of basic
formulas. This shows that every simple existential formula is equivalent to a
disjunction of primitive existential formulas, namely to
\(\bigvee_{i<n}(\exists y\rho_i)\). We can therefore assume that every simple
existential formula is, modulo \(T\), equivalent to a quantifier-free formula

We are now able to eliminate the quantifiers in arbitrary formulas in prenex
normal form (Exercise \ref{ex1.2.3})
\begin{equation*}
Q_1x_1\dots Q_nx_n\rho
\end{equation*}
if \(Q_n=\exists\), we choose a quantifier-free formula \(\rho_0\) which,
modulo \(T\), is equivalent to \(\exists x_n\rho\) and proceed with the
formula \(Q_1x_1\dots Q_{n-1}x_{n-1}\rho_0\). If \(Q_n=\forall\), we
find a quantifier-free \(\rho_1\) which is, modulo \(T\), equivalent to
\(\exists x_n\neg\rho\) and proceed with \(Q_1x_1\dots Q_{n-1}x_{n-1}\neg\rho_1\)
\end{proof}


\begin{theorem}[]
\label{thm3.2.5}
For a theory \(T\) the following are equivalent
\begin{enumerate}
\item \(T\) has quantifier elimination
\item For all models \(\fM^1\) and \(\fM^2\) of \(T\) with a common substructure
\(\fA\) we have
\begin{equation*}
\fM_A^1\equiv\fM_A^2
\end{equation*}
\item For all models \(\fM^1\) and \(\fM^2\) of \(T\) with a common substructure
\(\fA\) and for all primitive existential formulas \(\varphi(x_1,\dots,x_n)\)
and parameter \(a_1,\dots,a_n\) from \(A\) we have
\begin{equation*}
\fM^1\models\varphi(a_1,\dots,a_n)\Rightarrow\fM^2\models\varphi(a_1,\dots,a_n)
\end{equation*}
(this is exactly the equivalence relation)
\end{enumerate}

If \(L\) has no constants, \(\fA\) is allowed to be the empty "structure"
\end{theorem}

\begin{proof}
\(3\to1\). Let \(\varphi(\bbar{x})\) be a primitive existential formula. In order
to show that \(\varphi(\bbar{x})\) is equivalent, modulo \(T\), to a
quantifier-free formula \(\rho(\bbar{x})\) we extend \(L\) by an \(n\)-tuple
\(\bbar{c}\) of new constants \(c_1,\dots,c_n\). \textbf{We have to show that we can}
\textbf{separate \(T\cup\{\varphi(\bbar{c})\}\) and \(T\cup\{\neg\varphi(\bbar{c})\}\) by a}
\textbf{quantifier free sentence \(\rho(\bbar{c})\)}. We apply the Separation Lemma
(\(\calh\) hear is the set of quantifier-free sentence). Let
\(\fM^1\) and \(\fM^2\) be two models of \(T\) with two distinguished
\(n\)-tuples \(\bbar{a}^1\) and \(\bbar{a}^2\). Suppose that
\((\fM^1,\bbar{a}^1)\) and \((\fM^2,\bbar{a}^2)\) satisfy the same
quantifier-free \(L(\bbar{c})\)-sentences. We have to show that
\begin{equation*}
\fM^1\models\varphi(\bbar{a}^1)\Rightarrow
\fM^2\models\varphi(\bbar{a}^2)
\end{equation*}
then there is no \(L(\bbar{c})\)-sentence that can separate the models of
\(T\cup\{\varphi(\bbar{c})\}\) and the models of \(T\cup\{\neg\varphi(\bbar{c})\}\)
Consider the substructure \(\fA^i=\la\bbar{a}^i\ra^{\fM^i}\), generated by
\(\bbar{a}^i\). If we can show that there is an isomorphism
\begin{equation*}
f:\fA^1\to\fA^2
\end{equation*}
taking \(\bbar{a}\) to \(\bbar{a}\), we may assume that \(\fA^1=\fA^2=\fA\)
and \(\bbar{a}^1=\bbar{a}^2=\bbar{a}\).

Every element of \(\fA^1\) has the form \(t^{\fM^1}[\bbar{a}^1]\) for an
\(L\)-term \(t(\bbar{x})\). The isomorphism \(f\)to be constructed must
satisfy
\begin{equation*}
f(t^{\fM^1}[\bbar{a}^1])=t^{\fM^2}[\bbar{a}^2]
\end{equation*}
We define \(f\) by this equation and have to check that \(f\) is well defined
and injective. Assume
\begin{equation*}
s^{\fM^1}[\bbar{a}^1]=t^{\fM^1}[\bbar{af^1}]
\end{equation*}
Then \(\fM^1,\bbar{a}^1\models s(\bbar{c})\dot{=}t(\bbar{c})\), and by out
assumption it also holds in \((\fM^2,\bbar{a}^2)\), which means
\begin{equation*}
s^{\fM^2}[\bbar{a}^2]=t^{\fM^2}[\bbar{a}^2]
\end{equation*}
Swapping the two sides yields injectivity.

Surjectivity is clear. It remains to show that \(f\) commutes with the
interpretation of the relation symbols. Now
\begin{equation*}
\fM^1\models R\left[t_1^{\fM^1}[\bbar{a}^1],\dots,t_m^{\fM^1}[\bbar{a}^1]\right]
\end{equation*}
is equivalent to \((\fM^1,\bbar{a}^1)\models
   R(t_1(\bbar{c}),\dots,t_m(\bbar{c}))\), which is equivalent to
\((\fM^2,\bbar{a}^2)\models
   R(t_1(\bbar{c}),\dots,t_m(\bbar{c}))\), which in turn is equivalent to
\begin{equation*}
\fM^2\models R\left[t_1^{\fM^2}[\bbar{a}^2],\dots,t_m^{\fM^2}[\bbar{a}^2]\right]
\end{equation*}
\end{proof}

Note that (2) of Theorem \ref{thm3.2.5} is saying that \(T\) is \textbf{substructure
complete}; i.e., for any model \(\fM\models T\) and substructure
\(\fA\subseteq\fM\) the theory \(T\cup\Diag(\fA)\) is complete

\begin{definition}[]
We call \(T\) \textbf{model complete} if for all models \(\fM^1\) and \(\fM^2\) of
\(T\)
\begin{equation*}
\fM^1\subseteq\fM^2\Rightarrow\fM^1\prec\fM^2
\end{equation*}
\end{definition}

\(T\) is model complete iff for any \(\fM\models T\) the theory
\(T\cup\Diag(\fM)\) is complete

\begin{lemma}[Robinson's Test]
Let \(T\) be a theory. Then the following are equivalent
\begin{enumerate}
\item \(T\) is model complete
\item For all models \(\fM^1\subseteq\fM^2\) of \(T\) and all existential
sentences \(\varphi\) from \(L(M^1)\)
\begin{equation*}
\fM^2\models\varphi\Rightarrow\fM^1\models\varphi
\end{equation*}
\item Each formula is, modulo \(T\), equivalent to a universal formula
\end{enumerate}
\end{lemma}

\begin{proof}
\(1\leftrightarrow3\). Corollary \ref{cor3.1.5}

(2) implies that every existential formula is, modulo \(T\), equivalent to a
universal formula
\end{proof}

If \(\fM^1\subseteq\fM^2\) satisfies (2), we call \(\fM^1\) \textbf{existentially
closed} in \(\fM^2\). We denote this by
\begin{equation*}
\fM^1\prec_1\fM^2
\end{equation*}

\begin{definition}[]
Let \(T\) be a theory. A theory \(T^*\) is a \textbf{model companion} of \(T\) if the
following three conditions are satisfied
\begin{enumerate}
\item Each model of \(T\) can be extended to a model of \(T^*\)
\item Each model of \(T^*\) can be extended to a model of \(T\)
\item \(T^*\) is model complete
\end{enumerate}
\end{definition}

\begin{theorem}[]
\label{thm3.2.9}
A theory \(T\) has, up to equivalence, at most one model companion \(T^*\)
\end{theorem}

\begin{proof}
If \(T^+\) is another model companion of \(T\), every model of \(T^+\) is
contained in a model of \(T^*\) and conversely. Let \(\fA^0\models T^+\) .
Then \(\fA_0\) can be embedded in a model \(\fB_0\) of \(T^*\). In turn
\(\fB_0\) is contained in a model \(\fA^1\) of \(T^+\). In this way we find
two elementary chains \((\fA_i)\) and \((\fB_i)\), which have a common union
\(\fC\). Then \(\fA_0\prec\fC\) and \(\fB_0\prec\fC\) implies
\(\fA_0\equiv\fB_0\) since \(T\) are all sentences. Thus \(\fA_0\) is a model of \(T^*\)
\end{proof}
\subsubsection{Existentially closed structures and the Kaiser hull}
\label{sec:orgc6eea9b}
Let \(T\) be an \(L\)-theory. It follows from \ref{lemma3.1.2} that the models
of \(T_\forall\) are the substructures of models of \(T\). The conditions
(1) and (2) in the definition of "model companion" can therefore be
expressed as
\begin{equation*}
T_{\forall}=T_{\forall}^*
\end{equation*}
Hence the model companion of a theory \(T\) depends only on \(T_{\forall}\).
(Note that \(T_{\forall}\) is model complete)

\begin{definition}[]
An \(L\)-structure \(\fA\) is called \textbf{\(T\)-existentiallay closed} (or
\textbf{\(T\)-ec}) if
\begin{enumerate}
\item \(\fA\) can be embedded in a model of \(T\)
\item \(\fA\) is existentially closed in every extension which is a model of \(T\)
\end{enumerate}
\end{definition}

A structure \(\fA\) is \(T\)-ec exactly if it is \(T_{\forall}\)-ec. Since
every model of \(\fB\) of \(T_{\forall}\) can be embedded in a model \(\fM\)
of \(T\) and \(\fA\subseteq\fB\subseteq\fM\) and \(\fA\prec_1\fM\) implies \(\fA\prec_1\fB\)

\begin{lemma}[]
\label{lemma3.2.11}
Every model of a theory \(T\) can be embedded in a \(T\)-ec structure
\end{lemma}

\begin{proof}
Let \(\fA\) be a model of \(T_{\forall}\). We choose an enumeration
\((\varphi_\alpha)_{\alpha<\kappa}\) of all existential \(L(A)\)-sentences and
construct an ascending chain \((\fA_\alpha)_{\alpha\le\kappa}\) of models of
\(T_{\forall}\). We begin with \(\fA_0=\fA\). Let \(\fA_\alpha\) be
constructed. If \(\varphi_\alpha\) holds in an extension of \(\fA_\alpha\)
which is a model of \(T\) we let \(\fA_{\alpha+1}\) be such a model.
Otherwise we set \(\fA_{\alpha+1}=\fA_{\alpha}\). For limit ordinals \(\lambda\) we define
\(\fA_\lambda\) to be the union of all \(\fA_\alpha\). \(\fA_\lambda\) is
again a model of \(T_{\forall}\)
\end{proof}

Every elementary substructure \(\fN\) of a \(T\)-ec structure \(\fM\) is
again \(T\)-ec. Let \(\fN\subseteq\fA\) be a model of \(T\). Since
\(\fM_N\Rightarrow_{\exists}\fA_N\), there is an embedding of \(\fM\) in an
elementary extension \(\fB\) of \(\fA\) which is the identity on \(N\).
Since \(\fM\) is existentially closed in \(\fB\), it follows that \(\fN\) is
existentially closed in \(\fB\) and therefore also in \(\fA\)

\begin{center}\begin{tikzcd}
&\fB&\\
\fA\arrow[ur,"\prec"]&&\fM\arrow[ul,"\prec_1"']\\
&\fN\arrow[ul,"\prec_1"]\arrow[ur,"\prec"']&
\end{tikzcd}\end{center}

\begin{lemma}[]
Let \(T\) be a theory. Then there is a biggest inductive theory \(T^{\KH}\)
with \(T_{\forall}=T_{\forall}^{\KH}\). We call \(T^{\KH}\) the \textbf{Kaiser hull}
of \(T\)
\end{lemma}

\begin{proof}
Let \(T^1\) and \(T^2\) be two inductive theories with
\(T_{\forall}^1=T_{\forall}^2=T_{\forall}\). We have to show that \((T^1\cup
    T^2)_\forall=T_\forall\).  Let \(\fM\) be a model of \(T\), as in the proof
of \ref{thm3.2.9} we extend \(\fM\) by a chain
\(\fA_0\subseteq\fB_0\subseteq\fA_1\subseteq\fB_1\subseteq\cdots\) of models
of \(T^1\) and \(T^2\). The union of this chain is a model of \(T^1\cup
    T^2\)

(Both of \(T_{\forall}^1\) and \(T_{\forall}^2\) and model companion and
hence equivalent)
\end{proof}

\begin{lemma}[]
\label{lemma3.2.13}
The Kaiser hull \(T^{KH}\) is the \(\forall\exists\)-part of the theory of
all \(T\)-ec structures
\end{lemma}

\begin{proof}
Let \(T^*\) be the \(\forall\exists\)-part of the theory of all \(T\)-ec
structures. Since \(T\)-ec structures are models of \(T_{\forall}\), we have
\(T_\forall\subseteq T^*_\forall\). It follows from \ref{lemma3.2.11} that
\(T_\forall^*\subseteq T_\forall\). Hence \(T^*\) is contained in the Kaiser Hull.
\end{proof}

This implies that \(T\)-ec strctures are models of \(T_{\forall\exists}\)

\begin{theorem}[]
For any theory \(T\) the following are equivalent
\begin{enumerate}
\item \(T\) has a model companion \(T^*\)
\item All models of \(K^{\KH}\) are \(T\)-ec
\item The \(T\)-ec structures form an elementary class.
\end{enumerate}


If \(T^*\) exists, we have
\begin{equation*}
T^*=T^{\KH}=\text{ theory of all $T$-ec structures}
\end{equation*}
\end{theorem}


\begin{exercise}
\label{ex3.2.1}
Let \(L\) be the language containing a unary function \(f\) and a binary
relation symbol \(R\) and consider the \(L\)-theory \(T=\{\forall x\forall
    y(R(x,y)\to (R(x,f(y))))\}\). Showing the follow
\begin{enumerate}
\item For any \(T\)-structure \(\fM\) and \(a,b\in M\) with
\(b\not\in\{a,f^{\fM}(a),(f^{\fM})^2(a),\dots\}\) we have
\(\fM\models\exists z(R(z,a)\wedge\neg R(z,b))\)
\item Let \(\fM\) be a model of \(T\) and \(a\) an element of \(M\) s.t.
\(\{a,f^{\fM}(a),(f^{\fM})^2(a),\dots\}\) is infinite. Then in an
elementary extension \(\fM'\) there is an element \(b\) with
\(\fM'\models\forall z(R(z,a)\to R(z,b))\)
\item The class of \(T\)-ec structures is not elementary, so \(T\) does not
have a model companion
\end{enumerate}
\end{exercise}

\begin{exercise}
\label{ex3.2.3}
A theory \(T\) with quantifier elimination is axiomatisable by sentences of
the form
\begin{equation*}
\forall x_1\dots x_n\psi
\end{equation*}
where \(\psi\) is primitive existential formula
\end{exercise}
\subsection{Examples}
\label{sec:orgbfd58d0}
\textbf{Infinite sets}. The models of the theory  \(\Infset\) of \textbf{infinite sets} are all
infinite sets without additional structure. The language \(L_{\emptyset}\) is
empty, the axioms are (for \(n=1,2,\dots\))
\begin{itemize}
\item \(\exists x_0\dots x_{n-1}\bigwedge_{i<j<n}\neg x_i\dot{=}x_j\)
\end{itemize}
\begin{theorem}
The theory \(\Infset\) of infinite sets has quantifier elimination and is complete
\end{theorem}

\begin{proof}
Since the language is empty, the only basic formula is \(x_i=x_j\) and
\(\neg(x_i=x_j)\). By Lemma \ref{lemma3.2.5} we only need to consider primitive
existential formulas. 
\end{proof}

\textbf{Dense linear orderings}.
\begin{align*}
&\forall a,b(a\le b\wedge b\le a\to a\dot{=}b)\\
&\forall a,b,c(a\le b\wedge b\le c\to a\le c)\\
&\forall a,b(a\le b\vee b\le a)\\
&\forall a,b\exists c(a< b\to a< c< b)
\end{align*}
\begin{theorem}[]
\(\DLO\) has quantifier elimination
\end{theorem}

\begin{proof}
Let \(A\) be a finite common substructure of the two models \(O_1\) and
\(O_2\). We choose an ascending enumeration \(A=\{a_1,\dots,a_n\}\). Let
\(\exists y\rho (y)\) be a simple existential \(L(A)\)-sentence, which is
true in \(O_1\) and assume \(O_1\models\rho(b_1)\). We want to extend the
order preserving map \(a_i\mapsto a_i\) to an order preserving map
\(A\cup\{b_1\}\to O_2\). For this we have an image \(b_2\) of \(b_1\). There
are four cases
\begin{enumerate}
\item \(b_1\in A\), we set \(b_2=b_1\)
\item \(b_1\in(a_i,a_{i+1})\). We choose \(b_2\) in \(O_2\) with the same property
\item \(b_1\) is smaller than all elements of \(A\). We choose a \(b_2\in O_2\)
of the same kind
\item \(b_1\) is bigger than all \(a_i\). Choose \(b_2\) in the same manner
\end{enumerate}


This defines an isomorphism \(A\cup\{b_1\}\to A\cup\{b_2\}\), which show that \(O_2\models\rho(b_2)\)
\end{proof}

\textbf{Modules}. Let \(R\) be a (possibly non-commutative) ring with 1. An
\(R\)-module
\begin{equation*}
\fM=(,0,+,-,r)_{r\in R}
\end{equation*}
is an abelian group \((M,0,+,-)\) together with operations \(r:M\to M\) for
every ring element \(r\in R\). We formulate the axioms in the language
\(L_{Mod}(R)=L_{AbG}\cup\{r\mid r\in R\}\). The theory \(\sfMod(R)\) of
\(R\)-modules consists of
\begin{align*}
&\AbG\\
&\forall x,y\; r(x+y)\dot{=}rx+ry\\
&\forall x\;(r+s)x\dot{=}rx+sx\\
&\forall x\;(rs)x\dot{=}r(sx)\\
&\forall x\;1x\dot{=}x
\end{align*}
for all \(r,s\in R\). Then \(\Infset\cup\sfMod(R)\) is the theory of all
infinite \(R\)-modules

A module over fields is a vector space

\begin{theorem}[]
Let \(K\) be a field. Then the theory of all infinite \(K\)-vector spaces has
quantifier elimination and is complete
\end{theorem}

\begin{proof}
Let \(A\) be a common finitely generated substructure (i.e., a subspace) of
the two infinite \(K\)-vector spaces \(V_1\) and \(V_2\). Let \(\exists
   y\rho(y)\) be a simple existential \(L(A)\)-sentence which holds in \(V_1\).
Choose a \(b_1\) from \(V_1\) which satisfies \(\rho(y)\). If \(b_1\) belongs to
\(A\), we finished. If not, we choose a \(b_2\in V_2\setminus A\). Possibly
we have to replace \(V_2\) by an elementary extension. The vector spaces
\(A+Kb_1\) and \(A+Kb_2\) are isomorphic by an isomophism which maps \(b_1\)
to \(b_2\) and fixes \(A\) elementwise. Hence \(V_2\models\rho(b_2)\)

The theory is complete since a quantifier-free sentence is true in a vector
space iff it is true in the zero-vector space.
\end{proof}

\begin{definition}[]
An \textbf{equation} is an \(L_{Mod}(R)\)-formula \(\gamma(\bbar{x})\) of the form
\begin{equation*}
r_1x_1+\dots+r_mx_m=0
\end{equation*}
A \textbf{positive primitive} formula (\textbf{pp}-formula) is of the form
\begin{equation*}
\exists\bbar{y}(\gamma_1\wedge\dots\wedge\gamma_n)
\end{equation*}
where the \(\gamma_i(\bbar{xy})\) are equations
\end{definition}

\begin{theorem}[]
For every ring \(R\) and any \(R\)-module \(M\), every \(L_{Mod}(R)\)-formula
is equivalent (modulo the theory of \(M\)) to a Boolean combination of
positive primitive formulas
\end{theorem}

\textbf{Algebraically closed fields}.
\begin{theorem}[Tarski]
The theory \(\sfACF\) of algebraically closed fields has quantifier elimination
\end{theorem}

\begin{proof}
Let \(K_1\) and \(K_2\) be two algebraically closed fields and \(R\) a common
subring. Let \(\exists y\rho(y)\) be a simple existential sentence with
parameters in \(R\) which hold in \(K_1\). We have to show that \(\exists
   y\rho(y)\) is also true in \(K_2\).

Let \(F_1\) and \(F_2\) be the quotient fields of \(R\) in \(K_1\) and
\(K_2\), and let \(f:F_1\to F_2\) be an isomorphism which is the identity on
\(R\). Then \(f\) extends to an isomorphism \(g:G_1\to G_2\) between the
relative algebraic closures \(G_i\) of \(F_i\) in \(K_i\).
\end{proof}
\section{Countable Models}
\label{sec:orgaba7093}
\subsection{The omitting types theorem}
\label{sec:org6b60a51}
\begin{definition}[]
Let \(T\) be an \(L\)-theory and \(\Sigma(x)\) a set of \(L\)-formulas. A model
\(\fA\) of \(T\) not realizing \(\Sigma(x)\) is said to \textbf{omit} \(\Sigma(x)\). A
formula \(\varphi(x)\) \textbf{isolates} \(\Sigma(x)\) if
\begin{enumerate}
\item \(\varphi(x)\) is consistent with \(T\)
\item \(T\models\forall x(\varphi(x)\to\sigma(x))\) for all \(\sigma(x)\in\Sigma(x)\)
\end{enumerate}
\end{definition}

A set of formulas is often called a \textbf{partial type}.

\begin{theorem}[Omitting Types]
If \(T\) is countable and consistent and if \(\Sigma(x)\) is not isolated in
\(T\), then \(T\) has a model which omits \(\Sigma(x)\)
\end{theorem}

If \(\Sigma(x)\) is isolated by \(\varphi(x)\) and \(\fA\) is a model of \(T\), then
\(\Sigma(x)\) is realised in \(\fA\) by all realisations \(\varphi(x)\). Therefore the
converse of the theorem is true for \textbf{complete} theories \(T\): if \(\Sigma(x)\) is
isolated in \(T\), then it is realised in every model of \(T\) 

\begin{proof}
We choose a countable set \(C\) of new constants and extend \(T\) to a theory
\(T^*\) with the following properties
\begin{enumerate}
\item \(T^*\) is a Henkin theory: for all \(L(C)\)-formulas \(\psi(x)\) there
exists a constant \(c\in C\) with \(\exists x\psi(x)\to\psi(c)\in T^*\)
\item for all \(c\in C\) there is a \(\sigma(x)\in\Sigma(x)\) with \(\neg\sigma(c)\in
      T^*\)
\end{enumerate}


We construct \(T^*\) inductively as the union of an ascending chain
\begin{equation*}
T=T_0\subseteq T_1\subseteq T_1\subseteq\dots
\end{equation*}
of consistent extensions of \(T\) by finitely many axioms from \(L(C)\), in
each step making an instance of (1) or (2) true.

Enumerate \(C=\{c_i\mid i<\omega\}\) and let \(\{\psi_i(x)\mid i<\omega\}\)
be an enumeration of the \(L(C)\)-formulas

Assume that \(T_{2i}\) is the already constructed. Choose some \(c\in C\)
which doesn't occur in \(T_{2i}\cup\{\psi_i(x)\}\) and set
\(T_{2i+1}=T_{2i}\cup\{\exists x\psi_i(x)\to\psi_i(c)\}\).

Up to equivalence \(T_{2i+1}\) has the form \(T\cup\{\delta(c_i,\bbar{c})\}\) for
an \(L\)-formula \(\delta(x,\bbar{y})\) and a tuple \(\bbar{c}\in C\) which
doesn't contain \(c_i\). Since \(\exists\bbar{y}\delta(x,\bbar{y})\) doesn't
isolate \(\Sigma(x)\), for some \(\sigma\in\Sigma\) the formula
\(\exists\bbar{y}\delta(x,\bbar{y})\wedge\neg\sigma(x)\) is consistent with \(T\).
Thus \(T_{2i+2}=T_{2i+1}\cup\{\neg\sigma(c_i)\}\) is consistent

Take a model \((\fA',a_c)_{c\in C}\) of \(T^*\). Since \(T^*\) is a Henkin
theory, Tarski's Test \ref{thm2.1.2} shows that \(A=\{a_c\mid c\in C\}\) is the
universe of an elementary substructure \(\fA\) (Lemma \ref{lemma2.2.3}). By
property (2), \(\Sigma(x)\) is omitted in \(\fA\)
\end{proof}

\begin{corollary}[]
Let \(T\) be countable and consistent and let
\begin{equation*}
\Sigma_0(x_0,\dots,x_{n_0}),\Sigma_1(x_1,\dots,x_{n_1}),\dots
\end{equation*}
be a sequence of partial types. If all \(\Sigma_i\) are not isolated, then
\(T\) has a model which omits all \(\Sigma_i\)
\end{corollary}

\begin{proof}
If \(\Sigma_0(x),\Sigma_1(x),\dots\). Then
\(T_{2i+2}=T_{2i+1}\cup\{\neg\sigma_m(c_{mn})\}\)

If \(\Sigma(x_1,\dots,x_n)\), then
\(T_{2i+1}=T_{2i}\cup\{\exists\bbar{x}\psi_i(\bbar{x})\to\psi_i(\bbar{c})\}\).

Combine the two case
\end{proof}
\subsection{The space of types}
\label{sec:org81fdc22}
Fix a theory \(T\). An \textbf{\(n\)-type} is a maximal set of formulas
\(p(x_1,\dots,x_n)\) consistent with \(T\). We denote by \(S_n(T)\) the set
of all \(n\)-types of \(T\). We also write \(S(T)\) for \(S_1(T)\).
\(S_0(T)\) is all complete extensions of \(T\)

If \(B\) is a subset of an \(L\)-structure \(\fA\), we recover
\(S_n^{\fA}(B)\) as \(S_n(\Th(\fA_B))\). In particular, if \(T\) is complete
and \(\fA\) is any model of \(T\), we have \(S^{\fA}(\emptyset)=S(T)\)

For any \(L\)-formula \(\varphi(x_1,\dots,x_n)\), let \([\varphi]\) denote the set of all
types containing \(\varphi\).







\section{{\bfseries\sffamily TODO} Don't understand}
\label{sec:org4d58981}
Lemma \ref{lemma3.2.13}

Exercise \ref{ex3.2.3}
\end{document}