% Created 2020-10-30 五 15:09
% Intended LaTeX compiler: pdflatex
\documentclass[11pt]{article}
\usepackage[utf8]{inputenc}
\usepackage[T1]{fontenc}
\usepackage{graphicx}
\usepackage{grffile}
\usepackage{longtable}
\usepackage{wrapfig}
\usepackage{rotating}
\usepackage[normalem]{ulem}
\usepackage{amsmath}
\usepackage{textcomp}
\usepackage{amssymb}
\usepackage{capt-of}
\usepackage{hyperref}
%%%%%%%%%%%%%%%%%%%%%%%%%%%%%%%%%%%%%%
%% TIPS                                 %%
%%%%%%%%%%%%%%%%%%%%%%%%%%%%%%%%%%%%%%
% \substack{a\\b} for multiple lines text

\usepackage[utf8]{inputenc}

\usepackage[B1,T1]{fontenc}

% pdfplots will load xolor automatically without option
\usepackage[dvipsnames]{xcolor}
%%%%%%%%%%%%%%%%%%%%%%%%%%%%%%%%%%%%%%%
%% MATH related pacakge                  %%
%%%%%%%%%%%%%%%%%%%%%%%%%%%%%%%%%%%%%%%
% \usepackage{amsmath} mathtools loads the amsmath
\usepackage{amsmath}
\usepackage{mathtools}


\usepackage{amsthm}
\usepackage{amsbsy}

%\usepackage{commath}

\usepackage{amssymb}
\usepackage{mathrsfs}
%\usepackage{mathabx}
\usepackage{stmaryrd}
\usepackage{empheq}

\usepackage{scalerel}
\usepackage{stackengine}
\usepackage{stackrel}

\usepackage{nicematrix}
\usepackage{tensor}
\usepackage{blkarray}
\usepackage{siunitx}
\usepackage[f]{esvect}

\usepackage{unicode-math}
\setmainfont{TeX Gyre Pagella}
% \setmathfont{STIX}
% \setmathfont{texgyrepagella-math.otf}
% \setmathfont{Libertinus Math}
\setmathfont{Latin Modern Math}
\setmathfont[range={\mscra,\mscrb,\mscrc,\mscrd,\mscre,\mscrf,\mscrg,\mscrh,\mscri,\mscrj,\mscrk,\mscrl,\mscrm,\mscrn,\mscro,\mscrp,\mscrq,\mscrr,\mscrs,\mscrt,\mscru,\mscrv,\mscrw,\mscrx,\mscry,\mscrz,\mscrA,\mscrB,\mscrC,\mscrD,\mscrE,\mscrF,\mscrG,\mscrH,\mscrI,\mscrJ,\mscrK,\mscrL,\mscrM,\mscrN,\mscrO,\mscrP,\mscrQ,\mscrR,\mscrS,\mscrT,\mscrU,\mscrV,\mscrW,\mscrX,\mscrY,\mscrZ}]{Latin Modern Math}
\setmathfont[range={\smwhtdiamond,\enclosediamond,\varlrtriangle}]{Latin Modern Math}
\setmathfont[range={\rightrightarrows,\twoheadrightarrow,\leftrightsquigarrow,\triangledown}]{XITS Math}
\setmathfont[range={\int,\setminus}]{Libertinus Math}



%%%%%%%%%%%%%%%%%%%%%%%%%%%%%%%%%%%%%%%
%% TIKZ related packages                 %%
%%%%%%%%%%%%%%%%%%%%%%%%%%%%%%%%%%%%%%%

\usepackage{pgfplots}
\pgfplotsset{compat=1.15}
\usepackage{tikz}
\usepackage{tikz-cd}
\usepackage{tikz-qtree}

\usetikzlibrary{arrows,positioning,calc,fadings,decorations,matrix,decorations,shapes.misc}
%setting from geogebra
\definecolor{ccqqqq}{rgb}{0.8,0,0}


%%%%%%%%%%%%%%%%%%%%%%%%%%%%%%%%%%%%%%%
%% MISCLELLANEOUS packages               %%
%%%%%%%%%%%%%%%%%%%%%%%%%%%%%%%%%%%%%%%
\usepackage[most]{tcolorbox}
\usepackage{threeparttable}
\usepackage{tabularx}

\usepackage{enumitem}

% wrong with preview
\usepackage{subcaption}
\usepackage{caption}
% {\aunclfamily\Huge}
\usepackage{auncial}

\usepackage{float}

\usepackage{fancyhdr}

\usepackage{ifthen}
\usepackage{xargs}


\usepackage{imakeidx}
\usepackage{hyperref}
\usepackage{soul}


%\usepackage[xetex]{preview}
%%%%%%%%%%%%%%%%%%%%%%%%%%%%%%%%%%%%%%%
%% USEPACKAGES end                       %%
%%%%%%%%%%%%%%%%%%%%%%%%%%%%%%%%%%%%%%%

% \setlist{nosep}
% \numberwithin{equation}{subsection}
% \fancyhead{} % Clear the headers
% \renewcommand{\headrulewidth}{0pt} % Width of line at top of page
% \fancyhead[R]{\slshape\leftmark} % Mark right [R] of page with Chapter name [\leftmark]
% \pagestyle{fancy} % Set default style for all content pages (not TOC, etc)


% \newlength\shlength
% \newcommand\vect[2][0]{\setlength\shlength{#1pt}%
%   \stackengine{-5.6pt}{$#2$}{\smash{$\kern\shlength%
%     \stackengine{7.55pt}{$\mathchar"017E$}%
%       {\rule{\widthof{$#2$}}{.57pt}\kern.4pt}{O}{r}{F}{F}{L}\kern-\shlength$}}%
%       {O}{c}{F}{T}{S}}


\indexsetup{othercode=\small}
\makeindex[columns=2,options={-s /media/wu/file/stuuudy/notes/index_style.ist},intoc]
\makeatletter
\def\@idxitem{\par\hangindent 0pt}
\makeatother


%\newcounter{dummy} \numberwithin{dummy}{section}
\newtheorem{dummy}{dummy}[section]
\theoremstyle{definition}
\newtheorem{definition}[dummy]{Definition}
\theoremstyle{plain}
\newtheorem{corollary}[dummy]{Corollary}
\newtheorem{lemma}[dummy]{Lemma}
\newtheorem{proposition}[dummy]{Proposition}
\newtheorem{theorem}[dummy]{Theorem}
\theoremstyle{definition}
\newtheorem{examplle}{Example}[section]
\theoremstyle{remark}
\newtheorem*{remark}{Remark}
\newtheorem{exercise}{Exercise}[subsection]
\newtheorem{observation}{Observation}[section]


\newenvironment{claim}[1]{\par\noindent\textbf{Claim:}\space#1}{}

\makeatletter
\DeclareFontFamily{U}{tipa}{}
\DeclareFontShape{U}{tipa}{m}{n}{<->tipa10}{}
\newcommand{\arc@char}{{\usefont{U}{tipa}{m}{n}\symbol{62}}}%

\newcommand{\arc}[1]{\mathpalette\arc@arc{#1}}

\newcommand{\arc@arc}[2]{%
  \sbox0{$\m@th#1#2$}%
  \vbox{
    \hbox{\resizebox{\wd0}{\height}{\arc@char}}
    \nointerlineskip
    \box0
  }%
}
\makeatother

\setcounter{MaxMatrixCols}{20}
%%%%%%% ABS
\DeclarePairedDelimiter\abss{\lvert}{\rvert}%
\DeclarePairedDelimiter\normm{\lVert}{\rVert}%

% Swap the definition of \abs* and \norm*, so that \abs
% and \norm resizes the size of the brackets, and the
% starred version does not.
\makeatletter
\let\oldabs\abss
%\def\abs{\@ifstar{\oldabs}{\oldabs*}}
\newcommand{\abs}{\@ifstar{\oldabs}{\oldabs*}}
\newcommand{\norm}[1]{\left\lVert#1\right\rVert}
%\let\oldnorm\normm
%\def\norm{\@ifstar{\oldnorm}{\oldnorm*}}
%\renewcommand{norm}{\@ifstar{\oldnorm}{\oldnorm*}}
\makeatother

% \newcommand\what[1]{\ThisStyle{%
%     \setbox0=\hbox{$\SavedStyle#1$}%
%     \stackengine{-1.0\ht0+.5pt}{$\SavedStyle#1$}{%
%       \stretchto{\scaleto{\SavedStyle\mkern.15mu\char'136}{2.6\wd0}}{1.4\ht0}%
%     }{O}{c}{F}{T}{S}%
%   }
% }

% \newcommand\wtilde[1]{\ThisStyle{%
%     \setbox0=\hbox{$\SavedStyle#1$}%
%     \stackengine{-.1\LMpt}{$\SavedStyle#1$}{%
%       \stretchto{\scaleto{\SavedStyle\mkern.2mu\AC}{.5150\wd0}}{.6\ht0}%
%     }{O}{c}{F}{T}{S}%
%   }
% }

% \newcommand\wbar[1]{\ThisStyle{%
%     \setbox0=\hbox{$\SavedStyle#1$}%
%     \stackengine{.5pt+\LMpt}{$\SavedStyle#1$}{%
%       \rule{\wd0}{\dimexpr.3\LMpt+.3pt}%
%     }{O}{c}{F}{T}{S}%
%   }
% }

\newcommand{\bl}[1] {\boldsymbol{#1}}
\newcommand{\Wt}[1] {\stackrel{\sim}{\smash{#1}\rule{0pt}{1.1ex}}}
\newcommand{\wt}[1] {\widetilde{#1}}
\newcommand{\tf}[1] {\textbf{#1}}


%For boxed texts in align, use Aboxed{}
%otherwise use boxed{}

\DeclareMathSymbol{\widehatsym}{\mathord}{largesymbols}{"62}
\newcommand\lowerwidehatsym{%
  \text{\smash{\raisebox{-1.3ex}{%
    $\widehatsym$}}}}
\newcommand\fixwidehat[1]{%
  \mathchoice
    {\accentset{\displaystyle\lowerwidehatsym}{#1}}
    {\accentset{\textstyle\lowerwidehatsym}{#1}}
    {\accentset{\scriptstyle\lowerwidehatsym}{#1}}
    {\accentset{\scriptscriptstyle\lowerwidehatsym}{#1}}
  }


\newcommand{\cupdot}{\mathbin{\dot{\cup}}}
\newcommand{\bigcupdot}{\mathop{\dot{\bigcup}}}

\usepackage{graphicx}

\usepackage[toc,page]{appendix}

% text on arrow for xRightarrow
\makeatletter
%\newcommand{\xRightarrow}[2][]{\ext@arrow 0359\Rightarrowfill@{#1}{#2}}
\makeatother

% Arbitrary long arrow
\newcommand{\Rarrow}[1]{%
\parbox{#1}{\tikz{\draw[->](0,0)--(#1,0);}}
}

\newcommand{\LRarrow}[1]{%
\parbox{#1}{\tikz{\draw[<->](0,0)--(#1,0);}}
}


\makeatletter
\providecommand*{\rmodels}{%
  \mathrel{%
    \mathpalette\@rmodels\models
  }%
}
\newcommand*{\@rmodels}[2]{%
  \reflectbox{$\m@th#1#2$}%
}
\makeatother







\newcommand{\trcl}[1]{%
  \mathrm{trcl}{(#1)}
}



% Roman numerals
\makeatletter
\newcommand*{\rom}[1]{\expandafter\@slowromancap\romannumeral #1@}
\makeatother
% \\def \\b\([a-zA-Z]\) {\\boldsymbol{[a-zA-z]}}
% \\DeclareMathOperator{\\b\1}{\\textbf{\1}}


\DeclareMathOperator{\bx}{\textbf{x}}
\DeclareMathOperator{\bz}{\textbf{z}}
\DeclareMathOperator{\bff}{\textbf{f}}
\DeclareMathOperator{\ba}{\textbf{a}}
\DeclareMathOperator{\bk}{\textbf{k}}
\DeclareMathOperator{\bs}{\textbf{s}}
\DeclareMathOperator{\bh}{\textbf{h}}
\DeclareMathOperator{\bc}{\textbf{c}}
\DeclareMathOperator{\br}{\textbf{r}}
\DeclareMathOperator{\bi}{\textbf{i}}
\DeclareMathOperator{\bj}{\textbf{j}}
\DeclareMathOperator{\bn}{\textbf{n}}
\DeclareMathOperator{\be}{\textbf{e}}
\DeclareMathOperator{\bo}{\textbf{o}}
\DeclareMathOperator{\bU}{\textbf{U}}
\DeclareMathOperator{\bL}{\textbf{L}}
\DeclareMathOperator{\bV}{\textbf{V}}
\def \bzero {\mathbf{0}}
\def \btwo {\mathbf{2}}
\DeclareMathOperator{\bv}{\textbf{v}}
\DeclareMathOperator{\bp}{\textbf{p}}
\DeclareMathOperator{\bI}{\textbf{I}}
\DeclareMathOperator{\bM}{\textbf{M}}
\DeclareMathOperator{\bN}{\textbf{N}}
\DeclareMathOperator{\bK}{\textbf{K}}
\DeclareMathOperator{\bt}{\textbf{t}}
\DeclareMathOperator{\bb}{\textbf{b}}
\DeclareMathOperator{\bA}{\textbf{A}}
\DeclareMathOperator{\bX}{\textbf{X}}
\DeclareMathOperator{\bu}{\textbf{u}}
\DeclareMathOperator{\bS}{\textbf{S}}
\DeclareMathOperator{\bZ}{\textbf{Z}}
\DeclareMathOperator{\by}{\textbf{y}}
\DeclareMathOperator{\bw}{\textbf{w}}
\DeclareMathOperator{\bT}{\textbf{T}}
\DeclareMathOperator{\bF}{\textbf{F}}
\DeclareMathOperator{\bmm}{\textbf{m}}
\DeclareMathOperator{\bW}{\textbf{W}}
\DeclareMathOperator{\bR}{\textbf{R}}
\DeclareMathOperator{\bC}{\textbf{C}}
\DeclareMathOperator{\bD}{\textbf{D}}
\DeclareMathOperator{\bE}{\textbf{E}}
\DeclareMathOperator{\bQ}{\textbf{Q}}
\DeclareMathOperator{\bP}{\textbf{P}}
\DeclareMathOperator{\bY}{\textbf{Y}}
\DeclareMathOperator{\bH}{\textbf{H}}
\DeclareMathOperator{\bB}{\textbf{B}}
\DeclareMathOperator{\bG}{\textbf{G}}
\def \blambda {\symbf{\lambda}}
\def \boldeta {\symbf{\eta}}
\def \balpha {\symbf{\alpha}}
\def \bbeta {\symbf{\beta}}
\def \bgamma {\symbf{\gamma}}
\def \bxi {\symbf{\xi}}
\def \bLambda {\symbf{\Lambda}}

\newcommand{\bto}{{\boldsymbol{\to}}}
\newcommand{\Ra}{\Rightarrow}
\newcommand\und[1]{\underline{#1}}
\def \bPhi {\boldsymbol{\Phi}}
\def \btheta {\boldsymbol{\theta}}
\def \bTheta {\boldsymbol{\Theta}}
\def \bmu {\boldsymbol{\mu}}
\def \bphi {\boldsymbol{\phi}}
\def \bSigma {\boldsymbol{\Sigma}}
\def \lb {\left\{}
\def \rb {\right\}}
\def \la {\langle}
\def \ra {\rangle}
\def \caln {\mathcal{N}}
\def \dissum {\displaystyle\Sigma}
\def \dispro {\displaystyle\prod}
\def \E {\mathbb{E}}
\def \Q {\mathbb{Q}}
\def \N {\mathbb{N}}
\def \V {\mathbb{V}}
\def \R {\mathbb{R}}
\def \P {\mathbb{P}}
\def \A {\mathbb{A}}
\def \F {\mathbb{F}}
\def \Z {\mathbb{Z}}
\def \I {\mathbb{I}}
\def \C {\mathbb{C}}
\def \cala {\mathcal{A}}
\def \cale {\mathcal{E}}
\def \calb {\mathcal{B}}
\def \calq {\mathcal{Q}}
\def \calp {\mathcal{P}}
\def \cals {\mathcal{S}}
\def \calx {\mathcal{X}}
\def \caly {\mathcal{Y}}
\def \calg {\mathcal{G}}
\def \cald {\mathcal{D}}
\def \caln {\mathcal{N}}
\def \calr {\mathcal{R}}
\def \calt {\mathcal{T}}
\def \calm {\mathcal{M}}
\def \calw {\mathcal{W}}
\def \calc {\mathcal{C}}
\def \calv {\mathcal{V}}
\def \calf {\mathcal{F}}
\def \calk {\mathcal{K}}
\def \call {\mathcal{L}}
\def \calu {\mathcal{U}}
\def \calo {\mathcal{O}}
\def \calh {\mathcal{H}}
\def \cali {\mathcal{I}}

\def \bcup {\bigcup}

% set theory

\def \zfcc {\textbf{ZFC}^-}
\def \ac  {\textbf{AC}}
\def \gl  {\textbf{L }}
\def \gll {\textbf{L}}
\newcommand{\zfm}{$\textbf{ZF}^-$}

%\def \zfm {$\textbf{ZF}^-$}
\def \zfmm {\textbf{ZF}^-}
\def \wf {\textbf{WF }}
\def \on {\textbf{On }}
\def \cm {\textbf{M }}
\def \cn {\textbf{N }}
\def \cv {\textbf{V }}
\def \zc {\textbf{ZC }}
\def \zcm {\textbf{ZC}}
\def \zff {\textbf{ZF}}
\def \wfm {\textbf{WF}}
\def \onm {\textbf{On}}
\def \cmm {\textbf{M}}
\def \cnm {\textbf{N}}
\def \cvm {\textbf{V}}
\def \gchh {\textbf{GCH}}
\renewcommand{\restriction}{\mathord{\upharpoonright}}
\def \pred {\text{pred}}

\def \rank {\text{rank}}
\def \con {\text{Con}}
\def \deff {\text{Def}}


\def \uin {\underline{\in}}
\def \oin {\overline{\in}}
\def \uR {\underline{R}}
\def \oR {\overline{R}}
\def \uP {\underline{P}}
\def \oP {\overline{P}}

\def \Ra {\Rightarrow}

\def \e {\enspace}

\def \sgn {\operatorname{sgn}}
\def \gen {\operatorname{gen}}
\def \Hom {\operatorname{Hom}}
\def \hom {\operatorname{hom}}
\def \Sub {\operatorname{Sub}}

\def \supp {\operatorname{supp}}

\def \epiarrow {\twoheadarrow}
\def \monoarrow {\rightarrowtail}
\def \rrarrow {\rightrightarrows}

% \def \minus {\text{-}}
% \newcommand{\minus}{\scalebox{0.75}[1.0]{$-$}}
% \DeclareUnicodeCharacter{002D}{\minus}


\def \tril {\triangleleft}

\def \ACF {\text{ACF}}
\def \GL {\text{GL}}
\def \PGL {\text{PGL}}
\def \equal {=}
\def \deg {\text{deg}}
\def \degree {\text{degree}}
\def \app {\text{App}}
\def \FV {\text{FV}}
\def \conv {\text{conv}}
\def \cont {\text{cont}}
\DeclareMathOperator{\cl}{\textbf{CL}}
\DeclareMathOperator{\sg}{sg}
\DeclareMathOperator{\trdeg}{trdeg}
\def \Ord {\text{Ord}}

\DeclareMathOperator{\cf}{cf}
\DeclareMathOperator{\zfc}{ZFC}

%\DeclareMathOperator{\Th}{Th}
%\def \th {\text{Th}}
% \newcommand{\th}{\text{Th}}
\DeclareMathOperator{\type}{type}
\DeclareMathOperator{\zf}{\textbf{ZF}}
\def \fa {\mathfrak{a}}
\def \fb {\mathfrak{b}}
\def \fc {\mathfrak{c}}
\def \fd {\mathfrak{d}}
\def \fe {\mathfrak{e}}
\def \ff {\mathfrak{f}}
\def \fg {\mathfrak{g}}
\def \fh {\mathfrak{h}}
%\def \fi {\mathfrak{i}}
\def \fj {\mathfrak{j}}
\def \fk {\mathfrak{k}}
\def \fl {\mathfrak{l}}
\def \fm {\mathfrak{m}}
\def \fn {\mathfrak{n}}
\def \fo {\mathfrak{o}}
\def \fp {\mathfrak{p}}
\def \fq {\mathfrak{q}}
\def \fr {\mathfrak{r}}
\def \fs {\mathfrak{s}}
\def \ft {\mathfrak{t}}
\def \fu {\mathfrak{u}}
\def \fv {\mathfrak{v}}
\def \fw {\mathfrak{w}}
\def \fx {\mathfrak{x}}
\def \fy {\mathfrak{y}}
\def \fz {\mathfrak{z}}
\def \fA {\mathfrak{A}}
\def \fB {\mathfrak{B}}
\def \fC {\mathfrak{C}}
\def \fD {\mathfrak{D}}
\def \fE {\mathfrak{E}}
\def \fF {\mathfrak{F}}
\def \fG {\mathfrak{G}}
\def \fH {\mathfrak{H}}
\def \fI {\mathfrak{I}}
\def \fJ {\mathfrak{J}}
\def \fK {\mathfrak{K}}
\def \fL {\mathfrak{L}}
\def \fM {\mathfrak{M}}
\def \fN {\mathfrak{N}}
\def \fO {\mathfrak{O}}
\def \fP {\mathfrak{P}}
\def \fQ {\mathfrak{Q}}
\def \fR {\mathfrak{R}}
\def \fS {\mathfrak{S}}
\def \fT {\mathfrak{T}}
\def \fU {\mathfrak{U}}
\def \fV {\mathfrak{V}}
\def \fW {\mathfrak{W}}
\def \fX {\mathfrak{X}}
\def \fY {\mathfrak{Y}}
\def \fZ {\mathfrak{Z}}

\def \sfA {\textsf{A}}
\def \sfB {\textsf{B}}
\def \sfC {\textsf{C}}
\def \sfD {\textsf{D}}
\def \sfE {\textsf{E}}
\def \sfF {\textsf{F}}
\def \sfG {\textsf{G}}
\def \sfH {\textsf{H}}
\def \sfI {\textsf{I}}
\def \sfj {\textsf{J}}
\def \sfK {\textsf{K}}
\def \sfL {\textsf{L}}
\def \sfM {\textsf{M}}
\def \sfN {\textsf{N}}
\def \sfO {\textsf{O}}
\def \sfP {\textsf{P}}
\def \sfQ {\textsf{Q}}
\def \sfR {\textsf{R}}
\def \sfS {\textsf{S}}
\def \sfT {\textsf{T}}
\def \sfU {\textsf{U}}
\def \sfV {\textsf{V}}
\def \sfW {\textsf{W}}
\def \sfX {\textsf{X}}
\def \sfY {\textsf{Y}}
\def \sfZ {\textsf{Z}}
\def \sfa {\textsf{a}}
\def \sfb {\textsf{b}}
\def \sfc {\textsf{c}}
\def \sfd {\textsf{d}}
\def \sfe {\textsf{e}}
\def \sff {\textsf{f}}
\def \sfg {\textsf{g}}
\def \sfh {\textsf{h}}
\def \sfi {\textsf{i}}
\def \sfj {\textsf{j}}
\def \sfk {\textsf{k}}
\def \sfl {\textsf{l}}
\def \sfm {\textsf{m}}
\def \sfn {\textsf{n}}
\def \sfo {\textsf{o}}
\def \sfp {\textsf{p}}
\def \sfq {\textsf{q}}
\def \sfr {\textsf{r}}
\def \sfs {\textsf{s}}
\def \sft {\textsf{t}}
\def \sfu {\textsf{u}}
\def \sfv {\textsf{v}}
\def \sfw {\textsf{w}}
\def \sfx {\textsf{x}}
\def \sfy {\textsf{y}}
\def \sfz {\textsf{z}}



%\DeclareMathOperator{\ker}{ker}
\DeclareMathOperator{\im}{im}

\DeclareMathOperator{\inn}{Inn}
\DeclareMathOperator{\AC}{\textbf{AC}}
\DeclareMathOperator{\cod}{cod}
\DeclareMathOperator{\dom}{dom}
\DeclareMathOperator{\ran}{ran}
\DeclareMathOperator{\textd}{d}
\DeclareMathOperator{\td}{d}
\DeclareMathOperator{\id}{id}
\DeclareMathOperator{\LT}{LT}
\DeclareMathOperator{\Mat}{Mat}
\DeclareMathOperator{\Eq}{Eq}
\DeclareMathOperator{\irr}{irr}
\DeclareMathOperator{\Fr}{Fr}
\DeclareMathOperator{\Gal}{Gal}
\DeclareMathOperator{\lcm}{lcm}
\DeclareMathOperator{\alg}{\text{alg}}
\DeclareMathOperator{\Th}{Th}

\DeclareMathOperator{\DAG}{DAG}
\DeclareMathOperator{\ODAG}{ODAG}

% \varprod
\DeclareSymbolFont{largesymbolsA}{U}{txexa}{m}{n}
\DeclareMathSymbol{\varprod}{\mathop}{largesymbolsA}{16}
% \DeclareMathSymbol{\tonm}{\boldsymbol{\to}\textbf{Nm}}
\def \tonm {\bto\textbf{Nm}}
\def \tohm {\bto\textbf{Hm}}

% Category theory
\DeclareMathOperator{\Ab}{\textbf{Ab}}
\DeclareMathOperator{\Alg}{\textbf{Alg}}
\DeclareMathOperator{\Rng}{\textbf{Rng}}
\DeclareMathOperator{\Sets}{\textbf{Sets}}
\DeclareMathOperator{\Met}{\textbf{Met}}
\DeclareMathOperator{\Aut}{\textbf{Aut}}
\DeclareMathOperator{\RMod}{R-\textbf{Mod}}
\DeclareMathOperator{\RAlg}{R-\textbf{Alg}}
\DeclareMathOperator{\LF}{LF}
\DeclareMathOperator{\op}{op}
% Model theory
\DeclareMathOperator{\tp}{tp}
\DeclareMathOperator{\Diag}{Diag}
\DeclareMathOperator{\el}{el}
\DeclareMathOperator{\depth}{depth}
\DeclareMathOperator{\FO}{FO}
\DeclareMathOperator{\fin}{fin}
\DeclareMathOperator{\qr}{qr}
\DeclareMathOperator{\Mod}{Mod}
\DeclareMathOperator{\TC}{TC}
\DeclareMathOperator{\KH}{KH}
\DeclareMathOperator{\Part}{Part}
\DeclareMathOperator{\Infset}{\textsf{Infset}}
\DeclareMathOperator{\DLO}{\textsf{DLO}}
\DeclareMathOperator{\sfMod}{\textsf{Mod}}
\DeclareMathOperator{\AbG}{\textsf{AbG}}
\DeclareMathOperator{\sfACF}{\textsf{ACF}}
% Computability Theorem
\DeclareMathOperator{\Tot}{Tot}
\DeclareMathOperator{\graph}{graph}
\DeclareMathOperator{\Fin}{Fin}
\DeclareMathOperator{\Cof}{Cof}
\DeclareMathOperator{\lh}{lh}
% Commutative Algebra
\DeclareMathOperator{\ord}{ord}
\DeclareMathOperator{\Idem}{Idem}
\DeclareMathOperator{\zdiv}{z.div}
\DeclareMathOperator{\Frac}{Frac}
\DeclareMathOperator{\rad}{rad}
\DeclareMathOperator{\nil}{nil}
\DeclareMathOperator{\Ann}{Ann}
\DeclareMathOperator{\End}{End}
\DeclareMathOperator{\coim}{coim}
\DeclareMathOperator{\coker}{coker}
\DeclareMathOperator{\Bil}{Bil}
\DeclareMathOperator{\Tril}{Tril}
% Topology
\newcommand{\interior}[1]{%
  {\kern0pt#1}^{\mathrm{o}}%
}

% \makeatletter
% \newcommand{\vect}[1]{%
%   \vbox{\m@th \ialign {##\crcr
%   \vectfill\crcr\noalign{\kern-\p@ \nointerlineskip}
%   $\hfil\displaystyle{#1}\hfil$\crcr}}}
% \def\vectfill{%
%   $\m@th\smash-\mkern-7mu%
%   \cleaders\hbox{$\mkern-2mu\smash-\mkern-2mu$}\hfill
%   \mkern-7mu\raisebox{-3.81pt}[\p@][\p@]{$\mathord\mathchar"017E$}$}

% \newcommand{\amsvect}{%
%   \mathpalette {\overarrow@\vectfill@}}
% \def\vectfill@{\arrowfill@\relbar\relbar{\raisebox{-3.81pt}[\p@][\p@]{$\mathord\mathchar"017E$}}}

% \newcommand{\amsvectb}{%
% \newcommand{\vect}{%
%   \mathpalette {\overarrow@\vectfillb@}}
% \newcommand{\vecbar}{%
%   \scalebox{0.8}{$\relbar$}}
% \def\vectfillb@{\arrowfill@\vecbar\vecbar{\raisebox{-4.35pt}[\p@][\p@]{$\mathord\mathchar"017E$}}}
% \makeatother
% \bigtimes

\DeclareFontFamily{U}{mathx}{\hyphenchar\font45}
\DeclareFontShape{U}{mathx}{m}{n}{
      <5> <6> <7> <8> <9> <10>
      <10.95> <12> <14.4> <17.28> <20.74> <24.88>
      mathx10
      }{}
\DeclareSymbolFont{mathx}{U}{mathx}{m}{n}
\DeclareMathSymbol{\bigtimes}{1}{mathx}{"91}
% \odiv
\DeclareFontFamily{U}{matha}{\hyphenchar\font45}
\DeclareFontShape{U}{matha}{m}{n}{
      <5> <6> <7> <8> <9> <10> gen * matha
      <10.95> matha10 <12> <14.4> <17.28> <20.74> <24.88> matha12
      }{}
\DeclareSymbolFont{matha}{U}{matha}{m}{n}
\DeclareMathSymbol{\odiv}         {2}{matha}{"63}


\newcommand\subsetsim{\mathrel{%
  \ooalign{\raise0.2ex\hbox{\scalebox{0.9}{$\subset$}}\cr\hidewidth\raise-0.85ex\hbox{\scalebox{0.9}{$\sim$}}\hidewidth\cr}}}
\newcommand\simsubset{\mathrel{%
  \ooalign{\raise-0.2ex\hbox{\scalebox{0.9}{$\subset$}}\cr\hidewidth\raise0.75ex\hbox{\scalebox{0.9}{$\sim$}}\hidewidth\cr}}}

\newcommand\simsubsetsim{\mathrel{%
  \ooalign{\raise0ex\hbox{\scalebox{0.8}{$\subset$}}\cr\hidewidth\raise1ex\hbox{\scalebox{0.75}{$\sim$}}\hidewidth\cr\raise-0.95ex\hbox{\scalebox{0.8}{$\sim$}}\cr\hidewidth}}}
\newcommand{\stcomp}[1]{{#1}^{\mathsf{c}}}

\setlength{\baselineskip}{0.8in}

\stackMath
\newcommand\yrightarrow[2][]{\mathrel{%
  \setbox2=\hbox{\stackon{\scriptstyle#1}{\scriptstyle#2}}%
  \stackunder[0pt]{%
    \xrightarrow{\makebox[\dimexpr\wd2\relax]{$\scriptstyle#2$}}%
  }{%
   \scriptstyle#1\,%
  }%
}}
\newcommand\yleftarrow[2][]{\mathrel{%
  \setbox2=\hbox{\stackon{\scriptstyle#1}{\scriptstyle#2}}%
  \stackunder[0pt]{%
    \xleftarrow{\makebox[\dimexpr\wd2\relax]{$\scriptstyle#2$}}%
  }{%
   \scriptstyle#1\,%
  }%
}}
\newcommand\yRightarrow[2][]{\mathrel{%
  \setbox2=\hbox{\stackon{\scriptstyle#1}{\scriptstyle#2}}%
  \stackunder[0pt]{%
    \xRightarrow{\makebox[\dimexpr\wd2\relax]{$\scriptstyle#2$}}%
  }{%
   \scriptstyle#1\,%
  }%
}}
\newcommand\yLeftarrow[2][]{\mathrel{%
  \setbox2=\hbox{\stackon{\scriptstyle#1}{\scriptstyle#2}}%
  \stackunder[0pt]{%
    \xLeftarrow{\makebox[\dimexpr\wd2\relax]{$\scriptstyle#2$}}%
  }{%
   \scriptstyle#1\,%
  }%
}}

\newcommand\altxrightarrow[2][0pt]{\mathrel{\ensurestackMath{\stackengine%
  {\dimexpr#1-7.5pt}{\xrightarrow{\phantom{#2}}}{\scriptstyle\!#2\,}%
  {O}{c}{F}{F}{S}}}}
\newcommand\altxleftarrow[2][0pt]{\mathrel{\ensurestackMath{\stackengine%
  {\dimexpr#1-7.5pt}{\xleftarrow{\phantom{#2}}}{\scriptstyle\!#2\,}%
  {O}{c}{F}{F}{S}}}}

\newenvironment{bsm}{% % short for 'bracketed small matrix'
  \left[ \begin{smallmatrix} }{%
  \end{smallmatrix} \right]}

\newenvironment{psm}{% % short for ' small matrix'
  \left( \begin{smallmatrix} }{%
  \end{smallmatrix} \right)}

\newcommand{\bbar}[1]{\mkern 1.5mu\overline{\mkern-1.5mu#1\mkern-1.5mu}\mkern 1.5mu}

\newcommand{\bigzero}{\mbox{\normalfont\Large\bfseries 0}}
\newcommand{\rvline}{\hspace*{-\arraycolsep}\vline\hspace*{-\arraycolsep}}

\font\zallman=Zallman at 40pt
\font\elzevier=Elzevier at 40pt

\newcommand\isoto{\stackrel{\textstyle\sim}{\smash{\longrightarrow}\rule{0pt}{0.4ex}}}
\newcommand\embto{\stackrel{\textstyle\prec}{\smash{\longrightarrow}\rule{0pt}{0.4ex}}}
\setcounter{secnumdepth}{2}
\setcounter{tocdepth}{2}
\author{Katin Tent \& Martin Ziegler}
\date{\today}
\title{A Course in Model Theory}
\hypersetup{
 pdfauthor={Katin Tent \& Martin Ziegler},
 pdftitle={A Course in Model Theory},
 pdfkeywords={},
 pdfsubject={},
 pdfcreator={Emacs 27.1 (Org mode 9.3)}, 
 pdflang={English}}
\begin{document}

\maketitle
\tableofcontents


\section{The Basics}
\label{sec:org27109c0}

\subsection{Structures}
\label{sec:org2cd4d4f}
\begin{definition}[]
Let \(\fA,\fB\) be \(L\)-structures. A map \(h:A\to B\) is called a
\textbf{homomorphism} if for all \(a_1,\dots,a_n\in A\)
\begin{equation*}
 \begin{array}{rcl}
 h(c^{\fA})&=&c^{\fB}\\
 h(f^{\fA}(a_1,\dots,a_n))&=&f^{\fB}(h(a_1),\dots,h(a_n))\\
 R^{\fA}(a_1,\dots,a_n)&\Rightarrow&R^{\fB}(h(a_1),\dots,h(a_n))
 \end{array}
\end{equation*}

We denote this by
\begin{equation*}
 h:\fA\to\fB
\end{equation*}

If in addition \(h\) is injective and
\begin{equation*}
 R^{\fA}(a_1,\dots,a_n)\Leftrightarrow R^{\fB}(h(a_1),\dots,h(a_n))
\end{equation*}
for all \(a_1,\dots,a_n\in A\), then \(h\) is called an (isomorphic)
\textbf{embedding}. An \textbf{isomorphism} is a surjective embedding
\end{definition}

\begin{lemma}[]
\label{lemma1.1.8}
Let \(h:\fA \isoto\fA'\) be an isomorphism and \(\fB\) an
extension of \(\fA\). Then there exists an extension \(\fB'\) of \(\fA'\) and
an isomorphism \(g:\fB \isoto\fB'\) extending \(h\)
\end{lemma}

\begin{definition}[]
Let \((I,\le)\) be a \textbf{directed partial order}. This means that for all
\(i,j\in I\) there exists a \(k\in I\) s.t. \(i\le k\) and \(j\le k\). A
family \((\fA_i)_{i\in I}\) of \(L\)-structures is called \textbf{directed} if
\begin{equation*}
i\le j\Rightarrow\fA_i\subseteq\fA_j
\end{equation*}
If \(I\) is linearly ordered, we call \((\fA_i)_{i\in I}\) a \textbf{chain}
\end{definition}

If a structure \(\fA_1\) is isomorphic to a substructure \(\fA_0\) of itself,
\begin{equation*}
 h_0:\fA_0\isoto\fA_1
\end{equation*}
then Lemma \ref{lemma1.1.8} gives an extension
\begin{equation*}
 h_1:\fA_1\isoto\fA_2
\end{equation*}
Continuing in this way we obtain a chain 
\(\fA_0\subseteq \fA_1\subseteq\fA_2\subseteq\dots\)
and an increasing sequence
\(h_i:\fA_i\isoto\fA_{i+1}\) of isomorphism

\begin{lemma}[]
Let \((\fA_i)_{i\in I}\) be a directed family of \(L\)-structures. Then
\(A=\bigcup_{i\in I}A_i\) is the universe of a (uniquely determined)
\(L\)-structure
\begin{equation*}
\fA=\bigcup_{i\in I}\fA_i
\end{equation*}
which is an extension of all \(\fA_i\)
\end{lemma}

A subset \(K\) of \(L\) is called a \textbf{sublanguage}. An \(L\)-structure becomes a
\(K\)-structure, the \textbf{reduct}.
\begin{equation*}
\fA\restriction K=(A,(Z^{\fA})_{Z\in K})
\end{equation*}
Conversely we call \(\fA\) an \textbf{expansion} of \(\fA\restriction K\).
\begin{enumerate}
\item Let \(B\subseteq A\) , we obtain a new language
\begin{equation*}
L(B)=L\cup B
\end{equation*}
and the \(L(B)\)-structure 
\begin{equation*}
\fA_B=(\fA,b)_{b\in B}
\end{equation*}
Note that \(\Aut(\fA_B)\) is the group of automorphisms of \(\fA\) fixing
\(B\) elementwise. We denote this group by \(\Aut(\fA/B)\)
\end{enumerate}


Let \(S\) be a set, which we call the set of sorts. An \(S\)-sorted
language \(L\) is given by a set of constants for each sort in \(S\), and
typed function and relations. For any tuple \((s_1,\dots,s_n)\) and
\((s_1,\dots,s_n,t)\) there is a set of relation symbols and function
symbols respectively. An \(S\)-sorted structure is a pair
\(\fA=(A,(Z^{\fA})_{Z\in L})\), where 
\begin{alignat*}{2}      
&A&&\text{if a family $(A_s)_{s\in S}$ of non-empty sets}\\
&Z^{\fA}\in A_s&&\text{if $Z$ is a constant of sort $s\in S$}\\
&Z^{\fA}:A_{s_1}\times\dots\times A_{s_n}\to A_t&&\text{if $Z$ is a
function symbol of type $(s_1,\dots,s_n,t)$}\\
&Z^{\fA}\subseteq A_{s_1}\times\dots\times A_{s_n}&&\text{if $Z$ is a
relation symbol of type $(s_1,\dots,s_n)$}
\end{alignat*}

\begin{examplle}[]
Consider the two-sorted language \(L_{Perm}\) for permutation groups with a
sort \(x\) for the set and a sort \(g\) for the group. The constants and
function symbols for \(L_{Perm}\) are those of \(L_{Group}\) restricted to
the sort \(g\) and an additional function symbol \(\varphi\) of type \((x,g,x)\). Thus
an \(L_{Perm}\)-structure \((X,G)\) is given by a set \(X\) and an
\(L_{Group}\)-structure \(G\) together with a function \(X\times G\to X\)
\end{examplle}

\subsection{Language}
\label{sec:orge2a8efa}
\begin{lemma}[]
\label{lemma1.2.11}
Suppose \(\vv{b}\) and \(\vv{c}\) agree on all variables which are free
in \(\varphi\). Then 
\begin{equation*}
\fA\models\varphi[\vv{b}]\Leftrightarrow\fA\models\varphi[\vv{c}]
\end{equation*}
\end{lemma}

We define
\begin{equation*}
\fA\models\varphi[a_1,\dots,a_n]
\end{equation*}
by \(\fA\models\varphi[\vv{b}]\), where \(\vv{b}\) is an assignment
satisfying \(\vv{b}(x_i)=a_i\). Because of Lemma \ref{lemma1.2.11} this is
well defined.




Thus \(\varphi(x_1,\dots,x_n)\) defines an \(n\)-ary relation
\begin{equation*}
\varphi(\fA)=\{\bbar{a}\mid\fA\models\varphi[\bbar{a}]\}
\end{equation*}
on \(A\), the \textbf{realisation set} of \(\varphi\). Such realisation sets are called
\textbf{0-definable subsets} of \(A^n\), or 0-definable relations

Let \(B\) be a subset of \(A\). A \textbf{\(B\)-definable} subset of \(\fA\) is a set
of the form \(\varphi(\fA)\) for an \(L(B)\)-formula \(\varphi(x)\). We also say that \(\varphi\)
are defined \textbf{over} \(B\) and that the set \(\varphi(\fA)\) is defined by \(\varphi\). We call
two formulas \textbf{equivalent} if in every structure they define the same set.

Atomic formulas and their negations are called \textbf{basic}. Formulas without
quantifiers are Boolean combinations of basic formulas. It is convenient to
allow the empty conjunction and the empty disjunction. For that we introduce
two new formulas: the formula \(\top\), which is always true, and the formula
\(\bot\), which is always false. We define
\begin{align*}
&\bigwedge_{i<0}\pi_i=\top\\
&\bigvee_{i<0}\pi_i=\bot
\end{align*}

A formula is in \textbf{negation normal form} if it is built from basic formulas using
\(\wedge,\vee,exists,\forall\)

\begin{definition}[]
A formula in negation normal form which does not contain any existential
quantifier is called \textbf{universal}. Formulas in negation normal form without
universal quantifiers are called \textbf{existential}
\end{definition}


Let \(\fA\) be an \(L\)-structure. The \textbf{atomic diagram} of \(\fA\) is
\begin{equation*}
\Diag(\fA)=\{\varphi\text{ basic $L(A)$-sentence}\mid\fA_A\models\varphi\}
\end{equation*}

\begin{lemma}[]
The models of \(\Diag(\fa)\) are precisely those structures
\((\fB,h(a))_{a\in A}\) for embeddings \(h:\fA\to\fB\)
\end{lemma}

\subsection{Theories}
\label{sec:org81e7f35}
\begin{definition}[]
An \textbf{\(L\)-theory} \(T\) is a set of \(L\)-sentences
\end{definition}

A theory which has a model is a \textbf{consistent} theory. We call a set \(\Sigma\) of
\(L\)-formulas \textbf{consistent} if there is an \(L\)-structure and an assignment
\(\vv{b}\) s.t. \(\fA\models[\vv{b}]\) for all \(\varphi\in\Sigma\)

\begin{lemma}[]
\label{lemma1.3.4}
\begin{enumerate}
\item If \(T\models\varphi\) and \(T\models(\varphi\to\psi)\), then \(T\models\psi\)
\item If \(T\models\varphi(c_1,\dots,c_n)\) and the constants \(c_1,\dots,c_n\)
occur neither in \(T\) nor in \(\varphi(x_1,\dots,x_n)\), then \(T\models\forall
      x_1\dots x_n\varphi(x_1,\dots,x_n)\)
\end{enumerate}
\end{lemma}

\begin{proof}
\begin{enumerate}
\setcounter{enumi}{1}
\item Let \(L'=L\setminus\{c_1,\dots,c_n\}\). If the \(L'\)-structure is a
model of \(T\) and \(a_1,\dots,a_n\) are arbitrary elements, then
\((\fA,a_1,\dots,a_n)\models\varphi(c_1,\dots,c_n)\). This means
\(\fA\models\forall x_1\dots x_n\varphi(x_1,\dots,x_n)\).
\end{enumerate}
\end{proof}
\section{Elementary Extensions and Compactness}
\label{sec:org356ca42}
\subsection{Elementary substructures}
\label{sec:org1951443}
Let \(\fA,\fB\) be two \(L\)-structures. A map \(h:A\to B\) is called
\textbf{elementary} if for all \(a_1,\dots,a_n\in A\) we have
\begin{equation*}
\fA\models\varphi[a_1,\dots,a_n]\Leftrightarrow
\fB\models\varphi[h(a_1),\dots,h(a_n)]
\end{equation*}
We write
\begin{equation*}
h:\fA\embto\fB
\end{equation*}
\begin{lemma}[]
\label{lemma2.1.1}
The models of \(\Th(\fA_A)\) are exactly the structures of the form
\((\fB,h(a))_{a\in A}\) for elementary embeddings \(h:\fA\embto\fB\)
\end{lemma}

We call \(\Th(\fA_A)\) the \textbf{elemantary diagram} of \(\fA\)

A substructure \(\fA\) of \(\fB\) is called \textbf{elementary} if the inclusion map
is elementary. In this case we write
\begin{equation*}
\fA\prec\fB
\end{equation*}

\begin{theorem}[Tarski's Test]
Let \(\fB\) be an \(L\)-structure and \(A\) a subset of \(B\). Then \(A\) is
the universe of an elementary substructure iff every \(L(A)\)-formula
\(\varphi(x)\) which is satisfiable in \(\fB\) can be satisfied by an element of \(A\)
\end{theorem}

We use Tarski's Test to construct small elementary substructures

\begin{corollary}[]
Suppose \(S\) is a subset of the \(L\)-structure \(\fB\). Then \(\fB\) has a
elementary substructure \(\fA\) containing \(S\) and of cardinality at most
\begin{equation*}
\max(\abs{S},\abs{L},\aleph_0)
\end{equation*}
\end{corollary}

\begin{proof}
We construct \(A\) as the union of an ascending sequence \(S_0\subseteq
   S_1\subseteq\dots\) of subsets of \(B\). We start with \(S_0=S\). If \(S_i\)
is already defined, we choose an element \(a_\varphi\in B\) for every
\(L(S_i)\)-formula \(\varphi(x)\) which is satisfiable in \(\fB\) and define
\(S_{i+1}\) to be \(S_i\) together with these \(a_{\varphi}\).

An \(L\)-formula is a finite sequence of symbols from \(L\), auxiliary
symbols and logical symbols. These are
\(\abs{L}+\aleph_0=\max(\abs{L},\aleph_0)\) many symbols and there are
exactly\(\max(\abs{L},\aleph_0)\) many \(L\)-formulas

Let \(\kappa=\max(\abs{S},\abs{L},\aleph_0)\). There are \(\kappa\) many
\(L(S)\)-formulas: therefore \(\abs{S_1}\le\kappa\). Inductively it follows
for every \(i\) that \(\abs{S_i}\le\kappa\). Finally we have \(\abs{A}\le\kappa\cdot\aleph_0=\kappa\)
\end{proof}

A directed family \((\fA_i)_{i\in I}\) of structures is \textbf{elementary} if
\(\fA_i\prec\fA_j\) for all \(i\le j\)

\begin{theorem}[Tarski's Chain Lemma]
\label{thm2.1.4}
The union of an elementary directed family is an elementary extension of all
its members
\end{theorem}

\begin{proof}
Let \(\fA=\bigcup_{i\in I}(\fA_i)_{i\in I}\). We prove by induction on
\(\varphi(\bbar{x})\) that for all \(i\) and \(\bbar{a}\in\fA_i\)
\begin{equation*}
\fA_i\models\varphi(\bbar{a})\Leftrightarrow\fA\models\varphi(\bbar{a})
\end{equation*}
\end{proof}


\subsection{The Compactness Theorem}
\label{sec:orga9d7da3}
\begin{theorem}[Compactness Theorem]
Finitely satisfiable theories are consistent
\end{theorem}

Let \(L\) be a language and \(C\) a set of new constants. An \(L(C)\)-theory
\(T'\) is called a \textbf{Henkin theory} if for every \(L(C)\)-formula \(\varphi(x)\) there
is a constant \(c\in C\) s.t.
\begin{equation*}
\exists x\varphi(x)\to\varphi(c)\in T'
\end{equation*}

\begin{lemma}[]
Every finitely satisfiable \(L\)-theory \(T\) can be extended to a finitely
complete Henkin theory \(T^*\)
\end{lemma}

\begin{lemma}[]
Every finitely complete Henkin theory \(T^*\) has a model \(\fA\) (unique up
to isomorphism) consisting of constants; i.e.,
\begin{equation*}
(\fA,a_c)_{c\in C}\models T^*
\end{equation*}
with \(A=\{a_c\mid c\in C\}\)
\end{lemma}



\begin{corollary}[]
A set of formulas \(\Sigma(x_1,\dots,x_n)\) is consistent with \(T\) if and only
if every finite subset of \(\Sigma\) is consistent with \(T\)
\end{corollary}

\begin{proof}
Introduce new constants \(c_1,\dots,c_n\). Then \(\Sigma\) is consistent with \(T\) is
and only if \(T\cup\Sigma(c_1,\dots,c_n)\) is consistent. Now apply the
Compactness Theorem
\end{proof}

\begin{definition}[]
Let \(\fA\) be an \(L\)-structure and \(B\subseteq A\). Then \(a\in A\)
\textbf{realises} a set of \(L(B)\)-formulas \(\Sigma(x)\) if \(a\) satisfied all formulas
from \(\Sigma\). We write 
\begin{equation*}
\fA\models\Sigma(a)
\end{equation*}

We call \(\Sigma(x)\) \textbf{finitely satisfiable} in \(\fA\) if every finite subset of \(\Sigma\)
is realised in \(\fA\)
\end{definition}

\begin{lemma}[]
\label{lemma2.2.7}
The set \(\Sigma(x)\) is finitely satisfiable in \(\fA\) iff there is an
elementary extension of \(\fA\) in which \(\Sigma(x)\) is realised
\end{lemma}

\begin{proof}
By Lemma \ref{lemma2.1.1} \(\Sigma\) is realised in an elementary extension of \(\fA\)
iff \(\Sigma\) is consistent with \(\Th(\fA_A)\). So the lemma follows from the
observation that a finite set of \(L(A)\)-formulas is consistent with
\(\Th(\fA_A)\) iff it is realised in \(\fA\)
\end{proof}

\begin{definition}[]
Let \(\fA\) be an \(L\)-structure and \(B\) a subset of \(A\). A set \(p(x)\)
of \(L(B)\)-formulas is a \textbf{type} over \(B\) if \(p(x)\) is maximal finitely
satisfiable in \(\fA\). We call \(B\) the \textbf{domain} of \(p\). Let
\begin{equation*}
S(B)=S^{\fA}(B)
\end{equation*}
denote the set of types over \(B\).
\end{definition}

Every element \(a\) of \(\fA\) determines a type
\begin{equation*}
\tp(a/B)=tp^{\fA}(a/B)=\{\varphi(x)\mid\fA\models\varphi(a),\varphi\text{ an $L(B)$-formula}\}
\end{equation*}
So an element \(a\) realises the type \(p\in S(B)\) exactly if
\(p=\tp(a/B)\). If \(\fA'\) is an elementary extension of \(\fA\), then
\begin{equation*}
S^{\fA}(B)=S^{\fA'}(B)\quad\text{ and }\quad
\tp^{\fA'}(a/B)=\tp^{\fA}(a/B)
\end{equation*}
If \(\fA'\models p(x)\) then \(\fA'\models\exists xp(x)\), so
\(\fA\models\exists xp(x)\).

We use the notation \(\tp(a)\) for \(\tp(a/\emptyset)\)

Maximal finitely satisfiable sets of formulas in \(x_1,\dots,x_n\) are called
\textbf{\(n\)-types} and
\begin{equation*}
S_n(B)=S_N^{\fA}(B)
\end{equation*}
denotes the set of \(n\)-types over \(B\).
\begin{equation*}
\tp(C/B)=\{\varphi(x_{c_1},\dots,x_{c_n})\mid\fA\models\varphi(c_1,\dots,c_n),\varphi
\text{ an $L(B)$-formula}\}
\end{equation*}

\begin{corollary}[]
Every structure \(\fA\) has an elementary extension \(\fB\) in which all
types over \(A\) are realised
\end{corollary}

\begin{proof}
We choose for every \(p\in S(A)\) a new constant \(c_p\). We have to find a
model of
\begin{equation*}
\Th(\fA_A)\cup\bigcup_{p\in S(A)}p(c_p)
\end{equation*}
This theory is finitely satisfiable since every \(p\) is finitely satisfiable
in \(\fA\).

Or use Lemma \ref{lemma2.2.7}. Let \((p_\alpha)_{\alpha<\lambda}\) be an enumeration of
\(S(A)\). Construct an elementary chain
\begin{equation*}
\fA=\fA_0\prec\fA_1\prec\dots\prec\fA_\beta\prec\dots(\beta\le\lambda)
\end{equation*}
s.t. each \(p_\alpha\) is realised in \(\fA_{\alpha+1}\) (by recursion
theorem on ordinal numbers)

Suppose that the elementary chain \((\fA_{\alpha'})_{\alpha'<\beta}\) is already
constructed. If \(\beta\) is a limit ordinal, we let
\(\fA_\beta=\bigcup_{\alpha<\beta}\fA_\alpha\), which is elementary by Lemma \ref{thm2.1.4}. If
\(\beta=\alpha+1\) we  first note that \(p_\alpha\) is also finitely
satisfiable in \(\fA_\alpha\), therefore we can realise \(p_\alpha\) in a
suitable elementary extension \(\fA_\beta\succ\fA_\alpha\) by Lemma
\ref{lemma2.2.7}. Then \(\fB=\fA_\lambda\) is the model we were looking for
\end{proof}
\section{Quantifier Elimination}
\label{sec:org446532a}
\subsection{Preservation theorems}
\label{sec:org975cc22}
\begin{lemma}[Separation Lemma]
Let \(T_1, T_2\) be two theories. Assume \(\calh\) is a set of sentences
which is closed under \(\wedge,\vee\) and contains \(\bot\) and \(\top\).
Then the following are equivalent
\begin{enumerate}
\item There is a sentence \(\varphi\in\calh\) which separates \(T_1\) from
\(T_2\). This means
\begin{equation*}
  T_1\models\varphi \quad\text{ and }\quad
  T_2\models\neg\varphi
\end{equation*}
\item All models \(\fA_1\) of \(T_1\) can be separated from all models \(\fA_2\)
of \(T_2\) by a sentence \(\varphi\in\calh\). This means
\begin{equation*}
  \fA_1\models\varphi \quad\text{ and }\quad\fA_2\models\neg\varphi
\end{equation*}
\end{enumerate}
\end{lemma}

\begin{proof}
\(2\to1\). For any model \(\fA_1\) of \(T_1\) let \(\calh_{\fA_1}\) be the
set of all sentences from \(\calh\) which are true in \(\fA_1\). (2) implies
that \(\calh_{\fA_1}\) and \(T_2\) cannot have a common model. By the
Compactness Theorem there is a finite conjunction \(\varphi_{\fA_1}\) of
sentences from \(\calh_{\fA_1}\) inconsistent with \(T_2\). Clearly
\begin{equation*}
 T_1\cup\{\neg\varphi_{\fA_1}\mid\fA_1\models T_1\}
\end{equation*}
is inconsistent. Again by compactness \(T_1\) implies a disjunction \(\varphi\) of
finitely many of the \(\varphi_{\fA_1}\)
\end{proof}

For structures \(\fA,\fB\) and a map \(f:A\to B\) preserving all formulas
from a set of formulas \(\Delta\), we use the notation
\begin{equation*}
f:\fA\to_\Delta\fB
\end{equation*}
We also write
\begin{equation*}
\quad\fA\Rightarrow_\Delta\fB
\end{equation*}
to express that all sentences from \(\Delta\) true in \(\fA\) are also true in \(\fB\)

\begin{lemma}[]
\label{lemma3.1.2}
Let \(T\) be a theory, \(\fA\) a structure and \(\Delta\) a set of formulas, closed
under existential quantification, conjunction and substitution of variables.
Then the following are equivalent
\begin{enumerate}
\item All sentences \(\varphi\in\Delta\) which are true in \(\fA\) are
consistent with \(T\) (There is a model \(\fB\models\Delta\cup T\) and \(\fA\Rightarrow_\Delta\fB\))
\item There is a model \(\fB\models T\) and a map \(f:\fA\to_\Delta\fB\)
\end{enumerate}
\end{lemma}

\begin{proof}
\(1\to2\). Consider \(\Th_\Delta(\fA_A)\), the set of all sentences
\(\delta(\bbar{a})\)
(\(\delta(\bbar{x})\in\Delta\)), which are true in \(\fA_A\). The models
\((\fB,f(a)_{a\in A})\) of this theory correspond to maps
\(f:\fA\to_\Delta\fB\). \textbf{This means that we have to find a model of}
\(T\cup\Th_\Delta(\fA_A)\). To show finite satisfiability it is enough to
show that \(T\cup D\) is consistent for every finite subset \(D\) of
\(\Th_\Delta(\fA_A)\). Let \(\delta(\bbar{a})\) be the conjunction of the elements
of \(D\). Then \(\fA\) is a model of \(\varphi=\exists\bbar{x}\delta(\bbar{x})\)
\end{proof}

Lemma \ref{lemma3.1.2} applied to \(T=\Th(\fB)\) shows that
\(\fA\Rightarrow_\Delta\fB\) iff there exists a map \(f\) and a structure
\(\fB'\equiv\fB\) s.t. \(f:\fA\to_\Delta\fB'\)

\begin{theorem}[]
\label{thm3.1.3}
Let \(T_1\) and \(T_2\) be two theories. Then the following are equivalent
\begin{enumerate}
\item There is a universal sentence which separates \(T_1\) from \(T_2\)
\item No model of \(T_2\) is a substructure of a model of \(T_1\)
\end{enumerate}
\end{theorem}

\begin{proof}
\(2\to1\). If \(T_1\) and \(T_2\) cannot be separated by a universal
sentence, then they have models \(\fA_1\) and \(\fA_2\) which cannot be separated
by a universal sentence. This can be denoted by
\begin{equation*}
\fA_2\Rightarrow_\exists\fA_1
\end{equation*}
 Now Lemma \ref{lemma3.1.2} implies that \(\fA_2\) there is a map
 \(\fA_2\to_\exists\fA_1'\) where \(\fA_1'\models T_1\)
. Hence \(\fA_2\) has an extension \(\fA_2'\) s.t. \(\fA_2'\equiv\fA_1'\).
Then \(\fA'\) is gain a model of \(T_1\) contradicting (2)
\end{proof}

\begin{definition}[]
For any \(L\)-theory \(T\), the formulas \(\varphi(\bbar{x}),\psi(\bbar{x})\) are said
to be \textbf{equivalent} modulo \(T\) (or relative to \(T\)) if \(T\models\forall\bbar{x}(\varphi(\bbar{x})\leftrightarrow\psi(\bbar{x}))\)
\end{definition}

\begin{corollary}[]
Let \(T\) be a theory
\begin{enumerate}
\item Consider a formula \(\varphi(x_1,\dots,x_n)\). The following are equivalent
\begin{enumerate}
\item \(\varphi(x_1,\dots,x_n)\) is, modulo \(T\), equivalent to a universal formula
\item If \(\fA\subseteq\fB\) are models of \(T\) and \(a_1,\dots,a_n\in A\),
then \(\fB\models\varphi(a_1,\dots,a_n)\) implies \(\fA\models\varphi(a_1,\dots,a_n)\)
\end{enumerate}
\item We say that a theory which consists of universal sentences is universal.
Then \(T\) is equivalent to a universal theory iff all substructures of
models of \(T\) are again models of \(T\)
\end{enumerate}
\end{corollary}

\begin{proof}
\begin{enumerate}
\item Assume (2). We extend \(L\) by an \(n\)-tuple \(\bbar{c}\) of new
constants \(c_1,\dots,c_n\) and consider theory
\begin{equation*}
T_1=T\cup\{\varphi(\bbar{c})\}\quad\text{ and }\quad
T_2=T\cup\{\neg\varphi(\bbar{c})\}
\end{equation*}
Then (2) says the substructures of models of \(T_1\) cannot be models of
\(T_2\). By Theorem \ref{thm3.1.3} \(T_1\) and \(T_2\) can be separated by a
universal \(L(\bbar{c})\)-sentence \(\psi(\bbar{c})\). By Lemma
\ref{lemma1.3.4}, \(T_1\models\psi(\bbar{c})\) implies
\begin{equation*}
T\models\forall\bbar{x}(\varphi(\bbar{x})\to\psi(\bbar{x}))
\end{equation*}
and from \(T_2\models\neg\psi(\bbar{c})\) we see
\begin{equation*}
T\models\forall\bbar{x}(\neg\varphi(\bbar{x})\to\neg\psi(\bbar{x}))
\end{equation*}
\item Suppose a theory \(T\) has this property. Let \(\varphi\) be an axiom of \(T\). If
\(\fA\) is a substructure of \(\fB\), it is not possible for \(\fB\) to be
a model of \(T\) and for \(\fA\) to be a model of \(\neg\psi\) at the same
time. By Theorem \ref{thm3.1.3} there is a universal sentence \(\psi\) with
\(T\models\psi\) and \(\neg\varphi\models\neg\psi\). Hence all axioms of
\(T\) follow from
\begin{equation*}
T_\forall=\{\psi\mid T\models\psi,\psi\text{ universal}\}
\end{equation*}
\end{enumerate}
\end{proof}

An \(\forall\exists\)-formula is of the form
\begin{equation*}
\forall x_1\dots x_n\psi
\end{equation*}
where \(\psi\) is existential
\begin{lemma}[]
Suppose \(\varphi\) is an \(\forall\exists\)-sentence, \((\fA_i)_{i\in I}\) is a
directed family of models of \(\varphi\) and \(\fB\) the union of the \(\fA_i\). Then
\(\fB\) is also a model of \(\varphi\).
\end{lemma}

\begin{proof}
Write
\begin{equation*}
\varphi=\forall\bbar{x}\psi(\bbar{x})
\end{equation*}
where \(\psi\) is existential. For any \(\bbar{a}\in B\) there is an \(A_i\)
containing \(\bbar{a}\), clearly \(\psi(\bbar{a})\) holds in \(\fA_i\). As
\(\psi(\bbar{a})\) is existential it must also hold in \(\fB\)
\end{proof}

\begin{definition}[]
We call a theory \(T\) \textbf{inductive} if the union of any directed family of
models of \(T\) is again a model
\end{definition}

\begin{theorem}[]
Let \(T_1\) and \(T_2\) be two theories. Then the following are equivalent
\begin{enumerate}
\item there is an \(\forall\exists\)-sentence which separates \(T_1\) and \(T_2\)
\item No model of \(T_2\) is the union of a chain (or of a directed family) of
models of \(T_1\)
\end{enumerate}
\end{theorem}


\begin{proof}
\(2\to1\). If (1) is not true, \(T_1,T_2\) have models which cannot be
separated by an \(\forall\exists\)-sentence. Since
\(\exists\forall\)-formulas are equivalent to negated
\(\forall\exists\)-formulas, we have
\end{proof}
\end{document}